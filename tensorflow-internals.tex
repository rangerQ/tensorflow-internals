%%%%%%%%------------------------------------------------------------------------
% This program is free software: you can redistribute it and/or modify
% it under the terms of the GNU General Public License as published by
% the Free Software Foundation, either version 3 of the License, or
% (at your option) any later version.
% 
% This program is distributed in the hope that it will be useful,
% but WITHOUT ANY WARRANTY; without even the implied warranty of
% MERCHANTABILITY or FITNESS FOR A PARTICULAR PURPOSE.  See the
% GNU General Public License for more details.
% 
% You should have received a copy of the GNU General Public License
% along with this program.  If not, see <http://www.gnu.org/licenses/>.

%%%%%%%%------------------------------------------------------------------------
%%%% Introduction area
%% Document type is article
\documentclass[a4paper, 10pt]{book}
%1m = 39.4 inch
%大18开 (18.5cm * 23cm)
%\usepackage[left=3.25cm, right=3.25cm, top=2.3cm,bottom=1.4cm]{geometry}
\usepackage{geometry}
\geometry{left=3.75cm,right=3.25cm,top=3cm,bottom=2.5cm}

%% en_preamble Contains basic package configuration
%%%%%%%% ------------------------------------------ ------------------------------
%%%% daily package

%% sets the line spacing
\usepackage{setspace}

%% control item list
\usepackage{enumerate}

%% multi-column display
\usepackage{multicol}

%% hyperref package, generates hyperlinks that can locate clicks, and generates pdf bookmarks
\usepackage[%
    pdfstartview = FitH,%
    CJKbookmarks=true,%
    bookmarks=true,%
    bookmarksnumbered=true,%
    bookmarksopen=true,%
    colorlinks=true,%
    citecolor=blue,%
    linkcolor=blue,%
    anchorcolor=green,%
    urlcolor=blue%
]{hyperref}

%% control title
\usepackage{titlesec}

%% control table style
\ usepackage {booktabs}

%% control directory
\usepackage{titletoc}

%% controls font size
\usepackage{type1cm}

%% indent the first line, cancel some indentation with \noindent
\usepackage{indentfirst}

%% supports color text, background, text box, etc.
\usepackage{color,xcolor}

%% AMS LaTeX package
\usepackage{amsmath}
\usepackage{amssymb}

%% some special symbols
% \usepackage{bbding}

%% support reference
% \usepackage{cite}

%% LaTeX some special symbol package
% \usepackage{latexsym}

%% black italic in math formula
% \usepackage{bm}

%% adjust formula font size: \mathsmaller, \mathlarger
% \usepackage{relsize}

%% build index
% \ makeindex

%%%% basic illustration method
%% graphics package
\usepackage{graphicx}
\usepackage{float}
%% If the inserted image does not specify an extension, search for the file corresponding to the extension below.
\DeclareGraphicsExtensions{.pdf,.eps,.png,.jpg}
%% lets latex read Bounding Box information from .bb
% \ DeclareGraphicsRule {.jpg} {eps} {.bb} {}
%\DeclareGraphicsRule{.png}{eps}{.bb}{}
% \ DeclareGraphicsRule {.pdf} {eps} {.bb} {}

%% multiple graphics side by side, participate in lnotes.pdf
%\usepackage{subfig}
\usepackage{subfigure}


\usepackage{caption}
\captionsetup{font={sf, scriptsize}, labelfont={bf}, skip=15pt}
\DeclareCaptionLabelSeparator{colon}{~~}

\usepackage[perpage,stable]{footmisc}

\usepackage{longtable}
% \begin{figure}[htbp] %% Controls the position of the illustration
%   \setlength{\abovecaptionskip}{0pt}
%   \setlength{\belowcaptionskip}{10pt}
                                     %% controls the distance between graphics and context
% \centering %% centered the graphic
%   \includegraphics[width=0.8\textwidth]{CTeXLive2008.jpg}
                                     %% control graphic display width is 0.8\textwidth
% \caption{CTeXLive2008 installation process} \label{fig:CTeXLive2008}
                                     %% graphic title and cross-reference label
% \end{figure}
%%%% Basic illustration method ends

%%%% pgf/tikz drawing package settings
\ Usepackage {pgf, TikZ}
\usetikzlibrary{shapes,automata,snakes,backgrounds,arrows}
\usetikzlibrary{mindmap, trees,  calendar}
\ usetikzlibrary {positioning}
\usepackage{pgf-umlsd}
%% can use the graphviz/dot language directly in the latex documentation.
%% can also use the dot2tex tool to convert the dot file to a tex file and then include it.
%% \usepackage[shell,pgf,outputdir={docgraphs/}]{dot2texi}
%%%% pgf/tikz setting ends


%%%% fancyhdr set header and footer
%% header and footer package
\usepackage{fancyhdr}

%% header and footer style
\pagestyle{fancy}

%% these two lines of code are the classic mode for modifying \leftmark and \rightmark
\renewcommand{\chaptermark}[1]{\markboth{{\hei {Chapter \thechapter{}:}}\hspace 1  #1}{}}
\renewcommand{\sectionmark}[1]{\markright{\thesection{} #1}}

%% clears the default settings for the current header and footer
\fancyhf{}

%\fancyhead[L]{\scriptsize \fangsong \ascii{ZTE}ZTE}
%\fancyhead[R]{\scriptsize \fangsong Internal Public}

%\fancyhead[CE]{\scriptsize \fangsong \leftmark}
%\fancyhead[CO]{\scriptsize \fangsong \rightmark}

%\fancyfoot[RO, LE]{\scriptsize \thepage}
%\fancyfoot[C]{\scriptsize \fangsong All information in this article is internal information of ZTE Corporation and may not be transmitted outside}

\renewcommand{\headrulewidth}{0.4pt}
\renewcommand{\footrulewidth}{0.4pt}

% {\couriernew\thechapter{}} chapter
%% starts modifying headers and footers below
\fancyhead[RE]{\fangsong \leftmark}
\fancyhead[LO]{\fangsong \rightmark}
\fancyhead[RO, LE]{\small \thepage}
\fancypagestyle{plain}{%
  \fancyhead{} % get rid of headers
  \renewcommand{\headrulewidth}{0pt} % and the line.
}

%% defines a blank page
\makeatletter
\def\cleardoublepage{\clearpage\if@twoside \ifodd\c@page\else
  \hbox{}
  \vspace*{\fill}
  \begin{center}
   {\sffamily\large}
   \end{center}
   \vspace{\fill}
   \thispagestyle{empty}
   \newpage
   \if@twocolumn\hbox{}\newpage\fi\fi\fi}
\makeatother

\makeatletter
\def\cleardedicatepage{\clearpage
  \hbox{}
  \vspace*{\fill}
  \begin{center}
   {\sffamily\Large dedicated to my daughter Liu Chuxi}
   \end{center}
   \vspace{\fill}
   \thispagestyle{empty}
   \newpage
   \if@twocolumn\hbox{}\newpage\fi}
\makeatother

%% sometimes there is a warning of \headheight too small
\setlength{\headheight}{15pt}
%%%% fancyhdr set end

%% sets the distance between lines
\usepackage{framed}  
%%%% listings package end

%%%% Appendix settings
\usepackage[title,titletoc,header]{appendix}
%%%% End of appendix setting

%%%% End of daily macro package setting
%%%%%%%% ------------------------------------------ ------------------------------

%%%%%%%% ------------------------------------------ ------------------------------
%%%% English font setting ends
%% can add your own English font settings here.
%%%%%%%% ------------------------------------------ ------------------------------

%%%%%%%% ------------------------------------------ ------------------------------
%%%% Set common font size, corresponding to MS Word

%% number one, 1.4 times line spacing
\newcommand{\yihao}{\fontsize{26pt}{36pt}\selectfont}
%% number two, 1.25 times line spacing
\newcommand{\erhao}{\fontsize{22pt}{28pt}\selectfont}
%% small two, single line spacing
\newcommand{\xiaoer}{\fontsize{18pt}{18pt}\selectfont}
%% third, 1.5 times line spacing
\newcommand{\sanhao}{\fontsize{16pt}{24pt}\selectfont}
%% small three, 1.5 times line spacing
\newcommand{\xiaosan}{\fontsize{15pt}{22pt}\selectfont}
%% four, 1.5 times the line spacing
\newcommand{\sihao}{\fontsize{14pt}{21pt}\selectfont}
%% half four, 1.5 times line spacing
\newcommand{\bansi}{\fontsize{13pt}{19.5pt}\selectfont}
%% small four, 1.5 times line spacing
\newcommand{\xiaosi}{\fontsize{12pt}{18pt}\selectfont}
%% big five, single line spacing
\newcommand{\dawu}{\fontsize{11pt}{11pt}\selectfont}
%% number five, single line spacing
\newcommand{\wuhao}{\fontsize{10.5pt}{10.5pt}\selectfont}
%%%%%%%% ------------------------------------------ ------------------------------

%%%%%%%% ------------------------------------------ ------------------------------
%%%% some personality settings

%% set the page number mode, including arabic, roman, etc.
%% \pagenumbering{arabic}

%% Sometimes LaTeX has no line breaks, generating overfull errors. This command lowers LaTeX line break criteria.
%% \sloppy

%% set the directory display depth \tableofcontents
%% \setcounter{tocdepth}{2}
%% set the depth of \listoftables display
%% \setcounter{lotdepth}{2}
%% set the depth of the \listoffigures display
%% \setcounter{lofdepth}{2}

%% Chinese dash, said to come from Tsinghua template
\newcommand{\pozhehao}{\kern0.3ex\rule[0.8ex]{2em}{0.1ex}\kern0.3ex}

%% sets the symbol of the itemize environment item
\renewcommand{\labelitemi}{$\bullet$}

% \ Makeatletter
%\@addtoreset{lstlisting}{section} 
% \ Confectionery Other

\newenvironment{enum}
{
  \begin{spacing}{1.2}
  % \begin{enumerate}[1.]
  \begin{enumerate}
    \setlength{\itemsep}{0pt} 
    \ setlength {\ itemindent} {2em}
    % \ Setlength {\ rail dent par} {} 2em
} {%
  \end{enumerate}
  \end{spacing}
}

\newenvironment{nitemize}
{
  \begin{itemize}
    \setlength{\itemsep}{0pt} 
    \ setlength {\ itemindent} {2em}
} {%
  \end{itemize}
}


\newcommand{\suggest}[1]{
\tikzstyle{mybox} = [draw=black, very thick,
rectangle, rounded corners, inner sep=9pt, inner ysep=20pt]
\tikzstyle{fancytitle} =[fill=white, text=black, ellipse]
\noindent
\begin{tikzpicture}
\node [mybox] (box){%
\begin{minipage}{\textwidth}
\fangsong
#one
\end{minipage}
};
\node[fancytitle, right=10pt] at (box.north west) {\emph{建议}};
% \node[fancytitle, rounded corners] at (box.east) {$\clubsuit$};
\end{tikzpicture}
}

\newcommand{\notice}[1]{
\tikzstyle{mybox} = [draw=black, very thick,
rectangle, rounded corners, inner sep=9pt, inner ysep=20pt]
\tikzstyle{fancytitle} =[fill=white, text=black]
\noindent
\begin{tikzpicture}
\node [mybox] (box){%
\begin{minipage}{\textwidth}
\fangsong
#one
\end{minipage}
};
\node[fancytitle, right=10pt] at (box.north west) {\emph{注意}};
%\node[fancytitle, rounded corners] at (box.east) {$\clubsuit$};
\end{tikzpicture}
}

\newcommand{\tip}[1]{
\tikzstyle{mybox} = [draw=black, very thick,
rectangle, rounded corners, inner sep=9pt, inner ysep=20pt]
\tikzstyle{fancytitle} =[fill=white, text=black]
\noindent
\begin{tikzpicture}
\node [mybox] (box){%
\begin{minipage}{\textwidth}
\fangsong
#one
\end{minipage}
};
\node[fancytitle, right=10pt] at (box.north west) {\emph{提示}};
%\node[fancytitle, rounded corners] at (box.east) {$\clubsuit$};
\end{tikzpicture}
}

\newcommand\refch[1]{\ascii{第\ref{ch:#1}章(\nameref{ch:#1})}}
\newcommand\refsec[1]{\ascii{\ref{sec:#1}节(\nameref{sec:#1})}}

\ newcommand \ eitem [1] {\ item {\ itshape {# 1}}}
\ newcommand \ cpp {\ ascii {C \ nobreak + \ nobreak +}}
\newcommand\clang{\ascii{C}}
\newcommand\tf{\ascii{TensorFlow}}

\ Newcommand \ where [1] { "# 1"}

\newcommand\percent[1]{\ascii{#1\%}}

\newcommand{\trans}{\emph{transaction}}
\newcommand{\act}{\emph{operation}}
\newcommand{\seqact}{\emph{sequence operation}}
\newcommand{\conact}{\emph{concurrent operations}}
\newcommand{\atomact}{\emph{Basic Operations}}
\newcommand{\syncact}{\emph{sync operation}}
\newcommand{\asynact}{\emph{Asynchronous Operation}}
\ newcommand {\ action} [1] {\ emph {\ ascii {\ file \ _ \ _ # 1}}}
\newcommand{\sigwait}{\action{sig\_wait}}
\newcommand{\sigsync}{\action{sig\_sync}}
\newcommand{\sigreply}{\action{sig\_reply}}
\newcommand{\timerprot}{\action{timer\_prot}}
\newcommand{\unknownevet}{\ascii{UNKNOWN\_EVENT}}
\newcommand{\transdsl}{\ascii{Transaction DSL}}
\newcommand{\oper}[1]{\ascii{Action#1}}
\newcommand{\protproc}{\ascii{prot\_procedure}}

%\newcommand{\code}[1]{\ascii{\small{\texttt{#1}}}}
\newcommand{\code}[1]{\ascii{\footnotesize{\texttt{#1}}}}
\newcommand{\script}[1]{\ascii{\scriptsize{\texttt{#1}}}}


\newcommand{\inlinetitle}[1]{\large{\emph{#1}}}

%\newcommand{\Email}{\begingroup \def\UrlLeft{<}\def\UrlRight{>} \urlstyle{tt}\Url}
%\def\mailto|#1|{\href{mailto:#1}{Email|#1|}}
\newcommand{\contrib}[2]{#1\quad{\small\quad\textit{#2}}\\[1ex]}


\ Newcommand {\ upcite} [1] {\ textsuperscript {\ cite {# 1}}}
%% set the font size
% \renewcommand{\normalsize}{\sihao}

%%%% End of personality setting
%%%%%%%% ------------------------------------------ ------------------------------


%%%%%%%% ------------------------------------------ ------------------------------
%%%% bibtex settings

%% set reference display style

%%%% end of bibtex setting
%%%%%%%% ------------------------------------------ ------------------------------

%% If you don't write Chinese, you don't need to quote xecjk_preamble inside configuration.
%%%%%%%% ------------------------------------------ ------------------------------
%%%% xeCJK related package

\usepackage{xltxtra,fontspec,xunicode}

%% \ CJKsetecglue {\ hskip 0.15em plus 0.05em minus 0.05th}
%% slanfont: Allow italic
%% boldfont: Allow bold
%% CJKnormalspaces: Only whitespace between Chinese characters is ignored, but the gap between Chinese and English is preserved. 
%% CJKchecksingle: Avoid single characters in a single line.
\usepackage[slantfont, boldfont]{xeCJK} 
% \usepackage{ctex}

%% breaks the line for Chinese
\XeTeXlinebreaklocale "zh"             

%% gives TeX a certain degree of freedom
\XeTeXlinebreakskip = 0pt plus 1pt minus 0.1pt

%%%% xeCJK setting ends                                       
%%%%%%%% ------------------------------------------ ------------------------------

%%%%%%%% ------------------------------------------ ------------------------------
%%%% xeCJK font settings

%% Set Chinese punctuation style, support quanjiao, banjiao, kaiming, etc.
\punctstyle{quanjiao}                                        
                                                     
%% set default Chinese font
\setCJKmainfont[BoldFont={Adobe Heiti Std}, ItalicFont={Adobe Kaiti Std}]{Adobe Song Std}   %  FZBaoSongZ04
%% set Chinese sans serif font
\setCJKsansfont[BoldFont={Adobe Heiti Std}, ItalicFont={Adobe Kaiti Std}]{Adobe Kaiti Std} 
%% set the monospaced font
\setCJKmonofont{Adobe Heiti Std}                           
%\setCJKmonofont{Monaco}                           

%% English serif font
\setmainfont{Lucida Bright}                                 
%% English monospaced font
%\setmonofont{Courier}
\setmonofont{Monaco}                             
%\setmonofont{Consolas}                             
%% English sans serif font
\setsansfont{Optima}                                  

%% defines new font
\setCJKfamilyfont{song}{Adobe Song Std}                     
\setCJKfamilyfont{kai}{Adobe Kaiti Std}
\setCJKfamilyfont{hei}{Adobe Heiti Std}
\setCJKfamilyfont{fangsong}{Adobe Song Std}
\setCJKfamilyfont{lisu}{LiShu}
\setCJKfamilyfont{youyuan}{Adobe Kaiti Std}

%% custom English font
\newfontfamily\couriernew{Lucida Grande}
\newfontfamily\optima{Optima}
\newfontfamily\lucida{Lucida Bright}

\newcommand{\ascii}[1]{{\sffamily #1}}
\newcommand{\speak}[1]{{\itshape #1}}
\renewcommand{\emph}[1]{{\hei #1}}

%% custom Song
\newcommand{\song}{\CJKfamily{song}}                       
%% custom body
\newcommand{\kai}{\CJKfamily{kai}}                         
%% custom blackbody
\newcommand{\hei}{\CJKfamily{hei}}                        
%% custom imitation
\newcommand{\fangsong}{\CJKfamily{fangsong}}              
%% custom librarian
\newcommand{\lisu}{\CJKfamily{lisu}}                       
%% custom young round
\newcommand{\youyuan}{\CJKfamily{youyuan}}                 

%%%% xeCJK font setting ends
%%%%%%%% ------------------------------------------ ------------------------------

%%%%%%%% ------------------------------------------ ------------------------------
%%%% Some redefinitions about Chinese documents

%% Redefinition of the %% mathematical formula theorem

\newtheorem{example}{example}[section]                                   
\newtheorem{algorithm}{algorithm}
%% by section number
\newtheorem{theorem}{Theorem}[section]                         
\newtheorem{definition}{definition}
\newtheorem{axiom}{Axiom}
\newtheorem{property}{nature}
\newtheorem{proposition}{proposition}
\newtheorem{lemma}{lemma}
\newtheorem{corollary}{inference}
\newtheorem{condition}{conditions}
\newtheorem{conclusion}{Conclusion}
\newtheorem{assumption}{hypothesis}

\newtheorem{principle}{principle}[section]
\newtheorem{regulation}{rules}[section]
\newtheorem{advise}{recommendation}[section]
\newtheorem{concept}{concept}[section]

\usepackage{titlesec}

\renewcommand{\partname}{}
\renewcommand{\thepart}{Section \Roman{part}}

%% chapter and other name redefinition
\renewcommand{\contentsname}{catalog}
%\renewcommand{\abstractname}{Abstract}
\renewcommand{\indexname}{index}
\renewcommand{\listfigurename}{Illustration directory}
\renewcommand{\listtablename}{table directory}
\renewcommand{\figurename}{fig.~}
\renewcommand{\tablename}{table~}
\renewcommand{\appendixname}{Appendix}
\renewcommand{\appendixpagename}{Appendix}
\renewcommand{\appendixtocname}{Appendix}
%\renewcommand\refname{References} 

%% set content environment
\newenvironment{content}{
  \setlength{\parskip}{0.5\baselineskip}
  \begin{spacing}{1.5}
}{
  \end{spacing}
  \setlength{\parskip}{-0.5\baselineskip}
  \vskip -0.5\baselineskip
}

% The syntax for inserting a short story:
% \begin{story}
%   \begin{center}
% \inlinetitle{watershed}
%   \end{center}
% \end{story}
\newenvironment{story}
{
  \setlength{\parskip}{0.5\baselineskip}
  \hbox to \textwidth{\hfil\rule{\linewidth}{0.5mm}\hfil}
  \begin{spacing}{1.5}
}{%
  \end{spacing}
  \hbox to \textwidth{\hfil\rule{\linewidth}{0.5mm}\hfil}
  \setlength{\parskip}{-0.5\baselineskip}
  \vskip -0.5\baselineskip
}

%% sets the format of chapter, section and subsection
%\titleformat{\chapter}[display]{\flushright\yihao}{\thechapter{}}{1em}{\textbf}
\titleformat{\section}[block]{\flushleft\sanhao}{\optima{\thesection}}{1em}{\textbf}
\titleformat{\subsection}{\sihao}{\optima{\thesubsection}}{0.5em}{\textbf}
\titleformat{\subsubsection}{\xiaosi}{\thesubsubsection}{0.5em}{\textbf}

%\titlespacing{\chapter}{0pt}{0pt}{-\baselineskip}
%\titlespacing{\section}{0pt}{0pt}{0\baselineskip}
%\titlespacing{\subsection}{0pt}{0.5\baselineskip}{0\baselineskip}

%% set chapter format
\usepackage{quotchap}

\renewcommand\chapterheadstartvskip{
   \vspace*{-5\baselineskip}
}

\renewcommand\chapterheadendvskip{
   \vspace*{0.5\baselineskip}
}

\usepackage{helvet}
\renewcommand\sectfont{\rmfamily\bfseries}

\newcommand\refig[1] {{\itshape\figurename\ascii{\ref{fig:#1}(page \pageref{fig:#1})}}}
\newcommand\reftbl[1]{{\itshape\tablename\ascii {\ref{tbl:#1}(page \pageref{tbl:#1})}}}

\renewcommand{\footnoterule}{\vspace*{3pt}%
  \hrule width 0.382\textwidth height 0.4pt \vspace*{2.6pt}}

% Remark
\newenvironment{remark}{\par\vskip10pt\footnotesize\itshape % Vertical white space above the remark and smaller font size
\begin{list}{}{
\leftmargin=35pt % Indentation on the left
\rightmargin=25pt}\item\ignorespaces % Indentation on the right
\makebox[-2.5pt]{\begin{tikzpicture}[overlay]
\node[draw=red!60,line width=1pt,circle,fill=red!25,font=\sffamily\bfseries,inner sep=2pt,outer sep=0pt] at (-15pt,0pt){\textcolor{red}{R}};\end{tikzpicture}}
\advance\baselineskip -1pt}{\end{list}\vskip5pt}

%%%% Chinese redefinition end
%%%%%%%% ------------------------------------------ ------------------------------

%%%% set the listings package to paste the source code
%% easy to paste source code, some code highlighting function
\usepackage{listings}
\usepackage{color}

\DeclareCaptionFont{red}{\color{red}}

%% The programming language in which to paste the code
\lstloadlanguages{{[LaTeX]TeX}, {[ISO]C++}, {Java}, {Ruby}, {Python}, {Scala}}

%% set some global styles for the listings package
%% Reference http://hi.baidu.com/shawpinlee/blog/item/9ec431cbae28e41cbe09e6e4.html
\ lstset {
numberbychapter=true,
breakatwhitespace=true,
Showstringspaces=false, %% Set whether to display the space symbol between the codes
Basicstyle=\footnotesize\ttfamily, %% set font size\tiny, \scriptsize, \footnotesize, \small, \Large, etc.
keywordstyle=\bfseries,
commentstyle=\color{red!50!green!50!blue!50},                           
Escapechar=`, %% Chinese escape characters for Chinese and English mixed
xleftmargin=1.5em,xrightmargin=0em, aboveskip=1em,
Breaklines, %% This command allows LaTeX to automatically typeset long lines of code
Extendedchars=false, %% This command can solve the problem that the Chinese characters of chapter titles, headers, etc. are not displayed when the code is spread across pages.
frameround=fttt,
captionpos=top,
belowcaptionskip=1em
}

\lstdefinestyle{numbers}{
   numbers=left,
   numberstyle=\tiny,
   stepnumber=1,
   numbersep=1em
}

\lstdefinestyle{C++}{
   language=C++,
   texcl=true,
   prebreak = \ text backslash,
   breakindent=1em,
   Keywordstyle=\bfseries, %% keyword highlighting
   morekeywords={alignas, alignof, char16_t, char32_t, constexpr, decltype, noexcept, nullptr, static_assert, thread_local, override, OVERRIDE, INTERFACE, ABSTRACT, DEFINE_ROLE, ROLE, HAS_ROLE, USE_ROLE}
   style=numbers,
   %frame=leftline, %% to frame the code
   %framerule=2pt,
   %rulesep=5pt
}

\lstnewenvironment{c++}[1][]
  {\setstretch{1}
  \lstset{style=C++, #1}}
  {}

% \ Captionsetup [lstlisting] {text font = red}
%{labelfont=bf, singlelinecheck=off, labelsep=space, textfont=red}

\lstdefinestyle{Java}{
   language=Java,
   texcl=true,
   prebreak = \ text backslash,
   breakindent=1em,
   Keywordstyle=\bfseries, %% keyword highlighting
   morekeywords = {}
   style=numbers,
   %frame=leftline, %% to frame the code
   %framerule=2pt,
   %rulesep=5pt
}

\lstnewenvironment{java}[1][]
  {\setstretch{1}
  \lstset{style=Java, #1}}
  {}

\lstdefinestyle{Ruby}{
   language=Java,
   texcl=true,
   prebreak = \ text backslash,
   breakindent=1em,
   Keywordstyle=\bfseries, %% keyword highlighting
   morekeywords = {}
   style=numbers,
   %frame=leftline, %% to frame the code
   %framerule=2pt,
   %rulesep=5pt
}

\lstnewenvironment{ruby}[1][]
  {\setstretch{1}
  \lstset{style=Ruby, #1}}
  {}

\lstdefinestyle{Python}{
   language=Python,
   texcl=true,
   prebreak = \ text backslash,
   breakindent=1em,
   Keywordstyle=\bfseries, %% keyword highlighting
   morekeywords = {}
   style=numbers,
   %frame=leftline, %% to frame the code
   %framerule=2pt,
   %rulesep=5pt
}

\lstnewenvironment{python}[1][]
  {\setstretch{1}
  \lstset{style=Python, #1}}
  {}

\ Lstdefinestyle Scala {} {
   language = Scala,
   texcl=true,
   prebreak = \ text backslash,
   breakindent=1em,
   Keywordstyle=\bfseries, %% keyword highlighting
   morekeywords = {}
   style=numbers,
   %frame=leftline, %% to frame the code
   %framerule=2pt,
   %rulesep=5pt
}

\ Lstnewenvironment scale {} [1] []
  {\setstretch{1}
  \lstset{style=Scala, #1}}
  {}  

\renewcommand{\lstlistingname}{sample code}
\renewcommand\thefigure{\thechapter-\arabic{figure}}

\newcommand\refcode[1]{{\itshape \lstlistingname\ascii{\ref{code:#1}(第\pageref{code:#1}页)}}}

% \usepackage[
%   placement=center,
%   angle=45,
%   scale=20,
%   color=black!40,
%   %hshift=60,
%   %vshift=-5
% ]{background}

% \backgroundsetup{contents={Sample chapter}}
% \backgroundsetup{contents={\includegraphics[width=0.2\textwidth]{figures/cock.jpg}}}

\newcommand{\myclearpage}{\clearpage{\pagestyle{empty}\cleardoublepage}}
\newcommand{\mydedicate}{\clearpage{\pagestyle{empty}\cleardedicatepage}}

%%%% End of the introduction area
%%%%%%%%------------------------------------------------------------------------

%%%%%%%%------------------------------------------------------------------------
%%%% Body section

\begin{document}

\frontmatter
\pagestyle{empty}

%%Custom cover
\def\titlename{TensorFlow kernel analysis}
\def\subtitle{TensorFlow Internals}
\def\authors{Liu Guangcong}
% \def\orgnization{\ascii{}}


\newlength{\centeroffset}
\setlength{\centeroffset}{-0.5\oddsidemargin}
\addtolength{\centeroffset}{0.5\evensidemargin}
\thispagestyle{empty}

% \begin{tikzpicture}
%   \path[mindmap,concept color=black,text=white]
%     node[concept] {\ascii{Programming}}
%     [clockwise from=0]
%     child[concept color=green!50!black] {
%       node[concept] {\ascii{XP}}
%       [clockwise from=90]
%       child { node[concept] {\ascii{Simple Design}} }
%       child { node[concept] {\ascii{TDD}} }
%       child { node[concept] {\ascii{Refactoring}} }
%       child { node[concept] {\ascii{Pair Programming}} }
%     }
%     child[concept color=blue] {
%       node[concept] {\ascii{Oriented-Object}}
%       [clockwise from=-30]
%       child { node[concept] {\ascii{C++}} }
%       child { node[concept] {\ascii{Java}} }
%     }
%     child[concept color=red]
%     { node[concept] {\ascii{Evolutionary Design}} }
%     child[concept color=orange]
%     { node[concept] {\ascii{Clean Code}} };
%  \end{tikzpicture}

%title and subtitle
\vspace*{\stretch{1}}
\noindent\hspace*{\centeroffset}\makebox[0pt][l] {
  \begin{minipage}{\textwidth}
    \flushright{\Huge \hei \bfseries \titlename}
    \noindent\rule[-1ex]{\textwidth}{5pt}\\[2.5ex]
    \hfill\emph{\subtitle}
  \end{minipage}
}

%author
\vspace{\stretch{2}}
\noindent\hspace*{\centeroffset}\makebox[0pt][l] {
  \begin{minipage}{\textwidth}
    \center{\bfseries \authors} \\[1.5ex]
  \end{minipage}
}

\vspace{\stretch{1}}

\pagebreak

\myclearpage

\mydedicate

\input{contents/foreword}
\myclearpage

\chapter{Preface} 
\label{ch:preface}

\section*{Important notice}
\begin{content}
Original version of this book (\script{\url{https://github.com/horance-liu/tensorflow-internals}}) was translated semi-automatically with Google translate. It may and will contain typos in text, code and comments. It needs carefull proof-reading. The source code of this English version is hosted here - \script{\url{https://github.com/sergey-serebryakov/tensorflow-internals}}.
\end{content}

\section*{Book positioning}
\begin{content}
This is a book that analyzes how the \ascii{TensorFlow} kernel works. It doesn't tell you how to build a machine learning model using \ascii{TensorFlow} or the best practices for applying \ascii{TensorFlow}. This book will reveal the \ascii{TensorFlow} system architecture, domain model, working principle, and its implementation details by analyzing the \ascii{TensorFlow} source code to reveal the inner knowledge.
\end{content}


\section*{For readers}
\begin{content}
This book assumes that the reader already understands the basic concepts and theories related to machine learning, and understands the basic methodology related to machine learning. At the same time, a reader is assumed to be familiar with programming languages ​​such as \ascii{Python} and \ascii{C++}.
This book is suitable for system architects who are keen to learn more about \ascii{TensorFlow} kernel design, expect to improve \ascii{TensorFlow} system design and performance optimization, and explore the design and implementation of \ascii{TensorFlow} key technologies, \ascii{ AI} algorithm engineer, and \ascii{AI} software engineer.
\end{content}


\section*{Reading}
\begin{content}
For the first time reading this book, we recommend a step-by-step reading method; for advanced users, you can choose the chapter you are interested in reading. When using \ascii{TensorFlow} for the first time, it is recommended to build \ascii{TensorFlow} completely from the source code to understand how the system is built and to rationalize the basic component libraries it depends on.
In addition, it is recommended to use \ascii{TensorFlow} to personally practice some specific applications in order to deepen the understanding and understanding of \ascii{TensorFlow} system behavior, familiar with the common \ascii{API} usage and working principle. It is highly recommended to read the \ascii{TensorFlow} key code while reading this book; for the best practices for reading code, check out the appendix \ascii{A}.
\end{content}


\section*{Release notes}
\begin{content}
At the time of this writing, the stable release of \ascii{TensorFlow} is \ascii{1.2}. It is not excluded that the part \ascii{API} that this book explains will be discarded in the future, and there is no guarantee that some system implementations will change or even be deleted in future versions.
At the same time, in order to explain the nature of the problem more directly, some of the code in the book has been partially reconstructed; some exception handling branches have been deleted, or log printing, or even some optional parameter lists. However, such local reconstruction does not affect the reader's understanding of the main behavioral characteristics of the system, and is more conducive to the reader's mastery of the working principle of the system.
At the same time, in order to simplify the expression of the calculation graph, the calculation graph in this book is not from \ascii{TensorBoard}, but a simplified, equivalent graph structure. Similarly, the simplified diagram structure does not reduce the reader's understanding and understanding of the real graph structure.
\end{content}


% \section*{English terminology}
% \begin{content}
% Because the articles I have written are for reading by relevant professionals, not for popular science books, so I keep the catchy English terminology in the professional field and deliberately do not translate. For example, the term \ascii{OP} is used directly in the book instead of being translated as "operation."
% However, this will result in a large area of ​​Chinese-English mixed expression. Fortunately, the English terms used in the absolute part are nouns, and there are very few verbs or adjectives. However, the original subject semantics and logic will not be lost anyway.
% Everything has exceptions. For unambiguous, short-form, semantically explicit terms are expressed in Chinese terms. In particular, when there is ambiguity in the expression of Chinese terms, both Chinese terms and English terms are marked. For example, checkpoint (\ascii{Checkpoint}), coordinator (\ascii{Coordinator}).
% Generally, an unambiguous Chinese glossary is defined in \reftbl{glossary}.

% \begin{table}[!htb]
% \resizebox{0.95\textwidth}{!} {
% \begin{tabular*}{1.2\textwidth}{@{}ll@{}}
% \toprule
% \ascii{English} & \ascii{中文} \\
% \midrule
% \ascii{Variable} & \ascii{variable, parameter} \\
% \ascii{Session} & \ascii{session} \\ 
% \ascii{Device} & \ascii{device} \\ 
% \bottomrule
% \end{tabular*}
% }
% \caption{Specification Convention}
% \label{tbl:glossary}
% \end{table}

% \end{content}

% \section*{code refactoring}

% \begin{content}

% Read the code in \ascii{IDE} to help programmers read the code with contextual information. However, the source code displayed in the book, due to limited space, coupled with the lack of context, the effect of code reading will be greatly discounted.

% In general, in order to better assist readers in speeding up the speed and quality of code understanding, common methods include code comments, content explanations, algorithm descriptions, and data structure descriptions. But in the author's values, instead of annotating too much code comments, it's better to refactor the code and let the code directly express the semantics. So in this book, the author takes a different approach, trying local code refactoring to increase the comprehensibility of the code.

% For example, in \code{distributed\_runtime/master.cc}, the following code snippet exists. The semantics of the upper part is to linearly search for the corresponding \code{MasterSession} example according to \code{session\_handle}.

% \begin{leftbar}
% \begin{c++}
% void Master::ExtendSession(
% const ExtendSessionRequest* req,
% ExtendSessionResponse* resp, MyClosure done) {
% // Find master session by session handle.
% mu_.lock();
% MasterSession* session = nullptr;
% session = gtl::FindPtrOrNull(sessions_, req->session_handle());
% if (session == nullptr) {
% mu_.unlock();
% done(errors::Aborted(
% "Session ", req->session_handle(), " is not found."));
% return;
% }
% session->Ref();
% mu_.unlock();

% SchedClosure([session, req, resp, done]() {
% Status status = ValidateExternalGraphDefSyntax(req->graph_def());
% if (status.ok()) {
% status = session->Extend(req, resp);
% }
% session->Unref();
% done(status);
% });
% }

% void Master::PartialRunSetup(
% const PartialRunSetupRequest* req,
% PartialRunSetupResponse* resp, MyClosure done) {
% // Find master session by session handle.
% mu_.lock();
% MasterSession* session = nullptr;
% session = gtl::FindPtrOrNull(sessions_, req->session_handle());
% if (session == nullptr) {
% mu_.unlock();
% done(errors::Aborted(
% "Session ", req->session_handle(), " is not found."));
% return;
% }
% session->Ref();
% mu_.unlock();

% SchedClosure([this, session, req, resp, done]() {
% Status s = session->PartialRunSetup(req, resp);
% session->Unref();
% done(s);
% });
% }
% \end{c++}
% \end{leftbar}

% If you try to extract a function here, directly express the original semantics. The refactored effect not only removes redundant comments, but also enhances the comprehensibility of the code and indirectly eliminates duplicate code between the two. Of course, in order to ensure that the code examples in this book are consistent with the community code repository, the author will strive to incorporate the refactored code into the \ascii{master} branch.

% \begin{leftbar}
% \begin{c++}
% void Master::ExtendSession(
% const ExtendSessionRequest* req,
% ExtendSessionResponse* resp, MyClosure done) {
% auto session = FindMasterSession(req->session_handle());
% if (session == nullptr) {
% done(AbortedError(req));
% return;
% }

% SchedClosure([session, req, resp, done]() {
% Status status = ValidateExternalGraphDefSyntax(req->graph_def());
% if (status.ok()) {
% status = session->Extend(req, resp);
% }
% session->Unref();
% done(status);
% });
% }

% void Master::PartialRunSetup(
% const PartialRunSetupRequest* req,
% PartialRunSetupResponse* resp, MyClosure done) {
% auto session = FindMasterSession(req->session_handle());
% if (session == nullptr) {
% done(AbortedError(req));
% return;
% }

% SchedClosure([this, session, req, resp, done]() {
% Status s = session->PartialRunSetup(req, resp);
% session->Unref();
% done(s);
% });
% }
% \end{c++}
% \end{leftbar}

% \end{content}


\section*{Online help}
\begin{content}
In order to better communicate with readers, an errata has been created on \ascii{Github} with additional explanations. Due to limited personal experience and ability, it is inevitable to make mistakes in a limited time. If the reader is in the process of reading, if you find any errors, please help submit \ascii{Pull Request}, to prevent others from falling into the same trap, and to make knowledge sharing more smooth and easier, I will not be grateful.
At the same time, welcome to follow my brief book. I will continue to update related articles and learn and progress with more friends.
\begin{enum}
  \eitem{\ascii{Github: \script{\url{https://github.com/horance-liu/tensorflow-internals-errors}}}}
  \eitem{\ascii{Shorter version:\script{\url{http://www.jianshu.com/u/49d1f3b7049e}}}}
  \eitem{\ascii{Sources of this English version on Github: \script{\url{https://github.com/sergey-serebryakov/tensorflow-internals}}}}
\end{enum}
\end{content}


\section*{Thank you}
\begin{content}
I am grateful to my wife, Liu Meihong, for completing the review of the book after work and for making many revisions.
\end{content}

\myclearpage

\tableofcontents
\myclearpage

\def\thelstlisting{\thechapter-\arabic{lstlisting}}
%% Chinese rule of thumb is to set the first line indented to 2em.
%% Note that this setting must be in document. In the environment, this may be related to \setlength scope of action
\setlength{\parindent}{2em}

%%%%%%%%%%%%%%%%%%%%%%
%%Start the body, the page count starts from the body
\mainmatter
\setcounter{page}{1}
\pagestyle{fancy}

% \begin{savequote}[45mm]
\ascii{Any fool can write code that a computer can understand. Good programmers write code that humans can understand.}
\qauthor{\ascii{- Martin Flower}}
\end{savequote}

\chapter{linear model} 
\label{ch:linear-model}

\section{logical regression}

\begin{content}

In logistic regression (\ascii{Logistic Regression}), $y$ is a real value, and $y \in \{ 0,1\}$. However, logistic regression solves the problem of two classifications. Generally, if $y \ge 0.5$, it is judged to be a positive class; otherwise, it is judged to be a negative class.

\[c = \left\{ \begin{gathered}
  1,y \ge 0.5 \hfill \\
  0,y < 0.5 \hfill \\ 
\end{gathered} \right.\]

In addition, the above expression is often expressed using \emph{instruction function}.

\[
c = \mathbb I(y \ge 0.5)
\]

Where $\mathbb I(true) = 1$; otherwise $\mathbb I(false) = 0$.

\subsection{symbol definition}

To formalize the problem of logistic regression, some common symbols are defined here. It should be noted that in order to distinguish between the predicted value and the sign of the tag value, they are represented by $y, t$ respectively.

 \begin{itemize}
   \item \ascii{training sample set}: $ S = \{ ({x^{(i)}},{t^{(i)}});i = 1,2,...,m\} $
   \item \ascii{第$i$ training sample}: $ ({x^{(i)}},{t^{(i)}}) $
   \item \ascii{sample input}: $ x = ({x_1},{x_2},...,{x_n})^{T} \in {\mathbb{R}^n} $
   \item \ascii{sample tag}: $ t \in {\mathbb{R}} $
   \item \ascii{predicted value}: $ y \in {\mathbb{R}} $   
   \item \ascii{error value}: $ e = y - t \in {\mathbb{R}} $
   \item \ascii{weight}: $ w \in {\mathbb{R}^{n}} $   
   \item \ascii{offset}: $ b \in {\mathbb{R}} $
   \item \ascii{linear weighted sum}: $ z = w^Tx + b \in {\mathbb{R}} $   
   \item \ascii{activation function}: $ f(z) = \frac{1}{{1 + {e^{ - z}}}} \in {[0, 1]} $
   \item \ascii{regular item}: $ R(w) \in {\mathbb{R}} $
   \item \ascii{L2 regular item}: $ \parallel w\parallel _2^2 \in {\mathbb{R}} $
   \item \ascii{loss function (no regular items)}: $ L(w, b) \in {\mathbb{R}} $
   \item \ascii{loss function (with regular items)}: $ J(w, b) = L(w, b) + \lambda R(w)\in {\mathbb{R}} $
   \item \ascii{gradient function}: $ {\nabla _w}L( {w,b}) \in {\mathbb{R}^n} $
 \end{itemize}

\subsection{model definition}

As shown by \refig{logistic-regression-nn}, logistic regression is equivalent to a network model with only one neuron.

\begin{figure}[H]
\centering
\includegraphics[width=0.6\textwidth]{figures/logistic-regression-nn.png}
\caption{logical regression: neuron model}
 \label{fig:logistic-regression-nn}
\end{figure}

First, the logistic regression completes the linear weighted sum of $z = {w^T}x + b$; then, using the activation function of \ascii{sigmoid}, the nonlinear transformation is done.

\[\begin{aligned}
  y = & {h_{w,b}}(x) \\ 
   = & f({w^T}x + b) \\ 
   = & \frac{1}{{1 + {e^{ - {w^T}x + b}}}} \\ 
\end{aligned} \]

$w$ represents the weight of the model, which is a vector of $n$ dimensions; $b$ represents the bias of the neuron, which is a scalar.

\[\begin{gathered}
  w = \left[ {\begin{array}{*{20}{c}}
  {{w_1}} \\ 
  {{w_2}} \\ 
   \vdots \\ 
  {{w_n}} 
\end{array}} \right] \in {\mathbb{R}^n} \\ 
  b \in \mathbb{R} \\ 
\end{gathered} \]

$x$ represents the input to the model, which is a vector of $n$ dimensions; where $\forall {x_i} \in x,i = 1,2,...,n$, which represents the input vector $x$ An eigenvalue; $y$ represents the output of the model, which is a scalar.

\[\begin{gathered}
  x = \left[ {\begin{array}{*{20}{c}}
  {{x_1}} \\ 
  {{x_2}} \\ 
   \vdots \\ 
  {{x_n}} 
\end{array}} \right] \in {\mathbb{R}^n} \\ 
  {\text{y}} \in {\mathbb{R}} \\ 
\end{gathered}\]

$f(z)$ is the activation function of the neuron, using the \ascii{sigmoid} function here.

\subsection{Sigmoid function}

The \ascii{sigmoid} function compresses the infinite domain space into the range space of $[0, 1]$, so it has a natural probability interpretation.

\[
f(z) = \frac{1}{{1 + {e^{ - z}}}}
\]


\subsubsection{soft saturation}

As shown by \refig{sigmoid}, the \ascii{sigmoid} function has soft saturation, ie

\[\begin{gathered}
  \mathop {\lim }\limits_{z \to + \infty } f(z) = 1 \hfill \\
  \mathop {\lim }\limits_{z \to - \infty } f(z) = 0 \hfill \\ 
\end{gathered} \]

\begin{figure}[H]
\centering
\includegraphics[width=0.6\textwidth]{figures/sigmoid.png}
\caption{sigmoidfunction}
 \label{fig:sigmoid}
\end{figure}

\subsubsection{derivative}

The derivative of \ascii{sigmoid} has a special form, and its derivative function is a quadratic function of $y$. make

\[
y = \frac{1}{{1 + {e^{ - z}}}}
\]

The derivative of \ascii{sigmoid} can be easily derived from the chain-based derivation rule.

\[\begin{aligned}
  y' = & \frac{d}{{dz}}\left( {\frac{1}{{1 + {e^{ - z}}}}} \right) \\ 
   = & \frac{d}{{dz}}{\left( {1 + {e^{ - z}}} \right)^{ - 1}} \\ 
   = & - {\left( {1 + {e^{ - z}}} \right)^{ - 2}}\frac{d}{{dz}}\left( {{e^{ - z}} } \right) \\ 
   = & - {\left( {1 + {e^{ - z}}} \right)^{ - 2}}{e^{ - z}}\frac{d}{{dz}}\left( { - z} \right) \\ 
   = & - {\left( {1 + {e^{ - z}}} \right)^{ - 2}}{e^{ - z}}\left( { - 1} \right) \\ 
   = & \frac{{{e^{ - z}}}}{{{{\left( {1 + {e^{ - z}}} \right)}^2}}} \\ 
   = & \frac{1}{{1 + {e^{ - z}}}}\left( {1 - \frac{1}{{1 + {e^{ - z}}}}} \right) \\ 
   = & y(1 - y) \\ 
\end{aligned} \]

\subsection{probability explanation}

For any $(x,t) \in S$, according to the functional properties of \ascii{Sigmoid}, assume that $t|x$ obeys the probability distribution of $Bernoulli(y)$.

\[\begin{aligned}
  P(t = 1|x;w,b) = & y \\ 
  P(t = 0|x;w,b) = & 1 - y \\ 
\end{aligned} \]

As we all know, the $Bernoulli(y)$ probability distribution can be described as a more concise expression.

\[P(t|x;w,b) = {y^t}{(1 - y)^{1 - t}}\]

By extension, if there are $m$ independent and identically distributed training samples $\{ ({x^{(i)}}, {t^{(i)}}); i = 1,2,.. .,m\}$, the likelihood function with the parameter $(w,b)$ can be expressed as:

\[\begin{aligned}
  L(w,b) = & P\left( {\vec y|X} \right) \\ 
   = & \prod\limits_{i = 1}^m {P\left( {{t^{(i)}}|{x^{(i)}};w,b} \right)} \\ 
   = & \prod\limits_{i = 1}^m {{{\left( {{y^{(i)}}} \right)}^{{t^{(i)}}}}{{\ Left( {1 - {y^{(i)}}} \right)}^{1 - {t^{(i)}}}}} \\ 
\end{aligned} \]

among them,
${y^{(i)}} = {h_{w,b}}({x^{(i)}}), i=1,2,...,m $. Generally, according to the monotonous increment of the logarithmic function, it is transformed into a log likelihood function, so that the multiplication operation is converted into a continuous addition operation, which simplifies the complexity of the problem.

\[l(w,b) = \log L(w,b) = \sum\limits_{i = 1}^m {{t^{(i)}}\log {y^{(i)}} + \left( {1 - {t^{(i)}}} \right)\log \left( {1 - {y^{(i)}}} \right)} \]

Therefore, the logistic regression problem can be transformed into a parameter estimation problem that maximizes the log likelihood function.

\[\hat w,\hat b = \arg \mathop {\max }\limits_{w,b} \sum\limits_{i = 1}^m {{t^{(i)}}\log {y ^{(i)}} + \left( {1 - {t^{(i)}}} \right)\log \left( {1 - {y^{(i)}}} \right)} \ ]

You can use the gradient-rising optimization method to iteratively find the optimal parameter $(\hat w,\hat b)$.

\subsection{cross entropy loss function}

In the field of machine learning, \emph{loss function} is often used, and the optimization algorithm of \emph{gradient drop} is used to iteratively find the optimal parameter $(\hat w,\hat b)$. According to the derivation of the previous section, given a training sample $(x, t)$, logistic regression can use the loss function defined below, which is a special representation of the cross-entropy loss function under the two-class problem.

\[L(w,b;x,t) = - \left\{ {t\log y + \left( {1 - t} \right)\log \left( {1 - y} \right)} \ Right\}\]

By extension, given training data set $ S = \{ ({x^{(i)}}, {t^{(i)}});i = 1,2,...,m\} $, the loss function of logistic regression can be expressed as:

\[L(w,b; S) = - \sum\limits_{i = 1}^m {{t^{(i)}}\log {y^{(i)}} + \left( {1 - {t^{(i)}}} \right)\log \left( {1 - {y^{(i)}}} \right)} \]

Therefore, the learning problem of logistic regression is transformed into an optimization problem that minimizes the loss function $L(w,b)$.

\[\hat w,\hat b = \arg \mathop {\min }\limits_{w,b} L(w, b)\]

\subsection{regular item}

To control the complexity of the model, you can increase the regular term of $L2$, which is equal to the inner product of the vector $w$.

\[R(w) = \lambda \parallel w\parallel _2^2 = {w^T}w \]

Among them, $\lambda$ is called the regular term factor, which is used to control the degree of influence of the regular term; it is the hyperparameter of the model, and the best value can be obtained through the cross-validation test. After adding the regular term, the model's loss function can be expressed as:

\[J(w,b) = L(w,b) + \lambda \parallel w\parallel _2^2\]

The optimization problem is also expressed as:

\[\hat w,\hat b = \arg \mathop {\min }\limits_{w,b} J(w, b)\]

\subsection{gradient drop}

You can use the gradient descent algorithm to iteratively find the optimal $(\hat w,\hat b)$. Where $\alpha$ represents the learning rate, which represents the size of the parameter update.

\[\begin{aligned}
  w \leftarrow & w - \alpha {\nabla _w}L(w,b) \\ 
  b \leftarrow & b - \alpha {\nabla _b}L(w,b) \\ 
\end{aligned} \]

Among them, the gradients of $w and b$ are defined as:

\[\begin{aligned}
  {\nabla _w}L(w,b) = & \frac{{\partial L}}{{\partial w}} \\ 
  {\nabla _b}L(w,b) = & \frac{{\partial L}}{{\partial b}} \\ 
\end{aligned} \]

As shown by \refig{mnist-gd}, the loss function can be likened to a mountain, and the climber tries to find the best course of action to reach the valley. The climber stood on a hillside and decided to take a small step down the opposite direction of the gradient until a better solution was obtained. When the gradient descent update algorithm is implemented, the initial points are different and the minimum values ​​obtained may be different. Therefore, the gradient drop is only a local minimum.

\begin{figure}[H]
\centering
\includegraphics[width=0.8\textwidth]{figures/mnist-gd.jpeg}
\caption{gradient descent algorithm}
 \label{fig:mnist-gd}
\end{figure}

The step size of the gradient drop is very important. If it is too small, the speed of finding the minimum value of the function is very slow; if it is too large, it may exceed the extreme point. Generally, in the early stage of model training, because the target of convergence from the model is still far away, the learning rate is adjusted to be larger; as the number of iterations increases, the learning rate is adjusted to be smaller. Therefore, the learning rate $\alpha$ is adaptive, and there are theoretically many optimization algorithms that learn the rate of $\alpha$ decay with time, such as \ascii{Adagrad}. Therefore, the key to the optimization problem is how the gradient is calculated.

\subsection{calculation gradient}

According to the chain-based derivation rule, given any sample $ (x,t) \in S $, we can infer $ L(y, t) $ relative to $ w_i, i=1,2,...,n The gradient formula of $.

\[\begin{aligned}
  \frac{{\partial L}}{{\partial {w_i}}}
   = & \frac{{\partial L}}{{\partial y}}\frac{{\partial y}}{{\partial z}}\frac{{\partial z}}{{\partial {w_i} }} \\ 
   = & \left( {\frac{{1 - t}}{{1 - y}} - \frac{t}{y}} \right)y(1 - y){x_i} \\ 
   = & (y - t){x_i} \\ 
\end{aligned} \]

Similarly, a gradient formula for $L(W,b)$ relative to $b$ can be derived.

\[\begin{aligned}
  \frac{{\partial L}}{{\partial b}} 
   = & \frac{{\partial L}}{{\partial y}}\frac{{\partial y}}{{\partial z}}\frac{{\partial z}}{{\partial b}} \\ 
   = & \left( {\frac{{1 - t}}{{1 - y}} - \frac{t}{y}} \right)y(1 - y) \\ 
   = & y - t \\ 
\end{aligned} \]

Further, a gradient calculation formula of $w, b$ is obtained by vectorization. Where $\nabla _w}L( {w,b;x,t}) \in {\mathbb{R}^{n}}$,$\nabla _b}L( {w,b;x,t} ) \in {\mathbb{R}}$.

\[\begin{gathered}
  {\nabla _w}L( {w,b;x,t}) = \left({y - t} \right)x \\ 
  {\nabla _b}L( {w,b;x,t) = y - t \\ 
\end{gathered} \]

By extension, given training data set $ S = \{ ({x^{(i)}}, {t^{(i)}});i = 1,2,...,m\} $, get the vectorized gradient formula for $w,b$ as:

\[\begin{gathered}
  {\nabla _w}L(w,b;S) = \left( {\frac{1}{m}\sum\limits_{i = 1}^m {\left( {{y^{(i)} } - {t^{(i)}}} \right)} } \right)x \\ 
  {\nabla _b}L(w,b;S) = \frac{1}{m}\sum\limits_{i = 1}^m {\left( {{y^{(i)}} - {t ^{(i)}}} \right)} \\ 
\end{gathered} \]

\subsection{parameter update}

There are three basic algorithms for parameter updating. For a given training sample $ ({x^{(i)}}, {t^{(i)}}) $, according to the gradient formula of $w, b$, complete the parameter update of this iteration, the algorithm Often referred to as \emph{random gradient descent method} (\ascii{SGD}). Because the parameters are updated once, \ascii{SGD} only needs to read one training sample, and a single sample does not represent universality. There may be a lot of noise, so the convergence of the model is more jittery. However, \ascii{SGD} can achieve rapid convergence of the model and is more efficient.

\[\begin{gathered}
  w \leftarrow w - \alpha \left( {{y^{(i)}} - {t^{(i)}}} \right)x \\ 
  b \leftarrow b - \alpha \left( {{y^{(i)}} - {t^{(i)}}} \right) \\ 
\end{gathered} \]

At the other extreme, given the entire training sample data $ S $, according to the gradient formula of $w, b$, the parameter update of this iteration is completed. This algorithm is often called \emph{batch gradient descent method} (\ascii {BGD}). Because the parameters are updated once, the entire training data set needs to be traversed, so the model convergence is relatively stable. However, due to the large amount of computation, the model convergence speed is slower than \ascii{SGD}.

\[\begin{gathered}
  w \leftarrow w - \alpha \left( {\frac{1}{m}\sum\limits_{i = 1}^m {\left( {{y^{(i)}} - {t^{( i)}}} \right)} } \right)x \\ 
  b \leftarrow b - \alpha \left( {\frac{1}{m}\sum\limits_{i = 1}^m {\left( {{y^{(i)}} - {t^{( i)}}} \right)} } \right) \\ 
\end{gathered} \]

In real-world applications, neither \ascii{BGD} nor \ascii{SGD} will be used, but the \ascii{SGD} algorithm of \ascii{MiniBatch} will be used. What's more, due to the large-scale application of \ascii{MiniBatch}'s \ascii{SGD}, \ascii{SGD}, often referred to as \ascii{MiniBatch}, is the \ascii{SGD} algorithm.

\[\begin{aligned}
  w \leftarrow w - \alpha \left( {\frac{1}{{atch#\zie}}\sum\limits_{i = 1}^{batch\_size} {\left( {{y^{(i) }} - {t^{(i)}}} \right)} } \right)x \\ 
  b \leftarrow b - \alpha \left( {\frac{1}{{atch#\size}}\sum\limits_{i = 1}^{batch\_size} {\left( {{y^{(i) }} - {t^{(i)}}} \right)} } \right) \\ 
\end{aligned} \]

Compared to \ascii{BGD}, \ascii{MiniBatch}'s \ascii{SGD} does not need to traverse the real training data set with new parameters, just traverse a \ascii{batch\_size} training samples; therefore, relative \ascii{BGD}, \ascii{MiniBatch}'s \ascii{SGD} converges faster. And relative to \ascii{SGD}, \ascii{MiniBatch}'s \ascii{SGD} reads new \ascii{batch\_size} training samples with new parameters each time; therefore, relative \ascii {SGD}, \ascii{MiniBatch}'s \ascii{SGD}'s convergence is more stable.

\subsection{calculation chart}

In general, when a neural network is used to represent a learning model, its training process involves two basic steps:

\begin{itemize}
  \item \ascii{forward calculation}: Calculating loss
  \item \ascii{backpropagation}: Calculate the gradient
\end{itemize}

To achieve a positive loss calculation, and its inverse gradient calculation, the network can be translated into an equivalent computational map. In the forward computational graph, the node describes some abstract mathematical operations, such as matrix multiplication, vector inner product, activation function, etc.; the edge represents the output value of the function and serves as the input to the downstream node function.



Positive as logistic regression as shown by \refig{logistic-bp}

\begin{figure}[H]
\centering
\includegraphics[width=0.8\textwidth]{figures/logistic-bp.png}
\caption{logical regression: forward calculation and back propagation}
 \label{fig:logistic-bp}
\end{figure}


\end{content}

\section{Single layer perceptron}

\begin{content}

First, try to build a single layer perceptron for \ascii{10} neurons. As shown by \refig{mnist-slp}, for multi-classification problems such as handwritten digit recognition, the activation function of \ascii{softmax} is often used in theory.

\begin{figure}[H]
\centering
\includegraphics[width=0.8\textwidth]{figures/mnist-slp.png}
\caption{Single layer perceptron}
 \label{fig:mnist-slp}
\end{figure}

\subsection{theory basis}

In theory, the \ascii{softmax} regression is a generalized extension of the \ascii{logistic} regression. Among them, \ascii{logistic} regression is to solve the two classification problem, namely $y \in \{ 0,1\}$; and \ascii{softmax} regression is to solve the $k $ classification problem, namely $y \in \{ 1,2,...,k\}$.

\subsubsection{symbol definition}

To formalize the \ascii{softmax} regression problem, some common symbols are defined here.

 \begin{itemize}
   \item \ascii{training sample set}: $ S = \{ ({x^{(i)}}, {y^{(i)}});i = 1,2,...,m\} $
   \item \ascii{第$i$ training sample}: $ ({x^{(i)}}, {y^{(i)}}) $
   \item \ascii{sample input}: $ x = ({x_1},{x_2},...,{x_n})^{T} \in {\mathbb{R}^n} $
   \item \ascii{sample tag (one-hot)}: $ y = ({y_1},{y_2},...,{y_k})^{T} \in {\mathbb{R}^k} $
   \item \ascii{weight}: $ W \in {\mathbb{R}^{n \times k}} $   
   \item \ascii{offset}: $ b \in {\mathbb{R}^k} $   
   \item \ascii{softmax function}: $ 
Softmax {(z_i)} = \tfrac{{{e^{{z_i}}}}}{{\sum\limits_{j = 1}^k {{e^{{z_j}}}} }} \quad i = 1,2,...,k
$
 \end{itemize}

\subsubsection{softmax function}

As shown by \refig{softmax}, the model first obtains the linear weighted sum $z$, then evaluates to $e^z$, and finally implements the normalization operation.

\begin{figure}[H]
\centering
\includegraphics[width=0.8\textwidth]{figures/softmax.png}
\caption{softmax function}
 \label{fig:softmax}
\end{figure}

\subsubsection{weights and offsets}

The weight $W$ is a two-dimensional matrix of $n \times k$.

\[
W = \left( {{W_1},{W_2},...,{W_k}} \right) = \left( {\begin{array}{*{20}{c}}
  {{w_{11}}}& \ldots &{{w_{1k}}} \\ 
   \vdots & \ddots & \vdots \\ 
  {{w_{n1}}}& \cdots &{{w_{nk}}} 
\end{array}} \right) \in {\mathbb{R}^{n \times k}}
\]

Where $W_j$ is a vector of length $n$.

\[
{W_j} = {\left( {{w_{1j}},{w_{2j}},...,{w_{nj}}} \right)^T} \in {\mathbb{R}^n }, j = 1,2,...,k \\
\]

The offset $b$ is a \ascii{\quo{one-hot}} vector of length $k$.

\[
b = {({b_1},{b_2},...,{b_k})^T} \in {\mathbb{R}^k}
\]

\subsubsection{model definition}

The single-layer perceptron model for multi-classification problems, using the \ascii{softmax} activation function, can be defined as such.

\[\begin{aligned}
  y = & {h_{W,b}}(x) = softmax (z) = softmax ({W^T}x + b) \\ 
   = & {\left( {{y_1},{y_2},...,{y_k}} \right)^T} \\ 
   = & \frac{1}{{\sum\limits_{j = 1}^k {{e^{{z_j}}}} }}{\left( {{e^{{z_1}}},{e ^{{z_2}}},...,{e^{{z_k}}}} \right)^T} \\ 
   = & \frac{1}{{\sum\limits_{j = 1}^k {{e^{W_j^Tx + {b_j}}}} }}{\left( {{e^{W_1^Tx + {b_1}}},{e^{W_2^Tx + {b_2}}},...,{e^{W_k^Tx + {b_k}}}} \right)^T} \ 
\end{aligned} \]

Where, for any given sample $ (x, y) \in S $, $ z_i $ represents the linear weighted sum of $W_i^Tx+b_i$, and $y_i(i=1,2,...,k )$ indicates the probability of classifying it as class $i$.

\[\begin{gathered}
  P\left( {y = i|x;W,b} \right) = \frac{{{e^{W_i^Tx + b_i}}}}{{\sum\limits_{j = 1}^k { {e^{W_j^Tx + b_j}}} }} \hfill \\
  i = 1,2,...,k \hfill \\ 
\end{gathered} \]


\subsubsection{cross entropy function}

Based on the sample data set $ S = \{ ({x^{(i)}}, {y^{(i)}});i = 1,2,...,m\} $, cross entropy loss function Can be defined as such.

\[\begin{aligned}
  J(W,b) = & - \frac{1}{m}\sum\limits_{i = 1}^m {{y^{(i)}}\log \left( {{{\widehat y} ^{(i)}}} \right)} \\ 
   = & - \frac{1}{m}\sum\limits_{i = 1}^m {\sum\limits_{j = 1}^k {y_j^{(i)}\log \left( {\widehat Y_j^{(i)}} \right)} } \\
\end{aligned} \]

The \ascii{softmax} multi-classification problem is to find the optimal solution $(W^*,b^*)$, so that

\[W^*,b^* = \mathop {\arg \min }\limits_{W,b} J(W,b)\]

\subsubsection{calculation gradient}

For any sample $(x,y) \in S $, you can derive a gradient formula for $ J(W,b) $ relative to $ W $ and $ b $.

\[\begin{aligned}
  {\nabla _W}J\left( {W,b;x,y} \right) = & \left( {\widehat y - y} \right)x \\ 
  {\nabla _b}J\left( {W,b;{x^{(i)}},{y^{(i)}}} \right) = & \left( {\widehat y - y} \ Right) \\ 
\end{aligned} \]


\subsubsection{parameter update}

For the training sample data $S$, according to the gradient formula of $W, b$, the parameter update of this iteration is completed.

\[\begin{aligned}
  W \leftarrow & W - \alpha \frac{{\sum\limits_{i = 1}^m {{\nabla _W}J\left( {W,b;{x^{(i)}},{y ^{(i)}}} \right)} }}{m} \\ 
  b \leftarrow & b - \alpha \frac{{\sum\limits_{i = 1}^m {{\nabla _b}J\left( {W,b;{x^{(i)}},{y ^{(i)}}} \right)} }}{m} \\ 
\end{aligned} \]

\subsection{define model}

Next, use \tf{} to complete the construction and training of the model. It should be noted that there are subtle differences between the theoretical formula and the concrete implementation of \tf{}. In theory, $x$ in a formula often represents a sample, but \code{x} in \tf{} often represents a sample dataset of \ascii{mini-batch}. Therefore, when using \tf{} to design a network model, you need to pay special attention to whether the changes in the individual tensors are in line with expectations.

\subsubsection{input and tag}

First, use \code{tf.placeholder} to define the input and label of the training sample separately.

\begin{leftbar}
\begin{python}
x = tf.placeholder(tf.float32, [None, 28, 28, 1])
t = tf.placeholder(tf.float32, [None, 10])
\end{python}
\end{leftbar}

\code{tf.placeholder} defines a placeholder \ascii{OP}. \code{None} represents the number of undetermined samples, which represents the size of \code{batch\_size}; when \code{Session.run}, a \ascii is provided via the \code{feed\_dict} dictionary The sample dataset of {mini-batch} automatically derives the size of \code{tf.placeholder}.

In addition, each image is represented by a three-dimensional data of $ 28 \times 28 \times 1 $ (the scale is \ascii{1}). To simplify the problem, the input sample data is flattened here and transformed into a one-dimensional vector of length \ascii{784}. Where \ascii{-1} represents the number of samples of \ascii{mini-batch}, which is automatically deduced by the runtime.

\begin{leftbar}
\begin{python}
x = tf.reshape(x, [-1, 784])
\end{python}
\end{leftbar}

\subsubsection{define variable}

Then, define the model parameters using \code{tf.Variable}. When defining the training parameters, you must specify the initialization value of the parameters; the training parameters will automatically derive the type of the data and its size based on the initial values.

\begin{leftbar}
\begin{python}
w = tf.Variable(tf.zeros([784, 10]))
b = tf.Variable(tf.zeros([10]))
\end{python}
\end{leftbar}

In addition, the variables must be initialized before they are used. Here, \code{init\_op} will initialize all global training parameters.

\begin{leftbar}
\begin{python}
Init_op = tf.global_variables_initializer()
\end{python}
\end{leftbar}

\subsubsection{model definition}

Next, the single-layer perceptron model for multi-classification problems can be easily obtained.

\begin{leftbar}
\begin{python}
y = tf.nn.softmax(tf.matmul(x, w) + b)
\end{python}
\end{leftbar}

As shown in \refig{mnist-linear-sum}, first calculate the matrix multiplication of \code{x} and \code{w}, then let \code{b} broadcast (\ascii{broadcast}) to each of the matrices. Add a line and finally get a linear weighted sum of the training parameters.

\begin{figure}[H]
\centering
\includegraphics[width=0.8\textwidth]{figures/mnist-linear-sum.png}
\caption{linear weighted sum}
 \label{fig:mnist-linear-sum}
\end{figure}

As shown by \refig{mnist-softmax}, \ascii{softmax} will implement the operation line by line. Finally, the size of \code{y} is \code{[100, 10]}.

\begin{figure}[H]
\centering
\includegraphics[width=0.8\textwidth]{figures/mnist-softmax.png}
\caption{Activation function: softmax}
 \label{fig:mnist-softmax}
\end{figure}

\subsubsection{loss function}

For multi-classification problems, the loss function of cross entropy can be used.

\begin{leftbar}
\begin{python}
Cross_entropy = -tf.reduce_sum(t * tf.log(y))
\end{python}
\end{leftbar}

As shown by \refig{mnist-cross-entropy}, \code{t} and \code{y} are both \code{[100, 10]}; in particular, every line of \code{t} Is a \quo{\ascii{one-hot}} vector.

Implementing the \code{tf.log} operation on \code{y} will also result in a matrix of size \code{[100, 10]}. Then, \code{t} is multiplied bit by bit with \code{tf.log(y)} (not multiplied by the matrix), and a matrix of size \code{[100, 10]} will also be obtained. Finally, \code{tf.reduce\_sum} adds all the elements in the matrix to get a scalar (\ascii{scalar}) value.

\begin{figure}[H]
\centering
\includegraphics[width=0.8\textwidth]{figures/mnist-cross-entropy.png}
\caption{cross entropy loss function}
 \label{fig:mnist-cross-entropy}
\end{figure}

\subsubsection{accuracy}

\code{tf.argmax(y,1)} will calculate the index of the maximum value by the \ascii{1} dimension. The index value of the maximum value in each row is calculated for each row of $ y_{100 \times 10} $ . Therefore, \code{tf.argmax(y,1)} will get a matrix of size \code{[100, 1]}, or a vector of size \ascii{100}. Similarly, \code{tf.argmax(t,1)} is also a vector of size \ascii{100}.

Then, use \code{tf.equal} to compare them by element (\ascii{element-wise}) for equality, and get a Boolean vector of size \ascii{100}. In order to calculate the precision, the Boolean vector is first converted to a numerical vector, and finally the mean of the numerical vector is obtained.

\begin{leftbar}
\begin{python}
Is_correct = tf.equal(tf.argmax(y,1), tf.argmax(t,1))
Accuracy = tf.reduce_mean(tf.cast(is_correct, tf.float32))
\end{python}
\end{leftbar}

\subsection{Optimization Algorithm}

Next, a gradient descent algorithm is used to minimize the cross entropy loss function. Among them, \code{learning\_rate} indicates the learning rate, describes the speed of the parameter update and the size of the step, which is a typical super parameter.

\begin{leftbar}
\begin{python}
Optimizer = tf.train.GradientDescentOptimizer(learning_rate=0.003)
Train_step = optimizer.minimize(cross_entropy)
\end{python}
\end{leftbar}

\subsection{training model}

Prior to this, \tf{} only constructed the calculation graph and did not initiate the execution of the calculation graph. Next, the client creates a session, establishes a channel with the local or remote computing device set, and initiates the execution of the computation map.

First, the initialization of the training parameters is completed. The initial value is modified in situ to the corresponding training parameter by running the initialization subgraph of the model parameters and executing the initializer of each training parameter concurrently.

\begin{leftbar}
\begin{python}
With tf.Session() as sess:
  Sess.run(init_op)
\end{python}
\end{leftbar}

Then, iteratively executes \code{train\_step} to complete an iterative training of the model. Among them, every \ascii {100} iterations, calculate the accuracy and loss of the current model in the training data set and the test data set.

\begin{leftbar}
\begin{python}
With tf.Session() as sess:
  For step in range(1000):
    Batch_xs, batch_ys = mnist.train.next_batch(100)        
    Sess.run(train_step, feed_dict={x: batch_xs, t: batch_ys})
    
    If step % 100 == 0:
      Acc, loss = sess.run([accuracy, cross_entropy], 
        Feed_dict={x: batch_xs, t: batch_ys})
      Acc, loss = sess.run([accuracy, cross_entropy], 
        Feed_dict={x: mnist.test.images, t: mnist.test.labels}) 
\end{python}
\end{leftbar}

According to statistics, after \ascii{1000} iterations, you get an accuracy of about \percent{92}.

\begin{figure}[H]
\centering
\includegraphics[width=0.8\textwidth]{figures/mnist-slp-accuracy.png}
\caption{Visualization: Single layer perceptron, running 1000 times step}
 \label{fig:mnist-slp-accuracy}
\end{figure}

\end{content}

\part{Basic knowledge}
\begin{savequote}[45mm]
  \ascii{Any fool can write code that a computer can understand. Good programmers write code that humans can understand.}
  \qauthor{\ascii{- Martin Flower}}
\end{savequote}


\chapter{Introduction} 
\label{ch:introduction}

\begin{content}
\tf{} is an open source software library that uses \emph{dataflow graph} \ascii{(Dataflow Graph)} to express numerical calculations. It uses \emph{nodes} to represent abstract mathematical calculations and uses \ascii{operators} (\ascii{OP}) to express the computational logic; It uses \emph{edges} to represent the data flow passed between nodes, and uses \ascii{Tensors} to express the representation of the data \upcite{tf-white-paper}. The data flow graph is a \emph{directed acyclic graph} \ascii{(DAG)}. When the \ascii{OP} in the graph is sequentially executed according to a specific topological order, \ascii{Tensor} is in the graph. The flow forms the data  stream, and \tf{} gets its name.
In a distributed runtime, the data flow graph is split into multiple subgraphs and effectively deployed to multiple machines in the cluster for concurrent execution. Within a machine, registered subgraphs are split twice into smaller subgraphs that are deployed concurrently on \emph{local device set}. \tf{} supports distributed computing for a variety of heterogeneous devices, including  \ascii{CPU, GPU, ASIC}. \tf{}Peripheral performance makes it flexible to deploy on a variety of computing platforms, including desktops, servers, and mobile devices.
\tf{} was originally developed by \ascii{Google Brain} researchers and engineers for machine learning and deep neural network research, including speech recognition, computer vision, natural  language understanding, robotics, and information retrieval. However, the versatility and flexibility of the \tf{} system architecture makes it widely used for numerical calculations in other  scientific fields.
\end{content}


\section{Prehistoric life}
\begin{content}
The \ascii{Google Brain} project began in \ascii{2011} and was used to study ultra-large-scale deep neural networks. In the early stages of the project, \ascii{Google Brain} built the first generation of distributed deep learning framework \ascii{DistBelief} and got a lot of applications in the \ascii{Google} internal products.
Based on the experience of \ascii{DistBelief}, \ascii{Google Brain} has a more comprehensive and deeper understanding of the requirements for deep learning training and reasoning, and the system behavior and attributes of its deep learning framework, and at \ascii{2015.11 } Introducing the second generation distributed deep learning framework \tf{}. \tf{} As the successor of \ascii{DistBelief}, revolutionized the design and implementation of the existing system architecture, \tf{} once released, it has become a blockbuster in the field of deep learning, and has formed a huge Influence.


\subsection{DistBelief}
\ascii{DistBelief} is a distributed system for training large-scale neural networks. It is the first generation distributed machine learning framework from \ascii{Google}. Since \ascii{2011}, \ascii{DistBelief} has been used extensively within \ascii{Google} to train large-scale neural networks and is widely used in the research and application of machine learning and deep learning, including unsupervised learning, language representation, image classification, target detection, video classification, speech recognition, sequence prediction, pedestrian detection, reinforcement learning, etc.


\subsubsection{Programming Model}
The programming model for \ascii{DistBelief} is based on the \ascii{DAG} diagram of \emph{layer}. A layer can be thought of as a compound operator that combines multiple arithmetic operators to perform specific computational tasks. For example, the fully connected layer performs a composite calculation of $f({W^T}x + b)$, including matrix multiplication of input and weight, then adds it to the bias, and finally applies the activation function based on the linear weighted value, implementing a nonlinear transformation.


\subsubsection{Schema}
\ascii{DistBelief} uses the system architecture of \emph{parameter server}\ascii{(Parameter Server, often called PS)}. The training job consists of two separate processes: a stateless \ascii{Worker} process for Model training; stateful \ascii{PS} process for maintaining model parameters. As shown in \refig{parameter-server}, in the distributed training process, each \emph{model copy} asynchronously pulls the training parameter $w$ from \ascii{PS}, and after completing one-step iteration, pushes The gradient of the parameter $ \Delta w $ to \ascii{PS} goes up and the parameter is updated.
\begin{figure}[H]
  \centering
  \includegraphics[width=0.5\textwidth]{figures/parameter-server.png}
  \caption{DistBelief: Parameter Server Architecture}
  \label{fig:parameter-server}
\end{figure}


\subsubsection{Critisism}
However, for advanced users in the field of deep learning, the programming model of \ascii{DistBelief} and its system architecture based on \ascii{PS} lacks sufficient flexibility and scalability.
\begin{enum}
  \eitem{Optimization algorithm: Adding a new optimization algorithm, you must modify the implementation of \ascii{PS}; the abstract methods of \code{get(), put()} are not efficient for some optimization algorithms.}
  \eitem{Training algorithm: Supporting non-feedforward neural networks faces enormous challenges; for example, \ascii{RNN} with loops, alternately trained confrontation networks, and enhanced learning models whose loss functions are performed by separate agents.}
  \eitem{Acceleration device: \ascii{DistBelief} designed to support only multi-core\ascii{CPU}, does not support multi-card \ascii{GPU}, legacy system architecture lacks good scalability to support new computing devices.}
\end{enum}


\subsection{TensorFlow}
\ascii{DistBelief} The legacy architecture and design no longer meet the deeper learning and increasing demand changes. \ascii{Google} resolutely gave up the existing \ascii{DistBelief} implementation and decided to build a new system architecture based on it. Design, \ascii{TensorFlow} came into being, creating a new era of deep learning.


\subsubsection{Programming model}
\ascii{TensorFlow} uses a dataflow graph to represent the computational process and shared state, using nodes to represent abstract computations, and edges to represent data flows. As shown in \refig{tf-dataflow}, the data flow diagram of the \ascii{MNIST} handwriting recognition application is shown. In this model, the forward subgraph uses the \ascii{2} layer fully connected network, which is the \ascii{ReLU} layer and the \ascii{Softmax} layer. Subsequently, using the optimization algorithm of \ascii{SGD}, a reverse subgraph corresponding to the forward subgraph is constructed to calculate the gradient of the training parameters. Finally, according to the parameter update rule, the update subgraph of the training parameters is constructed, and the iterative update of the training parameters is completed.
\begin{figure}[H]
  \centering
  \includegraphics[width=0.4 \textwidth]{figures/tf-dataflow.png}
  \caption{TensorFlow: Data Flow Graph}
  \label{fig:tf-dataflow}
\end{figure}


\subsubsection{Design Principles}
The system architecture of \tf{} follows some basic design principles to guide the system implementation of \tf{}.
\begin{enum}
  \eitem{Delayed calculation: The construction of the graph is separated from the execution, and the execution of the graph is deferred;}
  \eitem{Atomic\ascii{OP}:\ascii{OP} is the smallest abstract computing unit that supports the construction of complex network models;}
  \eitem{Abstract device: support \ascii{CPU, GPU, ASIC} a variety of heterogeneous computing device types;}
  \eitem{Abstract Task: Task-based \ascii{PS}, with good scalability for new optimization algorithms and network models.}
\end{enum}


\subsubsection{Advantages}
Compared to other machine learning frameworks, \ascii{TensorFlow} has the following advantages.
\begin{enum}
  \eitem{High Performance: \tf{} Upgrade to \ascii{1.0} version performance improvement, single-machine multi-card (\ascii{8} card \ascii{GPU}) environment, \ascii{Inception v3} training implementation \ascii{7.3} times the speedup; in the distributed multi-machine multi-card (\ascii{64}card\ascii{GPU}) environment, \ascii{Inception v3} training achieved \ascii{58} times Acceleration ratio;}
  \eitem{Cross-platform: support multiple \ascii{CPU/GPU/ASIC} computing of multiple heterogeneous devices; support desktop, server, mobile device and other computing platforms; support \ascii{Windows, Linux, MacOS}, etc. Multiple operating systems;}
  \eitem{Distributed: Supports local and distributed model training and reasoning;}
  \eitem{Multi-language: support \ascii{Python, C++, Java, Go} and many other programming languages;}  
  \eitem{Universality: Supports the design and implementation of a variety of complex network models, including non-feedforward neural networks;}
  \eitem{Extensible: Support for \ascii{OP} extensions, \ascii{Kernel} extensions, \ascii{Device} extensions, extensions to communication protocols;}
  \eitem{Visualization: Visualize the entire training process with \ascii{TensorBoard}, greatly reducing the debugging process for \tf{}; }
  \eitem{Automatic differentiation: \tf{} automatically constructs the inverse calculation subgraph to complete the gradient calculation of the training parameters;}
  \eitem{Workflow: \ascii{TensorFlow} Seamless integration with \ascii{TensorFlow Serving}, support model training, import, export, release one-stop workflow, and automatically implement hot update and version management of the model. }
\end{enum}


\end{content}


\section{Community Development}
\begin{content}
\tf{} is the current hot deep learning framework. Since the open source, the \tf{} community has been quite active. Tens of thousands of code submissions from numerous non-associative {Google} employees were received on \ascii{Github}, and nearly a hundred \ascii{Issue} were submitted each week. There are also tens of thousands of questions about \tf{} on \ascii{Stack Overflow} that are questioned and answered. At various technical conferences, \tf{} is also a shining star, which is favored by many developers.


\subsection{Open source}
\ascii{2015.11}, \ascii{Google Research} post: \code{\href{https://research.googleblog.com/2015/11/tensorflow-googles-latest-machine\_9.html}{TensorFlow: Google's latest machine learning system, open sourced for everyone}}, officially announced a new generation of machine learning system \ascii{TensorFlow} open source. Subsequently, \ascii{TensorFlow} obtained a large number of \ascii{Star} and \ascii{Fork} in the code repository on \ascii{Github} for a short time. As shown by \refig{tf-commits}, \ascii{TensorFlow} has far more community activity than other competitors and is now the most popular deep learning framework.
\begin{figure}[H]
  \centering
  \includegraphics[width=1.0\textwidth]{figures/tf-commits.png}
  \caption{TensorFlow: Community activity}
  \label{fig:tf-commits}
\end{figure}

Undoubtedly, \ascii{TensorFlow}'s open source has had a huge impact on academia and industry, which greatly reduces the difficulty of deep learning in various industries. Numerous scholars, engineers, companies, and organizations have invested in the \ascii{TensorFlow} community, and together improve and improve \ascii{TensorFlow} to promote its continuous evolution and development.


\subsection{Milestones}
\tf{} since the open source of \ascii{2015.11}, an average version has been released for more than a month. As shown in \refig{tf-versions}, it shows the release time of several important features of \tf{}.
\begin{figure}[!htbp]
  \centering
  \includegraphics[width=1.0\textwidth]{figures/tf-versions.png}
  \caption{TensorFlow important milestone}
  \label{fig:tf-versions}
\end{figure}


\subsection{Industrial Applications}
\ascii{TensorFlow} has been used in a large number of applications in the production environment since its development in the open source for two years. In the medical field, help doctors predict the probability of skin cancer; in the field of music and painting, help humans better understand art; on the mobile side, a variety of mobile devices equipped with \ascii{TensorFlow} training machine learning model for translation jobs. \ascii{TensorFlow} is also growing rapidly within \ascii{Google}, with multiple heavyweight products having apps, including: \ascii{Google Search, Google Gmail, Google Translate, Google Maps}, etc. Refig{tf-google-apps}.
\begin{figure}[!htbp]
  \centering
  \includegraphics[width=1.0\textwidth]{figures/tf-google-apps.png}
  \caption{TensorFlow: Internal use within Google}
  \label{fig:tf-google-apps}
\end{figure}

\end{content}

\begin{savequote}[45mm]
  \ascii{Any fool can write code that a computer can understand. Good programmers write code that humans can understand.}
  \qauthor{\ascii{- Martin Flower}}
\end{savequote}


\chapter{Programming Environment} 
\label{ch:prog-env}

\begin{content}
In order to achieve a quick introduction to \tf{}, this chapter introduces the programming environment of \tf{}, including code structure, engineering construction, in order to establish a basic perceptual understanding of the \tf{} system architecture.
\end{content}


\section{Code structure}
\begin{content}


\subsection{Clone source}
First, clone the source code for \tf{} from \ascii{Github}.
\begin{leftbar}
\begin{python}
git clone git@github.com:tensorflow/tensorflow.git
\end{python}
\end{leftbar}

Then, switch to the latest stable branch. For example, the \code{r1.4} branch.
\begin{leftbar}
\begin{python}
$ cd tensorflow
$ git checkout r1.4
\end{python}%
\end{leftbar}


\subsection{Source Structure}
Run the following command to print out the organization structure of the \tf{} source:
\begin{leftbar}
\begin{python}[]
$ tree -d -L 1 ./tensorflow
\end{python}
\end{leftbar}

Among them, this book will focus on the \code{core, python} component, and some involve the \code{c, cc, stream\_executor} component.

\begin{leftbar}
\begin{c++}[caption={TensorFlowSource Structure}]
./tensorflow
├── c
├── cc
├── compiler
├── contrib
├── core
├── docs_src
├── examples
├── g3doc
├── go
├── java
├── python
├── stream_executor
├── tools
└── user_ops
\end{c++}
\end{leftbar}

As of the latest release of \ascii{1.4}, the \tf{} codebase has approximately \ascii{100} million lines of code. Among them, including \ascii{53} million lines of \ascii{C/C++} code, \ascii{37} million lines of \ascii{Python} code, and the scale of code is constantly expanding. Among them, the \ascii{API} provided by \ascii{Python} is the most complete; in contrast, the \ascii{API} of other programming languages are not yet mature, some even in their infancy.

\begin{leftbar}
\begin{python}[caption={TensorFlowCode Stats}]
-------------------------------------------------- -----
Language             files    blank    comment    code
-------------------------------------------------- -----
C++                   2238    77610     68275    443099
Python                1881    92018    151807    369399
C/C++ Header          1175    27392     46215     86691
Markdown 218 8859 2 30925
CMake                   50     2183       986     16398
Go                      28     1779     13290     15003
Java                    72     1789      3111      7779
Bourne Shell           103     1487      3105      6074
Protocol Buffers        87     1313      3339      3452
Objective C++            9      227       181      1201
C                        8      157       130       941
make                     4      105       136       612
XML                     25      135       265       315
Groovy                   3       46        74       246
Maven                    5       21         4       245
DOS Batch 9 30 0 143
Dockerfile               7       41        69       133
Perl                     2       29        38       130
Bourne Again Shell       3       24        63       111
JSON                     3        0         0        23
Objective C              1       10        13        21
YAML 1 3 24 15
-------------------------------------------------- -----
SUM:                  5932   215258    291127    982956
-------------------------------------------------- -----
\end{python}
\end{leftbar}


\subsection{Core}
The source code structure of the kernel is as follows, including platform, utility library, base framework, \ascii{Protobuf} definition, local runtime, distributed runtime, graph operation, \ascii{OP} definition, \ascii{Kernel} implementation, and other components. This is one of the key components of this book, will focus on unearthing the hidden domain model in the basic framework, tracking the life cycle of the entire runtime environment and the detailed process of the graph operation, and reveal the \ascii{Kernel} implementation principle and operating mechanism of common \ascii{OP}s.

\begin{leftbar}
\begin{c++}[caption={Core source structure}]
./tensorflow/core
├── common_runtime
├── debug
├── distributed_runtime
├── example
├── framework
├── graph
├── grappler
├── kernels
├── lib
├── ops
├── platform
├── profiles
├─ protobuf
├── public
├── user_ops
└── useful
\end{c++}
\end{leftbar}

Among them, \code{core} is mainly implemented in \code{C++}, which has about \ascii{26} million lines of code.

\begin{leftbar}
\begin{python}[caption={CoreCode Stats}]
-------------------------------------------------- -----
Language             files    blank   comment      code
-------------------------------------------------- -----
C++                   1368    44791     38968    259289
C/C++ Header           653    15040     24474     50506
Protocol Buffers        57      736      2371      1806
Markdown 11 327 0 1285
JSON                     2        0         0        18
-------------------------------------------------- -----
SUM:                  2091    60894     65813    312904
-------------------------------------------------- -----
\end{python}
\end{leftbar}


\subsection{Python}
\ascii{Python} defines and implements the programming model of \tf{}, and provides an \ascii{API} for programmers. The source structure is as follows, which is also the focus of this book.

\begin{leftbar}
\begin{c++}[caption={Python source structure}]
./tensorflow/python
├── client
├── debug
├── estimator
├── feature_column
├── framework
├── grappler
├── kernel_tests
├── layers
├── lib
├── ops
├── platform
├── profiles
├── saved_model
├── summary
├── tools
├── training
├── user_ops
└── useful
\end{c++}
\end{leftbar}

Among them, the component is implemented in \code{Python}, which has about \ascii{18} million lines of code.

\begin{leftbar}
\begin{python}[caption={Python Code Statistics}]
-------------------------------------------------- -----
Language            files     blank   comment      code
-------------------------------------------------- -----
Python                714     45769     69407    179565
C++                    20       496       506      3658
C/C++ Header           15       207       387       363
Markdown 4 48 0 200
Protocol Buffers        3        16        10        71
Bourne Shell            1        13        28        68
-------------------------------------------------- -----
SUM:                  757     46549     70338    183925
-------------------------------------------------- -----
\end{python}
\end{leftbar}


\subsection{Contrib}
\code{contrib} is a programming library contributed by third parties. It is also an experimental programming interface before \tf{} standardization, just like the relationship between the \ascii{Boost} community and the \ascii{C++} standard. When the interface of \code{contrib} matures, it will be standardized by \tf{}, and will be moved from \code{contrib} to \code{core, python} and officially released.

\begin{leftbar}
\begin{python}[caption={Contrib source structure}]
./tensorflow/contrib
├── android
├── batching
├── bayesflow
├── benchmark_tools
├── boosted_trees
├── cloud
├── cluster_resolve
├── cmake
├── compiler
├── copy_graph
├── crf
├── cudnn_rnn
├── data
├── decision_trees
├── deprecated
├── distributions
├── eager
├── factorization
├── ffmpeg
├── framework
├── fused_conv
├── gdr
├── graph_editor
├── grid_rnn
├── hooks
├── hvx
├── image
├── imperative
├── input_pipeline
├── integrate
├── hard
├── kernel_methods
├── labeled_tensor
├── layers
├── learn
├── legacy_seq2seq
├── linalg
├── linear_optimizer
├── lookup
├── losses
├── makefile
├── memory_stats
├── meta_graph_transform
├── metrics
├── mpi
├── nccl
├── ndlstm
├── nearest_neighbor
├── nn
├── opt
├── pi_examples
├── predictor
├── quantization
├── reduce_slice_ops
├── remote_fused_graph
├── resampler
├── rnn
├── saved_model
├── seq2seq
├── session_bundle
├── signal
├── slim
├── solvers
├── sparsemax
├── specs
├── staging
├── stat_summarizer
├── stateless
├── tensor_forest
├── tensorboard
├── testing
├── text
├── tfprof
├── timeseries
├── tpu
├── training
├── useful
├── verbs
└── xla_tf_graph
\end{python}
\end{leftbar}

Since the \tf{} community is quite active, the changes to \code{contrib} are quite frequent. As of \ascii{1.4}, there are about \ascii{23} million lines of code,  The interface is mainly designed and implemented in \ascii{Python}, and is partially implemented by \ascii{C++} when running.

\begin{leftbar}
\begin{python}[caption={Contrib Code Statistics}]
-------------------------------------------------- -----
Language            files     blank   comment      code
-------------------------------------------------- -----
Python               1007     41436     75096    170355
C++                   201      5500      5313     32944
CMake                  48      2172       955     16358
C/C++ Header           99      1875      2867      6583
Markdown 46 1108 0 3485
Bourne Shell           18       232       386      1272
C                       7       151       118       931
Protocol Buffers       20       224       454       680
make                    4       105       136       612
Java                    2        77       209       335
Groovy                  1        10        20        75
Bourne Again Shell      1         6        15        59
Dockerfile              1         2         1        14
XML                     2         3         0         9
-------------------------------------------------- -----
SUM:                 1457     52901     85570    233712
-------------------------------------------------- -----
\end{python}
\end{leftbar}


\subsection{StreamExecutor}
\ascii{StreamExecutor} is another open source component library of \ascii{Google} that provides a host-side programming model and runtime environment, and implements the unified packaging for \ascii{CUDA} and \ascii{OpenCL}. Thus, the \ascii{Kernel} function in the code on the host-side can be deployed seamlessly on the computing device with \ascii{CUDA} and \ascii{OpenCL}.

Currently, \ascii{StreamExecutor} is heavily used in the runtime of the \ascii{Google} internal\ascii{GPGPU} application. The \tf{} runtime also contains a snapshot version of \ascii{StreamExecutor} that encapsulates the runtimes of \ascii{CUDA} and \code{OpenCL}. This book will briefly introduce the programming model and threading model of \ascii{CUDA}, and introduce the system architecture and working principle of \ascii{StreamExecutor} in detail, revealing the implementation mode and idiom of \ascii{Kernel} function.

\begin{leftbar}
\begin{c++}[caption={StreamExecutorSource Structure}]
./tensorflow/stream_executor
├── miracles
├── host
├── lib
└── platform
\end{c++}
\end{leftbar}

Among them, \ascii{StreamExecutor} is implemented in \code{C++}, which has about \ascii{2.5} million lines of code.

\begin{leftbar}
\begin{python}[caption={StreamExecutorCode Statistics}]
-------------------------------------------------- -----
Language            files     blank   comment      code
-------------------------------------------------- -----
C++                    43      2440      1196     16577
C/C++ Header           81      2322      5080      8625
-------------------------------------------------- -----
SUM:                  124      4762      6276     25202
-------------------------------------------------- -----
\end{python}
\end{leftbar}


\subsection{Compiler}
As we all know, flexibility is the most important design goal and core advantage of \tf{}, so the system architecture of \tf{} has good scalability. \tf{} can be used to define arbitrary graph structures and executes efficiently on heterogeneous computing devices. However, you can't have your cake and eat it too. When the low-level \ascii{OP} is combined to calculate the subgraph and is expected to be effectively executed on \ascii{GPU}, the runtime will start more \ascii{Kernel} operations.

Therefore, the \tf{} method of decomposing and combining \ascii{OP} is not guaranteed to run in the most efficient way at runtime. At this point, \ascii{XLA} technology was born. It uses \ascii{JIT} compilation technology to analyze the graph calculation at runtime, which combines multiple \ascii{OP} and generates more efficient local machine code, to improve the execution efficiency of the graph calculation .

\begin{leftbar}
\begin{python}[caption={CompilerSource Structure}]
./tensorflow/compiler
├── aoT
├── jit
├── plugin
├── tests
├── tf2xla
└── xla
\end{python}
\end{leftbar}

\ascii{XLA} technology is currently in the initial stage of research and development, and is a more active research direction in the community. As of now, the code size is approximately \ascii{12.5} million lines, mainly in \ascii{C++}.

\begin{leftbar}
\begin{python}[caption={CompilerCode Stats}]
-------------------------------------------------- -----
Language            files     blank   comment      code
-------------------------------------------------- -----
C++                   455     19010     18334    102537
C/C++ Header          250      5939     10323     15510
Python                 37      1255      1416      6446
Protocol Buffers        5       312       501       781
Markdown 2 0 0 3
-------------------------------------------------- -----
SUM:                  749     26516     30574    125277
-------------------------------------------------- -----
\end{python}
\end{leftbar}

\end{content}


\section{Project Construction}

\begin{content}
Before you start, try the \tf{} source build process, understand the basic build methods of \tf{}, the tools, and their dependent component libraries, third-party toolkits, and have a great understanding of how \tf{} works. help. However, due to limited space, this chapter only uses the \ascii{Mac OS} system as an example to describe the source code compilation, installation, and verification process of \tf{}. For other operating systems, please check the official documentation published by \tf{}.


\subsection{Environment preparation}
The front end of \ascii{TensorFlow} is a programming interface that supports multiple languages. Therefore, before compiling the \ascii{TensorFlow} source code, you need to install the relevant compiler, interpreter, and its runtime environment. For example, to use \ascii{Python} as a programming interface, you need to install the \ascii{Python} interpreter beforehand. Second, before building the system, you also need to install \ascii{GCC} or \ascii{Clang} and other \ascii{C++} compilers to compile the backend system implementation. Since \ascii{TensorFlow} is implemented using the \ascii{C++11} syntax, it is important to ensure that the \ascii{C++} compiler is installed to support \ascii{C++11}. In addition, \ascii{TensorFlow} uses the \ascii{Bazel} build tool, which can be thought of as a more abstract \ascii{Make} tool. Unfortunately, \ascii{Bazel} is implemented using \ascii{Java8}, which relies on \ascii{JDK}. Therefore, before installing \ascii{Bazel}, you need to install \ascii{JDK} in \ascii{1.8} and above.


\subsubsection{Install JDK}
It is recommended to download \ascii{JDK} from the \ascii{Oracle} website and create the relevant environment variables and add them to the \code{~/.bashrc} configuration file.

\begin{leftbar}
\begin{python}
$ echo 'export JAVA_HOME=$(/usr/libexec/java_home)' >> ~/.bashrc
$ echo 'export PATH="$JAVA_HOME/bin:$PATH"' >> ~/.bashrc
\end{python}
\end{leftbar}


\subsubsection{Installing Bazel}
On \ascii{Mac OS}, you can install \ascii{Bazel} using \ascii{brew}.

\begin{leftbar}
\begin{python}
$ brew install bazel
\end{python}
\end{leftbar}

If \ascii{brew} is not installed on the system, you can install \ascii{brew} by executing the following command. Of course, installing \ascii{brew} requires the \ascii{Ruby} interpreter to be installed beforehand, and will not be redundant here.

\begin{leftbar}
\begin{python}
$ ruby -e "$(curl -fsSL https://raw.githubusercontent.com/Homebrew/install/master/install)"
\end{python}
\end{leftbar}


\subsubsection{Install Swig}
\ascii{TensorFlow} uses \ascii{Swig} to build a multi-language programming environment that automatically generates wrappers for related programming languages. Therefore, you need to install the \ascii{Swig} toolkit before you build it.

\begin{leftbar}
\begin{python}
$ brew install swig
\end{python}
\end{leftbar}


\subsubsection{Install Python dependency package}
Use \ascii{pip} to install the \ascii{Python} package that \ascii{TensorFlow} depends on.

\begin{leftbar}
\begin{python}
$ sudo pip install six numpy wheel autograd
\end{python}
\end{leftbar}

If \ascii{pip} is not installed on your system, you can pre-install \ascii{pip} using \ascii{brew}.

\begin{leftbar}
\begin{python}
$ brew install pip
\end{python}
\end{leftbar}


\subsubsection{Install CUDA Toolkit}

When the system installs \ascii{CUDA} to calculate the \ascii{GPU} card with compatibility greater than or equal to \ascii{3.0}, you need to install the \ascii{CUDA} toolkit and its \ascii{cuDNN} to implement\ The \tf{} runtime's \ascii{GPU} acceleration. It is recommended to download \ascii{CUDA Toolkit 8} and above from the \ascii{NVIDIA} website and install it into the system to configure related environment variables.

\begin{leftbar}
\begin{python}
$ echo 'export CUDA_HOME=/usr/local/cuda' >> ~/.bashrc
$ echo 'export LD_LIBRARY_PATH=$CUDA_HOME/lib:$LD_LIBRARY_PATH' >> ~/.bashrc
\end{python}
\end{leftbar}

Then, download \ascii{cuDNN 5.1} and above and extract it to the \code{CUDA\_HOME} related system directory.

\begin{leftbar}
\begin{python}
$ sudo tar -xvf cudnn-8.0-macos-x64-v5.1.tgz -C /usr/local
\end{python}
\end{leftbar}


\subsection{Configuration}
At this point, the build environment is ready, execute the \code{./configure} configuration\ascii{TensorFlow} build environment. When the system supports \ascii{GPU}, you need to configure the relevant \ascii{CUDA/cuDNN} build environment.

\begin{leftbar}
\begin{python}
$ ./configure
\end{python}
\end{leftbar}


\subsection{Build}
When the configuration is successful, use \ascii{Bazel} to start the compilation of \ascii{TensorFlow}. Before the compilation starts, it will try to download the source code of the relevant dependencies from the code repository, including \ascii{gRPC, Protobuf, Eigen}, etc., and compile automatically.

\begin{leftbar}
\begin{python}
$ bazel build --config=opt //tensorflow/tools/pip_package:build_pip_package
\end{python}
\end{leftbar}

Add the \code{--config=cuda} compile option when supporting \ascii{GPU} calculations.

\begin{leftbar}
\begin{python}
$ bazel build -c opt --config=cuda //tensorflow/tools/pip_package:build_pip_package
\end{python}
\end{leftbar}

After compiling successfully, you can build the \ascii{Wheel} package of \ascii{TensorFlow}.

\begin{leftbar}
\begin{python}
$ bazel-bin/tensorflow/tools/pip_package/build_pip_package /tmp/tensorflow_pkg
\end{python}
\end{leftbar}


\subsection{Installation}
When the \ascii{Wheel} package is successfully built, use \ascii{pip} to install \ascii{TensorFlow} into the system.

\begin{leftbar}
\begin{python}
$ sudo pip install /tmp/tensorflow_pkg/tensorflow-1.4.0-py2-none-any.whl
\end{python}
\end{leftbar}


\subsection{Verification}
Start the \ascii{Python} interpreter and verify that the \ascii{TensorFlow} installation was successful.

\begin{leftbar}
\begin{python}
$ python
>>> import tensorflow as tf
>>> hello = tf.constant('Hello, TensorFlow!')
>>> sess = tf.Session()
>>> print(sess.run(hello))
Hello, TensorFlow!
\end{python}
\end{leftbar}


\subsection{IDE}
Choosing a suitable \ascii{IDE} can improve the quality and speed of code reading before reading the code. It is recommended to use the \ascii{Eclipse CDT} to read the \ascii{C++} code and install the \code{PyDev} plugin to read the \ascii{Python} code. At the same time, it is also recommended to \ascii{Clion}\ascii{C++}, \ascii{PyCharm}\ascii{Python} from \ascii{JetBrains}. However, when reading the \ascii{C++} code, you need to configure the search directory for the \ascii{TensorFlow, CUDA, Eigen3} header files and add the relevant predefined macros so that \code{IDE} correctly parses the symbols in the code. This chapter uses \ascii{Eclipse CDT} as an example to describe the relevant configuration methods.


\subsubsection{Create Eclipse project}
Create a \ascii{Eclipse C++} project, as shown by \refig{setup-eclipse}. Determine the unique project name, manually specify the root directory of the \ascii{TensorFlow} source code, and select the empty project for \ascii{Makefile}. Then, follow the \ascii{Properties > C/C++ General > Paths and Symbols > Includes} configuration directory for the header files.

\begin{table}[!htb]
\resizebox{0.95\textwidth}{!} {
\begin{tabular*}{1.2\textwidth}{@{}ll@{}}
\toprule
\ascii{Configuration Item} & \ascii{Directory} \\
\midrule
\ascii{TensorFlow} & \code{/usr/local/lib/python2.7/site-packages/tensorflow/include} \\
\ascii{CUDA} & \code{/usr/local/cuda/include} \\ 
\bottomrule
\end{tabular*}
}
\caption{header file search directory}
\label{tbl:tf-includes}
\end{table}

\begin{figure}[!htb]
  \centering
  \includegraphics[width=0.75\textwidth]{figures/setup-eclipse.png}
  \caption{Create Eclipse C++ project}
  \label{fig:setup-eclipse}
\end{figure}


\subsubsection{Configure Eigen}
Unfortunately, \ascii{Eigen}'s publicly available header files lack the suffix of \code{.h}, and \ascii{CDT} cannot resolve related symbols. See \code{\href{http://eigen.tuxfamily.org/index.php?title=IDEs}{http://eigen.tuxfamily.org/index.php?title=IDEs}} for instructions, follow \ascii{Preferences > C/C++ > Coding Style > Organize Includes > Header Substitution} Import the \code{eigen-header-substitution.xml} file as shown by \refig{eclipse-eigen3}.

\begin{figure}[!htb]
  \centering
  \includegraphics[width=0.75\textwidth]{figures/eclipse-eigen3.png}
  \caption{Replace the header file of \ascii{Eigen}}
  \label{fig:eclipse-eigen3}
\end{figure}

\end{content}


\section{Code generation}
\begin{content}
When building the \ascii{TensorFlow} system, \ascii{Bazel} or \ascii{CMake} will automatically generate some source code. Understanding the output of the code generator can deepen understanding of the system's behavioral patterns.
\end{content}


\section{Technology stack}
\begin{content}
As shown in \refig{tf-stack}, the \tf{} technology stack is presented in terms of the hierarchy of the system, forming the core of the \tf{} ecosystem.

\begin{figure}[H]
  \centering
  \includegraphics[width=0.7\textwidth]{figures/tf-stack.png}
  \caption{TensorFlow Technology Stack}
  \label{fig:tf-stack}
\end{figure}

\end{content}
\begin{savequote}[45mm]
  \ascii{Any fool can write code that a computer can understand. Good programmers write code that humans can understand.}
  \qauthor{\ascii{- Martin Flower}}
\end{savequote}

\chapter{Icebreaking Tour} 
\label{ch:ice-breaker}
\begin{content}

Before starting to explore the \tf{} internals, hands-on training of the model, familiar with the basic methods of training and tuning techniques, will be of great benefit to understanding the content of the subsequent chapters. Through this study and practice, you will learn how to build and train a neural network \footnote{that recognizes handwritten numbers. This chapter is an excerpt from \ascii{Martin G\"{o}rner} published on \ascii{Codelabs}: \href{https://codelabs.developers.google.com/codelabs/cloud-tensorflow-mnist}{Tensorflow and deep learning, without a PhD}, with the consent of \ascii{Martin G\"{o}rner}, authorization This article is published in this book}.
This chapter will use the single-layer perceptron model, the multi-layer perceptron model, and finally try to use convolutional neural networks. In the training process, some common techniques of algorithm tuning are introduced, including selecting a better activation function, applying learning rate attenuation techniques, and implementing \ascii{Dropout} technology. In the end, the accuracy of the model is increased to above \ascii{99\%}.
Before introducing each network model, the algorithm theory knowledge of the model will be given simply to help you better understand the content of the program. However, this book is not a professional book on machine learning algorithms. For more information on related algorithms, please refer to related literature and papers.
\end{content}


\section{Problem statement}
\begin{content}
This chapter uses the \ascii{MNIST} dataset to complete the network model training of handwritten digits, which contains \ascii{60000} training instances; including \ascii{10000} test instances. As shown in \refig{mnist-x}, for any sample data $x$, use a $28 \times 28$ pixel numeric matrix representation. For simplicity, the matrix implementation of $28 \times 28$ is flattened to obtain a one-dimensional vector of length \ascii{784}.

\begin{figure}[H]
  \centering
  \includegraphics[width=0.9\textwidth]{figures/mnist-x.png}
  \caption{MNIST sample data representation}
  \label{fig:mnist-x}
\end{figure}


\subsection{Data set}
Therefore, in the \ascii{MNIST} training dataset, \code{mnist.train.images} is a two-dimensional matrix of \code{[60000, 784]}. Where each element in the matrix represents the intensity value of a pixel in the image, and its value is between \ascii{0} and \ascii{1}. As shown in \refig{mnist-train-xs}.

\begin{figure}[H]
  \centering
  \includegraphics[width=0.9\textwidth]{figures/mnist-train-xs.png}
  \caption{MNIST training data set: input data set}
  \label{fig:mnist-train-xs}
\end{figure}

Correspondingly, the label of the \ascii{MNIST} dataset is a number between \ascii{0} and \ascii{9}, and \code{mnist.train.labels} is a \code{[60000, 10]} The two-dimensional matrix, where each line is a \ascii{\quo{one-hot}} vector. As shown in \refig{mnist-train-ys}.

\begin{figure}[H]
  \centering
  \includegraphics[width=0.9\textwidth]{figures/mnist-train-ys.png}
  \caption{MNIST training data set: tag data set}
  \label{fig:mnist-train-ys}
\end{figure}


\subsection{Visualization}
\begin{content}

To better visualize the entire training process, draw \ascii{5} types of artboards using the \ascii{matplotlib} toolkit. As shown by \refig{mnist-training-digits}, it represents a training sample data set of \ascii{mini-batch}. Where \code{batch\_size = 100}, a white background indicates that the number is correctly recognized; a red background indicates that the number is misclassified, the left side of the handwritten number identifies the correct tag value, and the right side identifies the wrong predicted value.
\ascii{MNIST} has \ascii{50000} training samples. If \code{batch\_size} is \ascii{100}, it needs to iterate \ascii{500} times to completely traverse a training sample data set. Called an \ascii{epoch} cycle.

\begin{remark}
The sample code in this chapter does not use \ascii{TensorBoard}, but uses \ascii{matplotlib} to observe the curve of error and precision in real time during training.
\end{remark}

\begin{figure}[H]
  \centering
  \includegraphics[width=0.6\textwidth]{figures/mnist-training-digits.jpeg}
  \caption{A training sample data set for mini-batch, where \code{batch\_size=100}}
  \label{fig:mnist-training-digits}
\end{figure}

As shown by \refig{mnist-test-digits}, \ascii{MNIST} uses the test sample dataset of size \ascii{10000} to test the current accuracy of the model. Wherein, the left side indicates the approximate accuracy of the current model; likewise, the white background indicates that the numbers are correctly identified; the red background indicates that the numbers are misclassified, the left side of the handwritten digits identifies the correct label value, and the right side identifies the wrong label. Predictive value.

\begin{figure}[H]
  \centering
  \includegraphics[width=0.6\textwidth]{figures/mnist-test-digits.jpeg}
  \caption{Current model accuracy: based on test sample dataset}
 \label{fig:mnist-test-digits}
\end{figure}

As shown by \refig{mnist-cross-entropy-loss-fig}, the cross-entropy function is used to quantize the error before the predicted value and the tag value. The \ascii{x} axis represents the number of iterations, and the \ascii{y} axis represents the loss value. In addition, based on the training sample data set, the curve of the loss value has a large jitter; and based on the test sample data set, the curve jitter of the loss value is small.

\begin{figure}[H]
  \centering
  \includegraphics[width=0.6\textwidth]{figures/mnist-cross-entropy-loss-fig.jpeg}
  \caption{cross entropy loss for training and testing}
  \label{fig:mnist-cross-entropy-loss-fig}
\end{figure}

As shown by \refig{mnist-accuracy-fig}, the accuracy of the model on the current training dataset and testset can be calculated in real time. The \ascii{x} axis represents the number of iterations, and the \ascii{y} axis represents the precision value. Similarly, based on the training sample data set, the accuracy curve jitter is large; and based on the test sample data set, the precision curve jitter is small.

\begin{figure}[H]
  \centering
  \includegraphics[width=0.6\textwidth]{figures/mnist-accuracy-fig.jpeg}
  \caption{The precision of training and testing}
  \label{fig:mnist-accuracy-fig}
\end{figure}

As shown in \refig{mnist-weight-fig}, for each training parameter (including offset) of the model, its corresponding numerical distribution map can be statistically obtained. When the model cannot converge, the numerical distribution of the parameters can give helpful hints.

\begin{figure}[H]
  \centering
  \includegraphics[width=0.6\textwidth]{figures/mnist-weight-fig.png}
  \caption{weight distribution map}
  \label{fig:mnist-weight-fig}
\end{figure}

\end{content}


\section{Single layer perceptron}
\begin{content}
First, try to build a single layer perceptron for \ascii{10} neurons. As shown by \refig{mnist-slp}, for multi-classification problems such as handwritten digit recognition, the activation function of \ascii{softmax} is often used in theory.

\begin{figure}[H]
  \centering
  \includegraphics[width=0.8\textwidth]{figures/mnist-slp.png}
  \caption{Single layer perceptron}
  \label{fig:mnist-slp}
\end{figure}


\subsection{Theory basis}
In theory, the \ascii{softmax} regression is a generalized extension of the \ascii{logistic} regression. Among them, \ascii{logistic} regression is to solve the two classification problem, namely $y \in \{0,1\}$; and \ascii{softmax} regression is to solve the $k$ classification problem, namely $y \in \{1,2,...,k\}$.


\subsubsection{Symbol definition}
To formalize the \ascii{softmax} regression problem, some common symbols are defined here.
\begin{itemize}
  \item \ascii{Training sample set}: $ S = \{ ({x^{(i)}},{y^{(i)}});i = 1,2,...,m\} $
  \item \ascii{An $i$ training sample}: $ ({x^{(i)}},{y^{(i)}}) $
  \item \ascii{Sample input}: $ x = ({x_1},{x_2},...,{x_n})^{T}  \in {\mathbb{R}^n} $
  \item \ascii{Sample tag (one-hot)}: $ y = ({y_1},{y_2},...,{y_k})^{T} \in {\mathbb{R}^k} $
  \item \ascii{Weight}: $ W \in {\mathbb{R}^{n \times k}} $ 
  \item \ascii{Bias}: $ b \in {\mathbb{R}^k} $ 
  \item \ascii{Softmax function}: $softmax {(z_i)} = \tfrac{{{e^{{z_i}}}}}{{\sum\limits_{j = 1}^k {{e^{{z_j}}}}}} \quad i = 1,2,...,k$
\end{itemize}


\subsubsection{Softmax function}
As shown by \refig{softmax}, the model first obtains the linear weighted sum $z$, then evaluates to $e^z$, and finally implements the normalization operation.

\begin{figure}[H]
  \centering
  \includegraphics[width=0.8\textwidth]{figures/softmax.png}
  \caption{softmax function}
  \label{fig:softmax}
\end{figure}


\subsubsection{Weights and biases}
The weight $W$ is a two-dimensional matrix of $n \times k$.

\[
W = \left( {{W_1},{W_2},...,{W_k}} \right) = \left( {\begin{array}{*{20}{c}}
  {{w_{11}}}& \ldots &{{w_{1k}}} \\ 
   \vdots & \ddots & \vdots  \\ 
  {{w_{n1}}}& \cdots &{{w_{nk}}} 
\end{array}} \right) \in {\mathbb{R}^{n \times k}}
\]

Where $W_j$ is a vector of length $n$.

\[
{W_j} = {\left( {{w_{1j}},{w_{2j}},...,{w_{nj}}} \right)^T} \in {\mathbb{R}^n}, j = 1,2,...,k \\
\]

The bias $b$ is a \ascii{\quo{one-hot}} vector of length $k$.

\[
b = {({b_1},{b_2},...,{b_k})^T} \in {\mathbb{R}^k}
\]


\subsubsection{Model definition}
The single-layer perceptron model for multi-classification problems, using the \ascii{softmax} activation function, can be defined as such.

\[\begin{aligned}
  y =  & {h_{W,b}}(x) = softmax (z) = softmax ({W^T}x + b) \\ 
   =  & {\left( {{y_1},{y_2},...,{y_k}} \right)^T} \\ 
   =  & \frac{1}{{\sum\limits_{j = 1}^k {{e^{{z_j}}}} }}{\left( {{e^{{z_1}}},{e^{{z_2}}},...,{e^{{z_k}}}} \right)^T} \\ 
   =  & \frac{1}{{\sum\limits_{j = 1}^k {{e^{W_j^Tx + {b_j}}}} }}{\left( {{e^{W_1^Tx + {b_1}}},{e^{W_2^Tx + {b_2}}},...,{e^{W_k^Tx + {b_k}}}} \right)^T} \ 
\end{aligned} \]

Where, for any given sample $ (x, y) \in S $, $ z_i $ represents the linear weighted sum of $W_i^Tx+b_i$, and $y_i(i=1,2,...,k )$ indicates the probability of classifying it as class $i$.

\[\begin{gathered}
  P\left( {y = i|x;W,b} \right) = \frac{{{e^{W_i^Tx + b_i}}}}{{\sum\limits_{j = 1}^k {{e^{W_j^Tx + b_j}}} }} \hfill \\
  i = 1,2,...,k \hfill \\ 
\end{gathered} \]


\subsubsection{Cross entropy function}
Based on the sample data set $ S = \{ ({x^{(i)}}, {y^{(i)}});i = 1,2,...,m\} $, cross entropy loss function can be defined as such.

\[\begin{aligned}
  J(W,b) =  &  - \frac{1}{m}\sum\limits_{i = 1}^m {{y^{(i)}}\log \left( {{{\widehat y}^{(i)}}} \right)}  \\ 
   =  &  - \frac{1}{m}\sum\limits_{i = 1}^m {\sum\limits_{j = 1}^k {y_j^{(i)}\log \left( {\widehat y_j^{(i)}} \right)} }  \\
\end{aligned} \]

The \ascii{softmax} multi-classification problem is to find the optimal solution $(W^*,b^*)$, so that

\[W^*,b^* = \mathop {\arg \min }\limits_{W,b} J(W,b)\]


\subsubsection{Calculating gradient}
For any sample $(x,y) \in S $, you can derive a gradient formula for $ J(W,b) $ relative to $ W $ and $ b $.

\[\begin{aligned}
  {\nabla _W}J\left( {W,b;x,y} \right) =  & \left( {\widehat y - y} \right)x \\ 
  {\nabla _b}J\left( {W,b;{x^{(i)}},{y^{(i)}}} \right) =  & \left( {\widehat y - y} \right) \\ 
\end{aligned} \]


\subsubsection{Parameters update}
For the training sample data $S$, according to the gradient formula of $W, b$, the parameter update of this iteration is completed.

\[\begin{aligned}
  W \leftarrow  & W - \alpha \frac{{\sum\limits_{i = 1}^m {{\nabla _W}J\left( {W,b;{x^{(i)}},{y^{(i)}}} \right)} }}{m} \\ 
  b \leftarrow  & b - \alpha \frac{{\sum\limits_{i = 1}^m {{\nabla _b}J\left( {W,b;{x^{(i)}},{y^{(i)}}} \right)} }}{m} \\ 
\end{aligned} \]


\subsection{Define a model}
Next, use \tf{} to complete the construction and training of the model. It should be noted that there are subtle differences between the theoretical formula and the concrete implementation of \tf{}. In theory, $x$ in a formula often represents a sample, but \code{x} in \tf{} often represents a sample dataset of \ascii{mini-batch}. Therefore, when using \tf{} to design a network model, you need to pay special attention to whether the changes in the individual tensors are in line with expectations.


\subsubsection{Inputs and labels}
First, use \code{tf.placeholder} to define the input and label of the training sample separately.

\begin{leftbar}
\begin{python}
x = tf.placeholder(tf.float32, [None, 28, 28, 1])
t = tf.placeholder(tf.float32, [None, 10])
\end{python}
\end{leftbar}

\code{tf.placeholder} defines a placeholder \ascii{OP}. \code{None} represents the number of undetermined samples, which represents the size of \code{batch\_size}; when \code{Session.run}, a \ascii is provided via the \code{feed\_dict} dictionary The sample dataset of {mini-batch} automatically derives the size of \code{tf.placeholder}.

In addition, each image is represented by a three-dimensional data of $ 28 \times 28 \times 1 $ (the scale is \ascii{1}). To simplify the problem, the input sample data is flattened here and transformed into a one-dimensional vector of length \ascii{784}. Where \ascii{-1} represents the number of samples of \ascii{mini-batch}, which is automatically deduced by the runtime.

\begin{leftbar}
\begin{python}
x = tf.reshape(x, [-1, 784])
\end{python}
\end{leftbar}


\subsubsection{Define variables}
Then, define the model parameters using \code{tf.Variable}. When defining the training parameters, you must specify the initialization value of the parameters; the training parameters will automatically derive the type of the data and its size based on the initial values.

\begin{leftbar}
\begin{python}
w = tf.Variable(tf.zeros([784, 10]))
b = tf.Variable(tf.zeros([10]))
\end{python}
\end{leftbar}

In addition, the variables must be initialized before they are used. Here, \code{init\_op} will initialize all global training parameters.

\begin{leftbar}
\begin{python}
init_op = tf.global_variables_initializer()
\end{python}
\end{leftbar}


\subsubsection{Model definition}
Next, the single-layer perceptron model for multi-classification problems can be easily obtained.

\begin{leftbar}
\begin{python}
y = tf.nn.softmax(tf.matmul(x, w) + b)
\end{python}
\end{leftbar}

As shown in \refig{mnist-linear-sum}, first calculate the matrix multiplication of \code{x} and \code{w}, then let \code{b} broadcast (\ascii{broadcast}) to each of the matrices. Add a line and finally get a linear weighted sum of the training parameters.

\begin{figure}[H]
  \centering
  \includegraphics[width=0.8\textwidth]{figures/mnist-linear-sum.png}
  \caption{linear weighted sum}
  \label{fig:mnist-linear-sum}
\end{figure}

As shown by \refig{mnist-softmax}, \ascii{softmax} will implement the operation line by line. Finally, the size of \code{y} is \code{[100, 10]}.

\begin{figure}[H]
  \centering
  \includegraphics[width=0.8\textwidth]{figures/mnist-softmax.png}
  \caption{Activation function: softmax}
  \label{fig:mnist-softmax}
\end{figure}


\subsubsection{Loss function}
For multi-classification problems, the loss function of cross entropy can be used.

\begin{leftbar}
\begin{python}
cross_entropy = -tf.reduce_sum(t * tf.log(y))
\end{python}
\end{leftbar}

As shown by \refig{mnist-cross-entropy}, \code{t} and \code{y} are both \code{[100, 10]}; in particular, every line of \code{t} Is a \quo{\ascii{one-hot}} vector.

Implementing the \code{tf.log} operation on \code{y} will also result in a matrix of size \code{[100, 10]}. Then, \code{t} is multiplied bit by bit with \code{tf.log(y)} (not multiplied by the matrix), and a matrix of size \code{[100, 10]} will also be obtained. Finally, \code{tf.reduce\_sum} adds all the elements in the matrix to get a scalar (\ascii{scalar}) value.

\begin{figure}[H]
  \centering
  \includegraphics[width=0.8\textwidth]{figures/mnist-cross-entropy.png}
  \caption{cross entropy loss function}
  \label{fig:mnist-cross-entropy}
\end{figure}


\subsubsection{Accuracy}
\code{tf.argmax(y,1)} will calculate the index of the maximum value by the \ascii{1} dimension. The index value of the maximum value in each row is calculated for each row of $ y_{100 \times 10} $ . Therefore, \code{tf.argmax(y,1)} will get a matrix of size \code{[100, 1]}, or a vector of size \ascii{100}. Similarly, \code{tf.argmax(t,1)} is also a vector of size \ascii{100}.

Then, use \code{tf.equal} to compare them by element (\ascii{element-wise}) for equality, and get a Boolean vector of size \ascii{100}. In order to calculate the precision, the Boolean vector is first converted to a numerical vector, and finally the mean of the numerical vector is obtained.

\begin{leftbar}
\begin{python}
is_correct = tf.equal(tf.argmax(y,1), tf.argmax(t,1))
accuracy = tf.reduce_mean(tf.cast(is_correct, tf.float32))
\end{python}
\end{leftbar}


\subsection{Optimization Algorithm}
Next, a gradient descent algorithm is used to minimize the cross entropy loss function. Among them, \code{learning\_rate} indicates the learning rate, describes the speed of the parameter update and the size of the step, which is a typical super parameter.

\begin{leftbar}
\begin{python}
Optimizer = tf.train.GradientDescentOptimizer(learning_rate=0.003)
Train_step = optimizer.minimize(cross_entropy)
\end{python}
\end{leftbar}

As shown by \refig{mnist-gd}, the loss function can be likened to a mountain, and the climber tries to find the best course of action to reach the valley. The climber stands on a hillside and decides to take a small step down the opposite direction of the gradient until it reaches a local optimum.

When the gradient descent update algorithm is implemented, the initial points are different, and the minimum values ​​obtained are also different, so the gradient descent is only a local minimum. In addition, the closer to the minimum value, the slower the falling speed. The size of the descending step is also very important. If it is too small, the speed of finding the minimum value of the function is very slow; if it is too large, it may exceed the extreme point.

\begin{figure}[H]
  \centering
  \includegraphics[width=0.8\textwidth]{figures/mnist-gd.jpeg}
  \caption{gradient descent algorithm}
  \label{fig:mnist-gd}
\end{figure}


\subsection{Training a model}
Prior to this, \tf{} only constructed the calculation graph and did not initiate the execution of the calculation graph. Next, the client creates a session, establishes a channel with the local or remote computing device set, and initiates the execution of the computation map.
First, the initialization of the training parameters is completed. The initial value is modified in situ to the corresponding training parameter by running the initialization subgraph of the model parameters and executing the initializer of each training parameter concurrently.

\begin{leftbar}
\begin{python}
With tf.Session() as sess:
  sess.run(init_op)
\end{python}
\end{leftbar}

Then, iteratively executes \code{train\_step} to complete an iterative training of the model. Among them, every \ascii {100} iterations, calculate the accuracy and loss of the current model in the training data set and the test data set.

\begin{leftbar}
\begin{python}
with tf.Session() as sess:
  for step in range(1000):
    batch_xs, batch_ys = mnist.train.next_batch(100)        
    sess.run(train_step, feed_dict={x: batch_xs, t: batch_ys})
    
    if step % 100 == 0:
      acc, loss = sess.run([accuracy, cross_entropy], 
        feed_dict={x: batch_xs, t: batch_ys})
      acc, loss = sess.run([accuracy, cross_entropy], 
        feed_dict={x: mnist.test.images, t: mnist.test.labels}) 
\end{python}
\end{leftbar}

According to statistics, after \ascii{1000} iterations, you get an accuracy of about \percent{92}.

\begin{figure}[H]
  \centering
  \includegraphics[width=0.8\textwidth]{figures/mnist-slp-accuracy.png}
  \caption{Visualization: Single layer perceptron, running 1000 times step}
  \label{fig:mnist-slp-accuracy}
\end{figure}

\end{content}


\section{Multilayer Perceptron}
\begin{content}
To further improve the accuracy, next try to build a multi-layer perceptron model of the \ascii{5} layer.

\begin{figure}[H]
  \centering
  \includegraphics[width=0.8\textwidth]{figures/mnist-5-layer.png}
  \caption{5 layer perceptron}
  \label{fig:mnist-5-layer}
\end{figure}


\subsection{Theory basis}

\subsubsection{Symbol definition}
To formally describe the multilayer perceptron model, some common symbols are defined here.

\begin{itemize}
   \item \alert{$ {n_{\ell}} $}: Number of network layers, where the $0$ layer is the input layer and the $n_{\ell}$ layer is the output layer
   \item \alert{$ {s_{\ell}} $}: Number of nodes in the $\ell$ layer, $ \ell = 0, 1, ..., n_{\ell} $
   \item \alert{$ w_{ji}^{(\ell)} $}: The weight between the $(\ell-1)$ layer node $i$ and the $\ell$ layer node $j$, $ \ell = 1, ..., n_{\ell} $
   \item \alert{$ b_i^{(\ell)} $}: The offset of the $\ell$ layer node $i$, $ \ell = 1, ..., n_{\ell} $
   \item \alert{$ a_i^{(\ell)} $}: Output of $\ell$ layer node $i$, $ \ell = 1, ..., n_{\ell}, x = a^ {(0)}, y = a^{(n_{\ell})} $
   \item \alert{$ z_i^{(\ell)} $}: The weight of the $\ell$ layer node $i$, $ \ell = 1, ..., n_{\ell} $
   \item \alert{$ \delta _i^{(\ell)} $}: Error term for $\ell$ layer node $i$, $ \ell = 1, ..., n_{\ell} $
   \item \alert{$ S = \{ ({x^{(t)}},{y^{(t)}});t = 1,2,...,m\} $}: sample space
 \end{itemize}


\subsubsection{Forward propagation}
$z^{(\ell )}$ represents the linear weighted sum of the $\ell$ layer, which is output by $\ell - 1$ layer $a^{(\ell - 1)}$ and $\ell The weight matrix of the $ layer is multiplied by $w^{(\ell )}$, plus the offset vector of the $\ell$ layer.
By extension, the output of the $\ell$ layer is derived from the activation function $f({z^{(\ell )}})$. Where ${a^{(0)}} = x, y = {a^{({n_\ell })}}$.

\[\begin{gathered}
  {z^{(\ell )}} = {w^{(\ell )}}{a^{(\ell  - 1)}} + {b^{(\ell )}} \hfill \\
  {a^{(\ell )}} = f({z^{(\ell )}}) \hfill \\
  {a^{(0)}} = x \hfill \\
  y = {a^{({n_\ell })}} \hfill \\ 
\end{gathered} \]

\subsubsection{Backward propagation}
Then, the error of each layer is calculated in the reverse direction. The error of the $\ell$ layer is calculated from the error of the $\ell + 1$ layer. Specifically, at the output layer, the error between the predicted value $a^{({n_\ell })}$ and $y$ can be directly calculated.

\[{\delta ^{(\ell)}} = \left\{ \begin{gathered}
  {({w^{(\ell + 1)}})^T}{\delta ^{(\ell + 1)}} \circ f\,'({z^{(\ell)}});{\text{  }}\ell \ne {n_\ell} \hfill \\
  ({a^{(\ell)}} - y) \circ f\,'({z^{(\ell)}}); {\text{  }}\ell = {n_\ell} \hfill \\ 
\end{gathered}  \right.\]

The loss function $J(w,b)$ can be calculated relative to the gradient matrix of the layers and the gradient of the offset vector.

\[\begin{gathered}
  {\nabla _{{w^{(\ell )}}}}J(w,b;x,y) = {\delta ^{(\ell )}}{\left( {{a^{(\ell  - 1)}}} \right)^T} \hfill \\
  {\nabla _{{b^{(\ell )}}}}J(w,b;x,y) = {\delta ^{(\ell )}} \hfill \\
  \ell  = 1,2,...,{n_\ell } \hfill \\ 
\end{gathered} \]

Generally, in the actual system implementation, the downstream layer transfers the gradient to the upper layer, and the upper layer directly completes the calculation of the gradient.


\subsubsection{Parameters update}
For a given sample dataset $ S = \{ ({x^{(t)}}, {y^{(t)}});t = 1,2,...,m\} $, according to the gradient The back-propagation formula can calculate the amount of change in the parameter update.

\[\begin{aligned}
  \Delta {w^{(\ell )}} \leftarrow \Delta {w^{(\ell )}} + {\nabla _{{w^{(\ell )}}}}J\left( {w,b;{x^{(t)}},{y^{(t)}}} \right) \\ 
  \Delta {b^{(\ell )}} \leftarrow \Delta {b^{(\ell )}} + {\nabla _{{b^{(\ell )}}}}J\left( {w,b;{x^{(t)}},{y^{(t)}}} \right) \\ 
  t = 1,2,...,m;\ell  = 1,2,...,{n_\ell } \\ 
\end{aligned} \]

Finally, the gradient descent algorithm is executed to complete an iterative update of the training parameters.

\[\begin{aligned}
  {w^{(\ell )}} \leftarrow  & {w^{(\ell )}} - \alpha \left( {\frac{{\Delta {w^{(\ell )}}}}{m}} \right) \\ 
  {b^{(\ell )}} \leftarrow  & {b^{(\ell )}} - \alpha \frac{{\Delta {b^{(\ell )}}}}{m} \\ 
  \ell  = & 1,2,...,{n_\ell }  \\
\end{aligned} \]


\subsection{Define a model}
Compared to a single-layer perceptron described in the previous section, when defining the weight of each implicit layer, the constant value of the variable is not used to define the initial value of the variable, but a random value that satisfies a certain data distribution feature is used.

\begin{leftbar}
\begin{python}
K = 200
L = 100
M = 60
N = 30

w1 = tf.Variable(tf.truncated_normal([28*28, K] ,stddev=0.1)) 
b1 = tf.Variable(tf.zeros([K]))

w2 = tf.Variable(tf.truncated_normal([K, L], stddev=0.1))
b2 = tf.Variable(tf.zeros([L]))

w3 = tf.Variable(tf.truncated_normal([L, M], stddev=0.1)) 
b3 = tf.Variable(tf.zeros([M]))

w4 = tf.Variable(tf.truncated_normal([M, N], stddev=0.1)) 
b4 = tf.Variable(tf.zeros([N]))

w5 = tf.Variable(tf.truncated_normal([N, 10], stddev=0.1)) 
b5 = tf.Variable(tf.zeros([10]))
\end{python}
\end{leftbar}

The activation function of \ascii{sigmoid} is used when defining each implicit layer. In the final output layer, the activation function of \ascii{softmax} is used.

\begin{leftbar}
\begin{python}
y1 = tf.nn.sigmoid(tf.matmul(x,  w1) + b1)
y2 = tf.nn.sigmoid(tf.matmul(y1, w2) + b2)
y3 = tf.nn.sigmoid(tf.matmul(y2, w3) + b3)
y4 = tf.nn.sigmoid(tf.matmul(y3, w4) + b4)
y  = tf.nn.softmax(tf.matmul(y4, w5) + b5)
\end{python}
\end{leftbar}

After iterative model training, you can get an accuracy of about \percent{97}. However, as the level of the network increases, the model becomes more and more difficult to converge. Next, try some common tuning techniques to improve the performance of your network.


\subsection{Optimization Technology}

\subsubsection{Activation function: ReLU}
In the depth model, it is not appropriate to use the \ascii{sigmoid} activation function. It will squeeze all the values ​​between \ascii{0} and \ascii{1}; as the network level increases, the gradient disappears.

\begin{figure}[H]
  \centering
  \includegraphics[width=0.8\textwidth]{figures/mnist-relu.png}
  \caption{ReLU activation function}
  \label{fig:mnist-relu}
\end{figure}

You can use \ascii{ReLU(Rectified Linear Unit)} instead of \ascii{sigmoid}, which not only avoids some problems caused by \ascii{sigmoid}, but also speeds up the initial convergence.

\begin{leftbar}
\begin{python}
y1 = tf.nn.relu(tf.matmul(x,  w1) + b1)
y2 = tf.nn.relu(tf.matmul(y1, w2) + b2)
y3 = tf.nn.relu(tf.matmul(y2, w3) + b3)
y4 = tf.nn.relu(tf.matmul(y3, w4) + b4)
y  = tf.nn.softmax(tf.matmul(y4, w5) + b5)
\end{python}
\end{leftbar}

In addition, if the \ascii{ReLU} activation function is used, a bias vector is often initialized to a small positive value so that the neuron will work in the non-zero region of \ascii{ReLU} at the beginning.

\begin{leftbar}
\begin{python}
K = 200
L = 100
M = 60
N = 30

w1 = tf.Variable(tf.truncated_normal([28*28, K] ,stddev=0.1)) 
b1 = tf.Variable(tf.ones([L])/10)

w2 = tf.Variable(tf.truncated_normal([K, L], stddev=0.1))
b2 = tf.Variable(tf.ones([L])/10)

w3 = tf.Variable(tf.truncated_normal([L, M], stddev=0.1)) 
b3 = tf.Variable(tf.ones([L])/10)

w4 = tf.Variable(tf.truncated_normal([M, N], stddev=0.1)) 
b4 = tf.Variable(tf.ones([L])/10)

w5 = tf.Variable(tf.truncated_normal([N, 10], stddev=0.1)) 
b5 = tf.Variable(tf.ones([L])/10)
\end{python}
\end{leftbar}

As shown by \refig{mnist-sigmoid-to-relu}, the previous \ascii{300} iterations, using \ascii{ReLU} relative to the use of \ascii{sigmoid}, the initial convergence speed is significantly improved.

\begin{figure}[H]
  \centering
  \includegraphics[width=0.8\textwidth]{figures/mnist-sigmoid-to-relu.png}
  \caption{Apply ReLU activation function: initial convergence speed is significantly improved}
  \label{fig:mnist-sigmoid-to-relu}
\end{figure}


\subsubsection{Undefined values (NaNs)}
In order to obtain a stable numerical calculation result, it is avoided that the accuracy dip is \ascii{0}. Tracing the implementation code, it is possible to introduce \code{log(0)} to calculate the \code{NaN} indefinite value. You can use \code{softmax\_cross\_entropy\_with\_logits} to calculate the cross entropy loss and use the linear weighted sum as its input (often called \ascii{logits}).

\begin{leftbar}
\begin{python}
logits = tf.matmul(y4, w5) + b5
y = tf.nn.softmax(logits)
cross_entropy = tf.nn.softmax_cross_entropy_with_logits(logits=logits, labels=t)
\end{python}
\end{leftbar}


\subsubsection{Learning rate decay}
With the increase of the network level and the application of relevant optimization techniques, the accuracy of the model can be obtained by \percent{98}, but it is difficult to obtain a stable accuracy. As shown by \refig{mnist-lr-too-larger}, the accuracy and loss jitter are quite obvious.

\begin{figure}[H]
  \centering
  \includegraphics[width=0.8\textwidth]{figures/mnist-lr-too-larger.png}
  \caption{noise jitter: learning rate is too large}
  \label{fig:mnist-lr-too-larger}
\end{figure}

Better optimization algorithms can be used, such as \code{AdamOptimizer}. With the number of iterations, the learning rate is exponentially attenuated, and a more stable accuracy and loss curve can be obtained later in the model training.

\begin{leftbar}
\begin{python}
lr = tf.placeholder(tf.float32)
train_step = tf.train.AdamOptimizer(lr).minimize(cross_entropy)
\end{python}
\end{leftbar}

In each iterative training process, the learning rate of the current iteration \code{lr} is calculated in real time according to the value of the current \code{step}, and then passed to \code{Session.run} via \code{feed\_dict} . Among them, the learning rate decay equation is shown in the following code. As the number of iterations increases, the learning rate exponentially decays.

\begin{leftbar}
\begin{python}
def lr(step):
  max_lr, min_lr, decay_speed = 0.003, 0.0001, 2000.0
  return min_lr + (max_lr - min_lr) * math.exp(-step/decay_speed)

with tf.Session() as sess:
  for step in range(10000):
    batch_xs, batch_ys = mnist.train.next_batch(100)
    sess.run(train_step, 
      feed_dict={x: batch_xs, t: batch_ys, pkeep: 0.75, lr: lr(step)})
\end{python}
\end{leftbar}

As shown in \refig{mnist-apply-learning-rate-decay}, after applying the learning rate fading method, a more stable accuracy and loss curve can be obtained.

\begin{figure}[H]
  \centering
  \includegraphics[width=0.8\textwidth]{figures/mnist-apply-learning-rate-decay.png}
  \caption{After applying the Adam optimization algorithm, the accuracy and loss tend to be stable}
  \label{fig:mnist-apply-learning-rate-decay}
\end{figure}


\subsubsection{Apply Dropout}
However, the loss curve is separated from the training set and the test set, and there is a significant over-fitting phenomenon. That is, the model performs well on the training dataset, but there is a rebound in the test dataset, and the model lacks sufficient generalization ability.

\begin{figure}[H]
  \centering
  \includegraphics[width=0.8\textwidth]{figures/mnist-overfitting.png}
  \caption{over-fitting}
  \label{fig:mnist-overfitting}
\end{figure}

As shown in \refig{mnist-dropout}, the \ascii{dropout} operation is performed on the output of the hidden layer during training, and the output of the neuron is randomly discarded with the probability of \code{1 - pkeep}, and the gradient is propagated in the reverse direction. The corresponding weights are no longer updated. Restoring the output of all neurons during reasoning indirectly improves the generalization ability of the network.

\begin{figure}[H]
  \centering
  \includegraphics[width=0.8\textwidth]{figures/mnist-dropout.png}
  \caption{Dropout method}
  \label{fig:mnist-dropout}
\end{figure}

When using \tf{} to implement \ascii{dropout} operation, first define a super parameter \code{pkeep}, indicating that the hidden layer neurons are randomly reserved with probability \code{pkeep}, with probability \code{1 - pkeep} Discard randomly.

\begin{leftbar}
\begin{python}
pkeep = tf.placeholder(tf.float32)

y1 = tf.nn.relu(tf.matmul(x,  w1) + b1)
y1d = tf.nn.dropout(y1, pkeep)

y2 = tf.nn.relu(tf.matmul(y1d, w2) + b2)
y2d = tf.nn.dropout(y2, pkeep)

y3 = tf.nn.relu(tf.matmul(y2d, w3) + b3)
y3d = tf.nn.dropout(y3, pkeep)

y4 = tf.nn.relu(tf.matmul(y3d, w4) + b4)
y4d = tf.nn.dropout(y4, pkeep)

logits = tf.matmul(y4d, w5) + b5
y = tf.nn.softmax(logits)
\end{python}
\end{leftbar}

In training, the value of the super parameter \code{pkeep} is less than \ascii{1}; in the case of reasoning, the value of the super parameter \code{pkeep} is \ascii{1}.

\begin{leftbar}
\begin{python}
with tf.Session() as sess:
  for step in range(10000):
    batch_xs, batch_ys = mnist.train.next_batch(100)
    sess.run(train_step, 
      feed_dict={x: batch_xs, t: batch_ys, pkeep: 0.75, lr: lr(step)})

    if step % 100 == 0:
      acc, loss = sess.run([accuracy, cross_entropy], 
        feed_dict={x: batch_xs, t: batch_ys, pkeep: 1})
      acc, loss = sess.run([accuracy, cross_entropy], 
        feed_dict={x: mnist.test.images, t: mnist.test.labels, pkeep: 1})
\end{python}
\end{leftbar}

After implementing the \ascii{dropout} operation at each hidden layer, the training set and the test set's loss curve intersect again. However, there is a small jitter in the accuracy and loss curve, and the degree of coincidence between the training set and the loss curve of the test set is not very satisfactory. The over-fitting problem is still outstanding.

That is to say, there are other more profound reasons for the overfitting problem. For example, flattening a $28\times 28$ image and transforming it into a one-dimensional vector of length \ascii{784} will completely lose the spatial arrangement of the pixels.

Next, by attempting to construct a convolutional neural network, features are extracted from the original image, thereby preserving the spatial arrangement information of the pixels, thereby improving the performance of the network.

\begin{figure}[H]
  \centering
  \includegraphics[width=0.8\textwidth]{figures/mnist-apply-dropout-result.png}
  \caption{After implementing Dropout, the loss curve of the training set and the test set coincide again}
  \label{fig:mnist-apply-dropout-result}
\end{figure}

\end{content}


\section{Convolution network}
\begin{content}


\subsection{Features and Advantages}

As the network level increases, the problem of the gradient of the fully connected network disappears, and the convergence speed becomes slower and slower. Compared with fully connected networks, convolutional networks have \ascii{3} main features, which reduce the number of network parameters and improve the generalization ability of the network.


\subsubsection{Local connectivity}
The convolutional network implements a local connection relative to a fully connected network, ie each neuron does not have a connection to the neurons of the previous layer. As shown on the left side of \refig{mnist-conv-local-conn}, if there is a $1000\times 1000 $pixel image, and its $10^6$ hidden layer of neurons. In a fully connected network, you will have $10^3\times 10^3 \times 10^6 = 10^{12} $ training parameters.

In fact, it is not necessary for each neuron to be connected to the upper layer of neurons. As shown on the right side of \refig{mnist-conv-local-conn}, if each hidden layer of neurons is only connected to the local image of the previous $10\times 10 $, $10^6 $ hidden layer The neurons need $10^6\times 10^2 = 10^8$ network connections, compared to the reduced \ascii{4} orders of magnitude.

\begin{figure}[H]
  \centering
  \includegraphics[width=0.9\textwidth]{figures/mnist-conv-local-conn.png}
  \caption{local connection}
  \label{fig:mnist-conv-local-conn}
\end{figure}


\subsubsection{Weight sharing}
To further reduce network connectivity, the convolutional network also implements weight sharing; that is, each group of connections shares the same weight, rather than having different weights for each connection. As shown on the right side of \refig{mnist-conv-local-conn-2}, each hidden layer of neurons is only connected to a partial image of $10\times 10 $ and shares the weight of $ 10 \times 10 $ Matrix, independent of the number of neurons in the hidden layer. Relative to the local connection network on the left side of \refig{mnist-conv-local-conn-2}, $10^8$ parameters are required, and the convolution layer only needs $10^2$ parameters.

As shown on the right side of \refig{mnist-conv-local-conn-2}, multiple filters can be used in order to extract different features, such as image features of different edges. For example, if there are \ascii{100} filters, you need $10^4$ parameters.

\begin{figure}[H]
  \centering
  \includegraphics[width=0.9\textwidth]{figures/mnist-conv-local-conn-2.png}
  \caption{weight sharing, multiple filters}
  \label{fig:mnist-conv-local-conn-2}
\end{figure}


\subsubsection{Downsampling}
As shown in \refig{mnist-subsample}, downsampling is optionally implemented to further reduce network parameters and improve the robustness of the model.

\begin{figure}[H]
  \centering
  \includegraphics[width=0.9\textwidth]{figures/mnist-subsample.png}
  \caption{downsampling}
  \label{fig:mnist-subsample}
\end{figure}


\subsection{Convolution operation}

The convolution operation is a computationally intensive \ascii{OP}. As shown in \refig{mnist-conv2d-gif}, there is a weight vector \code{w[3,3,3,2]}, where the number of input channels is \ascii{3} and the number of output channels is \ascii{ 2}, the convolution kernel size is $3 \times 3$.

Obviously, the number of channels of the input image is equivalent to the depth of the convolution kernel; the number of convolution kernels is equal to the number of output channels of \ascii{Feature Map}. In addition, in order to capture the edge features of the image, a circle of zero values ​​(\ascii{padding}) is added to the periphery of the original image. For each convolution calculation, the step size of the move (\ascii{stride}) is \ascii{2}. Therefore, the final output of the \ascii{Feature Map} is $3 \times 3$.

\begin{figure}[H]
  \centering
  \includegraphics[width=0.8\textwidth]{figures/mnist-conv2d-gif.png}
  \caption{convolution operation}
  \label{fig:mnist-conv2d-gif}
\end{figure}


\subsubsection{Example}
If there is a $32 \times 32 \times 3$ image, the convolution kernel size is $5 \times 5 \times 3$. The depth of the convolution kernel is equal to the number of input channels of the picture. As shown in \refig{mnist-conv-1dot}, the convolution kernel performs a dot product with a block of size $5 \times 5 \times 3$ in the image to get a value.

\begin{figure}[H]
  \centering
  \includegraphics[width=0.8\textwidth]{figures/convolutional-layer-2.png}
  \caption{Convolution operation: dot product operation of convolution kernel and picture block}
  \label{fig:mnist-conv-1dot}
\end{figure}

As shown in \refig{mnist-conv-ndot}, the convolution kernel traverses the entire image space, resulting in a \ascii{Feature Map} with a size of $28 \times 28 \times 1$.

\begin{figure}[H]
  \centering
  \includegraphics[width=0.8\textwidth]{figures/convolutional-layer-3.png}
  \caption{Convolution operation: Convolution kernel traverses the picture in steps of 1}
  \label{fig:mnist-conv-ndot}
\end{figure}

As shown in \refig{mnist-conv-multi-filters}, if there are multiple convolution kernels, get multiple \ascii{Feature Map}.

\begin{figure}[H]
  \centering
  \includegraphics[width=0.8\textwidth]{figures/convolutional-layer-4.png}
  \caption{Convolution operation: multiple convolution kernels}
  \label{fig:mnist-conv-multi-filters}
\end{figure}


\subsection{Formula derivation}

\subsubsection{Forward propagation}
$Z^{(\ell )}$ represents the linear weighted sum of the $\ell$ layer, which is output by $\ell - 1$ layer $A^{(\ell - 1)}$ and $\ell The weight matrix of the $ layer is $W^{(\ell )}$ convolution, plus the offset vector of the $\ell$ layer.

By extension, the output of the $\ell$ layer is derived from the activation function $f({Z^{(\ell )}})$. Where ${A^{(0)}} = x, y = {A^{({n_\ell })}}$.

\[\begin{gathered}
  {Z^{(\ell )}} = {A^{(\ell  - 1)}} * {W^{(\ell )}} + {b^{(\ell )}} \hfill \\
  {A^{(\ell )}} = f\left( {{Z^{(\ell )}}} \right) \hfill \\ 
\end{gathered} \]


\subsubsection{Backward propagation}
Then, the error of each layer is calculated in the reverse direction. The error of the $\ell$ layer is calculated from the error of the $\ell + 1$ layer. Compared to a fully connected network, here is a convolution operation, not a matrix multiplication operation.

\[
{\delta ^{(\ell )}} = {\delta ^{(\ell  + 1)}} * {W^{(\ell  + 1)}} \circ f\,'\left( {{z^{(\ell )}}} \right)
\]

The loss function $J(w,b)$ can be calculated relative to the gradient matrix of the layers and the gradient of the offset vector.

\[\begin{aligned}
  {\nabla _{{W^{(\ell )}}}}J(W,b) =  & {A^{(\ell  - 1)}} * {\delta ^{(\ell )}} \\ 
  {\nabla _{{b^{(\ell )}}}}J(W,b) =  & {\delta ^{(\ell )}} \\ 
\end{aligned} \]


\subsection{Implement convolutional network}
To implement a convolutional network, you first need to define a weight matrix for each layer of filters to extract features of the image. The weight matrix appears in the image as a filter that extracts specific information from the original image matrix. A weight matrix may be used to extract image edge information, one weight matrix may be used to extract a particular color, and another weight matrix may be used to blur unwanted noise.

When there are multiple convolutional layers, the initial layer tends to extract more general features; as the network structure becomes deeper, the features extracted by the weight matrix become more and more complex, and are more and more applicable to the specific problems at hand.

In general, filters often use a tensor representation of the \ascii{4} dimension. The first two dimensions represent the size of the filter, the third dimension represents the number of channels input, and the fourth dimension represents the number of channels output. As shown in \refig{mnist-filter}.

\begin{figure}[H]
  \centering
  \includegraphics[width=0.4\textwidth]{figures/mnist-filter.png}
  \caption{Convolutional layer filter}
  \label{fig:mnist-filter}
\end{figure}

As shown in \refig{mnist-conv2d-1}, \ascii{3} convolutional layers and \ascii{2} fully connected layers are constructed. Among them, the middle hidden layer uses the activation function of \ascii{ReLU}, and the last output layer uses the activation function of \ascii{softmax}.

\begin{figure}[H]
  \centering
  \includegraphics[width=0.9\textwidth]{figures/mnist-conv2d-1.png}
  \caption{implementation of convolutional neural networks}
  \label{fig:mnist-conv2d-1}
\end{figure}

Use \tf{} to implement a convolutional network, as shown in the following code.

\begin{leftbar}
\begin{python}
K = 4 
L = 8
M = 12
N = 200

w1 = tf.Variable(tf.truncated_normal([5, 5, 1, K], stddev=0.1))
b1 = tf.Variable(tf.ones([K])/10)

w2 = tf.Variable(tf.truncated_normal([5, 5, K, L], stddev=0.1))
b2 = tf.Variable(tf.ones([L])/10)

w3 = tf.Variable(tf.truncated_normal([4, 4, L, M], stddev=0.1))
b3 = tf.Variable(tf.ones([M])/10)

w4 = tf.Variable(tf.truncated_normal([7 * 7 * M, N], stddev=0.1))
b4 = tf.Variable(tf.ones([N])/10)

w5 = tf.Variable(tf.truncated_normal([N, 10], stddev=0.1))
b5 = tf.Variable(tf.ones([10])/10)

y1 = tf.nn.relu(tf.nn.conv2d(
       x,  w1, strides=[1, 1, 1, 1], padding='SAME') + b1)
y2 = tf.nn.relu(tf.nn.conv2d(
       y1, w2, strides=[1, 2, 2, 1], padding='SAME') + b2)
y3 = tf.nn.relu(tf.nn.conv2d(
       y2, w3, strides=[1, 2, 2, 1], padding='SAME') + b3)

yy = tf.reshape(Y3, shape=[-1, 7 * 7 * M])
y4 = tf.nn.relu(tf.matmul(yy, w4) + b4)

logits = tf.matmul(y4, w5) + b5
y = tf.nn.softmax(logits)
\end{python}
\end{leftbar}

As shown in \refig{mnist-conv2d-1-result}, after $10^4$ training, you can get an accuracy of about \percent{98.9}.

\begin{figure}[H]
  \centering
  \includegraphics[width=0.9\textwidth]{figures/mnist-conv2d-1-result.png}
  \caption{Implementation of Convolutional Network: Can get the accuracy of \percent{98.9}}
  \label{fig:mnist-conv2d-1-result}
\end{figure}


\subsection{Enhanced Convolution Network}
As shown in \refig{mnist-conv2d-2}, the previous network hierarchy is preserved, and \ascii{3} convolutional layers and \ascii{2} fully connected layers are constructed. Among them, the middle hidden layer uses the activation function of \ascii{ReLU}, and the last output layer uses the activation function of \ascii{softmax}.

However, more channels are used to extract more features than previous convolutional networks. At the same time, the \ascii{dropout} operation is implemented in the hidden layer of the full connection to enhance the generalization ability of the network.

\begin{figure}[H]
  \centering
  \includegraphics[width=0.9\textwidth]{figures/mnist-conv2d-2.png}
  \caption{Improve Convolutional Neural Networks}
  \label{fig:mnist-conv2d-2}
\end{figure}

Use \tf{} to implement a larger convolutional network, as shown in the following code.

\begin{leftbar}
\begin{python}
K = 6
L = 12
M = 24
N = 200

w1 = tf.Variable(tf.truncated_normal([6, 6, 1, K], stddev=0.1))
b1 = tf.Variable(tf.ones([K])/10)

w2 = tf.Variable(tf.truncated_normal([5, 5, K, L], stddev=0.1))
b2 = tf.Variable(tf.ones([L])/10)

w3 = tf.Variable(tf.truncated_normal([4, 4, L, M], stddev=0.1))
b3 = tf.Variable(tf.ones([M])/10)

w4 = tf.Variable(tf.truncated_normal([7 * 7 * M, N], stddev=0.1))
b4 = tf.Variable(tf.ones([N])/10)

w5 = tf.Variable(tf.truncated_normal([N, 10], stddev=0.1))
b5 = tf.Variable(tf.ones([10])/10)

y1 = tf.nn.relu(tf.nn.conv2d(
       x,  w1, strides=[1, 1, 1, 1], padding='SAME') + b1)
y2 = tf.nn.relu(tf.nn.conv2d(
       y1, w2, strides=[1, 2, 2, 1], padding='SAME') + b2)
y3 = tf.nn.relu(tf.nn.conv2d(
       y2, w3, strides=[1, 2, 2, 1], padding='SAME') + b3)

yy = tf.reshape(Y3, shape=[-1, 7 * 7 * M])
y4 = tf.nn.relu(tf.matmul(yy, w4) + b4)
y4d = tf.nn.dropout(y4, pkeep)

logits = tf.matmul(y4d, w5) + b5
y = tf.nn.softmax(logits)
\end{python}
\end{leftbar}

As shown in \refig{mnist-conv2d-2-result}, after $10^4$ training, you can get an accuracy of about \percent{99.3}.

\begin{figure}[H]
  \centering
  \includegraphics[width=0.9\textwidth]{figures/mnist-conv2d-2-result.png}
  \caption{Enhanced convolutional network: Get the accuracy of \percent{99.3}}
  \label{fig:mnist-conv2d-2-result}
\end{figure}

At the same time, the over-fitting problem has been significantly improved over previously implemented convolutional networks, as shown by \refig{mnist-conv2d-3-result}.

\begin{figure}[H]
  \centering
  \includegraphics[width=0.9\textwidth]{figures/mnist-conv2d-3-result.png}
  \caption{Enhanced convolutional network: Over-fitting problems are significantly improved}
  \label{fig:mnist-conv2d-3-result}
\end{figure}

\end{content}


\part{System structure}
\begin{savequote}[45mm]
\ascii{Any fool can write code that a computer can understand. Good programmers write code that humans can understand.}
\qauthor{\ascii{- Martin Flower}}
\end{savequote}

\chapter{System Architecture} 
\label{ch:architecture}

\begin{content}

This chapter will explain the system architecture of \tf{}, and a simple example of the transformation process of the graph structure. Finally, by mining the working mechanism of session management, deepen understanding of the working mechanism of \tf{} runtime.

\end{content}

\section{System Architecture}
	
\begin{content}

As shown in \refig{tf-architecture}, the system structure of \tf{} is bounded by \ascii{C API}, dividing the whole system into \emph{front end} and \emph{backend} two subsystems\footnote {In fact, the code of \ascii{Client} also exists in the backend system, and the front-end system is the external programming interface of \tf{}. This issue will be discussed in detail in later chapters. }.

\begin{enum}
  \eitem{front-end system: provides a programming model responsible for constructing computational graphs;}
  \eitem{Backend system: Provides a runtime environment responsible for executing calculation diagrams. }
\end{enum}

The system design of \tf{} follows a good layered architecture, and the design and implementation of the backend system can be further broken down into \ascii{4} layers.

\begin{enum}
  \eitem{Runtime: Provide local mode and distributed mode, respectively, and share most of the design and implementation;}
  \eitem{Calculation layer: consists of the \ascii{Kernel} implementation of each \ascii{OP}; at runtime, \ascii{Kernel} implements the specific mathematical operations of \ascii{OP}; 
  \eitem{Communication layer: implements data exchange between components based on \ascii{gRPC} and can implement \ascii{RDMA} communication between nodes supporting \ascii{IB} network;
  \eitem{Device layer: The computing device is the main carrier executed by \ascii{OP}, and \tf{} supports a variety of heterogeneous computing device types. }
\end{enum}

From the point of view of the operation of the system, the \tf{} runtime is to complete the construction, layout, and operation of the calculation diagram.

\begin{enum}
  \eitem{Expression diagram: constructing a calculation diagram, but not executing a diagram;}
  \eitem{Organization diagram: The nodes that calculate the graph are deployed on the various computing devices in the cluster with the best execution plan; } 
  \eitem{Run graph: Execute the nodes in the graph by topology and start the \ascii{Kernel} calculation for each \ascii{OP}. }   
\end{enum}

\begin{figure}[H]
\centering
\includegraphics[width=1.0\textwidth]{figures/tf-architecture.png}
\caption{TensorFlow System Architecture}
 \label{fig:tf-architecture}
\end{figure}

\subsection{Client}

\ascii{Client} is the main component of the front-end system, which is a programming environment that supports multiple languages. \ascii{Client} constructs a calculation graph based on the programming interface of \ascii{TensorFlow}. Currently, \ascii{TensorFlow} supports the \ascii{Python} and \ascii{C++} programming interfaces, especially the \ascii{API} support for \ascii{Python} is the most comprehensive. Also, \ascii{API} support for other programming languages ​​is getting better.

At this point, \ascii{TensorFlow} does not perform any graph calculation until \ascii{Session} is established with the background calculation engine, and \ascii{Session} is used as a bridge to establish \ascii{Client} and \ascii{Master} The channel between, and \ascii{GraphDef} in \ascii{Protobuf} format is serialized and passed to \ascii{Master} to start the execution of the calculation graph.

\subsection{Master}

In a distributed runtime environment, \ascii{Client} executes \code{Session.run} and passes the entire calculation graph to the backend's \ascii{Master}. At this point, the calculation graph is complete, often called \emph{\ascii{Full Graph}}. Subsequently, \ascii{Master} passes the \code{fetches, feeds} parameter list passed to it according to \code{Session.run}, traverses \ascii{Full Graph} in reverse, and prunes it according to the dependency. Finally, the smallest dependent subgraph is calculated, often called \ascii{Client Graph}.

Next, \ascii{Master} is responsible for splitting \ascii{Client Graph} by task name (\code{SplitByTask}) into multiple \ascii{Graph Partition}; each \ascii{Worker} corresponds to a \ascii {Graph Partition}. Subsequently, \ascii{Master} registers \ascii{Graph Partition} to the corresponding \ascii{Worker} to execute these \ascii{Graph Partition} concurrently on different \ascii{Worker}. Finally, \ascii{Master} will notify all \ascii{Work} to start the corresponding \ascii{Graph Partition} implementation.

Among them, there may be data dependencies between \ascii{Work}, \ascii{Master} does not participate in the data exchange between the two, they communicate with each other, and independently exchange data until all calculations are completed.

\subsection{Worker}

For each task, \tf{} will launch an \ascii{Worker} instance. \ascii{Worker} is mainly responsible for the following \ascii{3} aspects:

\begin{enum}
  \eitem{Processing requests from \ascii{Master};}
  \eitem{Performs the registered \ascii{Graph Partition} to perform a second split (\code{SplitByDevice}) according to the local computing device set, and notifies each computing device to execute each \ascii{Graph Partition} concurrently;
  \eitem{Execute local submap on a computing device according to the topology sorting algorithm, and schedule \ascii{Kernel} implementation of \ascii{OP}; 
  \eitem{Data communication between collaborative tasks. }
\end{enum}

First, \ascii{Worker} receives the graph execution command sent by \ascii{Master}. The calculation graph at this time is complete with respect to \ascii{Worker}, also known as \ascii{Full Graph}, which corresponds to \ascii{Graph Partition} of \ascii{Master}. Subsequently, \ascii{Worker} according to the currently available hardware environment, including \ascii{(GPU/CPU)} resources, according to the constraint specification of \ascii{OP} device, then split the graph \code{(SplitByDevice)} \ascii{Graph Partition}; where each computing device corresponds to a \ascii{Graph Partition}. Next, \ascii{Worker} starts the execution of all \ascii{Graph Partition}. Finally, for each computing device, \ascii{Worker} will perform the topology sorting algorithm according to the dependencies between the nodes in the calculation graph, and then call \ascii{Kernel} of \ascii{OP} to complete the \ascii{OP The operation of } (a typical polymorphic implementation technique).

Among them, \ascii{Worker} is also responsible for sending the results of the \ascii{OP} operation to other \ascii{Worker}, or accepting the results from other \ascii{Worker} to achieve \ascii Data interaction between {Worker}. The \tf{} specialization implements \ascii{Send/Recv} between the source device and the target device.

\begin{enum}
  Between \eitem{local\ascii{CPU} and \ascii{GPU}, use \code{cudaMemcpyAsync} to implement asynchronous copy;}
  Between \eitem{local\ascii{GPU}, use the end-to-end \ascii{DMA} operation to avoid copying of the host\ascii{CPU}. }
\end{enum}

For communication between tasks, \tf{} supports multiple communication protocols.

\begin{enum}
  \eitem{\ascii{gRPC over TCP;}}
  \eitem{\ascii{RDMA over Converged Ethernet. }}
\end{enum}

In addition, \tf{} has initially begun to support the \ascii{cuNCCL} library to improve communication between multiple \ascii{GPU}.

\subsection{Kernel}

\ascii{Kernel} is a specific implementation of \ascii{OP} in a hardware device that is responsible for executing the specific operations of \ascii{OP}. Currently, the \ascii{TensorFlow} system contains \ascii{200} multiple standard \ascii{OP}, including numerical calculations, multidimensional array operations, control flow, state management, and more.

Generally, every \ascii{OP} will have an optimized \ascii{Kernel} implementation depending on the device type. At runtime, the runtime selects a specific \ascii{Kernel} implementation for \ascii{OP} based on the device constraint specification of \ascii{OP} and its local device type, completing the calculation of \ascii{OP}.

Among them, most \ascii{Kernel} is based on \code{Eigen::Tensor}. \code{Eigen::Tensor} is an efficient concurrency code for multicore \ascii{CPU/GPU} using \ascii{C++} template technology. However, \ascii{TensorFlow} also has the flexibility to use \ascii{cuDNN, cuNCCL, cuBLAS} to implement more efficient \ascii{Kernel}.

In addition, \ascii{TensorFlow} implements vectorization technology to enable more efficient reasoning in high-throughput, data-centric application requirements and in mobile devices. If the sub-calculation process for composite \ascii{OP} is difficult to represent, or the execution is inefficient, \ascii{TensorFlow} even supports more efficient \ascii{Kernel} registration, and its scalability is very good.

\end{content}

\section{图控制}

\begin{content}

Through a simple example, further stripping and stripping, and gradually unearthing the control and operation mechanism of the \tf{} calculation graph.

\subsection{Building a cluster}

As shown in \refig{tf-1ps-1worker}. If there is a simple distributed environment: \ascii{1 PS + 1 Worker}, and divide it into two tasks:

\begin{enum}
  \eitem{\ascii{ps0}: Use the \code{/job:ps/task:0} tag to be responsible for storing and updating model parameters;}
  \eitem{\ascii{worker0}: The \code{/job:worker/task:0} tag is responsible for the training of the model. }
\end{enum}

\begin{figure}[!htbp]
\centering
\includegraphics[width=0.9\textwidth]{figures/tf-1ps-1worker.png}
\caption{TensorFlow cluster: \ascii{1 PS + 1 Worker}}
 \label{fig:tf-1ps-1worker}
\end{figure}

\subsection{Figure construction}

As shown in \refig{tf-graph-construction}. \ascii{Client} builds a simple calculation graph; first, multiply the $w$ and $x$ matrices, add them in bits with the intercept $b$, and finally update them to $s$.

\begin{figure}[!htbp]
\centering
\includegraphics[width=0.9\textwidth]{figures/tf-graph-construction.png}
\caption{Figure construction}}
 \label{fig:tf-graph-construction}
\end{figure}

\subsection{图Execution}

As shown in \refig{tf-graph-execution}. First, \ascii{Client} creates a \code{Session} instance and establishes a channel with \ascii{Master}; then, \ascii{Client} passes the calculation graph to \ by calling \code{Session.run} Ascii{Master}.

Subsequently, \ascii{Master} starts the graph calculation process of \ascii{Step}. Before execution, \ascii{Master} implements a series of optimization techniques such as \emph{common expression elimination}, \emph{constant folding} and so on. Finally, \ascii{Master} is responsible for the collaboration between the tasks and performs the optimized calculation diagram.

\begin{figure}[!htbp]
\centering
\includegraphics[width=0.9\textwidth]{figures/tf-graph-execution.png}
\caption{图作}}
 \label{fig:tf-graph-execution}
\end{figure}

\subsubsection{Figure split}

As shown in \refig{tf-graph-split-by-task}, there is a reasonable graph partitioning algorithm. \ascii{Master} divides the \ascii{OP} related to the model parameters into a group and places them on the \ascii{ps0} task; the other \ascii{OP} is divided into another group and placed in \ascii{worker0} Execute on the task.

\begin{figure}[!htbp]
\centering
\includegraphics[width=1.0\textwidth]{figures/tf-graph-split-by-task.png}
\caption{Figure split: by task}}
 \label{fig:tf-graph-split-by-task}
\end{figure}

\subsubsection{Submap registration}

As shown in \refig{tf-register-graph}. During the graph splitting process, if the edge of the computed graph spans the node or device, \ascii{Master} splits the edge and inserts the \ascii{Send} and \ascii{Recv} nodes between the two nodes or devices. The delivery of data.

The \code{Send} and \code{Recv} nodes are also \ascii{OP}, except that they are two special \ascii{OP}, which are managed and controlled by the internal runtime and are invisible to the user; They are only used for data communication and do not have any logic for data calculation.

Finally, \ascii{Master} registers the submap to the corresponding \ascii{Worker} by calling the \code{RegisterGraph} interface, and the corresponding \ascii{Worker} is responsible for performing the operation.

\begin{figure}[!htbp]
\centering
\includegraphics[width=1.0\textwidth]{figures/tf-register-graph.png}
\caption{Submap registration: Insert Send and Recv nodes}}
 \label{fig:tf-register-graph}
\end{figure}

\subsubsection{Submap operation}

As shown in \refig{tf-run-graph}. \ascii{Master} notifies all \ascii{Worker} to perform submap operations by calling the \code{RunGraph} interface. Among them, \ascii{Worker} can exchange data by calling \code{RecvTensor} interface.

\begin{figure}[!htbp]
\centering
\includegraphics[width=1.0\textwidth]{figures/tf-run-graph.png}
\caption{Submap execution}}
 \label{fig:tf-run-graph}
\end{figure}

\end{content}

\section{session management}
	
\begin{content}

Next, the internal runtime mechanism of the runtime is further uncovered by an overview of the entire lifecycle process of the session and its association with the graph control.

\subsection{Create session}

First, when \ascii{Client}\emph{first} executes \code{tf.Session.run}, the entire image will be serialized and the \code{CreateSessionRequest} message will be sent via \ascii{gRPC} to pass the image to \ascii{Master}.

Subsequently, \ascii{Master} creates a \code{MasterSession} instance, which is identified by the globally unique \code{handle} and finally returned to \ascii{Client} via \code{CreateSessionResponse}. As shown in \refig{tf-create-session-overview}.

\begin{figure}[!h]
\centering
\includegraphics[width=0.7\textwidth]{figures/tf-create-session-overview.png}
\caption{Create session}}
 \label{fig:tf-create-session-overview}
\end{figure}

\subsection{iteration run}

Subsequently, \ascii{Client} initiates the iterative execution process and calls each iteration a \ascii{Step}. At this point, \ascii{Client} sends a \code{RunStepRequest} message to \ascii{Master}, and the message carries the \code{handle} identifier for the corresponding \code{MasterSession} instance of \ascii{Master}. As shown in \refig{tf-run-step-overview}.

\begin{figure}[!h]
\centering
\includegraphics[width=1.0\textwidth]{figures/tf-run-step-overview.png}
\caption{iteration execution}}
 \label{fig:tf-run-step-overview}
\end{figure}

\subsubsection{Register Submap}

After \ascii{Master} receives the \code{RunStepRequest} message, it will perform operations such as pruning, splitting, and optimization. Finally, according to the task \ascii{(Task)}, the graph is divided into multiple sub-picture segments \ascii{(Graph Partition)}. Subsequently, \ascii{Master} sends a \code{RegisterGraphRequest} message to each \ascii{Worker}, registering the sub-picture segments to each \ascii{Worker} node in turn.

When \ascii{Worker} receives the \code{RegisterGraphRequest} message, it performs the split operation again, and finally divides the graph into multiple sub-picture segments \ascii{(Graph Partition)} according to the device \ascii{(Device)}. \footnote{When distributed running, the graph splits through a two-stage splitting process. Split by task on \ascii{Master} and split by device in \ascii{Worker}. Therefore, the results are referred to as sub-picture segments, which only have a range and a difference in their size. }

When \ascii{Worker} completes the subgraph registration, it returns the \code{RegisterGraphReponse} message and carries the \code{graph\_handle} flag. This is because \ascii{Worker} can register and run multiple subgraphs concurrently, each using a unique identifier of \code{graph\_handle}.

\subsubsection{Run Submap}

After \ascii{Master} completes the subgraph registration, all \ascii{Worker} will be broadcast and all subgraphs will be executed concurrently. This process is done by \ascii{Master} sending a \code{RunGraphRequest} message to \ascii{Worker}. The message carries the identifier information of the \code{(session\_handle, graph\_handle, step\_id)} triplet for the corresponding subgraph of the \ascii{Worker} index.

After \ascii{Worker} receives the message \code{RunGraphRequest} message, \ascii{Worker} indexes the corresponding submap according to \code{graph\_handle}. Finally, \ascii{Worker} starts all local computing devices and executes all submaps concurrently. Among them, each subgraph is placed in a separate \code{Executor}, and \code{Executor} will calculate the subpicture segment according to the topological sorting algorithm. The above algorithm can be formally described as the following code.


\begin{leftbar}
  \begin{python}
Def run_partitions(rendezvous, executors_and_partitions, inputs, outputs):
  Rendezvous.send(inputs)
  For (executor, partition) in executors_and_partitions: 
    Executor.run(partition)
  Rendezvous.recv(outputs)
  \end{python}
\end{leftbar}

\subsubsection{Exchange data}

If there is a need to exchange data between the two devices, it is done by inserting the \ascii{Send/Recv} node. In particular, if there is a need to exchange data between two \ascii{Worker}, then cross-process communication is required.

At this point, you need to send the \code{RecvTensorRequest} message to the sender through the receiver, and then take the corresponding \ascii{Tensor} from the sender's mailbox and return it via \code{RecvTensorResponse}. As shown in \refig{tf-recv-tensor-overview}.

\begin{figure}[!h]
\centering
\includegraphics[width=0.7\textwidth]{figures/tf-recv-tensor-overview.png}
\caption{Data exchange between workers}}
 \label{fig:tf-recv-tensor-overview}
\end{figure}

\subsection{Close session}

When the calculation is complete, \ascii{Client} sends a \code{CloseSessionReq} message to \ascii{Master}. After \ascii{Master} receives the message, it starts releasing all the resources held by \code{MasterSession}. As shown in \refig{tf-close-session-overview}.

\begin{figure}[!h]
\centering
\includegraphics[width=0.7\textwidth]{figures/tf-close-session-overview.png}
\caption{Close session}}
 \label{fig:tf-close-session-overview}
\end{figure}

\end{content}
\begin{savequote}[45mm]
  \ascii{Any fool can write code that a computer can understand. Good programmers write code that humans can understand.}
  \qauthor{\ascii{- Martin Flower}}
\end{savequote}

\chapter{C API: Watershed} 
\label{ch:c-api}
\begin{content}
This chapter uses the implementation of the client\ascii{Session} lifecycle as an example to reveal the implementation channel of the front-end \ascii{Python} and the backend \cpp{} system, revealing the mystery of \ascii{TensorFlow} multi-language programming.
\end{content}


\section{Swig: Behind the scenes}
\begin{content}
The front-end multi-language programming environment and the backend \cpp{} implementation of the system's channel are attributed to the \ascii{Swig} wrapper. \ascii{TensorFlow} uses the \ascii{Bazel} build tool to start the \ascii{Swig} code generation process before the system is compiled, and automatically generates two adaptations via \code{tensorflow.i} (\ascii{Wrapper})file:

\begin{enum}
  \eitem{\code{pywrap\_tensorflow\_internal.py}: Responsible for docking the upper \ascii{Python} call;}
  \eitem{\code{pywrap\_tensorflow\_internal.cc}: Responsible for docking the underlying \ascii{C API} call. }
\end{enum}

As shown by \refig{swig}, the \code{pywrap\_tensorflow\_internal.py} module automatically loads the dynamic link library of \code{\_pywrap\_tensorflow\_internal.so} when it is first imported; where \code{\_pywrap\_tensorflow\_internal.so} contains all the symbols for the entire \tf{} runtime. In the implementation of \code{pywrap\_tensorflow\_internal.cc}, a function symbol table is statically registered, and the binary relationship of the \ascii{Python} function name to the \ascii{C} function name is implemented. At runtime, according to the function name of \ascii{Python}, the matching finds the corresponding \ascii{C} function implementation, and finally realizes the calling relationship of \ascii{Python} to \code{c\_api.c}.

\begin{figure}[H]
  \centering
  \includegraphics[width=1.0\textwidth]{figures/swig.png}
  \caption{SwigCode Generator}
  \label{fig:swig}
\end{figure}

The generation rules for \ascii{Bazel} are defined in \code{//tensorflow/python:pywrap\_tensorflow\_internal} as shown in the following code.

\begin{leftbar}
\begin{python}
tf_py_wrap_cc(
    name = "pywrap_tensorflow_internal",
    srcs = ["tensorflow.i"],
    swig_includes = [
        "client/device_lib.i",
        "client/events_writer.i",
        "client/tf_session.i",
        "client/tf_sessionrun_wrapper.i",
        "framework/cpp_shape_inference.i",
        "framework/python_op_gen.i",
        "grappler/cost_analyzer.i",
        "grappler/model_analyzer.i",
        "grappler/tf_optimizer.i",
        "lib/core/py_func.i",
        "lib/core/strings.i",
        "lib/io/file_io.i",
        "lib/io/py_record_reader.i",
        "lib/io/py_record_writer.i",
        "platform/base.i",
        "pywrap_tfe.i",
        "training/quantize_training.i",
        "training/server_lib.i",
        "util/kernel_registry.i",
        "util/port.i",
        "util/py_checkpoint_reader.i",
        "util/stat_summarizer.i",
        "util/tfprof.i",
        "util/transform_graph.i",
    ]
)
\end{python}
\end{leftbar}

The following takes the implementation of the client\ascii{Session} lifecycle as an example to reveal the implementation channel of the front-end \ascii{Python} and backend \cpp{} systems.

\end{content}


\section{Session control}
\begin{content}
Strictly speaking, \ascii{C API} is not the dividing line between \ascii{Client} and \ascii{Master}. As shown by \refig{tf-client-session}, \ascii{Client} has a partial \cpp{} implementation, ie \code{tensorflow::Session}. The \code{tf.Session} instance directly holds the handle of the \code{tensorflow::Session} instance. In the actual runtime environment, there may be multiple implementations of \code{tensorflow::Session}. For example, \code{DirectSession} is responsible for session control for \emph{local mode}. And \code{GrpcSession} is responsible for session control of \emph{distributed mode} based on \ascii{gRPC} protocol. In general, users use \code{tf.Session} to implement programming instead of \code{tensorflow::Session}.

\begin{figure}[!htbp]
  \centering
  \includegraphics[width=0.9\textwidth]{figures/tf-client-session.png}
  \caption{client: tensorflow::Session instance creation process}
  \label{fig:tf-client-session}
\end{figure}

\end{content}


\section{Session life cycle}
\begin{content}
The life cycle of a session includes the creation of a session, the creation of a calculation graph, the expansion of a computation graph, the execution of a computation graph, the closure of a session, and the destruction of a session. The front-end \ascii{Python} and backend\cpp{} behave as two sets of compatible interface implementations.


\subsection{Python frontend}
As shown in \refig{py-session-lifecycle}, in the \ascii{Python} front end, the life cycle of \code{Session} is mainly reflected in:

\begin{enum}
  \eitem{Create \code{Session(target)};}
  \eitem{Iteration execution \code{Session.run(fetches, feed\_dict)};}
    \begin{enum}
      \eitem{\code{Session.\_extend\_graph(graph)};}
      \eitem{\code{Session.TF\_Run(feeds, fetches, targets)};}
    \end{enum}
  \eitem{close \code{Session};}
  \eitem{destroy \code{Session};}
\end{enum}

\begin{figure}[H]
  \centering
  \includegraphics[width=0.5\textwidth]{figures/py-session-lifecycle.png}
  \caption{Python: Session lifecycle}
  \label{fig:py-session-lifecycle}
\end{figure}

For example, here the local mode \code{Session} instance is created and the training process for \code{mnist} is started.

\begin{leftbar}
\begin{python}
sess = tf.Session()
for _ in range(1000):
  batch_xs, batch_ys = mnist.train.next_batch(100)
  sess.run(train_step, feed_dict={x: batch_xs, y_: batch_ys})
sess.close()
\end{python}
\end{leftbar}


\subsection{C++Backend}
Accordingly, in the \cpp{} backend, the life cycle of \code{Session} is mainly reflected in:

\begin{enum}
  \eitem{Create \code{Session} based on \code{target} polymorphism;}
  \eitem{\code{Session.Create(graph)}: Yes and only once;}
  \eitem{\code{Session.Extend(graph)}: zero or more times;}
  \eitem{Iterative execution \code{Session.Run(inputs, outputs, targets)};}
  \eitem{Shut down \code{Session.Close};}
  \eitem{Destroy the \code{Session} object. }
\end{enum}

\begin{figure}[H]
  \centering
  \includegraphics[width=0.9\textwidth]{figures/cc-session-lifecycle.png}
  \caption{C++: Session lifecycle}
  \label{fig:cc-session-lifecycle}
\end{figure}

For example, here the local mode \code{DirectSession} instance is created and the execution of the calculation graph is started.

\begin{leftbar}
\begin{c++}
// create/load graph ...
tensorflow::GraphDef graph;

// local runtime, target is ""
tensorflow::SessionOptions options;

// create Session
std::unique_ptr<tensorflow::Session> 
sess(tensorflow::NewSession(options));

// create graph at initialization.
tensorflow::Status s = sess->Create(graph);
if (!s.ok()) { ... }

// run step
std::vector<tensorflow::Tensor> outputs;
s = session->Run(
  {},               // inputs is empty
  {"output:0"},     // outputs names
  {"update_state"}, // target names
  &outputs);        // output tensors
if (!s.ok()) { ... }

// close
session->Close();
\end{c++}
\end{leftbar}

\end{content}


\section{Create session}
\begin{content}
The following is a detailed process of session creation. Starting from the \ascii{Python} front end, the \ascii{Python-C++} wrapper automatically generated by \ascii{Swig} is used as a medium to implement \ascii{Python } Call to \ascii{C API} of \ascii{TensorFlow}. Among them, \ascii{C API} is the watershed of the front-end system and the back-end system.

\begin{figure}[H]
  \centering
  \includegraphics[width=1.0\textwidth]{figures/py-create-session.png}
  \caption{Create session}
  \label{fig:py-create-session}
\end{figure}


\subsection{Programming interface}
When \ascii{Client} wants to start the execution of the calculation graph, it first creates a \code{Session} instance, and then calls the constructor of the parent class \code{BaseSession}.

\begin{leftbar}
\begin{python}[caption={tensorflow/python/client/session.py}]
class Session(BaseSession):
  def __init__(self, target='', graph=None, config=None):
    super(Session, self).__init__(target, graph, config=config)
    self._default_graph_context_manager = None
    self._default_session_context_manager = None
\end{python}
\end{leftbar}

In the constructor of \code{BaseSession}, the function in the \code{pywrap\_tensorflow} module will be called. Among them, \code{TF\_NewDeprecatedSession} is a legacy interface that has been deprecated. In the new \ascii{API}, passing the graph instance directly to the backend \ascii{C++} avoids the overhead of serializing the front and back graph instances.

\begin{leftbar}
\begin{python}[caption={tensorflow/python/client/session.py}]
from tensorflow.python import pywrap_tensorflow as tf_session

class BaseSession(SessionInterface):
  def __init__(self, target='', graph=None, config=None):
    # default graph
    if graph is None:
      self._graph = ops.get_default_graph()
    else:
      self._graph = graph

    # handle to tensorflow::Session
    self._session = None
    opts = tf_session.TF_NewSessionOptions(target=self._target, 
                                           config=config)
    try:
      with errors.raise_exception_on_not_ok_status() as status:
        if self._created_with_new_api:
          self._session = tf_session.TF_NewSession(
              self._graph._c_graph, opts, status)
        else:
          self._session = tf_session.TF_NewDeprecatedSession(opts, status)
    finally:
      tf_session.TF_DeleteSessionOptions(opts)
\end{python}
\end{leftbar}

As shown by \refig{tf-graph-inst}, \code{ScopedTFGraph} is a wrapper for \code{TF\_Graph} that does the work of \ascii{RAII} like \ascii{C++}. And \code{TF\_Graph} holds the \code{ternsorflow::Graph} instance. Where \code{self.\_graph.\_c\_graph} returns a \code{TF\_Graph} instance, which is created by \ascii{C API}.

\begin{leftbar}
\begin{python}[caption={tensorflow/python/framework/ops.py}]
class Graph(object):
  def __init__(self):
    if _USE_C_API:
      self._scoped_c_graph = c_api_util.ScopedTFGraph()
    else:
      self._scoped_c_graph = None

  def _c_graph(self):
    if self._scoped_c_graph:
      return self._scoped_c_graph.graph
    return None
\end{python}
\end{leftbar}

\begin{leftbar}
\begin{python}[caption={tensorflow/python/framework/c\_api\_util.py}]
class ScopedTFGraph(object):
  def __init__(self):
    self.graph = c_api.TF_NewGraph()

  def __del__(self):
    if c_api.TF_DeleteGraph is not None:
      c_api.TF_DeleteGraph(self.graph)
\end{python}
\end{leftbar}

\begin{figure}[H]
  \centering
  \includegraphics[width=0.3\textwidth]{figures/tf-graph-inst.png}
  \caption{front and rear: pass the graph instance}
  \label{fig:tf-graph-inst}
\end{figure}


\subsubsection{Python wrapper}
In the \code{pywrap\_tensorflow} module, through \code{\_pywrap\_tensorflow\_internal} forwarding, from \ascii{Python} to the dynamic link library \code{\_pywrap\_tensorflow\_internal.so} Function call.

\begin{leftbar}
\begin{python}[caption={tensorflow/bazel-bin/tensorflow/python/pywrap\_tensorflow\_internal.py}]
def TF_NewDeprecatedSession(opts, status):
  return _pywrap_tensorflow_internal.TF_NewDeprecatedSession(opts, status)

def TF_NewSession(graph, opts, status):
  return _pywrap_tensorflow_internal.TF_NewSession(graph, opts, status)
\end{python}
\end{leftbar}


\subsubsection{C++ wrapper}
In the concrete implementation of \code{pywrap\_tensorflow\_internal.cc}, the symbol table of the function call is statically registered, and the function name of \ascii{Python} is implemented to the specific mapping of the \cpp{} function implementation.

\begin{leftbar}
\begin{c++}[caption={tensorflow/bazel-bin/tensorflow/python/pywrap\_tensorflow\_internal.cc}]
static PyMethodDef SwigMethods[] = {
  // ...
  { (char *)"TF_NewDeprecatedSession", 
    _wrap_TF_NewDeprecatedSession, METH_VARARGS, NULL},

  { (char *)"TF_NewSession", 
    _wrap_TF_NewSession, METH_VARARGS, NULL},
};
\end{c++}
\end{leftbar}

Finally, \code{\_wrap\_TF\_NewSession/\_wrap\_TF\_NewDeprecatedSession} will call \code{c\_api.h} to open the \ascii{API} interface: \code{TF\_NewSession/TF \_NewDeprecatedSession}. In other words, the automatically generated \code{pywrap\_tensorflow\_internal.cc} is only responsible for forwarding the \ascii{Python} function to the \ascii{C/C++} function call, and will eventually call the underlying \ascii{C} system up. The \ascii{API} interface provided.


\subsection{C API}
\code{c\_api.h} is the public \ascii{API} interface open to the front end of the \ascii{TensorFlow} backend execution system. Among them, the new interface implementation uses the technique of reference counting, and the implementation of the graph instance is shared among multiple \code{Session} instances.

\begin{leftbar}
\begin{c++}[caption={tensorflow/c/c\_api.c}]
TF_Session* TF_NewSession(TF_Graph* graph, const TF_SessionOptions* opt,
                          TF_Status* status) {
  Session* session;
  status->status = NewSession(opt->options, &session);
  if (status->status.ok()) {
    if (graph != nullptr) {
      mutex_lock l(graph->mu);
      graph->num_sessions += 1;
    }
    return new TF_Session(session, graph);
  } else {
    return nullptr;
  }
}

TF_DeprecatedSession* TF_NewDeprecatedSession(const TF_SessionOptions* opt,
                                              TF_Status* status) {
  Session* session;
  status->status = NewSession(opt->options, &session);
  if (status->status.ok()) {
    return new TF_DeprecatedSession({session});
  } else {
    DCHECK_EQ(nullptr, session);
    return nullptr;
  }
}
\end{c++}
\end{leftbar}


\subsection{Backend system}
\code{NewSession} will create different types of \code{tensorflow::Session} instances using \code{SessionFactory} polymorphism based on the \code{target} passed to the frontend.

\begin{leftbar}
\begin{c++}[caption={tensorflow/c/c\_api.c}]
Status NewSession(const SessionOptions& options, Session** out_session) {
  SessionFactory* factory;
  Status s = SessionFactory::GetFactory(options, &factory);
  if (!s.ok()) {
    *out_session = nullptr;
    return s;
  }
  *out_session = factory->NewSession(options);
  if (!*out_session) {
    return errors::Internal("Failed to create session.");
  }
  return Status::OK();
}
\end{c++}
\end{leftbar}


\subsubsection{Factory method}
In the backend \cpp{} implementation, the creation of \code{tensorflow::Session} uses the abstract factory method. If \code{target} in \code{SessionOptions} is an empty string (the default), create a \code{DirectSession} instance and start the local run mode; if \code{target} in \code{SessionOptions} At the beginning of \code{grpc://}, create a \code{GrpcSession} instance and start a distributed run mode based on \code{RPC}. As shown in \refig{cc-session-factory}.

\begin{figure}[H]
  \centering
  \includegraphics[width=1.0\textwidth]{figures/cc-session-factory.png}
  \caption{tensorflow::Session creation: abstract factory method}
  \label{fig:cc-session-factory}
\end{figure}

\end{content}


\section{Create/Extension Diagram}
\begin{content}
In the existing interface implementation, the graph constructed by the graph construction period needs to be serialized and passed to the backend \ascii{C++} system. In the new interface implementation, there is no need to implement the creation or extension of the graph. Because when you create \ascii{OP}, the node is added to the graph instance of the backend \ascii{C++} system in real time. However, the implementation of the existing interface is more important for understanding the behavior of the system.

The \ascii{Python} frontend will iteratively call the \code{Session.run} interface, and the constructed calculation graph will be sent to the \cpp{} backend as \code{GraphDef}. Each time the front end calls the \code{Session.run} interface, it will try to send the calculation graph of the newly added node to the backend system, so as to add the calculation graph \ascii{Extend} of the newly added node to the original calculation diagram. Specifically, the first computational graph will be sent to the backend system when \code{Session.run} is first called.

The first time the backend system calls \code{Session.Extend}, it is transposed (or equivalent) \code{Session.Create}. Later, each time the backend system calls \code{Session.Extend}, the semantics of \code{Extend} will be actually executed, and the nodes of the newly added calculation graph will be added to the original calculation graph.

\begin{figure}[H]
  \centering
  \includegraphics[width=0.9\textwidth]{figures/py-session-create-graph.png}
  \caption{Create a diagram}
  \label{fig:py-session-create-graph}
\end{figure}


\subsection{Programming interface}
In the existing interface implementation, the extension of the graph instance is implemented by \code{\_extend\_graph}.

\begin{leftbar}
\begin{python}[caption={tensorflow/python/client/session.py}]
class Session(BaseSession):
  def run(self, fetch_list, feed_dict=None, options=None, run_metadata=None):
    # ignores implements...
    self._extend_graph()
    # ignores implements...

\end{python}
\end{leftbar}

When \code{self.\_extend\_graph} is called for the first time, or when a new node is added to the calculation graph, serialization is performed on the calculation graph \code{GraphDef}, which finally triggers \code{tf\_session.TF\_ExtendGraph}.

\begin{leftbar}
\begin{python}[caption={tensorflow/python/client/session.py}]
from tensorflow.python import pywrap_tensorflow as tf_session

class Session(BaseSession):
  def _extend_graph(self):
    if self._created_with_new_api: return

    with self._extend_lock:
      if self._graph.version > self._current_version:
        graph_def, self._current_version = self._graph._as_graph_def(
            from_version=self._current_version,
            add_shapes=self._add_shapes)

        with errors.raise_exception_on_not_ok_status() as status:
          tf_session.TF_ExtendGraph(
              self._session, graph_def.SerializeToString(), status)
\end{python}
\end{leftbar}


\subsubsection{Python wrapper}
\begin{leftbar}
\begin{python}[caption={tensorflow/bazel-bin/tensorflow/python/pywrap\_tensorflow\_internal.py}]
def TF_ExtendGraph(sess, graph_def, status):
  return _pywrap_tensorflow.TF_ExtendGraph(sess, graph_def, status)
\end{python}
\end{leftbar}


\subsubsection{C++ wrapper}
\begin{leftbar}
\begin{c++}[caption={tensorflow/bazel-bin/tensorflow/python/pywrap\_tensorflow\_internal.cc}]
static PyMethodDef SwigMethods[] = {
  // ignore implements...
  { (char *)"TF_ExtendGraph", 
    _wrap_TF_ExtendGraph, METH_VARARGS, NULL},
};
\end{c++}
\end{leftbar}


\subsection{C API}
\code{TF\_ExtendGraph} is the interface for the \ascii{C API} docking of the upper programming environment. First, it completes the deserialization of the computed graph \code{GraphDef} and finally calls the \code{Extend} interface of \code{tensorflow::Session}.

\begin{leftbar}
\begin{c++}[caption={tensorflow/c/c\_api.c}]
void TF_ExtendGraph(TF_DeprecatedSession* sess, 
  const void* proto, size_t proto_len, TF_Status* status) {
  GraphDef g;
  if (! tensorflow :: ParseProtoUnlimited (& g, therefore, proto_len)) {
    status->status = InvalidArgument("Invalid GraphDef");
    return;
  }
  status->status = sess->session->Extend(g);
}
\end{c++}
\end{leftbar}


\subsection{Backend system}
\code{tensorflow::Session} will polymorphically call the implementation of the corresponding subclass at runtime based on the dynamic type of \code{Session}.

\begin{leftbar}
\begin{c++}[caption={tensorflow/core/common\_runtime/session.h}]
class Session {
public:
  virtual Status Create(const GraphDef& graph) = 0;
  virtual Status Extend(const GraphDef& graph) = 0;
};
\end{c++}
\end{leftbar}

Where \code{Create} means to register the calculation graph on the current \code{tensorflow::Session} instance. If you want to register a new calculation graph, you need to close the \code{tensorflow::Session} object. \code{Extend} indicates that the node is appended to the computed graph on the \code{tensorflow::Session} instance. When \code{Extend} is first executed, it is equivalent to the semantics of \code{Create}, because for the first time \code{Extend}, the registered calculation graph is empty. In fact, the system is implemented according to the above scheme, here is an example of the \code{GrpcSession} implementation.


\subsubsection{First expansion map: GrpcSession}
If it is judged that \code{handle} of \code{Master} is not empty, execute \code{Extend}; otherwise, execute the semantics of \code{Create}, establish a connection with \code{Master}, and hold \code{handle} of \code{MasterSession}.

\begin{leftbar}
\begin{c++}[caption={tensorflow/core/distributed\_runtime/rpc/grpc\_session.cc}]
Status GrpcSession::Extend(const GraphDef& graph) {
  CallOptions call_options;
  call_options.SetTimeout(options_.config.operation_timeout_in_ms());
  return ExtendImpl(&call_options, graph);
}

Status GrpcSession::ExtendImpl
  (CallOptions* call_options, const GraphDef& graph) {
  if (handle_is_empty()) {
    // Session was unitialized, 
    // so simply initialize the session with 'graph'.
    return Create(graph);
  }
  // ignore implements...  
}
\end{c++}
\end{leftbar}

\end{content}


\section{Iteration run}
\begin{content}
As shown in \refig{py-session-run}, the \ascii{Python} frontend\code{Session.run} implementation passes \code{fetches, feed\_dict} to the backend system, and the backend system calls \code{ Session.Run} interface. A \code{Session.Run} execution of the backend system is often referred to as \ascii{Step}. Among them, the execution process of \ascii{Step} is the key path of \ascii{TensorFlow} runtime.

\begin{figure}[H]
  \centering
  \includegraphics[width=1.0\textwidth]{figures/py-session-run.png}
  \caption{iteration execution}
  \label{fig:py-session-run}
\end{figure}


\subsection{Programming interface}
When \ascii{Client} calls \code{Session.run}, the function in the \code{pywrap\_tensorflow\_internal} module will eventually be called.

\begin{leftbar}
\begin{python}[caption={tensorflow/python/client/session.py}]
from tensorflow.python import pywrap_tensorflow as tf_session

class Session(BaseSession):
  def run(self, fetch_list, feed_dict=None, options=None, run_metadata=None):
    # ignores other implements...
    self._extend_graph()
    with errors.raise_exception_on_not_ok_status() as status:
      if self._created_with_new_api:
        return tf_session.TF_SessionRun_wrapper(
            session, options, feed_dict, fetch_list, target_list,
            run_metadata, status)
      else:
        return tf_session.TF_Run(session, options,
                                 feed_dict, fetch_list, target_list,
                                 status, run_metadata)
\end{python}
\end{leftbar}


\subsubsection{Python wrapper}

\begin{leftbar}
\begin{python}[caption={tensorflow/bazel-bin/tensorflow/python/pywrap\_tensorflow\_internal.py}]
def TF_SessionRun_wrapper(session, run_options, inputs, 
  outputs, targets, run_metadata, out_status):
  return _pywrap_tensorflow_internal.TF_SessionRun_wrapper(
    session, run_options, inputs, outputs, targets, run_metadata, out_status)

def TF_Run(sess, options, feeds, outputs, 
  targets, status, run_metadata):
  return _pywrap_tensorflow.TF_Run(
    sess, options, feeds, outputs, targets, status, run_metadata)
\end{python}
\end{leftbar}


\subsubsection{C++ wrapper}

\begin{leftbar}
\begin{c++}[caption={tensorflow/bazel-bin/tensorflow/python/pywrap\_tensorflow\_internal.cc}]
static PyMethodDef SwigMethods[] = {
  // ...
  { (char *)"TF_Run", 
    _wrap_TF_Run, METH_VARARGS, NULL},

  { (char *)"TF_SessionRun_wrapper", 
    _wrap_TF_SessionRun_wrapper, METH_VARARGS, NULL},
};
\end{c++}
\end{leftbar}

Finally, \code{\_wrap\_TF\_Run/\_wrap\_TF\_SessionRun\_wrapper} will be transferred to the \code{TF\_Run/TF\_SessionRun} interface function corresponding to \ascii{C API}.


\subsection{C API}
In the existing interface, \code{TF\_Run} is the interface of the \ascii{C API} docking upper-level programming environment. First, it completes the format conversion of input data from \ascii{C} to \cpp{} and starts the execution of the background \code{tensorflow::Session}. When the execution is complete, convert the output data of \code{outputs} from \cpp{} to \ascii{C}. \code{TF\_SessionRun} is similar to \code{TF\_Run} and will not be redundant here.

\begin{leftbar}
\begin{c++}[caption={tensorflow/c/c\_api.c}]
void TF_Run(TF_DeprecatedSession* s, 
  // session options
  const TF_Buffer* run_options,
  // Input tensors
  const char** c_input_names, TF_Tensor** c_inputs, int ninputs,
  // Output tensors
  const char** c_output_names, TF_Tensor** c_outputs, int noutputs,
  // Target nodes
  const char** c_target_oper_names, int ntargets,
  // run\_metadata
  TF_Buffer* run_metadata, TF_Status* status) {
  // convert data format, ignore implements...
  s->session->Run(options_proto, input_names, output_names,
                  target_names, &outputs, &run_metadata); 
  // store results in c\_outputs...
}

void TF_SessionRun(TF_Session* session, 
  const TF_Buffer* run_options,
  // Input tensors
  const TF_Output* inputs, TF_Tensor* const * input_values, int ninputs, 
  // Output tensors
  const TF_Output* outputs, TF_Tensor** output_values, int noutputs,
  // Target nodes
  const TF_Operation* const* target_opers, int ntargets,
  // run\_metadata
  TF_Buffer* run_metadata, TF_Status* status) {
  // ignore implements.
}
\end{c++}
\end{leftbar}


\subsection{Backend system}
\code{tensorflow::Session} will call the corresponding subclass implementation polymorphically at runtime according to its dynamic type.

\begin{leftbar}
\begin{c++}[caption={tensorflow/core/common\_runtime/session.h}]
class Session {
public:
  virtual Status Run(
    const RunOptions& options,
    const vector<pair<string, Tensor> >& inputs,
    const vector<string>& output_names,
    const vector<string>& target_names,
    vector<Tensor>* outputs, RunMetadata* run_metadata) {
      return errors::Unimplemented(
        "Run with options is not supported for this session.");
  }
};
\end{c++}
\end{leftbar}

Inputs include:

\begin{enum}
  \eitem{\code{options}:\code{Session} running configuration parameters;}
  \eitem{\code{inputs}: Enter a list of names for \code{Tensor};}
  \eitem{\code{output\_names}: Output a list of names for \code{Tensor};}
  \eitem{\code{targets}: No output, list of names of \ascii{OP} to be executed. }
\end{enum}

The output includes:

\begin{enum}
  \eitem{\code{outputs}: list of output \code{Tensor};}
  \eitem{\code{run\_metadata}: a collector for runtime metadata.}
\end{enum}

Among them, the output \code{outputs} list has a one-to-one correspondence with the input \code{output\_names}. If the runtime is executed concurrently, the \code{outputs} will be executed out of order, and the final return will need to be compared with the input \code. {output\_names} name list, sorting \code{outputs}.

\end{content}


\section{Close session}
\begin{content}
After the calculation graph is executed, you need to close \code{tf.Session} to release the backend system resources, including the queue, \ascii{IO}, and so on. The session closure process is simpler, as shown by \refig{py-session-close}. .

\begin{figure}[H]
  \centering
  \includegraphics[width=0.9\textwidth]{figures/py-session-close.png}
  \caption{Close session}
  \label{fig:py-session-close}
\end{figure}


\subsection{Programming interface}
When \ascii{Client} calls \code{Session.close}, the function in the \code{pywrap\_tensorflow} module will eventually be called: \code{TF\_CloseDeprecatedSession}.

\begin{leftbar}
\begin{python}[caption={tensorflow/python/client/session.py}]
from tensorflow.python import pywrap_tensorflow as tf_session

class Session(BaseSession):
  def close(self):
    if self._created_with_new_api:
      if self._session and not self._closed:
        self._closed = True
        with errors.raise_exception_on_not_ok_status() as status:
          tf_session.TF_CloseSession(self._session, status)
    else:
      with self._extend_lock:
        if self._opened and not self._closed:
          self._closed = True
          with errors.raise_exception_on_not_ok_status() as status:
            tf_session.TF_CloseDeprecatedSession(self._session, status)
\end{python}
\end{leftbar}


\subsubsection{Python wrapper}

\begin{leftbar}
\begin{python}[caption={tensorflow/bazel-bin/tensorflow/python/pywrap\_tensorflow\_internal.py}]
def TF_CloseSession(sess, status):
    return _pywrap_tensorflow_internal.TF_CloseSession(sess, status)

def TF_CloseDeprecatedSession(sess, status):
  return _pywrap_tensorflow.TF_CloseDeprecatedSession(sess, status)
\end{python}
\end{leftbar}


\subsubsection{C++ wrapper}
\code{\_wrap\_TF\_CloseSession/\_wrap\_TF\_CloseDeprecatedSession} will be transferred to the \code{TF\_CloseSession/TF\_CloseDeprecatedSession} interface function corresponding to \ascii{C API}.

\begin{leftbar}
\begin{c++}[caption={tensorflow/bazel-bin/tensorflow/python/pywrap\_tensorflow\_internal.cc}]
static PyMethodDef SwigMethods[] = {
  // ...
  { (char *)"TF_CloseSession", 
    _wrap_TF_CloseSession, METH_VARARGS, NULL},

  { (char *)"TF_CloseDeprecatedSession", 
    _wrap_TF_CloseDeprecatedSession, METH_VARARGS, NULL},
};
\end{c++}
\end{leftbar}


\subsection{C API}
\code{TF\_CloseSession/TF\_CloseDeprecatedSession} directly completes the closing of \code{tensorflow::Session}.

\begin{leftbar}
\begin{c++}[caption={tensorflow/c/c\_api.c}]
void TF_CloseSession(TF_Session* s, TF_Status* status) {
  status->status = s->session->Close();
}

void TF_CloseDeprecatedSession(TF_DeprecatedSession* s, TF_Status* status) {
  status->status = s->session->Close();
}
\end{c++}
\end{leftbar}


\subsection{Backend system}
\code{Session(C++)} At runtime, its dynamic type will polymorphically call the corresponding subclass implementation.

\begin{leftbar}
\begin{c++}[caption={tensorflow/core/common\_runtime/session.h}]
class Session {
public:
  virtual Status Close() = 0;
};
\end{c++}
\end{leftbar}

\end{content}


\section{Destroy session}
\begin{content}
When \code{tf.Session} is not being used, it is released by \ascii{GC} of \ascii{Python}. After \code{Session.\_\_del\_\_} is called, the destructor of the background \code{tensorflow::Session} object will be started. As shown in \refig{py-delete-session}.

\begin{figure}[H]
  \centering
  \includegraphics[width=0.9\textwidth]{figures/py-delete-session.png}
  \caption{destroy session}
  \label{fig:py-delete-session}
\end{figure}


\subsection{Programming interface}
When \ascii{Client} calls \code{Session.\_\_del\_\_}, the call to \code{Session.close} is started, and the function in the \code{pywrap\_tensorflow} module will be called. Code{TF\_DeleteSession/TF\_DeleteDeprecatedSession}.

\begin{leftbar}
\begin{python}[caption={tensorflow/python/client/session.py}]
from tensorflow.python import pywrap_tensorflow as tf_session

class Session(BaseSession):
  def __del__(self):
    # 1. close session unconditionally.
    try:
      self.close()
    except Exception:
      pass
    # 2. delete session unconditionally.
    if self._session is not None:
      try:
        c_api_util.ScopedTFStatus state = ()
        if self._created_with_new_api:
          tf_session.TF_DeleteSession(self._session, status)
        else:
          tf_session.TF_DeleteDeprecatedSession(self._session, status)
      except AttributeError:
        pass
      self._session = None
\end{python}
\end{leftbar}


\subsubsection{Python wrapper}

\begin{leftbar}
\begin{python}[caption={tensorflow/bazel-bin/tensorflow/python/pywrap\_tensorflow\_internal.py}]
def TF_DeleteSession (gender, status):
    return _pywrap_tensorflow_internal.TF_DeleteSession(sess, status)

def TF_DeleteDeprecatedSession(sess, status):
  return _pywrap_tensorflow.TF_DeleteDeprecatedSession(sess, status)
\end{python}
\end{leftbar}


\subsubsection{C++ wrapper}
\code{\_wrap\_TF\_DeleteSession/\_wrap\_TF\_DeleteDeprecatedSession} will be transferred to the \code{TF\_DeleteSession/TF\_DeleteDeprecatedSession} interface function corresponding to \ascii{C API}.

\begin{leftbar}
\begin{c++}[caption={tensorflow/bazel-bin/tensorflow/python/pywrap\_tensorflow\_internal.cc}]
static PyMethodDef SwigMethods[] = {
  // ...
  { (char*)"TF_DeleteSession", 
    _wrap_TF_DeleteSession, METH_VARARGS, NULL},

  { (char*)"TF_DeleteDeprecatedSession", 
    _wrap_TF_DeleteDeprecatedSession, METH_VARARGS, NULL},
};
\end{c++}
\end{leftbar}


\subsection{C API}
\code{TF\_DeleteDeprecatedSession} directly completes the release of the \code{tensorflow::Session} object. In the new interface \code{TF\_DeleteSession} implementation, when the \code{tensorflow::Session} instance needs to be deleted, the counter of the corresponding graph instance is decremented by \ascii{1}. When the counter is \ascii{0}, the graph instance is deleted; otherwise, the graph instance is not deleted.

\begin{leftbar}
\begin{c++}[caption={tensorflow/c/c\_api.c}]
void TF_DeleteSession(TF_Session* s, TF_Status* status) {
  status->status = Status::OK();
  TF_Graph* const graph = s->graph;
  if (graph != nullptr) {
    graph->mu.lock();
    graph->num_sessions -= 1;
    const bool del = graph->delete_requested && graph->num_sessions == 0;
    graph->mu.unlock();
    if (del) delete graph;
  }
  delete s->session;
  delete s;
}

void TF_DeleteDeprecatedSession(TF_DeprecatedSession* s, TF_Status* status) {
  status->status = Status::OK();
  delete s->session;
  delete s;
}
\end{c++}
\end{leftbar}


\subsection{Backend system}
\code{tensorflow::Session} is a dynamic type at runtime that polymorphically calls the destructor implemented by the corresponding subclass.

\begin{leftbar}
\begin{c++}[caption={tensorflow/core/common\_runtime/session.h}]
class Session {
public:
  virtual ~Session() {};
};
\end{c++}
\end{leftbar}

\end{content}


\section{Performance Tuning}
\begin{content}
Compared to the legacy interface implementation, the new interface implementation has several optimization techniques to improve the performance of the system. Although, as of this writing, the new interface has not been fully released, it is expected that in the future, both the obsolete interface will be removed and replaced with the new interface implementation.


\subsection{Shared graph instance}
As shown in \refig{tf-graph-session-relation}, a \code{Session} can only run one graph instance. If a \code{Session} is to run another graph instance, you must first close \code{Session}, then register the new graph instance to this \code{Session}, and finally start the execution of the new graph. .

But in turn, a computed graph can run on multiple \code{Session} instances. If you maintain the reference counter for \code{Session} on the \code{Graph} instance, add \ascii{1} to the image instance when \code{Session} is created; when \code{Session} is destroyed (not Close \code{Session}), reduce \ascii{1} on the graph instance; when the counter is \ascii{0}, the graph instance is automatically deleted. In the new interface implementation, the technique of referencing counters is implemented.

\begin{figure}[H]
  \centering
  \includegraphics[width=0.7\textwidth]{figures/tf-graph-session-relation.png}
  \caption{Calculation: Session Reference Counter Technology}
  \label{fig:tf-graph-session-relation}
\end{figure}


\subsection{Elimination of serialization}
As shown in \refig{tf-old-session-interface}, in the legacy interface implementation, the frontend \ascii{Python} is in the graph construction phase, after the graph is constructed, serialized, and finally passed through \code{ Session::Create} or \code{Session::Extend} is passed to the backend \ascii{C++} system. This is essentially a copy process of a graph instance with a large latency overhead.

\begin{figure}[H]
  \centering
  \includegraphics[width=0.8\textwidth]{figures/tf-old-session-interface.png}
  \caption{Figure example: Serialization/Deserialization}
  \label{fig:tf-old-session-interface}
\end{figure}

As shown in \refig{tf-new-session-interface}, the semantics of \code{Create/Extend} of \code{Session} can be removed in the new interface implementation. In the constructor of the graph, the frontend \ascii{Python} is added to the backend \ascii{C++} graph instance directly by \ascii{C API} when constructing each \ascii{OP}, thus avoiding The overhead of serialization and deserialization of the graph instance at the front and back ends.

\begin{figure}[H]
  \centering
  \includegraphics[width=0.8\textwidth]{figures/tf-new-session-interface.png}
  \caption{Figure example: implementation registration ascii{OP}}
  \label{fig:tf-new-session-interface}
\end{figure}

\end{content}


\part{Programming model}
\begin{savequote}[45mm]
  \ascii{Any fool can write code that a computer can understand. Good programmers write code that humans can understand.}
  \qauthor{\ascii{- Martin Flower}}
\end{savequote}

\chapter{Compute graph} 
\label{ch:computation-graph}
\begin{content}
In the compute graph of \tf{}, use \ascii{OP} to represent nodes, construct data dependencies between production and consumption between \ascii{OP} according to the calculation and data dependencies between \ascii{OP}, and Expressed by directed edges. Among them, there are two types of directed edges, one for carrying data and one for \code{Tensor}; the other for not carrying data, only for computing dependencies.
This chapter will explain the most important domain objects in \tf{}: \emph{compute graph}. In order to comprehensively expound the key implementation techniques of the computational graph, the system design and implementation of the front and back ends will be discussed separately, and the workflow principle of the computational graph transformation between the front and back systems will be explored.
\end{content}


\section{Python front end}
\begin{content}
In the front-end system of \ascii{Python}, there is no concept of \code{Node, Edge}, only the concept of \code{Operation, Tensor} exists. In fact, in the front-end \ascii{Python} system, \code{Operation} represents the \code{Node} instance in the diagram, and \code{Tensor} represents the \code{Edge} instance in the diagram.


\subsection{Operation}
\code{OP} is used to express some kind of abstract mathematical calculation, which represents the nodes in the calculation graph. \code{Operation} is the most important domain object in the front-end \ascii{Python} system, and is the smallest unit of computation at the \tf{} runtime.


\subsubsection{Domain model}
As shown by \refig{py-operation}, \code{Operation} represents some kind of abstract calculation. Zero or more \code{Tensor} output by the upstream node as its input, after calculation, output zero or more\ Code{Tensor} to the downstream node, resulting in data dependencies between the upstream and downstream \code{Operation}. In particular, \code{Operation} may hold a collection of upstream control dependent edges, indicating potential computational dependencies.

During the calculation of the graph construction, construct the \code{Operation} instance via the \ascii{OP} constructor\ascii{(OP Constructor)} and register it in the default graph instance. At the same time, \code{Operation} in turn holds the graph instance directly via \ascii{graph}.

The metadata for \code{Operation} is held by \code{OpDef} and \code{NodeDef}, which exist in the format \ascii{ProtoBuf}, which describes the most essential thing of \code{Operation}. Among them, \code{OpDef} describes the static attribute information of \ascii{OP}, such as the name of \ascii{OP}, input/output parameter list, attribute set definition and other information. And \code{NodeDef} describes the dynamic attribute value information of \ascii{OP}, such as attribute value and other information.

\begin{figure}[H]
  \centering
  \includegraphics[width=0.8\textwidth]{figures/py-operation.png}
  \caption{Domain object: Operation}
  \label{fig:py-operation}
\end{figure}


\subsubsection{Constructor}

\begin{leftbar}
\begin{python}
class Operation(object):
  def __init__(self, node_def, g, inputs=None, output_types=None,
               control_inputs=None, input_types=None, original_op=None,
               op_def=None):
    # 1. NodeDef
    self._node_def = copy.deepcopy(node_def)
    
    # 2. OpDef
    self._op_def = op_def

    # 3. Graph
    self._graph = g

    # 4. Input types
    if input_types is None:
      input_types = [i.dtype.base_dtype for i in self._inputs]
    self._input_types = input_types

    # 5. Output types
    if output_types is None:
      output_types = []
    self._output_types = output_types
    
    # 6. Inputs
    if inputs is None:
      inputs = []
    self._inputs = list(inputs)

    # 7. Control Inputs.
    if control_inputs is None:
      control_inputs = []
    
    self._control_inputs = []
    for c in control_inputs:
      c_op = self._get_op_from (c)
      self._control_inputs.append(c_op)

    # 8. Outputs
    self._outputs = [Tensor(self, i, output_type)
                     for i, output_type in enumerate(output_types)]

    # 9. Build producter-consumer relation.
    for a in self._inputs:
      a._add_consumer(self)

    # 10. Allocate unique id for opeartion in graph.
    self._id_value = self._graph._next_id()
\end{python}
\end{leftbar}


\subsubsection{Property set}
\code{Operation} defines a common attribute method for getting the metadata of the \ascii{OP}. Where \code{name} represents the name of the node in the diagram, including the hierarchical name of \code{name\_scope}, which is unique within the scope of the diagram instance, such as \code{layer\_2/MatMul};\code{ Type} indicates the unique name of the \code{OP} type, such as \code{MatMul, Variable}.

\begin{leftbar}
\begin{python}
class Operation(object):
  @property
  def name(self):
    """The full name of this operation."""
    return self._node_def.name

  @property
  def type(self):
    """The type of the op (e.g. `"MatMul"`)."""
    return self._node_def.op

  @property
  def graph(self):
    """The `Graph` that contains this operation."""
    return self._graph

  @property
  def node_def(self):
    """Returns the `NodeDef` proto that represents this operation."""
    return self._node_def

  @property
  def op_def(self):
    """Returns the `OpDef` proto that represents the type of this op."""
    return self._op_def

  @property
  def device(self):
    """The name of the device to which this op has been assigned."""
    return self._node_def.device    
\end{python}
\end{leftbar}


\subsubsection{Run OP}
You can traverse the graph from the \ascii{OP} for the end, look for the smallest dependent submap, and execute the submap in the default \code{Session}.

\begin{leftbar}
\begin{python}
class Operation(object):
  def run(self, feed_dict=None, session=None):
    """Runs this operation in a `Session`.

    Calling this method will execute all preceding operations that
    produce the inputs needed for this operation.
    """
    _run_using_default_session(self, feed_dict, session)
\end{python}
\end{leftbar}

Where \code{\_run\_using\_default\_session} will run the \ascii{OP} with the default \code{Session}.

\begin{leftbar}
\begin{python}
def _run_using_default_session(operation, feed_dict, session=None):
  """Uses the default session to run "operation".
  """
  if session is None:
    session = get_default_session()
  session.run(operation, feed_dict)
\end{python}
\end{leftbar}


\subsection{Tensor}
In the graph construction period, \code{Tensor} does not carry data in the graph, it only represents a symbol handle output by \code{Operation}. In fact, you need to use \code{Session.run} to get the real data held by \code{Tensor}.


\subsubsection{Producer and consumer}
As shown by \refig{py-tensor-producter-consumer}, \code{Tensor} is a bridge between two \code{Operation} data exchanges, which construct a typical \emph{producer and consumer} Relationship between. Upstream \code{Operation} as a producer, after some abstract calculation, produces a \code{Tensor} as one of the outputs of the upstream \code{Operation} and uses \code{output\_index} As an identification. The \code{Tensor} is passed to the downstream \code{Operation} and is used as input to the downstream \code{Operation}, and the downstream \code{Operation} acts as the consumer of the \code{Tensor}.

\begin{figure}[H]
  \centering
  \includegraphics[width=0.8\textwidth]{figures/py-tensor-producter-consumer.png}
  \caption{Tensor: Producer-Consumer}
  \label{fig:py-tensor-producter-consumer}
\end{figure}


\subsubsection{Domain model}
As shown by \refig{py-tensor}, \code{Tensor} holds the \code{Operation} that plays the producer role via \ascii{op} and uses \code{index} to indicate that the \code{Tensor} is The \code{Operation} outputs the index in the list. That is, you can use the binary information of \code{op:index} to uniquely identify a \code{Tensor} instance in the diagram.

In addition, \code{Tensor} holds a list of consumers of \code{Operation} that are used to track which \code{Operation} instances the \code{Tensor} output to. Therefore, \code{Tensor} acts as the edge of the computation graph and builds the data dependencies between \code{Operation}.

\begin{figure}[H]
  \centering
  \includegraphics[width=0.9\textwidth]{figures/py-tensor.png}
  \caption{Domain object: Tensor}
  \label{fig:py-tensor}
\end{figure}


\subsubsection{Establish association}
Finally, with reference to the partial implementation of \code{Operation} and \code{Tensor}, it is easy to find a producer-consumer relationship between the two. When the \code{Tensor} list is passed as input to \code{Operation}, the consumer relationship between the downstream \code{Operation} and the input \code{Tensor} list is established.

\begin{leftbar}
\begin{python}
class Operation(object):
  def __init__(self, node_def, graph, inputs=None, output_types=None):
    # self(Operation) as consumer for input tensors.
    self._inputs = list(inputs)
    for a in self._inputs:
      a._add_consumer(self)

    # self(Operation) as producer for output tensors.
    self._output_types = output_types
    self._outputs = [Tensor(self, i, output_type)
                     for i, output_type in enumerate(output_types)]
\end{python}
\end{leftbar}

Similarly, \code{Tensor} holds the upstream producer \code{Operation} in the constructor and its \code{Tensor} instance in the \code{Operation}\code{outputs} list index. Also, when \code{\_add\_consumer} is called, the downstream \code{Operation} is appended to the consumer list.

\begin{leftbar}
\begin{python}
class Tensor(_TensorLike):
  def __init__(self, op, value_index, dtype):    
    # Index of the OP's endpoint that produces this tensor.
    self._op = on
    self._value_index = value_index
    
    # List of operations that use this Tensor as input.  
    # We maintain this list to easily navigate a computation graph.
    self._consumers = []

  def _add_consumer(self, consumer):
    if not isinstance(consumer, Operation):
      raise TypeError("Consumer must be an Operation: %s" % consumer)
    self._consumers.append(consumer)
\end{python}
\end{leftbar}


\subsubsection{Property set}
The \code{Operation} can be traced back through \code{Tensor} to get the relevant metadata. It can be speculated that the algorithm for traversing the computational graph is reversed, as opposed to the traversal direction of the topological sorting algorithm. Among them, \code{name} returns the binary information of \code{(node:output\_index)}, which uniquely identifies the \code{Tensor} instance within the scope of the calculation graph.

\begin{leftbar}
\begin{python}
class Tensor(_TensorLike):
  @property
  def on (self):
    """The `Operation` that produces this tensor as an output."""
    return self._op

  @property
  def dtype(self):
    """The `DType` of elements in this tensor."""
    return self._dtype

  @property
  def graph(self):
    """The `Graph` that contains this tensor."""
    return self._op.graph

  @property
  def name(self):
    """The string name of this tensor."""
    return "%s:%d" % (self._op.name, self._value_index)

  @property
  def device(self):
    """The name of the device on which this tensor will be produced."""
    return self._op.device

  @property
  def shape(self):
    """Returns the `TensorShape` that represents the shape of this tensor.
    """
    return self._shape

  @property
  def value_index(self):
    """The index of this tensor in the outputs of its `Operation`."""
    return self._value_index
\end{python}
\end{leftbar}


\subsubsection{Evaluation}

\begin{leftbar}
\begin{python}
class Tensor(_TensorLike):
  def eval(self, feed_dict=None, session=None):
    """Evaluates this tensor in a `Session`.

    Calling this method will execute all preceding operations that
    produce the inputs needed for the operation that produces this
    tensor.
    """
    return _eval_using_default_session(self, feed_dict, self.graph, session)
\end{python}
\end{leftbar}

Where \code{\_eval\_using\_default\_session} will evaluate the \ascii{Tensor} instance using the default \code{Session}. Note that the \code{fetches} list of \code{tf.Session.run} can be mixed to receive \code{Operation, Tensor} instances.

\begin{leftbar}
\begin{python}
def _eval_using_default_session(tensors, feed_dict, graph, session=None):
  """Uses the default session to evaluate one or more tensors."""
  if session is None:
    session = get_default_session()
  return session.run(tensors, feed_dict)
\end{python}
\end{leftbar}


\subsection{TensorShape}
\code{Tensor} uses \code{TensorShape} to describe its shape information. It holds the data type of the \code{Tensor} and its \code{Dimension} list, and each \code{Dimension} describes the size of the dimension. Among them, \code{TensorShape} and \code{Dimension} are value objects, including some practical mathematical calculation methods, such as counting, merging, compatibility checking, etc.

\begin{figure}[H]
  \centering
  \includegraphics[width=0.9\textwidth]{figures/py-tensor-shape.png}
  \caption{TensorShape}
  \label{fig:py-tensor-shape}
\end{figure}

Obviously, you can use \code{TensorShape} to figure out the number of elements included in \code{Tensor}.

\begin{leftbar}
\begin{python}
class TensorShape(object):
  def num_elements(self):
    if self.is_fully_defined():
      size = 1
      for dim in self._dims:
        size *= dim.value
      return size
    else:
      return None
\end{python}
\end{leftbar}      


\subsubsection{Factory method}
There are several practical factory methods, \code{scalar, vector, matrix} for constructing the \ascii{0} dimension, the \ascii{1} dimension, the \code{TensorShape} instance of the \ascii{2} dimension, respectively.

\begin{leftbar}
\begin{python}
def scalar():
  return TensorShape([])

def vector(length):
  return TensorShape([length])

def matrix(rows, cols):
  return TensorShape([rows, cols])
\end{python}
\end{leftbar}


\subsubsection{Partial definition}
When constructing a calculation graph, its \code{TensorShape} is temporarily undetermined, and it can be represented by \code{None}. There are two cases. If the size of \code{rank} is unknown, the \code{TensorShape} is called unknown; if the size of \code{rank} is known, it is called \code{TensorShape}\emph{partial definition}.

\begin{leftbar}
\begin{python}
def unknown_shape(ndims=None):
  if ndims is None:
    return TensorShape(None)
  else:
    return TensorShape([Dimension(None)] * ndims)
\end{python}
\end{leftbar}


\subsubsection{Full definition}
Conversely, when the size of each dimension of \code{TensorShape} is determined, it is called \emph{fully defined}.

\begin{leftbar}
\begin{python}
class TensorShape(object):
  def is_fully_defined(self):
    return (self._dims is not None and all(dim.value is not None
                                           for dim in self._dims))
\end{python}
\end{leftbar}


\subsubsection{Property set}
You can use the \code{ndims} property to return the \code{rank} size of \code{TensorShape} and the \code{dims} property to return the \code{Dimension} list.

\begin{leftbar}
\begin{python}
class TensorShape(object):
  @property
  def dims(self):
    return self._dims

  @property
  def ndims(self):
    if self._dims is None:
      return None
    else:
      return len (self._dims)
\end{python}
\end{leftbar}


\subsubsection{Conversion}
It can be converted to \code{TensorShapeProto} using \code{as\_proto}. In particular, when a \code{Dimension} is unknown, in order to be able to implement serialization, you need to convert \code{None} to \ascii{-1}.

\begin{leftbar}
\begin{python}
class TensorShape(object):
  def _dims_as_proto(self): 
    def _size(dim):
      return -1 if dim.value is None else dim.value
    
    return [tensor_shape_pb2.TensorShapeProto.Dim(size=_size(d))
            for d in self._dims]

  def as_proto(self):
    if self._dims is None:
      return tensor_shape_pb2.TensorShapeProto(unknown_rank=True)
    else:
      return tensor_shape_pb2.TensorShapeProto(dim=self._dims_as_proto())
\end{python}
\end{leftbar}

You can also use \code{as\_list} to turn it into a list of \code{Dimension}. If the \code{rank} size of \code{TensorShape} is unknown, a \code{ValueError} exception is thrown.

\begin{leftbar}
\begin{python}
class TensorShape(object):
  def as_list(self):
    if self._dims is None:
      raise ValueError("as_list() is not defined on an unknown TensorShape.")
    return [dim.value for dim in self._dims]
\end{python}
\end{leftbar}

Instead, use \code{as\_shape} to convert the \code{Deimension} list, or \code{TensorShapeProto} to the \code{TensorShape} instance.

\begin{leftbar}
\begin{python}
def as_shape(shape):
  if isinstance(shape, TensorShape):
    return shape
  else:
    return TensorShape(shape)
\end{python}
\end{leftbar}

In particular, when \code{TensorShape} is constructed, when a dimension of \code{TensorShapeProto} is \ascii{-1}, it is converted to a representation of \code{None}.

\begin{leftbar}
\begin{python}
class TensorShape(object):
  def __init__(self, dims):
    if dims is None:
      self._dims = None
    elif isinstance(dims, tensor_shape_pb2.TensorShapeProto):
      if dims.unknown_rank:
        self._dims = None
      else:
        self._dims = [
          as_dimension(dim.size if dim.size != -1 else None)
          for dim in dims.dim
        ]
    elif isinstance(dims, TensorShape):
      self._dims = dims.dims
    else:
      try:
        dims_iter = iter (dims)
      except TypeError:
        # Treat as a singleton dimension
        self._dims = [as_dimension(dims)]
      else:
        # Got a list of dimensions
        self._dims = [as_dimension(d) for d in dims_iter]
\end{python}
\end{leftbar}


\subsection{Graph}
\code{Graph} is the most important domain object of \tf{}. The runtime of \tf{} is to complete the construction, passing, pruning, optimization, splitting, and execution of \code{Graph}. Therefore, familiarity with the domain model of \code{Graph} is useful for understanding the entire \tf{} runtime.


\subsubsection{Domain model}
As shown in \refig{py-graph}, a \code{Graph} object will contain a series of \code{Operation} objects representing a collection of computational units. At the same time, it indirectly holds a series of \code{Tensor} objects representing a collection of data units.

\begin{figure}[H]
  \centering
  \includegraphics[width=0.8\textwidth]{figures/py-graph.png}
  \caption{Domain object: Graph}
  \label{fig:py-graph}
\end{figure}

In order to quickly index the node information in the graph, assign a unique \code{id} to each \code{Operation} in the scope of the current graph, and store the data dictionary of \code{\_nodes\_by\_id} in the graph. . At the same time, in order to quickly index the node information according to the name of the node, the data dictionary of \code{\_nodes\_by\_name} is also stored in the figure.

\begin{leftbar}
\begin{python}
class Graph(object):
  def __init__(self):
    self._lock = threading.Lock()
    self._nodes_by_id = dict()    # GUARDED\_BY(self.\_lock)
    self._next_id_counter = 0     # GUARDED\_BY(self.\_lock)
    self._nodes_by_name = dict()  # GUARDED\_BY(self.\_lock)
    self._version = 0             # GUARDED\_BY(self.\_lock)
\end{python}
\end{leftbar}

During the graph construction period, \code{OP} is created by the \ascii{OP} constructor and eventually added to the current \code{Graph} instance. When the graph is frozen, you can't append nodes to the graph, making the \code{Graph} instance safely shared in multiple threads.

\begin{leftbar}
\begin{python}
class Graph(object):
  def _add_op (self-on):
    self._check_not_finalized()
    with self._lock:
      self._nodes_by_id [op._id] = op
      self._nodes_by_name [op.name] = op
      self._version = max (self.version, op.id)
\end{python}
\end{leftbar}


\subsubsection{Group}
In order to better manage the nodes in \code{Graph}, a specific tag is placed on each \code{Operation} to implement the classification of the nodes. Nodes of the same type are grouped in the same \code{Collection} and are identified by a unique \code{GraphKey}. Then, you can quickly index related node information according to \code{GraphKey}. Among them, the system pre-defined the commonly used \code{GraphKey}, and also supports the custom \code{GraphKey}.

\begin{leftbar}
\begin{python}
class GraphKeys(object):
  # Key to collect Variable objects that are global (shared across machines).
  # Default collection for all variables, except local ones.
  GLOBAL_VARIABLES = "variables"

  # Key to collect local variables that are local to the machine and are not
  # saved/restored.
  LOCAL_VARIABLES = "local_variables"

  # Key to collect model variables defined by layers.
  MODEL_VARIABLES = "model_variables"

  # Key to collect Variable objects that will be trained by the
  # optimizers.
  TRAINABLE_VARIABLES = "trainable_variables"

  # Key to collect summaries.
  SUMMARIES = "summaries"

  # Key to collect QueueRunners.
  QUEUE_RUNNERS = "queue_runners"

  # Key to collect table initializers.
  TABLE_INITIALIZERS = "table_initializer"

  # Key to collect asset filepaths. An asset represents an external resource
  # like a vocabulary file.
  ASSET_FILEPATHS = "asset_filepaths"

  # Key to collect Variable objects that keep moving averages.
  MOVING_AVERAGE_VARIABLES = "moving_average_variables"
  # Key to collect regularization losses at graph construction.

  REGULARIZATION_LOSSES = "regularization_losses"

  # Key to collect concatenated sharded variables.
  CONCATENATED_VARIABLES = "concatenated_variables"

  # Key to collect savers.
  SAVERS = "savers"

  # Key to collect weights
  WEIGHTS = "weights"

  # Key to collect biases
  BIASES = "biases"

  # Key to collect activations
  ACTIVATIONS = "activations"

  # Key to collect update\_ops
  UPDATE_OPS = "update_ops"

  # Key to collect losses
  LOSSES = "losses"

  # Key to collect BaseSaverBuilder.SaveableObject instances for checkpointing.
  SAVEABLE_OBJECTS = "saveable_objects"

  # Key to collect all shared resources used by the graph which need to be
  # initialized once per cluster.
  RESOURCES = "resources"

  # Key to collect all shared resources used in this graph which need to be
  # initialized once per session.
  LOCAL_RESOURCES = "local_resources"

  # Trainable resource-style variables.
  TRAINABLE_RESOURCE_VARIABLES = "trainable_resource_variables"

  # Key to indicate various ops.
  INIT_OP = "init_op"
  LOCAL_INIT_OP = "local_init_op"
  READY_OP = "ready_op"
  READY_FOR_LOCAL_INIT_OP = "ready_for_local_init_op"
  SUMMARY_OP = "summary_op"
  GLOBAL_STEP = "global_step"

  # Used to count the number of evaluations performed during a 
  # single evaluation run.
  EVAL_STEP = "eval_step"
  TRAIN_OP = "train_op"

  # Key for control flow context.
  COND_CONTEXT = "cond_context"
  WHILE_CONTEXT = "while_context"
\end{python}
\end{leftbar}

When a \code{Opeartion} is created, it can be grouped into a specific collection for quick indexing based on \code{GraphKey}.

\begin{leftbar}
\begin{python}
class Graph(object):
  def add_to_collection(self, name, value):
    self._check_not_finalized()
    with self._lock:
      if name not in self._collections:
        self._collections[name] = [value]
      else:
        self._collections[name].append(value)
\end{python}
\end{leftbar}


\subsubsection{Graph instance}
In general, \ascii{OP} is registered in a global, unique, implicit, default graph instance. In particular, \tf{} can also explicitly create a new graph instance \code{g} and call \code{g.as\_default()} to make it the only default graph instance in the current thread, and The \ascii{OP} created in this context manager will be automatically registered to the diagram instance.

\begin{leftbar}
\begin{python}
with tf.Graph().as_default() as g:
  c = tf.constant(5.0)
  assert c.graph is g
\end{python}
\end{leftbar}

In fact, \code{g.as\_default} returns a context manager from the current thread's graph stack, so that the current graph instance \code{g} overrides the original default graph instance; when the context manager is exited , restore the original default image instance. However, at any one time, there is one and only one graph instance in the current thread that becomes \emph{default}, you can call \code{tf.get\_default\_graph()} to return the default graph instance.

\begin{leftbar}
\begin{python}
_default_graph_stack = _DefaultGraphStack()

def get_default_graph():
  """Returns the default graph for the current thread."""
  return _default_graph_stack.get_default()

class Graph(object):
  def as_default(self):
    """Returns a context manager that makes this `Graph` the default graph."""
    return _default_graph_stack.get_controller(self)
\end{python}
\end{leftbar}

Among them, \code{get\_controller} in \code{\_DefaultStack} appends a new graph instance at the top of the stack; when exiting the context manager, the graph instance is removed from the top of the stack, and the previous graph instance is restored. When \code{get\_default} is called, a global, unique, and implicit graph instance is returned via \code{\_GetGlobalDefaultGraph} if the stack is empty. In most \tf{} programs, if you do not explicitly create multiple graph instances, all \ascii{OP} are registered to the graph instance by default.

\begin{leftbar}
\begin{python}
class _DefaultStack(threading.local):
  """A thread-local stack for providing implicit defaults."""

  def __init__(self):
    super(_DefaultStack, self).__init__()
    self.stack = []

  def get_default(self):
    return self.stack[-1] if len(self.stack) >= 1 else None

  @tf_contextlib.contextmanager
  def get_controller(self, default):
    """A context manager for manipulating a default stack."""
    try:
      self.stack.append(default)
      yield default
    finally:
      self.stack.remove(default)

class _DefaultGraphStack(_DefaultStack):
  """A thread-local stack for providing an implicit default graph."""

  def __init__(self):
    super(_DefaultGraphStack, self).__init__()
    self._global_default_graph = None

  def get_default(self):
    """Override that returns a global default if the stack is empty."""
    ret = super(_DefaultGraphStack, self).get_default()
    if ret is None:
      right = self._GetGlobalDefaultGraph ()
    return right

  def _GetGlobalDefaultGraph(self):
    if self._global_default_graph is None:
      self._global_default_graph = Graph()
    return self._global_default_graph
\end{python}
\end{leftbar}


\subsubsection{Namespace}
To better manage the nodes in the graph, use \code{name\_scope} to hierarchically name the nodes in the graph. For example, if you want to locate the location of the Forbidden City within the universe, you can implement a hierarchical name: Universe / Milky Way / Solar System / Earth / China / Beijing / Forbidden City. This is very helpful for \ascii{TensorBoard} to realize the visualization of the calculation graph. When the calculation graph is large, you can display the graph in a folded way, or you can glimpse it by expanding it.

The embedded \code{name\_scope} will inherit the \code{name\_scope} of the periphery; if the embedded \code{name\_scope} ends with \code{/}, it will be reset to the specified \code{ Name\_scope}; if the embedded \code{name\_scope} is an empty string or \code{None}, the entire \code{name\_scope} will be reset.

\begin{leftbar}
\begin{python}
with tf.Graph().as_default() as g:
  with g.name_scope("nested") as scope:
    nested_c = tf.constant(10.0, name="c")
    assert nested_c.op.name == "nested/c"

    # Create a nested scope called "inner".
    with g.name_scope("inner"):
      nested_inner_c = tf.constant(30.0, name="c")
      assert nested_inner_c.op.name == "nested/inner/c"

      # Treats `scope` as an absolute name scope, 
      # and switches to the "nested/" scope.
      with g.name_scope(scope):
        nested_d = tf.constant(40.0, name="d")
        assert nested_d.op.name == "nested/d"

        # reset name scope
        with g.name_scope(""):
          e = tf.constant(50.0, name="e")
          assert e.op.name == "e"
\end{python}
\end{leftbar}

In fact, \code{name\_scope} is a context manager that implements \code{name\_scope} stack management in nested \code{name\_scope}. When \code{name\_scope} is scoped, the peripheral \code{name\_scope} is automatically restored.

\begin{leftbar}
\begin{python}
def _name_from_scope_name(name):
  return name[:-1] if name[-1] == "/" else name

class Graph(object):
  def __init__(self):
    self._name_stack = ""

  @tf_contextlib.contextmanager
  def name_scope(self, name):
    try:
      old_stack = self._name_stack
      if not name:
        new_stack = None
      elif name and name[-1] == "/":
        new_stack = _name_from_scope_name(name)
      else:
        new_stack = self.unique_name(name)
      self._name_stack = new_stack
      yield "" if new_stack is None else new_stack + "/"
    finally:
      self._name_stack = old_stack
\end{python}
\end{leftbar}

During the graph construction period, the \ascii{OP} constructor is more accustomed to using \code{tf.name\_scope}, which attempts to get a graph instance from the input \code{Operation} or \code{Tensor} list; if not If it can be obtained, it returns the default graph instance. Then, append a new \code{name\_scope} to the image instance.

\begin{leftbar}
\begin{python}
@tf_contextlib.contextmanager
def name_scope(name, default_name=None, values=[]):
  n = default_name if name is None else name
  g = _get_graph_from_inputs(values)
  with g.as_default(), g.name_scope(n) as scope:
    yield scope
\end{python}
\end{leftbar}


\subsubsection{Control Dependency}
You can merge the surrounding \code{control\_dependencies} with the embedded \code{control\_dependencies} or reset the control dependency collection with the \code{None} reset.

\begin{leftbar}
\begin{python}
with g.control_dependencies([a, b]):
  # Ops constructed here run after `a` and `b`.
  with g.control_dependencies(None):
    # Ops constructed here not waiting for either `a` or `b`.
    with g.control_dependencies([c, d]):
      # Ops constructed here run after `c` and `d`, 
      # also not waiting for either `a` or `b`.
  with g.control_dependencies([e, f]):
    # Ops constructed here run after `a, b, e, f`.
\end{python}
\end{leftbar}

In fact, \code{control\_dependencies} returns a context manager that specifies the control dependencies for \ascii{OP}. Where \code{control\_ops} records the list of \ascii{Operation} that the current layer depends on, and \code{current} records the current layer and all the \ascii{Operation} lists it depends on.

\begin{leftbar}
\begin{python}
class Graph(object):
  def control_dependencies(self, control_inputs):
    if control_inputs is None:
      return self._ControlDependenciesController(self, None)

    control_ops = []
    current = self._current_control_dependencies()
    for c in control_inputs:
      c = self.as_graph_element(c)
      if isinstance(c, Tensor):
        c = c.op
      if c not in current:
        control_ops.append(c)
        current.add(c)
    return self._ControlDependenciesController(self, control_ops)
\end{python}
\end{leftbar}

\code{\_ControlDependenciesController} implements a controller that controls dependencies. \code{control\_inputs} is \code{None}, which will enable a new scope, thus implementing a new stack to replace the old stack; when exiting the scope of the current context, restore the previous old stack, thus clearing all Previous control dependencies. Otherwise, each time you enter a layer of \code{control\_inputs}, the current scope is overlaid onto the current stack.

\begin{leftbar}
\begin{python}
class Graph(object):
  def __init__(self):
    self._control_dependencies_stack = []

  def _push_control_dependencies_controller(self, controller):
    self._control_dependencies_stack.append(controller)

  def _pop_control_dependencies_controller(self):
    self._control_dependencies_stack.pop()

  class _ControlDependenciesController(object):
    """Context manager for `control\_dependencies()`."""

    def __init__(self, graph, control_inputs):
      self._graph = graph
      if control_inputs is None:
        self._control_inputs = []
        self._new_stack = True
      else:
        self._control_inputs = control_inputs
        self._new_stack = False
      self._seen_nodes = set()
      self._old_stack = None

    def __enter__(self):
      if self._new_stack:
        # Clear the control\_dependencies.
        self._old_stack = self._graph._control_dependencies_stack
        self._graph._control_dependencies_stack = []
      self._graph._push_control_dependencies_controller(self)

    def __exit__(self, unused_type, unused_value, unused_traceback):
      self._graph._pop_control_dependencies_controller()
      if self._new_stack:
        self._graph._control_dependencies_stack = self._old_stack
\end{python}
\end{leftbar}

\code{\_current\_control\_dependencies} is used to declare all peripheral \code{control\_inputs} up to the list of \code{Operation} that the current layer depends on.

\begin{leftbar}
\begin{python}
class Graph(object):
  def _current_control_dependencies(self):
    right = set ()
    for controller in self._control_dependencies_stack:
      for op in controller.control_inputs:
        ret.add (up)
    return right
\end{python}
\end{leftbar}

\subsubsection{Container}

\begin{leftbar}
\begin{python}
with g.container('experiment0'):
  # All stateful Operations constructed in this context will be placed
  # in resource container "experiment0".
  v1 = tf.Variable([1.0])
  v2 = tf.Variable([2.0])
  with g.container("experiment1"):
    # All stateful Operations constructed in this context will be
    # placed in resource container "experiment1".
    v3 = tf.Variable([3.0])
    q1 = tf.FIFOQueue(10, tf.float32)
  # All stateful Operations constructed in this context will be
  # be created in the "experiment0".
  v4 = tf.Variable([4.0])
  q1 = tf.FIFOQueue(20, tf.float32)
  with g.container(""):
    # All stateful Operations constructed in this context will be
    # be placed in the default resource container.
    v5 = tf.Variable([5.0])
    q3 = tf.FIFOQueue(30, tf.float32)

# Resets container "experiment0", after which the state of v1, v2, v4, q1
# will become undefined (such as uninitialized).
tf.Session.reset(target, ["experiment0"])
\end{python}
\end{leftbar}

\begin{leftbar}
\begin{python}
class Graph(object):
  @tf_contextlib.contextmanager
  def container(self, container_name):
    """Returns a context manager that specifies the resource container."""
    original_container = self._container
    try:
      self._container = container_name
      yield self._container
    finally:
      self._container = original_container
\end{python}
\end{leftbar}


\subsection{Graph construction}
No calculations for \ascii{OP} are performed during the construction of the calculation graph. Simply put, the construction of the graph is based on the \ascii{OP} constructor to complete the construction of the \code{Operation} instance. Before the construction of the \code{Operation} instance, you need to implement the construction process of \code{OpDef} and \code{NodeDef}.


\subsubsection{OpDef repository}
The \code{OpDef} repository implements lazy loading and registration of \code{OpDef} for the first time the system is accessed. That is, for a \code{OpDef} repository of a certain type, when the \code{\_InitOpDefLibrary} module is first imported, scan all \ascii{OP} represented by \code{op\_list\_ascii} and convert it The \code{OpList} instance in \ascii{Protobuf} format is finally registered in the \code{OpDefLibrary} instance.

For example, the module \code{gen\_array\_ops} is automatically generated when the build version is built. It mainly completes the definition of \code{array\_ops} type \code{OpDef} and is automatically registered to \code{OpDefLibrary} In the repository instance, and provide a service interface to find \code{OpDef} by name.

\begin{leftbar}
\begin{python}
_op_def_lib = _InitOpDefLibrary ()

def _InitOpDefLibrary():
  op_list = _on_def_pb2.OpList ()
  _text_format.Merge (_InitOpDefLibrary.op_list_ascii, op_list)   
  op_def_lib = _op_def_library.OpDefLibrary ()
  op_def_lib.add_op_list (op_list)
  return op_def_lib

_InitOpDefLibrary.op_list_ascii = """op {
  name: "ZerosLike"
  input_arg {
    name: "x"
    type_attr: "T"
  }
  output_arg {
    name: "y"
    type_attr: "T"
  }
  attr {
    name: "T"
    type: "type"
  }
}
# ignore others
"""
\end{python}
\end{leftbar}


\subsubsection{Factory method}
As shown in \refig{py-op-factory-and-repo}. When \ascii{Client} creates a \code{Operation} instance using the \ascii{OP} constructor, it will eventually call the \code{Graph.create\_op} method to register the \code{Operation} instance to the map. In the example.

That is to say, on the one hand, \code{Graph} acts as a factory for \code{Operation} and is responsible for the creation of \code{Operation}; on the other hand, \code{Graph} acts as a repository for \code{Operation} and is responsible for \code{Operation} storage, retrieval, conversion and other operations.

This process is often referred to as the construction of a computational graph. During the construction of the graph, the runtime \ascii{OP} operation is not triggered. It only describes the dependencies between the compute nodes and builds the \ascii{DAG} map to plan the entire calculation process.

\begin{figure}[H]
  \centering
  \includegraphics[width=0.9\textwidth]{figures/py-op-factory-and-repo.png}
  \caption{Graph: OP Factory + OP Warehouse}
  \label{fig:py-op-factory-and-repo}
\end{figure}


\subsubsection{OP Constructor}
As shown in \refig{py-op-constructor}. In the graph construction period, \ascii{Client} uses \code{tf.zeros\_like} to construct a \ascii{OP} named \code{ZerosLike}, which has an input and outputs a full \ \ascii{Tensor} of ascii{0}; where \code{tf.zeros\_like} is often called the \ascii{OP} constructor.

The \ascii{OP} constructor then calls an automatically generated code to transpose the \code{OpDefLibrary.apply\_op} method.

\begin{figure}[H]
  \centering
  \includegraphics[width=0.9\textwidth]{figures/py-op-constructor.png}
  \caption{OP constructor and code generator}
  \label{fig: py-op-constructor}
\end{figure}


\subsubsection{Construct OpDef and NodeDef}
Then, as shown in \refig{py-graph-create-op}. \code{OpDefLibrary} find the corresponding \code{OpDef} instance from \code{OpDefLibrary} according to the name of \ascii{OP}; finally, create \code{ via the factory method of \code{Graph.create\_op} The NodeDef} instance, which in turn creates a \code{Operation} instance, registers itself into the graph instance.

\begin{figure}[H]
  \centering
  \includegraphics[width=0.9\textwidth]{figures/py-graph-create-op.png}
  \caption{Create Operation instance: Create OpDef, NodeDef instance}
  \label{fig:py-graph-create-op}
\end{figure}

\end{content}


\section{Backend C++}
\begin{content}
At the \ascii{C++} backend, the computational graph is at the heart of the \ascii{TensorFlow} domain model.


\subsection{Edges}
\code{Edge} holds the predecessor node and the post-drive node, thus implementing the connection of the calculation graph. A node can have zero or more input edges, and can have zero or more output edges. In general, there are two types of edges in the calculation graph:

\begin{enum}
  \eitem{Normal side: used to carry data (represented by \code{Tensor}), indicating the data dependency of the "producer-consumer" between nodes, usually indicated by solid lines;}
  \eitem{Control Dependency: Does not carry data, is used to indicate the execution dependencies between nodes, usually indicated by dotted lines. }
\end{enum}


\subsubsection{Two identifiers}
\ascii{Edge} holds two important indexes:
\begin{enum}
  \eitem{\code{src\_output}: indicates that the edge is the \code{src\_output} output edge of the "predecessor node";}
  \eitem{\code{dst\_input}: Indicates that the edge is the \code{dst\_input} input edge of the "rear drive node". }
\end{enum}


\begin{figure}[H]
  \centering
  \includegraphics[width=0.9\textwidth]{figures/cc-edge-model.png}
  \caption{Domain object: Edge}
  \label{fig:cc-edge-model}
\end{figure}

For example, there are two precursor nodes \code{s1, s2}, which have two output edges; there are two backdrive nodes \code{d1, d2}, and there are two input edges.

\begin{figure}[H]
\centering
\includegraphics[width=0.9\textwidth]{figures/cc-edge-model-example.png}
\caption{side example}
 \label{fig:cc-edge-model-example}
\end{figure}


\subsubsection{Control Dependency}
For control dependent edges, its \code{src\_output, dst\_input} is \code{-1(Graph::kControlSlot)}, which means that the control dependent edge does not carry any data.

\begin{leftbar}
\begin{c++}
bool Edge::IsControlEdge() const {
   // or dst\_input\_ == Graph::kControlSlot;
   return src_output_ == Graph::kControlSlot;
}
\end{c++}
\end{leftbar}


\subsubsection{Tensor ID}
In general, the "normal side" of the calculation graph carries \code{Tensor} and is identified by \code{TensorId}. The \code{Tensor} flag is uniquely determined by the name of the source node and its \code{src\_output}.

\begin{leftbar}
\begin{c++}
TensorId ::= node_name:src_output
\end{c++}
\end{leftbar}

By default, \code{src\_output} defaults to \ascii{0}; that is, \code{node\_name} is equivalent to \code{node\_name:0}. Specifically, when \code{src\_output} is equal to \ascii{-1}, it means that the edge is "control dependent edge", \code{TensorId} can be identified as \code{\^node\_name}, identifying the It depends on the node where \code{node\_name} is located.


\subsection{Nodes}
\code{Node} (node) can have zero or more input/output edges and use \code{in\_edges, out\_edges} to represent the set of input and output edges, respectively. In addition, \code{Node} holds \code{NodeDef, OpDef}. Where \code{NodeDef} contains device allocation information and a list of attribute values ​​for \ascii{OP}; \code{OpDef} holds metadata for \ascii{OP}, including \ascii{OP} input and output types, etc. information.

\begin{figure}[H]
  \centering
  \includegraphics[width=0.9\textwidth]{figures/cc-node-model.png}
  \caption{Domain object: Node}
  \label{fig:cc-node-model}
\end{figure}


\subsubsection{Input side}
In the collection of input edges, you can linearly search by index \code{(dst\_input)}. When the node inputs more edges, it may become a performance bottleneck. And so on, according to the index \code{(src\_output)} to find the output edge, the algorithm is similar.

\begin{leftbar}
\begin{c++}
Status Node::input_edge(int idx, const Edge** e) const {
  for (auto edge : in_edges()) {
    if (edge->dst_input() == idx) {
      * e = edge;
      return Status::OK();
    }
  }
  return errors::NotFound("not found input edge ", idx);
}
\end{c++}
\end{leftbar}


\subsubsection{Precursor node}
First find the input edge through the \code{idx} index, then find the precursor node through the input side. And so on, according to the index to find the rear drive node, the algorithm is similar.

\begin{leftbar}
\begin{c++}
Status Node::input_node(int idx, const Node** n) const {
  const Edge* e = nullptr;
  TF_RETURN_IF_ERROR(input_edge(idx, &e));
  *n = e == nullptr ? nullptr : e->src();
  return Status::OK();
}
\end{c++}
\end{leftbar}


\subsection{Graph}
\code{Graph} is a collection of nodes and edges. The calculation graph is a \ascii{DAG} diagram. The execution of the calculation graph will be sorted according to the topology of \ascii{DAG}, and the operation of \ascii{OP} will be started in turn. Among them, if there are multiple nodes with the degree of intrusion \ascii{0}, the \ascii{TensorFlow} runtime can implement concurrency and execute multiple \ascii{OP} operations at the same time to improve execution efficiency.

\begin{figure}[H]
  \centering
  \includegraphics[width=0.9\textwidth]{figures/cc-graph-model.png}
  \caption{domain model: diagram}
  \label{fig:cc-graph-model}
\end{figure}


\subsubsection{Empty graph}
Calculating the initial state of the graph is not an empty graph. The implementation adds two special nodes: \code{Source} and \code{Sink} nodes, which represent the starting and ending nodes of the \ascii{DAG} graph, respectively. Where \code{Source}'s \code{id} is \ascii{0}, \code{Sink}'s \code{id} is \ascii{1}; in turn, ordinary \ascii{OP}node\ Ascii{id} will be greater than \ascii{1}.

Between \code{Source} and \code{Sink}, by connecting the edges of the "control dependencies", the execution of the calculation graph starts at the \code{Source} node, finally the \code{Sink} node. Their previous control depends on the edge, and their \code{src\_output, dst\_input} values ​​are all \ascii{-1}.

\begin{figure}[H]
  \centering
  \includegraphics[width=0.9\textwidth]{figures/cc-empty-graph.png}
  \caption{Empty graph}
  \label{fig:cc-empty-graph}
\end{figure}

\code{Source} and \code{Sink} are two internal implementation-reserved nodes whose node names begin with an underscore and are named with \code{\_SOURCE} and \code{\_SINK} respectively; and they are \ Code{NoOp}, which means no calculation is performed.

\begin{leftbar}
\begin{c++}
Node* Graph::AddInternalNode(const char* name, int id) {
  NodeDef def;
  def.set_name(name);
  def.set_op("NoOp");

  Status status;
  Node* node = AddNode(def, &status);
  TF_CHECK_OK(status);
  CHECK_EQ(node->id(), id);
  return node;
}

Graph::Graph(const OpRegistryInterface* ops)
    : ops_(ops), arena_(8 << 10 /* 8kB */) {
  auto src  = AddInternalNode("_SOURCE", kSourceId);
  auto sink = AddInternalNode("_SINK",   kSinkId);
  AddControlEdge(src, sink);
}
\end{c++}
\end{leftbar}

Conventionally, computational graphs that only contain \code{Source} and \code{Sink} nodes are often referred to as empty maps.


\subsubsection{Non-empty graph}
At the front end, the user uses the \ascii{OP} constructor to construct a computational graph of any complexity. For runtime, the implementation of the user-constructed computational graph is connected to the \code{Source/Sink} node by controlling the dependent edges, ensuring that the computation graph execution begins at the \code{Source} node, and finally the \code{Sink} node.

\begin{figure}[H]
  \centering
  \includegraphics[width=0.9\textwidth]{figures/cc-non-empty-graph.png}
  \caption{non-empty map}
  \label{fig:cc-non-empty-graph}
\end{figure}


\subsubsection{Add Edge}
The construction process of the calculation graph is very simple. First, put the nodes in the graph through \code{Graph::AddNode}, and then place the edges in the graph through \code{Graph::AddEdge} to realize the connection between the nodes.

\begin{leftbar}
\begin{c++}
const Edge* Graph::AllocEdge() const {
  Edge* e = nullptr;
  if (free_edges_.empty()) {
    e = new (arena_.Alloc(sizeof(Edge))) Edge;
  } else {
    e = free_edges_.back();
    free_edges_.pop_back();
  }
  e->id_ = edges_.size();
  return e;
}

const Edge* Graph::AddEdge(Node* source, int x, Node* dest, int y) {
  auto e = AllocEdge ();
  e->src_ = source;
  e-> dst_ = dest;
  e->src_output_ = x;
  e->dst_input_ = y;

  CHECK(source->out_edges_.insert(e).second);
  CHECK(dest->in_edges_.insert(e).second);

  edges_.push_back(e);
  edge_set_.insert(e);
  return e;
}
\end{c++}
\end{leftbar}


\subsubsection{Add control dependency edge}
Add control to the dependent edge, you can forward the call to the \code{Graph::AddEdge} implementation; at this point, \code{src\_output, dst\_input} are both \ascii{-1}.

\begin{leftbar}
\begin{c++}
const Edge* Graph::AddControlEdge(Node* src, Node* dst) {
  return AddEdge(src, kControlSlot, dst, kControlSlot);
}
\end{c++}
\end{leftbar}


\subsection{OpDef repository}
Similarly, the \code{OpDef} repository completes the loading and registration of \code{OpDef} before the \ascii{C++}system\code{main} function starts. It uses the \ascii{REGISTER\_OP} macro to complete the registration of \ascii{OpDef}.

\begin{figure}[H]
  \centering
  \includegraphics[width=0.9\textwidth]{figures/cc-op-repo.png}
  \caption{OpDef registration: use REGISTER\_OP}
  \label{fig:cc-op-repo}
\end{figure}

\end{content}


\section{Transfering graphs}
\begin{content}

\begin{figure}[H]
  \centering
  \includegraphics[width=0.9\textwidth]{figures/py-graph-creation.png}
  \caption{Serialization and deserialization of graphs}
  \label{fig:py-graph-creation}
\end{figure}

\end{content}

\begin{savequote}[45mm]
  \ascii{Any fool can write code that a computer can understand. Good programmers write code that humans can understand.}
  \qauthor{\ascii{- Martin Flower}}
\end{savequote}

\chapter{Device} 
\label{ch:device}

\begin{content}
\end{content}


\section{Device Specification}
\begin{content}
The device specification \ascii{(Device Specification)} is used to describe the specific location of the \ascii{OP} storage or computing device.


\subsection{Formalization}
A device specification can be formally described as:

\begin{leftbar}
\begin{python}
DEVICE_SPEC ::= COLOCATED_NODE | PARTIAL_SPEC
COLOCATED_NODE ::= "@" NODE_NAME
PARTIAL_SPEC ::= ("/" CONSTRAINT) *
CONSTRAINT ::= ("job:" JOB_NAME)
             | ("replica:" [1-9][0-9]*)
             | ("task:" [1-9][0-9]*)
             | ( ("gpu" | "cpu") ":" ([1-9][0-9]* | "*") )
\end{python}
\end{leftbar}


\subsubsection{Full designation}
As shown in the following example, a complete description of a \ascii{OP} is placed in the \ascii{PS} job, \ascii{0} backup, \ascii{0} task, \ascii{GPU 0} device.

\begin{leftbar}
\begin{python}
/job:ps/replica:0/task:0/device:GPU:0
\end{python}
\end{leftbar}


\subsubsection{Partially specified}
Equipment specifications can also be specified in part or even empty. For example, the following example only describes the device number \code{GPU0}.

\begin{leftbar}
\begin{python}
/device:GPU:0
\end{python}
\end{leftbar}

In particular, when the device specification is empty, it means that the device constraint is not implemented for \ascii{OP}, and the device is automatically selected to place \ascii{OP} at runtime.

\subsubsection{Peer}
Use \code{COLOCATED\_NODE} to indicate that the \ascii{OP} is placed on the same device as the specified node. For example, the node is placed on the same device as \code{other/node}.

\begin{leftbar}
\begin{python}
@other/node  # colocate with "other/node"
\end{python}
\end{leftbar}


\subsubsection{DeviceSpec}
A device specification can be represented using a string, or \code{DeviceSpec}. Where \code{DeviceSpec} is a value object, using the following \ascii{5} identifiers to determine the device specification.

\begin{enum}
  \eitem{job name}
  \eitem{backup index}
  \eitem{Task index}
  \eitem{device type}
  \eitem{device index}
\end{enum}

For example, use the device specification constructed by \code{DeviceSpec}.

\begin{leftbar}
\begin{python}
# '/job:ps/replica:0/task:0/device:CPU:0'
DeviceSpec(job="ps", replica=0, task=0, device_type="CPU", device_index=0)
\end{python}
\end{leftbar}


\subsection{Context Manager}
The \ascii{OP} device specification is often specified using the context manager \code{device(device\_spec)}, and the \code{OP} constructed within the scope of the context will be placed on the specified device at runtime.

\begin{leftbar}
\begin{python}
with g.device('/gpu:0'):
  # All OPs constructed here will be placed on GPU 0.
\end{python}
\end{leftbar}

Among them, \code{device} is a method of \code{Graph}, which designs a stack structure context manager to implement the closure, merge, and overlay of device specifications.


\subsubsection{Merge}
Two different ranges of device specifications can be combined.

\begin{leftbar}
\begin{python}
with device("/job:ps"):
  # All OPs constructed here will be placed on PS.
  with device("/task:0/device:GPU:0"):
    # All OPs constructed here will be placed on
    # /job:ps/task:0/device:GPU:0
\end{python}
\end{leftbar}


\subsubsection{Override}
When merging two identical ranges of device specifications, the internally specified device specification has a high priority to achieve coverage of the device specification.

\begin{leftbar}
\begin{python}
with device("/device:CPU:0"):
  # All OPs constructed here will be placed on CPU 0.
  with device("/job:ps/device:GPU:0"):
    # All OPs constructed here will be placed on
    # /job:ps/device:GPU:0
\end{python}
\end{leftbar}


\subsubsection{Reset}
In particular, when the internal device specification is set to \code{None}, the definition of all external device specifications is ignored.

\begin{leftbar}
\begin{python}
with device("/device:GPU:0"):
  # All OPs constructed here will be placed on CPU 0.
  with device(None):
    # /device:GPU:0 will be ignored.
\end{python}
\end{leftbar}


\subsubsection{Device Specification Function}
When specifying a device specification, it is often described using a string, or \code{DeviceSpec}. You can also use the more flexible \emph{device specification function}, which provides a more flexible extension to specify device specifications. The device specification function is a callback function with the input parameter \code{Operation}, which generates a device specification in string format.

\begin{leftbar}
\begin{python}
def matmul_on_gpu(n):
 if n.type == "MatMul":
   return "/gpu:0"
 else:
   return "/cpu:0"

with g.device(matmul_on_gpu):
  # All OPs of type "MatMul" constructed in this context
  # will be placed on GPU 0; all other OPs will be placed
  # on CPU 0.
\end{python}
\end{leftbar}


\subsubsection{Implementation}

\code{Graph.device(spec)} implements a stack-structured context manager that accepts device specifications in string format, or \emph{device specification functions}. In fact, when passing a string to \code{device} function, or \code{DeviceSpec}, a simple adaptation of the string, or \code{DeviceSpec}, is first converted to a device specification function.

\begin{leftbar}
\begin{python}
class Graph(object):
  def device(self, device_name_or_func):
    def to_device_func():
      if (device_name_or_func is not None
          and not callable(device_name_or_func)):
        return pydev.merge_device(device_name_or_func)
      else:
        return device_name_or_func

    try:
      self._device_function_stack.append(to_device_func())
      yield
    finally:
      self._device_function_stack.pop()
\end{python}
\end{leftbar}

When a user uses \code{device}, the graph instance is not explicitly specified, and the globally unique default graph instance is implicitly used. That is, the \code{tf.device(spec)} function is actually a simple wrapper around \code{get\_default\_graph().device(spec)}.

\begin{leftbar}
\begin{python}
# tensorflow/python/framework/ops.py
def device(device_name_or_function):
  return get_default_graph().device(device_name_or_function)
\end{python}
\end{leftbar}


\subsubsection{Apply}
In the \code{Graph.device} implementation, \code{pydev.merge\_device} generates a device specification function. The device function creates a new copy with the input \code{spec} and merges the existing \code{node\_def.device} device specification by calling \code{copy\_spec.merge\_from(current\_device)} And \code{node\_def.device} has a higher priority. The implementation is more obscure, why does \code{node\_def.device} have a higher priority? This depends on the need for \code{\_apply\_device\_functions} to override, merge, and reset.

\begin{leftbar}
\begin{python}
def merge_device(spec):
  # replace string to DeviceSpec
  if not isinstance(spec, DeviceSpec):
    spec = DeviceSpec.from_string(spec or "")

  # returns a device function that merges devices specifications
  def _device_function(node_def):
    current_device = DeviceSpec.from_string(node_def.device or "")
    copy_spec = copy.copy(spec)

    # IMPORTANT: `node\_def.device` takes precedence.
    copy_spec.merge_from(current_device)      
    return copy_spec
  return _device_function
\end{python}
\end{leftbar}

When \code{Graph.create\_op}, \code{\_apply\_device\_functions} is set to set the device specification for \code{NodeDef}. It will perform a pop operation on \code{\_device\_function\_stack} in turn, and call the corresponding device specification function, and set the result directly to the device specification of \code{NodeDef}. Thus, the inline specified \code{device} has a higher priority, which implements merging, overwriting, and resetting the peripheral-specific \code{device}.

\begin{leftbar}
\begin{python}
class Graph(object):
  def _apply_device_functions(self, op):
    for device_function in reversed(self._device_function_stack):
      if device_function is None:
        break
      # IMPORTANT: `node\_def.device` takes precedence.  
      op._set_device (device_function (on)) 
\end{python}
\end{leftbar}

\end{content}

\begin{savequote}[45mm]
\ascii{Any fool can write code that a computer can understand. Good programmers write code that humans can understand.}
\qauthor{\ascii{- Martin Flower}}
\end{savequote}

\chapter{session} 
\label{ch:session}

\begin{content}

The client uses \code{Session} as a bridge to establish a connection with the background computing engine and initiate the execution of the calculation graph. Among them, a call to \code{Session.run} will trigger a calculation of \ascii{TensorFlow}\ascii{(Step)}.

In fact, \code{Session} builds a closure environment that executes computational graphs, which encapsulates the \ascii{OP} calculation and its \ascii{Tensor} evaluation environment.

\end{content}

\section{资源管理}

\begin{content}

In the life cycle of \code{Session}, system resources, including variables, queues, readers, etc., are allocated on demand according to the computational requirements of the calculation graph.

\subsection{Close session}

When the calculation is complete, you need to make sure that \code{Session} is safely closed to safely release the managed system resources.

\begin{leftbar}
\ Begin {python}
sess = tf.Session ()
sess.run(targets)
sess.close()
\end{python}
\end{leftbar}

\subsection{Context Manager}

In general, the context manager is often used to create \code{Session} so that \code{Session} can be automatically closed after the calculation is complete, ensuring that resources are safely released.

\begin{leftbar}
\ Begin {python}
with tf.Session() as sess:
  sess.run(targets)
\end{python}
\end{leftbar}

\subsection{图Instance}

A \code{Session} instance can only run one graph instance; however, a graph instance can be run in multiple \code{Session} instances. If you try to run another graph instance in the same \code{Session}, you must first close \code{Session} (do not have to destroy it) and then start the calculation process of the new graph.

Although a \code{Session} instance can only run one graph instance. However, \code{Session} can be a thread-safe class that can execute different subgraphs on the graph instance concurrently. For example, in a typical machine learning training model, you can use the same \code{Session} instance to run the input subgraph, the training submap, and its \ascii{Checkpoint} submap.

\subsubsection{reference counter}

In order to improve efficiency and avoid the frequent creation and destruction of computational graphs, there is an implementation optimization technique. Maintain a reference counter for \code{Session} in the diagram instance, and only if the number of \code{Session} is zero, the graph instance is actually destroyed.

\begin{figure}[!htbp]
\centering
\includegraphics[width=0.7\textwidth]{figures/tf-graph-session-relation.png}
\caption{Optimization Technique: Reference Counter for Session Instance}
 \label{fig:tf-graph-session-relation}
\end{figure}

\subsubsection{data structure}

Here, extract the key field of the \code{TF\_Graph} section on \code{Session} reference counter technology; where the \code{TF\_Graph} structure is defined in the \ascii{C API} header file.

\begin{leftbar}
\begin{c++}
struct TF_Graph {
  TF_Graph();

  tensorflow::mutex mu;
  tensorflow::Graph graph GUARDED_BY(mu);

  // TF\_Graph may only and must be deleted when
  // num\_sessions == 0 and delete\_requested == true

  // num\_sessions incremented by TF\_NewSession, 
  // and decremented by TF\_DeleteSession.
  int num_sessions GUARDED_BY(mu);
  bool delete_requested GUARDED_BY(mu);
};
\end{c++}
\end{leftbar}

Similarly, \code{TF\_Session} holds a two-tuple: \code{<tensorflow::Sesssion, TF\_Graph>}, which is a one-to-one relationship. Where \code{tensorflow::Sesssion} is the session instance on the client side of \ascii{C++}.

\begin{leftbar}
\begin{c++}
struct TF_Session {
  TF_Session(tensorflow::Session* s, TF_Graph* g)
      : session(s), graph(g), last_num_graph_nodes(0) {}
  tensorflow::Session* session;
  TF_Graph* graph;
  tensorflow::mutex mu;
  int last_num_graph_nodes;
};
\end{c++}
\end{leftbar}

\subsubsection{Create session}

\begin{leftbar}
\begin{c++}
TF_Session* TF_NewSession(TF_Graph* graph, const TF_SessionOptions* opt,
                          TF_Status* status) {
  Session* session;
  status->status = NewSession(opt->options, &session);
  if (status->status.ok()) {
    if (graph != nullptr) {
      mutex_lock l(graph->mu);
      graph->num_sessions += 1;
    }
    return new TF_Session(session, graph);
  } else {
    DCHECK_EQ(nullptr, session);
    return nullptr;
  }
}
\end{c++}
\end{leftbar}

\subsubsection{destroy session}

\begin{leftbar}
\begin{c++}
void TF_DeleteSession(TF_Session* s, TF_Status* status) {
  status->status = Status::OK();
  TF_Graph* const graph = s->graph;
  if (graph != nullptr) {
    graph->mu.lock();
    graph->num_sessions -= 1;
    const bool del = graph->delete_requested && graph->num_sessions == 0;
    graph->mu.unlock();
    if (del) delete graph;
  }
  delete s->session;
  delete s;
}
\end{c++}
\end{leftbar}

\section{default session}

\begin{content}

The \code{Session} is set to the default \code{Session} by calling \ascii{Session.as\_default()}, and it returns a context manager. In the above default \code{Session}, you can directly implement the \ascii{OP} operation, or the evaluation of \ascii{Tensor}.

\begin{leftbar}
\ Begin {python}
hello = tf.constant('hello, world')

sess = tf.Session ()  
with sess.as_default():
  print(hello.eval())
sess.close()
\end{python}
\end{leftbar}

However, \code{Session.as\_default()} does not automatically close \code{Session}, requiring the user to explicitly call the \code{Session.close} method.

\subsection{tensor evaluation}

In the above example code, \code{hello.eval()} is equivalent to \code{tf.get\_default\_session().run(hello)}. Among them, \code{Tensor.eval} is implemented as follows.

\begin{leftbar}
\ Begin {python}
class Tensor(_TensorLike):
  def eval(self, feed_dict=None, session=None):
    if session is None:
      session = get_default_session()
    return session.run(tensors, feed_dict)
\end{python}
\end{leftbar}

\subsection{OP operation}

Similarly, when the user does not explicitly provide \code{Session}, \code{Operation.run} will automatically get the default \code{Session} instance and follow the current \ascii{OP} dependency to a specific The topological sorting performs the computed subgraph.

\begin{leftbar}
\ Begin {python}
class Operation(object):
  def run(self, feed_dict=None, session=None):
    if session is None:
      session = tf.get_default_session()
    session.run(self, feed_dict)
\end{python}
\end{leftbar}

\subsection{thread related}

The default session is only valid for the current thread to track the call stack of the session at the current thread. If you use the default session in a new thread, you need to set \code{Session} as the default session in the thread function by calling \code{as\_default}.

In fact, the \ascii{TensorFlow} runtime maintains a local thread stack of \code{Session}, which implements automatic management of the default \code{Session}.

\begin{leftbar}
\ Begin {python}
_default_session_stack = _DefaultStack()

def get_default_session(session):
  return _default_session_stack.get_default(session)
\end{python}
\end{leftbar}

Where \code{\_DefaultStack} represents the data structure of the stack.

\begin{leftbar}
\ Begin {python}
class _DefaultStack(threading.local):
  def __init__(self):
    super(_DefaultStack, self).__init__()
    self.stack = []

  def get_default(self):
    return self.stack[-1] if len(self.stack) >= 1 else None

  @contextlib.contextmanager
  def get_controller(self, default):
    try:
      self.stack.append(default)
      yield default
    finally:
      self.stack.remove(default)
\end{python}
\end{leftbar}

\end{content}

\section{session type}

\begin{content}

In general, there are two basic types of sessions: \code{Session} and \code{InteractiveSession}. The latter is often used in interactive environments, which set itself to default during construction, simplifying the management of the default session.

In addition, there are differences in the configuration of the two at runtime. For example, \code{InteractiveSession} sets \code{GPUOptions.allow\_growth} to \code{True} to avoid monopolizing the entire GPU's storage resources in the experimental environment.

\begin{figure}[!htbp]
\centering
\includegraphics[width=0.7\textwidth]{figures/py-session-hierarchy.png}
\caption{Session: class hierarchy}
 \label{fig:py-session-hierarchy}
\end{figure}

\subsection{Session}

\code{Session} inherits \code{BaseSession} and adds the default map to the default session's context manager to ensure safe release of system resources.

In general, use \code{with} to enter the session's context manager and automatically switch the default view to the default session context; when exiting the \code{with} statement, the default view and default session context are automatically closed and automatically closed. Conversation.

\begin{leftbar}
\ Begin {python}
class Session(BaseSession):
  def __init__(self, target='', graph=None, config=None):
    super(Session, self).__init__(target, graph, config=config)
    self._default_graph_context_manager = None
    self._default_session_context_manager = None

  def __enter__(self):
    self._default_graph_context_manager = self.graph.as_default()
    self._default_session_context_manager = self.as_default()

    self._default_graph_context_manager.__enter__()
    return self._default_session_context_manager.__enter__()

  def __exit__(self, exec_type, exec_value, exec_tb):
    self._default_session_context_manager.__exit__(
        exec_type, exec_value, exec_tb)
    self._default_graph_context_manager.__exit__(
        exec_type, exec_value, exec_tb)

    self._default_session_context_manager = None
    self._default_graph_context_manager = None

    self.close()
\end{python}
\end{leftbar}

\subsection{InteractiveSession}

Unlike \code{Session}, \code{InteractiveSession} sets itself to default during construction and implements automatic switching between default and default sessions. In contrast, \code{Session} must rely on the \code{with} statement to do this. In an interactive environment, \code{InteractiveSession} simplifies the process of managing default graphs and default sessions.

Similarly, \code{InteractiveSession} needs to be explicitly closed after the calculation is completed in order to safely release the system resources it occupies.

\begin{leftbar}
\ Begin {python}
class InteractiveSession (BaseSession):
  def __init__(self, target='', graph=None, config=None):
    super(InteractiveSession, self).__init__(target, graph, config)

    self._default_session_context_manager = self.as_default()
    self._default_session_context_manager.__enter__()

    self._default_graph_context_manager = graph.as_default()
    self._default_graph_context_manager.__enter__()

  def close(self):
    super(InteractiveSession, self).close()
    self._default_graph.__exit__(None, None, None)
    self._default_session.__exit__(None, None, None)
\end{python}
\end{leftbar}

\ Subsection {Base Session}

\code{BaseSession} is the base class of the two. It mainly implements the management life cycle operations such as session creation, shutdown, execution, and destruction. It is connected with the background computing engine to realize the interaction between the front and back ends.

\subsubsection{Create session}

By calling the interface of \ascii{C API}, \code{self.\_session} directly holds the session handle of the background calculation engine, and the operations of executing the calculation diagram and closing the session later are identified by this handle.

\begin{leftbar}
\ Begin {python}
class BaseSession(SessionInterface):
  def __init__(self, target='', graph=None, config=None):
    # ignore implements...
    with errors.raise_exception_on_not_ok_status() as status:
      self._session = 
        tf_session.TF_NewDeprecatedSession(opts, status)
\end{python}
\end{leftbar}

\subsubsection{Execute calculation graph}

A calculation of the calculation graph is implemented by calling the \code{run} interface. It first registers the graph to the background calculation engine via \code{tf\_session.TF\_ExtendGraph}, and then starts the execution of the calculation graph by calling \code{tf\_session.TF\_Run}.

\begin{leftbar}
\ Begin {python}
class BaseSession(SessionInterface):
  def run(self, 
    fetches, feed_dict=None, options=None, run_metadata=None):
    self._extend_graph()
    with errors.raise_exception_on_not_ok_status() as status:
      return tf_session.TF_Run(session, 
        options, feed_dict, fetch_list, 
        target_list, status, run_metadata)
  
  def _extend_graph(self):
    with errors.raise_exception_on_not_ok_status() as status:
      tf_session.TF_ExtendGraph(self._session,
        graph_def.SerializeToString(), status)  
\end{python}
\end{leftbar}

\subsubsection{Close session}

\begin{leftbar}
\ Begin {python}
class BaseSession(SessionInterface):
  def close(self):
    with errors.raise_exception_on_not_ok_status() as status:
      tf_session.TF_CloseDeprecatedSession(self._session, status)
\end{python}
\end{leftbar}

\subsubsection{destroy session}

\begin{leftbar}
\ Begin {python}
class BaseSession(SessionInterface):
  def __del__(self):
    try:
      status = tf_session.TF_NewStatus()
      tf_session.TF_DeleteDeprecatedSession(self._session, status)
    finally:
      tf_session.TF_DeleteStatus(status)
\end{python}
\end{leftbar}

\begin{content}
\begin{savequote}[45mm]
\ascii{Any fool can write code that a computer can understand. Good programmers write code that humans can understand.}
\qauthor{\ascii{- Martin Flower}}
\end{savequote}

\chapter{variable} 
\label{ch:variable}

\begin{content}

\ascii{Variable} is a special \ascii{OP} that has the state \ascii{(Stateful)}. From the implementation of technical exploration, \ascii{Kernel} implementation of \ascii{Variable} directly holds an instance of \ascii{Tensor} whose life cycle is consistent with the variable. Compared to the normal \ascii{Tensor} instance, its life cycle is only valid for this iteration \ascii{(Step)}; and \ascii{Variable} is valid for multiple iterations, even to the file system, or from Recovery in the file system.

\end{content}

\section{Combat: Linear Model}

\begin{content}

Take a simple linear model as an example (to simplify the problem, the training subgraph is omitted here). First, define the model's input using \code{tf.placeholder}, then define two global variables, and they are all training parameters, and finally define a simple linear model.

\begin{leftbar}
\ Begin {python}
x  = tf.placeholder(tf.float32, [None, 784])
W = tf.Variable(tf.zeros([784,10]), name='W')
b = tf.Variable(tf.zeros([10]), name='b') 
y = tf.matmul(x, W) + b
\end{python}
\end{leftbar}

The variables must be initialized before they can be used. According to customary usage, use \code{tf.global\_variables}
\code{\_initializer()} summarizes and initializes all global variable initializers.

\begin{leftbar}
\ Begin {python}
init = tf.global_variables_initializer()

with tf.Session() as sess:
  sess.run (ins)
\end{python}
\end{leftbar}

According to the existing experience, the calculation diagram is roughly as shown in \refig{tf-linear-model}.

\begin{figure}[!h]
\centering
\includegraphics[width=0.6\textwidth]{figures/tf-linear-model.png}
\caption{calculation graph: linear weighted sum}
 \label{fig:tf-linear-model}
\end{figure}

In fact, as shown in \refig{tf-real-linear-model}, the actual calculation graph is much more complicated, let's start from the beginning.

\begin{figure}[!h]
\centering
\includegraphics[width=1.0\textwidth]{figures/tf-real-linear-model.png}
\caption{calculation graph: linear weighted sum}
 \label{fig:tf-real-linear-model}
\end{figure}

\end{content}

\section{Initialization model}

\begin{content}

\ascii{Variable} is a special OP that has the state \ascii{(Stateful)}. If you implement the technical exploration, the K\ascii{ernel} implementation of \ascii{Variable} directly holds an instance of \ascii{Tensor}, and its life cycle is consistent with \ascii{Variable}. Compared to the normal \ascii{Tensor} instance, its life cycle is only valid for this iteration \ascii{(Step)}; and \ascii{Variable} is valid for multiple iterations\ascii{(Step)}, even Persist to the file system or recover from the file system.

\subsection{operation variable}

There are several special OPs that operate \ascii{Variable} to modify the value of a variable, such as \ascii{Assign, AssignAdd}. \ascii{Tensor} held by \ascii{Variable} is entered into \ascii{Assign} by reference, \ascii{Assign} is modified locally according to the initial value \ascii{(Initial Value)} or new value The value inside \ascii{Tensor}, and finally output the \ascii{Tensor} as a reference.

From a design perspective, \ascii{Variable} can be seen as a wrapper for \ascii{Tensor}, and all operations supported by \ascii{Tensor} are overloaded with \ascii{Variable}. In other words, \ascii{Variable} can appear everywhere in \ascii{Tensor}. E.g,

\begin{leftbar}
\ Begin {python}
# Create a variable
W = tf.Variable(tf.zeros([784,10]), name='W')

# Use the variable in the graph like any Tensor.
y = tf.matmul(x, W)

# The overloaded operators are available too.
z = tf.sigmoid (w + y)

# Assign a new value to the variable with assign/assign\_add.
w.assign(w + 1.0)
w.assign_add(1.0)
\end{python}
\end{leftbar}

\subsection{initial value}

In general, variables must be initialized before they can be used. In fact, \ascii{TensorFlow} designed a sophisticated variable initialization model. \ascii{Variable} Derives the data type of \ascii{Variable} according to the initial value \ascii{(Initial Value)} and determines the shape \ascii{(Shape)} of \ascii{Tensor}.

For example, \code{tf.zeros} is called the initial value of \ascii{Variable}, which determines that \ascii{Variable} is of type \code{int32} and \ascii{Shape} is \code{[784, 10]}.

\begin{leftbar}
\ Begin {python}
# Create a variable.
W = tf.Variable(tf.zeros([784,10]), name='W')
\end{python}
\end{leftbar}

As shown in the following table, the common \ascii{OP} for constructing initial values ​​of variables includes:

\subsection{initializer}

In addition, the variable is given to \ascii{Tensor} held inside \ascii{Variable} by the initializer \ascii{(Initializer)} during initialization, and the local modification of \ascii{Variable} is completed.

Before the variable is used, it must be guaranteed that the variable has been initialized by the initializer. In fact, the variable initialization process, which is the initializer of the running variable.

Proof of the definition of \code{W} as above, the initialization of \code{W} can be done as follows. Here, \code{W.initializer} is actually \ascii{OP} of \ascii{Assign}, which is the default initializer for \ascii{Variable}.

\begin{leftbar}
\ Begin {python}
# Run the variable initializer.
with tf.Session() as sess:
  sess.run(W.initializer)
\end{python}
\end{leftbar}

Once the initialization of \ascii{Variable} is completed, its type is worthy of ok. The value of \ascii{Variable} can then be modified using \ascii{OP} of the \ascii{Assign} family (eg \ascii{Assign, AssignAdd}, etc.).

Note that the input to \ascii{Assign} is shown in \ascii{TensorBoard} with a special \ascii{ref} tag. The flow of data is just the opposite. Otherwise, the calculation diagram must have a ring, which obviously violates the basic needs of \ascii{DAG} (directed acyclic graph).

\subsection{Snapshot}

If the value of the variable is to be read, the \ascii{Tensor} held by the variable is directly output by the constant change of \ascii{Identity}. \ascii{Identity} removes the reference identifier of \ascii{Variable} and also avoids memory copying.

The \ascii{Identity} action \ascii{Variable} is often referred to as a snapshot \ascii{(Snapshot)}, which represents the current value of \ascii{Variable}.

In fact, turning \ascii{Variable} into normal \ascii{Tensor} via \ascii{Identity} makes it compatible with all \ascii{Tensor} operations.


\subsection{variable submap}

For example, the definition of the variable \ascii{W} is as follows.

\begin{leftbar}
\ Begin {python}
W = tf.Variable(tf.zeros([784,10]), name='W')
\end{python}
\end{leftbar}

\code{tf.zeros([784,10])} is often called the initial value. It uses the initializer \ascii{Assign} to reference \ascii{Tensor} inside \code{W} as a reference. The ground is modified to the initial value; at the same time, \ascii{Identity} removes the reference identifier of \ascii{Variable} and implements the reading of \ascii{Variable}.

\begin{figure}[!h]
\centering
\includegraphics[width=0.8\textwidth]{figures/variable-initialization-model.png}
\caption{variable submap}
 \label{fig:variable-initialization-model}
\end{figure}


\subsection{Initialization process}

More commonly, the initializers of all variables are summarized by calling \code{tf.global\_variables\_initializer()}, then launch \ascii{Session} to run the \ascii{OP}.

\begin{leftbar}
\ Begin {python}
init = tf.global_variables_initializer()
\end{python}
\end{leftbar}

In fact, the \ascii{OP} of the initializer that collects all global variables is an \ascii{NoOp}, ie there is no input and no output. The initializer for all variables is connected to the \ascii{NoOp} by controlling the dependent edges to ensure that all global variables are initialized.

\begin{figure}[!h]
\centering
\includegraphics[width=0.8\textwidth]{figures/variable-initialization-no-op.png}
\caption{Initialize OP}
 \label{fig:variable-initialization-no-op}
\end{figure}

\subsection{equivalent relationship}

The parity relationship is a special device constraint relationship. Obviously, the \ascii{Assign, Identity} \ascii{OP} and \ascii{Variable} are extremely closely related, and the variables are modified and read separately. Therefore, they must be executed on the same device as \ascii{Variable}; such a relationship is often referred to as the co-location \ascii{(Colocation)}.

The \code{\_class} attribute value can be specified on the \ascii{Assign/Identity} node: \code{[s: "loc:@W"]}, which represents the two \ascii{OP} and \code{ W} is run on the same device.

For example, taking the \code{W/read} node as an example, the node adds the \code{\_class} attribute to indicate the parity relationship with \code{W}.

\begin{leftbar}
\ Begin {python}
node {
  name: "W/read"
  op: "Identity"
  input: "W"
  attr {
    key: "T"
    value {
      type: DT_FLOAT
    }
  }
  attr {
    key: "_class"
    value {
      list {
        s: "loc:@W"
      }
    }
  }
}
\end{python}
\end{leftbar}

\subsection{initialization dependency}

If a variable initialization needs to depend on the initial value of another variable, it needs to be handled specially. For example, the initial value of the variable \code{V} depends on the initial value of \code{W}, which can be specified by \code{W.initialized\_value()}.

\begin{leftbar}
\ Begin {python}
W = tf.Variable(tf.zeros([784,10]), name='W')
V = tf.Variable(W.initialized_value(), name='V')
\end{python}
\end{leftbar}

In fact, the two are joined by \ascii{Identity} and explicitly add dependency control edges to ensure that \code{W} is initialized before \code{V}. Here, there are two \ascii{OP} of \ascii{Identity}, but the responsibilities are different, they complete initialization dependencies and variable reads respectively.


\begin{figure}[!h]
\centering
\includegraphics[width=0.8\textwidth]{figures/variable-initialization-dependency-1.png}
\caption{initialization dependency}
 \label{fig:variable-initialization-dependency-1}
\end{figure}

Similarly, you can summarize all the initializers of a variable by calling \code{tf.global\_variables\_initializer()} and then start \ascii{Session} to initialize all variables.

\begin{leftbar}
\ Begin {python}
init = tf.global_variables_initializer()
\end{python}
\end{leftbar}

According to the dependency, because the control dependency edge between \code{W/Assign} and \code{Identity} is added, it is cleverly implemented that \code{W} is initialized before \code{V} and passed \ The current initialization value of code{W} finally completes the initialization of \code{V}.

\begin{figure}[!h]
\centering
\includegraphics[width=0.8\textwidth]{figures/variable-initialization-dependency-2.png}
\caption{Initialize OP}
 \label{fig:variable-initialization-dependency-2}
\end{figure}

\subsection{Initializer List}

You can use \code{variables\_initializer} to build a list of initializers for variable lists. Among them, \code{group} will construct a control only \ascii{NoOP} that depends on \code{\_initialier\_list()}.

\begin{leftbar}
\ Begin {python}
def variables_initializer(var_list, name="init"):
  def _initialier_list():
    return *[v.initializer for v in var_list]
  return control_flow_ops.group(_initialier_list(), name=name)
\end{python}
\end{leftbar}

For example, the initializer list of the global variable list can be constructed as follows.

\begin{leftbar}
\ Begin {python}
def global_variables_initializer():
  return variables_initializer(global_variables())
\end{python}
\end{leftbar}

\end{content}

\section{variable grouping}

\begin{content}

By default, \ascii{Variable} is divided into a collection of global variables and training variables. As in the previous example, \code{W, V} is automatically divided into a collection of global variables and training variables.

\subsection{global variable}

A collection of global variables can be easily retrieved via \code{tf.global\_variables()}. In a distributed environment, global variables enable parameter sharing between different processes.

\begin{leftbar}
\ Begin {python}
def global_variables():
  return ops.get_collection(ops.GraphKeys.GLOBAL_VARIABLES)
\end{python}
\end{leftbar}

\subsection{local variable}

A collection of local variables can be easily retrieved via \code{tf.local\_variables()}.

\begin{leftbar}
\ Begin {python}
def local_variables():
  return ops.get_collection(ops.GraphKeys.LOCAL_VARIABLES)
\end{python}
\end{leftbar}

You can build a local variable using the syntactic sugar of \code{local\_variable}.

\begin{leftbar}
\ Begin {python}
def local_variable(initial_value, validate_shape=True, name=None):
  return variables.Variable(
      initial_value, trainable=False,
      collections=[ops.GraphKeys.LOCAL_VARIABLES],
      validate_shape=validate_shape, name=name)
\end{python}
\end{leftbar}

Local variables represent shared variables within a process, and it usually does not need to be a breakpoint recovery \ascii{(Checkpoint)}, only for the purpose of temporary counters. For example, in a distributed environment, use local variables to record the number of \ascii{Epoch} of the process's read data.

\subsection{training variable}

A collection of training variables can be retrieved via \code{tf.trainable\_variables()}. In machine learning, training variables represent model parameters.

\begin{leftbar}
\ Begin {python}
def trainable_variables():
  return ops.get_collection(ops.GraphKeys.TRAINABLE_VARIABLES)
\end{python}
\end{leftbar}

\subsection{global\_step}

\code{global\_step} is a special \ascii{Variable}, which is not a training variable, but it is a global variable. In a distributed environment, \code{global\_step} is often used to track the number of times \ascii{step} has been run and to synchronize data between processes.

To create a \code{global\_step} you can use the following function:

\begin{leftbar}
\ Begin {python}
def create_global_step(graph=None):
  graph = ops.get_default_graph() if graph is None else graph
  with graph.as_default() as g, g.name_scope(None):
    collections = [GLOBAL_VARIABLES, GLOBAL_STEP]
    return variable(
        GLOBAL_STEP,
        shape=[],
        dtype=dtypes.int64,
        initializer=init_ops.zeros_initializer(),
        trainable=False,
        collections=collections)
\end{python}
\end{leftbar}

\end{content}

\section{Source code analysis: constructor variables}

\begin{content}

To simplify the code implementation, a simple refactoring of \ascii{Variable} is done here.

\begin{leftbar}
\ Begin {python}
class Variable(object):
  def __init__(self, initial_value=None, trainable=True,
    collections=None, name=None, dtype=None):
    with ops.name_scope(name, "Variable", [initial_value]) as name:
      self._cons_initial_value(initial_value, dtype)
      self._cons_variable(name)
      self._cons_initializer()
      self._cons_snapshot()
    self._cons_collections(trainable, collections)
\end{python}
\end{leftbar}

Constructing an instance of \ascii{Variable} basically consists of the following steps:

\subsection{construct initial value}

\begin{leftbar}
\ Begin {python}
  def _cons_initial_value(self, initial_value, dtype):
    self._initial_value = ops.convert_to_tensor(
        initial_value, name="initial_value", dtype=dtype)
\end{python}
\end{leftbar}

\subsection{construction variable OP}

\ascii{Variable} Completes the automatic derivation based on the type and size of the initial value.

\begin{leftbar}
\ Begin {python}
  def _cons_variable(self, name):
    self._variable = state_ops.variable_op_v2(
      self._initial_value.get_shape(),
      self._initial_value.dtype.base_dtype,
      name=name)
\end{python}
\end{leftbar}

\subsection{construct initializer}

The initializer for \ascii{Variable} is essentially a \ascii{Assign} that holds a reference to \ascii{Variable} and modifies the variable itself in place using the initial value.

\begin{leftbar}
\ Begin {python}
  def _cons_initializer(self):
    self._initializer_op = state_ops.assign(
      self._variable,
      self._initial_value).op
\end{python}
\end{leftbar}

\subsection{structuring snapshot}

The snapshot of \ascii{Variable} is essentially a \ascii{Identity} representing the current value of \ascii{Variable}.

\begin{leftbar}
\ Begin {python}
  def _cons_snapshot(self):
    with ops.colocate_with(self._variable.op):
      self._snapshot = array_ops.identity(
        self._variable, name="read")
\end{python}
\end{leftbar}

\subsection{variable grouping}

By default, \ascii{Variable} is divided into a collection of global variables; if \code{trainable} is true, it indicates that the variable is a training parameter and is divided into a collection of training variables.

\begin{leftbar}
\ Begin {python}
  def _cons_collections(self, trainable, collections)
    if collections is None:
      collections = [GLOBAL_VARIABLES]
    if trainable and TRAINABLE_VARIABLES not in collections:
      collections = list(collections) + [TRAINABLE_VARIABLES]
    ops.add_to_collections(collections, self)
\end{python}
\end{leftbar}

\end{content}
\begin{savequote}[45mm]
  \ascii{Any fool can write code that a computer can understand. Good programmers write code that humans can understand.}
  \qauthor{\ascii{- Martin Flower}}
\end{savequote}


\chapter{Queue} 
\label{ch:queue}
\begin{content}
\ascii{TensorFlow}'s \code{Session} is thread-safe. That is, multiple threads can use the same \code{Session} instance to concurrently execute the same \ascii{OP} of the same graph instance; the \ascii{TensorFlow} execution engine prunes the graph based on input and output. , get a sub-graph with minimal dependence.

Therefore, by multithreading and using the same \code{Session} instance, concurrently executing the different \ascii{OP} of the same graph instance, the final effect is the concurrent execution between the subgraphs.

For typical model training, you can take full advantage of the \code{Session} multi-threaded concurrency and improve the performance of the training. For example, the input submap is run in a separate thread to prepare the sample data; the training submap is run in a separate thread and is taken in a batch according to the size of \code{batch\_size} Sample and start the iterative training process.

This article will explain the infrastructure in the concurrency model above, including queues, multithreaded coordinators, and \ascii{QueueRunner} that controls \code{Enqueue OP} execution.

\end{content}


\section{Queue}
\begin{content}
In the execution engine of \ascii{TensorFlow}, \ascii{Queue} is a powerful tool for controlling asynchronous computing. In particular, \ascii{Queue} is a special \ascii{OP}, similar to \ascii{Variable}, which is a class of stateful \ascii{OP}.

Similarly, \ascii{Variable} has the associated \ascii{Assign} and other \ascii{OP}, \ascii{OP}, and \ascii{OP}, such as \code{Enqueue, Dequeue, EnqueueMany, DequeueMany} and other \ascii{OP}, they can directly modify the state of \ascii{Queue}.


\subsection{FIFOQueue}
Give a simple example. First, a \code{FIFOQueue} queue is constructed; then, a \code{EnqueueMany} is added to the calculation graph, and the \ascii{OP} is used to append \ascii{1} or more elements to the head of the queue. Secondly, add a dequeue of \code{Dequeue}; finally, increase the value of the dequeue element by \ascii{1} and then join the result. The construction of the calculation graph is shown in the figure below before starting the calculation diagram execution.

\begin{figure}[!h]
  \centering
  \includegraphics[width=0.9\textwidth]{figures/py-queue-example-1.png}
  \caption{Figure construction period}
  \label{fig:py-queue-example-1}
\end{figure}

After executing the \code{EnqueueMany} operation, the state of the calculation graph is as shown below.

\begin{figure}[!h]
  \centering
  \includegraphics[width=0.9\textwidth]{figures/py-queue-example-2.png}
  \caption{Figure execution period: Execute EnqueueMany}
  \label{fig:py-queue-example-2}
\end{figure}

After executing the first step \code{Enqueue}, the state of the calculation graph is as shown below.

\begin{figure}[!h]
  \centering
  \includegraphics[width=0.9\textwidth]{figures/py-queue-example-3.png}
  \caption{Figure execution period: Execute Enqueue once}
  \label{fig:py-queue-example-3}
\end{figure}


\subsection{Use}
Queues play an important role in model training. Later, we will cover the \ascii{Pipeline} of data loading. The training model often uses \code{RandomShuffleQueue} to prepare sample data for it. To improve the throughput of \ascii{IO}, you can use multithreading to append sample data to the sample queue concurrently; at the same time, the thread of the training model iterates through \code{train\_op} and gets \code once. Batch sample data of {batch\_size} size.

Obviously, the queue plays the role of asynchronous coordination and data exchange in the \ascii{Pipeline} process, which brings a lot of flexibility to the design and implementation of \ascii{Pipeline}.

It's important to note that in order for the queue to maximize its multithreading, two tricky issues need to be addressed:

\begin{enum}
  \eitem{How do I stop all threads at the same time, and how to handle exception reports?}
  \eitem{How to append sample data to the queue concurrently?}
\end{enum}

Therefore, \ascii{TensorFlow} designed two classes, \code{tf.train.Coordinator} and \code{tf.train.QueueRunner}, to solve the above two problems.

These two classes complement each other, \code{Coordinator} coordinates multiple threads to stop running at the same time, and reports an exception to the main program waiting for the stop notification; and \code{QueueRunner} creates a set of threads and collaborates multiple enqueues\ascii Execution of {OP} (eg \code{Enqueue, EnqueueMany}).

\end{content}


\section{Coordinator}
\begin{content}
\code{Coordinator} provides a simple mechanism to stop the execution of a group of threads at the same time. It has \ascii{3} important methods:

\begin{enum}
\eitem{\code{should\_stop}: Determine if the current thread should exit}
\eitem{\code{request\_stop}: Request all threads to stop executing}
\eitem{\code{join}: Wait for all threads to stop executing}
\end{enum}


\subsection{Usage method}
In general, the main program often uses the following pattern to use \code{Coordinator}.

\begin{leftbar}
\begin{python}
# Create a coordinator.
coord = tf.train.Coordinator()

# Create 10 threads that run 'MyLoop()'
threads = [threading.Thread(target=MyLoop, args=(coord,)) 
          for i in xrange(10)]

# Start the threads.
for t in threads:
  t.start()
  
# wait for all of them to stop
coord.join(threads)
\end{python}
\end{leftbar}

Any child thread can notify other threads to stop execution by calling \code{coord.request\_stop}. Therefore, in the iterative execution of each thread, check \code{coord.should\_stop()} beforehand. Once \code{coord.request\_stop} is called, \code{coord.request\_stop()} of other threads will immediately return \code{True}.

In general, the iterative execution method of a child thread follows the implementation pattern as follows.

\begin{leftbar}
\begin{python}
def MyLoop(coord):
  try
    while not coord.should_stop():
      # ...do something...
  except Exception as e:
    coord.request_stop(e)
\end{python}
\end{leftbar}


\subsection{Exception handling}
When an exception occurs on a thread, the exception can be reported via \code{coord.request\_stop(e)}.

\begin{leftbar}
\begin{python}
try:
  while not coord.should_stop():
    # ...do some work...
except Exception as e:
  coord.request_stop(e)
\end{python}
\end{leftbar}

To eliminate duplicate code for exception code handling, you can use the context manager of \code{coord.stop\_on\_exception()}.

\begin{leftbar}
\begin{python}
with coord.stop_on_exception():
  while not coord.should_stop():
    # ...do some work...
\end{python}
\end{leftbar}

Among them, the exception will be re-thrown in \code{coord.join}. Therefore, the main program also needs to handle exceptions reasonably.

\begin{leftbar}
\begin{python}
try:
  # Create a coordinator.
  coord = tf.train.Coordinator()

  # Create 10 threads that run 'MyLoop()'
  threads = [threading.Thread(target=MyLoop, args=(coord,)) 
            for i in xrange(10)]

  # Start the threads.
  for t in threads:
    t.start()

  # wait for all of them to stop
  coord.join(threads)
except Exception as e:
  # ...exception that was passed to coord.request\_stop(e)
\end{python}
\end{leftbar}


\subsection{Combat: LoopThread}
\end{content}


\section{QueueRunner}
\begin{content}
A \code{QueueRunner} instance holds one or more \code{Enqueue} enqueues\ascii{OP}, which starts a thread for each \code{Enqueue OP}.

\begin{figure}[!htbp]
  \centering
  \includegraphics[width=0.9\textwidth]{figures/tf-queue-runner-model.png}
  \caption{TensorFlow System Architecture}
  \label{fig:tf-queue-runner-model}
\end{figure}


\subsection{Register QueueRunner}
You can call \code{tf.train.add\_queue\_runner} to register the \code{QueueRunner} instance into the calculation diagram and add it to the \code{GraphKeys.QUEUE\_RUNNERS} collection.

\begin{leftbar}
\begin{python}
def add_queue_runner(qr, collection=ops.GraphKeys.QUEUE_RUNNERS):
  ops.add_to_collection(collection, qr)
\end{python}
\end{leftbar}


\subsection{Execute QueueRunner}
When \code{tf.train.start\_queue\_runners} can be called, it will find all \code{QueueRunner} instances from the calculation graph and fetch all \code{Enqueue OP} from the \code{QueueRunner} instance. Start a thread for each \ascii{OP}.

\begin{leftbar}
\begin{python}
def start_queue_runners(sess, coord, daemon=True, start=True,
                        collection=ops.GraphKeys.QUEUE_RUNNERS):
  with sess.graph.as_default():
    threads = []
    for qr in ops.get_collection(collection):
      threads.extend(qr.create_threads(
          sess, coord=coord, daemon=daemon, start=start))
  return threads
\end{python}
\end{leftbar}

In the \code{QueueRunner.create\_threads} method, start a separate thread for each \code{Enqueue} type of \ascii{OP} it contains.

\begin{leftbar}
\begin{python}
class QueueRunner(object):
  def create_threads(self, sess, coord, daemon, start):
    """Create threads to run the enqueue ops.
    """
    threads = [threading.Thread(
        target = self._run, args = (sess, op, coord))
        for in self._enqueue_ops]
    if coord:
      threads.append(threading.Thread(
          target=self._close_on_stop, 
          args = (sess, self.cancel_op, coord)))
    for t in threads:
      if coord:
        coord.register_thread(t)
      if daemon:
        t.daemon = daemon
      if start:
        t.start()
    return threads
\end{python}
\end{leftbar}


\subsubsection{Iterative execution Enqueue}
Each \code{Enqueue} child thread will iterate through \code{Enqueue OP}. When a \code{OutOfRangeError} exception occurs, the queue is automatically closed and the child thread is exited; however, if other types of exceptions occur, \code{Coordinator} is actively notified to stop all threads from running and exit the child thread.

\begin{leftbar}
\begin{python}
class QueueRunner(object):
  def _run (self, sess, enqueue_op, coord):
    try:
      enqueue_callable = sess.make_callable(enqueue_op)
      while True:
        if coord.should_stop():
          break
        try:
          enqueue_callable()
        except errors.OutOfRangeError:  
          sess.run (self._close_op)
          return
    except Exception as e:
      coord.request_stop(e)
\end{python}
\end{leftbar}


\subsubsection{Listening queue off}
In addition, if a \code{Coordinator} instance is given, \code{QueueRunner} will additionally start a thread; when the \code{Coordinator} instance is triggered to call the \code{request\_stop} method, the thread will automatically close. queue.

\begin{leftbar}
\begin{python}
class QueueRunner(object):
  def _close_on_stop(self, sess, cancel_op, coord):
    """Close the queue, and cancel pending enqueue ops
       when the Coordinator requests stop.
    """
    coord.wait_for_stop()
    try:
      sess.run (cancel_op)
    except Exception:
      pass
\end{python}
\end{leftbar}

Among them, \code{Cancel OP} and \code{Close OP} of \code{Queue} will close the queue, but \code{Cancel OP} will undo the cached list of \code{Enqueue OP}, but \code{Close OP} keeps a list of cached \code{Enqueue OP}.


\subsection{Close Queue}
When the queue is closed, an error will be generated for any attempt on \code{Enqueue}. However, for any attempt to \code{Dequeue} it is still successful, as long as the elements are left in the queue; otherwise, \code{Dequeue} will fail immediately, throwing a \code{OutOfRangeError} exception without blocking waiting for more elements to be Enter the team.

\end{content}

\begin{savequote}[45mm]
  \ascii{Any fool can write code that a computer can understand. Good programmers write code that humans can understand.}
  \qauthor{\ascii{- Martin Flower}}
\end{savequote}


\chapter{OP Essentials} 
\label{ch: essential-op}
\begin{content}
\end{content}


\section{OP registration}
\begin{content}
In the \cpp{} backend system, the system completes the registration of all \ascii{OP} when the system is initialized. The registration of \ascii{OP} is done via the \code{REGISTER\_OP} macro.


\subsection{REGISTER\_OP}
In practice, \code{REGISTER\_OP} defines a set of exquisite internal \ascii{DSL}, the system automatically completes the translation of the string representation, and converts it to the internal representation of \code{OpDef}, and finally saves it in \code{OpDef} in the repository.

\begin{figure}[!h]
  \centering
  \includegraphics[width=0.9\textwidth]{figures/cc-op-repo.png}
  \caption{REGISTER\_OP: Utility macro for registering OP}
  \label{fig:cc-op-repo}
\end{figure}


\subsection{Query Interface}

\begin{leftbar}
\begin{c++}
struct OpRegistryInterface {
  virtual ~OpRegistryInterface() {}

  virtual Status LookUp(
    const string& op_name,
    const OpRegistrationData ** on_reg_data) const = 0;
  Status LookUpOpDef(const string& op_name, const OpDef** op_def) const;
};
\end{c++}
\end{leftbar}

Among them, \code{OpRegistrationData} describes the two basic metadata of \ascii{OP}: \code{OpDef} and \code{OpShapeInferenceFn}; the former is used to describe the input/output parameter information of \ascii{OP}, attribute Name list, and its constraint relationship. The latter is used to describe the deduction rules for \ascii{Shape} of \ascii{OP}.

\begin{leftbar}
\begin{c++}
struct OpRegistrationData {
  OpDef op_def;
  OpShapeInferenceFn shape_inference_fn;
};

using OpRegistrationDataFactory = 
  std::function<Status(OpRegistrationData*)>;
\end{c++}
\end{leftbar}


\subsection{OpDef repository}
In the implementation, the technique of delay initialization is adopted. To simplify the description of the problem, here is a simple code refactoring to help you understand how \code{OpRegistry} works.

\begin{leftbar}
\begin{c++}
struct OpRegistry : OpRegistryInterface {  
  OpRegistry();
  ~OpRegistry() override;

  void Register(const OpRegistrationDataFactory& factory);

 private:
  Status LookUp(
     const string& op_name,
     const OpRegistrationData** op_reg_data) const override;

 private:
  using Registry = 
    std::unordered_map<string, OpRegistrationData*>;

  mutex mu_;
  Registry registry_;
};
\end{c++}
\end{leftbar}

\begin{leftbar}
\begin{c++}
Status OpRegistry::Register(
  const OpRegistrationDataFactory& factory) {
  auto op_reg_data(std::make_unique<OpRegistrationData>());
  Status s = factory(op_reg_data.get());
  if (s.ok()) {
    gtl::InsertIfNotPresent(&registry_, 
      op_reg_data-> op_def.name (),
      op_reg_data.get ())
  }
  if (s.ok()) {
    op_reg_data.release ();
  } else {
    op_reg_data.reset ();
  }
  return watcher_status;
}
\end{c++}
\end{leftbar}

\end{content}


\part{Runtime model}
\begin{savequote}[45mm]
  \ascii{Any fool can write code that a computer can understand. Good programmers write code that humans can understand.}
  \qauthor{\ascii{- Martin Flower}}
\end{savequote}


\chapter{Local execution} 
\label{ch:local}
\begin{content}
\tf{} can be run independently in a process to complete the execution of the calculation graph. This chapter will focus on the basic architecture and operational mechanisms of the local runtime; focus on the implementation details of computational graph pruning, splitting, optimization, execution, etc.; and explore in detail the data between \ascii{OP} across devices. The working mechanism of the interaction, and its \ascii{OP} orchestration on the device set (\ascii{placement}) algorithm.
\end{content}


\section{Local Mode}
\label{sec:local-runtime}
\begin{content}
As shown in \refig{local}, in local mode, \ascii{Client, Master, Worker} is deployed in the same process on the same machine, and these three roles are played by \code{DirectSession} at the same time. \code{DirectSession} runs in a separate process, with function call relationships between service entities.

\begin{figure}[H]
  \centering
  \includegraphics[width=0.7\textwidth]{figures/local.png}
  \caption{local mode}
  \label{fig:local}
\end{figure}

\ascii{Client} is responsible for computing the construction of the graph, starting the execution of the graph by calling \code{Session.run}. As shown in \refig{local-runtime}, during the execution of \code{run\_step}, there are three important stages involved in the pruning, splitting, and execution of computational graphs.

\begin{figure}[H]
  \centering
  \includegraphics[width=0.8\textwidth]{figures/local-runtime.png}
  \caption{Local mode: Figure operation}
  \label{fig:local-runtime}
\end{figure}


\subsection{Partial execution}
\ascii{Master} starts the pruning operation of the calculation graph after receiving the calculation diagram execution command. It reverses the traversal map based on the input and output of the computed graph, looking for a minimally dependent subgraph, often called \code{ClientGraph}.

That is to say, each time you execute \code{run\_step}, the entire calculation graph (\code{FullGraph}) is not executed, but a partial subgraph is executed. The pruning reflects the design concept of the \tf{} part of the implementation.


\subsection{Concurrent execution}
Then, the runtime completes the splitting of the graph according to the current device set, and generates a lot of subgraphs, each of which is called \code{PartitionGraph}; then triggers each \ascii{Worker} to execute each \code{PartitionGraph} concurrently; For each \ascii{PartitionGraph}, the runtime will launch a \ascii{Executor} to complete the execution of \code{PartitionGraph} according to its topology.

In other words, splitting and execution embody the design philosophy of \tf{} concurrent execution.

\end{content}


\section{Session control}
In local mode, its runtime is controlled by \code{DirectSession}. In general, when \code{DirectSession} executes a calculation graph, each component is a function call relationship. However, \code{DirectSession} also has a clear lifecycle management mechanism, as shown by \refig{local-direct-session-lifecycle}.

\begin{figure}[H]
  \centering
  \includegraphics[width=0.6\textwidth]{figures/local-direct-session-lifecycle.png}
  \caption{DirectSession lifecycle}
  \label{fig:local-direct-session-lifecycle}
\end{figure}


\subsection{Domain model}
As shown by \refig{local-direct-session-model}, \code{DirectSession} holds the \code{SimpleGraphExecutionState} instance, which is responsible for computing the pruning of the graph and generating the \code{ClientGraph} instance.

\code{DirectSession} holds a set of thread pools at the same time, but when \code{DirectSession.run} is not used, it selects one of the thread pool groups to provide services according to the externally configured index. Because \code{DirectSession} is thread-safe and supports multiple concurrent executions of \code{DirectSession.run}, multiple thread pool instances can be run simultaneously.

\begin{figure}[H]
  \centering
  \includegraphics[width=0.9\textwidth]{figures/local-direct-session-model.png}
  \caption{DirectSession Domain Model}
  \label{fig:local-direct-session-model}
\end{figure}


\subsection{Create session}
As shown by \refig{local-direct-session-factory}, \code{DirectSession} is created by \code{DirectSessionFactory} polymorphism. Where \code{DeviceFactory::AddDevices} will create a local device set.

Among them, \code{DirectSession} mainly completes the creation of the thread pool group.

\begin{figure}[H]
  \centering
  \includegraphics[width=0.6\textwidth]{figures/local-direct-session-factory.png}
  \caption{Polymorphic creation of DirectSession}
  \label{fig:local-direct-session-factory}
\end{figure}

\begin{leftbar}
\begin{c++}
struct DirectSessionFactory : SessionFactory {
  bool AcceptsOptions(const SessionOptions& options) override {
    return options.target.empty();
  }

  Session* NewSession(const SessionOptions& options) override {
    std::vector<Device*> devices;
    DeviceFactory::AddDevices(
        options, "/job:localhost/replica:0/task:0", &devices);
    return new DirectSession(options, new DeviceMgr(devices));
  }
};
\end{c++}
\end{leftbar}

Where \code{DirectSessionFactory::NewSession} is called by \ascii{C API}.

\begin{leftbar}
\begin{c++}
Status NewSession(const SessionOptions& options, Session** out_session) {
  SessionFactory* factory;
  Status s = SessionFactory::GetFactory(options, &factory);
  if (!s.ok()) {
    *out_session = nullptr;
    return s;
  }
  *out_session = factory->NewSession(options);
  if (!*out_session) {
    return errors::Internal("Failed to create session.");
  }
  return Status::OK();
}

TF_DeprecatedSession* TF_NewDeprecatedSession(
  const TF_SessionOptions* opt, TF_Status* status) {
  Session* session;
  status->status = NewSession(opt->options, &session);
  if (status->status.ok()) {
    return new TF_DeprecatedSession({session});
  } else {
    return nullptr;
  }
}
\end{c++}
\end{leftbar}

In the constructor of \code{DirectSession}, it is mainly responsible for the initialization of its domain model, including the creation of the thread pool, and the construction of the \code{CancellationManager} instance.

\begin{leftbar}
\begin{c++}
DirectSession::DirectSession(
    const SessionOptions& options,
    const DeviceMgr* device_mgr)
    : options_(options),
      device_mgr_(device_mgr),
      cancellation_manager_(new CancellationManager()) {
  // thread\_pools\_ = ... 
}
\end{c++}
\end{leftbar}


\subsection{Destroy session}
\code{DirectSession} from \ascii{SessionFactory}\code{new} is responsible for \code{delete} by \ascii{C API}.

\begin{leftbar}
\begin{c++}
void TF_DeleteDeprecatedSession(TF_DeprecatedSession* s, TF_Status* status) {
  status->status = Status::OK();
  delete s->session;  // delete DirectSession
  delete s;
}
\end{c++}
\end{leftbar}

Subsequently, the destructor of \code{DirectSession} is called, which is responsible for cleaning up the system resources it is responsible for. It mainly includes the \code{Executor} list, the \code{ThreadPool} list, and the \code{CancellationManager} instance.

\begin{leftbar}
\begin{c++}
DirectSession::~DirectSession() {
  for (auto& it : partial_runs_) {
    it.second.reset(nullptr);
  }
  
  for (auto& it : executors_) {
    it.second.reset();
  }
  
  for (auto d : device_mgr_->ListDevices()) {
    d->op_segment()->RemoveHold(session_handle_);
  }
  
  delete cancellation_manager_;
  
  for (const auto& p_and_owned : thread_pools_) {
    if (p_and_owned.second) delete p_and_owned.first;
  }

  execution_state_.reset(nullptr);
  flib_def_.reset(nullptr);
}
\end{c++}
\end{leftbar}


\subsection{Creating/Extending a graph}
The first extension of a graph is equivalent to creating a graph. The extended graph is based on the original compute graph, and a new subgraph is added. Of course, the nodes included in the added subgraph should not exist in the original compute graph.

\begin{leftbar}
\begin{c++}
Status DirectSession::Create(const GraphDef& graph) {
  if (graph.node_size() > 0) {
    mutex_lock l(graph_def_lock_);
    return ExtendLocked(graph);
  }
  return Status::OK();
}

Status DirectSession::Extend(const GraphDef& graph) {
  mutex_lock l(graph_def_lock_);
  return ExtendLocked(graph);
}
\end{c++}
\end{leftbar}

When creating a calculation graph, \code{DirectSession} mainly completes the creation of the \code{SimpleGraphExecutionState} instance. As shown in \refig{local-simple-graph-execution-state-model}, the \code{SimpleGraphExecutionState} instance holds instances of \code{FullGraph}: \code{Graph} and \code{GraphDef}, It is responsible for managing and maintaining the lifecycle of \code{FullGraph}.

\begin{figure}[H]
  \centering
  \includegraphics[width=0.5\textwidth]{figures/local-simple-graph-execution-state-model.png}
  \caption{Create a SimpleGraphExecutionState instance}
  \label{fig:local-simple-graph-execution-state-model}
\end{figure}

Among them, the main responsibilities of \code{SimpleGraphExecutionState} include:

\begin{enum}
  \eitem{construct \code{FullGraph}: Occurs in \code{DirectSession.Create};}
  \eitem{Execute simple \ascii{OP} orchestration algorithm: Occurs in \code{DirectSession.Create};}
  \eitem{Execute pruning operation: Occurs in \code{DirectSession.Run}.}
\end{enum}

When \code{DirectSession::Create} is executed, the \code{SimpleGraphExecutionState} instance is created and the build and initialization of the \code{FullGraph} instance is completed.

\begin{leftbar}
\begin{c++}
Status SimpleGraphExecutionState::MakeForBaseGraph(
    GraphDef* graph_def, const SimpleGraphExecutionStateOptions& opts,
    std::unique_ptr<SimpleGraphExecutionState>* out_state) {
  auto ret = std::make_unique<SimpleGraphExecutionState>(graph_def, opts));

  AddDefaultAttrsToGraphDef(&ret->original_graph_def_, *ret->flib_def_, 0));
  if (!ret->session_options_->config.graph_options().place_pruned_graph()) {
    direction> InitBaseGraph ();
  }
  *out_state = std::move(ret);
  return Status::OK();
}
\end{c++}
\end{leftbar}

Among them, \code{SimpleGraphExecutionState::InitBaseGraph} completes the format conversion of \code{FullGraph} from \code{GraphDef} to \code{Graph} and starts the \ascii{OP} orchestration algorithm of \code{SimplePlacer}.

\begin{leftbar}
\begin{c++}
Status SimpleGraphExecutionState::InitBaseGraph() {
  auto ng = std::make_unique<Graph>(OpRegistry::Global());

  GraphConstructorOptions opts;
  ConvertGraphDefToGraph(opts, *original_graph_def_, ng.get());

  SimplePlacer placer(ng.get(), device_set_, session_options_);
  placer.Run ();

  this-> graph_ = ng.release ();
  return Status::OK();
}
\end{c++}
\end{leftbar}


\subsubsection{Graph construction: GraphDef -> Graph}
At the beginning, \code{SimpleGraphExecutionState} got \code{GraphDef}, which is the original primitive structure. It is serialized by \ascii{Client} and passed to the backend \ascii{C++}, which is then deserialized by the backend.

As shown in \refig{local-graph-def-to-graph}, convert the \code{GraphDef} instance to the equivalent \code{Graph} instance by calling \code{ConvertGraphDefToGraph}; similarly, you can call \code {Graph.ToGraphDef} transforms the \code{Graph} instance into an equivalent \code{GraphDef} instance.

Where \code{GraphDef} is a graph structure that exists in the \ascii{protobuf} format, which contains all the metadata of the graph; and \code{Graph} is the domain object used to describe the graph structure in the runtime system. Not only holds the metadata of \code{GraphDef}, but also contains other information about other graph structures.

\begin{figure}[H]
  \centering
  \includegraphics[width=0.6\textwidth]{figures/local-graph-def-to-graph.png}
  \caption{Format conversion between \code{GraphDef} and \code{Graph}}
  \label{fig:local-graph-def-to-graph}
\end{figure}


\subsubsection{OP orchestration: SimplePlacer}
The \ascii{OP} orchestration (\ascii{placement}) refers to the computation of the resources in the most efficient way by placing the \ascii{OP} contained in the calculation graph on the appropriate computing device. Utilization, which can be formally described as \refig{local-cost-model}.

\begin{figure}[H]
  \centering
  \includegraphics[width=0.6\textwidth]{figures/local-cost-model.png}
  \caption{cost model}
  \label{fig:local-cost-model}
\end{figure}

To find the optimal arrangement, I guess this is a \ascii{NP} problem. This problem depends on the characteristics of the computational graph, network topology and bandwidth, the number of samples, and many other complex factors, which is one of the most active issues in the community.


\subsection{Iteration execution}
\code{DirectSession.Run} is the key path to the \tf{} runtime and is responsible for completing an iterative calculation. First, \code{DirectSession} prunes the \code{FullGraph} based on the input/output to generate \code{ClientGraph}; then, splits \code{ClientGraph} into multiple \code{PartitionGraph} based on the set of local devices held. The runtime starts a \code{Executor} instance for each of its \code{PartitionGraph}, which executes the topological sorting algorithm of \code{PartitionGraph} to complete the execution of the computed graph.

For specific implementation, please refer to \refsec{graph-operation-prune}, \refsec{graph-operation-split}, \refsec{graph-operation-exec}.


\subsubsection{Graph operation}
As shown in \refig{local-graph-transformation}, in local mode, the computational graph undergoes three forms of transformation and is ultimately decomposed into computational devices to enable concurrent execution of subgraphs on various computing devices.

\begin{figure}[H]
  \centering
  \includegraphics[width=0.9\textwidth]{figures/local-graph-transformation.png}
  \caption{Graph transformation}
  \label{fig:local-graph-transformation}
\end{figure}

\begin{itemize}
  \item \code{FullGraph}: \ascii{Client} is responsible for constructing the complete calculation graph, often called \code{FullGraph}; however, once \code{Session.run} does not execute the entire calculation graph;
  \item \code{ClientGraph}: \ascii{Master} passes the \code{feeds, fetches} input and output list according to \code{Session.run}, implements pruning operation on \ascii{FullGraph}, and calculates the local iteration execution. The least dependent subgraph, often called \code{ClientGraph};
  \item \code{PartitionGraph}: \ascii{Master} splits \code{ClientGraph} into multiple \code{PartitionGraph} based on the current set of computing devices and its \ascii{OP} device constraint specification; The computing device corresponds to a \code{PartitionGraph}, and the computing device is responsible for the execution of \code{PartitionGraph}.
\end{itemize}

However, the data structures of \code{FullGraph, ClientGraph, PartitionGraph} are the same, they are all three different manifestations of \code{Graph}, only the size and scope are different.


\subsubsection{Formalization}
In real system implementations, the local mode runtime is implemented using \ascii{C++}. The key path to the \tf{} runtime is \code{run\_step}. Because the real system implementation involves too much detail, it is not easy to find the backbone and logic of the algorithm. To simplify the description of the problem, the implementation of \code{run\_step} will be described formally.

\begin{leftbar}
\begin{python}
def do_run_partitions(executors_and_partitions):
  barrier = ExecutorBarrier(executors_and_partitions.size())
  for (executor, partition) in executors_and_partitions:
    executor.run(partition, barrier)  
  barrier.wait()

def run_partitions(executors_and_partitions, inputs, outputs):
  frame = FunctionCallFrame()
  frame.set_args(inputs)
  do_run_partitions(executors_and_partitions)
  frame.get_ret_vals(outputs)

def run_step(devices, full_graph, inputs, outputs):
  client_graph = prune(full_graph, inputs, outputs)
  executors_and_partitions = split(client_graph, devices)
  run_partitions(executors_and_partitions, inputs, outputs)
\end{python}
\end{leftbar}

Among them, on each computing device, start a \code{Executor} to execute the \code{PartitionGraph} assigned to it. After a certain computing device executes the allocated \code{PartitionGraph}, the counter of \code{ExecutorBarrier} is added to \ascii{1} until all devices complete the execution of the \code{PartitionGraph} list, \code{barrier.wait ()} Blocking operation exits.

There may be data dependencies between \code{PartitionGraph} across devices, and they interact by inserting \code{Send/Recv} nodes. In fact, in local mode, \code{Send/Recv} does the data exchange via \code{Rendezvous}. \code{Send} puts the data on \code{Rendezvous}, and \code{Recv} takes it from \code{Rendezvous} according to the identifier. Where \code{Send} is not blocked, and \code{Recv} is blocked.


\subsection{Close session}

\begin{leftbar}
\begin{c++}
Status DirectSession::Close() {
  cancellation_manager_->StartCancel();
  {
    mutex_lock l(closed_lock_);
    if (closed_) return Status::OK();
    closed_ = true;
  }
  return Status::OK();
}
\end{c++}
\end{leftbar}

Register \ascii{Step} to \code{CancellationManager} of \code{DirectSession} as shown in \refig{local-cancellation-manager}. When \code{DirectSession} is closed, the \code{CancellationManager} of \code{DirectSession} will cancel the execution of this \ascii{step}.

\begin{figure}[H]
  \centering
  \includegraphics[width=0.9\textwidth]{figures/local-cancellation-manager.png}
  \caption{How does CancellationManager work}
  \label{fig:local-cancellation-manager}
\end{figure}

\begin{leftbar}
\begin{c++}
Status DirectSession::Run(
   const NamedTensorList& inputs,
   const std::vector<string>& output_names,
   const std::vector<string>& target_nodes,
   std::vector<Tensor>* outputs) {
  // step\_cancellation\_manager is passed to `OpKernelContext`
  CancellationManager step_cancellation_manager;

  // Register this step with session's cancellation manager, so that
  // `Session::Close()` will cancel the step.
  CancellationToken cancellation_token =
      cancellation_manager_->get_cancellation_token();
  bool already_cancelled = !cancellation_manager_->RegisterCallback(
      cancellation_token, [&step_cancellation_manager]() {
        step_cancellation_manager.StartCancel();
      });
  // ignore others...
}
\end{c++}
\end{leftbar}

The \code{CancellationManager} of the current \ascii{Step} will eventually be passed to \code{OpKernelContext}. When \ascii{Kernel} implements the calculation, if the intermediate state is saved, you can register the corresponding callback hook with it. Among them, each callback hook has a unique \code{token} identifier.

When \ascii{Step} is canceled, the callback hook is called, and the \ascii{Kernel} can cancel the calculation of the \ascii{OP}. For example, when \code{FIFOQueue} implements \code{TryEnqueue}, a callback hook is registered to the \code{CancellationManager} of this \ascii{Step} to cancel the status information in the middle of \ascii{Kernel}.

\begin{leftbar}
\begin{c++}
void FIFOQueue::TryEnqueue(const Tuple& tuple, OpKernelContext* ctx,
                           DoneCallback callback) {
  CancellationManager* cm = ctx->cancellation_manager();
  CancellationToken token = cm->get_cancellation_token();
  bool already_cancelled;
  {
    mutex_lock l(mu_);
    already_cancelled = !cm->RegisterCallback(
        token, [this, cm, token]() { Cancel(kEnqueue, cm, token); });
  }
  // ignore others...
}
\end{c++}
\end{leftbar}


\section{Pruning}
\label{sec:graph-operation-prune}
When \code{DirectSession::Run} is executed, the construct of \code{ClientGraph} is completed first. In fact, the construction process of \code{ClientGraph} mainly completes the pruning algorithm of \code{FullGraph} and generates \code{ClientGraph}.


\subsection{Build ClientGraph}
As shown in \refig{local-simple-graph-execution-state}, the \code{SimpleGraphExecutionState} instance holds the \code{FullGraph} instance and generates \code{ClientGraph} based on the input/output list.

\begin{figure}[H]
  \centering
  \includegraphics[width=0.55\textwidth]{figures/local-simple-graph-execution-state.png}
  \caption{Generate\code{ClientGraph}}
  \label{fig:local-simple-graph-execution-state}
\end{figure}

Where \code{BuildGraphOptions} contains the input/output list and calls \code{SimpleGraphExecutionState::BuildGraph} to generate the \code{ClientGraph} instance.

\begin{leftbar}
\begin{c++}
namespace {
  BuildGraphOptions build_graph_options(
    const NamedTensorList& inputs,
    const std::vector<string>& outputs,
    const std::vector<string>& targets) {
    // sort inputs/outputs/targets
    std::vector<string> inputs_sorted(inputs.begin(), inputs.end());
    std::sort(inputs_sorted.begin(), inputs_sorted.end());

    std::vector<string> outputs_sorted(outputs.begin(), outputs.end());
    std::sort(outputs_sorted.begin(), outputs_sorted.end());

    std::vector<string> tn_sorted(targets.begin(), targets.end());
    std::sort(tn_sorted.begin(), tn_sorted.end());

    // build graph options
    BuildGraphOptions options;
    options.feed_endpoints = inputs_sorted;
    options.fetch_endpoints = outputs_sorted;
    options.target_nodes = tn_sorted;
    options.use_function_convention = !run_state_args->is_partial_run;
    return options;
  }
}

Status DirectSession::Run(
  const RunOptions& run_options,
  const NamedTensorList& inputs,
  const std::vector<string>& output_names,
  const std::vector<string>& target_nodes,
  std::vector<Tensor>* outputs,
  RunMetadata* run_metadata) {

  // 1. prune graph
  // client\_graph = prune(full\_graph, inputs, outputs)
  std :: unique_ptr <SimpleClientGraph> client_graph;
  execution_state_->BuildGraph(
    build_graph_options(inputs, output_names, target_nodes), 
    &client_graph);
   
  // 2. split graph into partition by devices 
  // executors\_and\_partitions = split(client\_graph, devices)
  
  // 3. lauch executor per partition
  // def run\_partitions(executors\_and\_partitions, inputs, outputs):
  // \ \ frame = FunctionCallFrame()
  // \ \ frame.set\_args(inputs)
  // \ \ for (executor, partition) in executors\_and\_partitions: 
  // \ \ \ \ exec.run(part)
  // \ \ frame.get\_ret\_vals(outputs)

  return Status::OK();
}
\end{c++}
\end{leftbar}

\code{ClientGraph} is initially from the original \code{FullGraph}, calling the \code{RewriteGraphForExecution} function, which will rewrite the \code{ClientGraph} according to the input/output, including adding nodes, or deleting nodes, and finally generating\ Code{SimpleClientGraph} instance.

\begin{leftbar}
\begin{c++}
const DeviceAttributes& 
SimpleGraphExecutionState::local_device_attr() const {
  return device_set_->client_device()->attributes();
}

Status SimpleGraphExecutionState::BuildGraph(
  const BuildGraphOptions& options, 
  std::unique_ptr<SimpleClientGraph>* out) {
  // 1. create new\_graph from origin graph, 
  // which is client graph.
  std::unique_ptr<Graph> ng;
  ng.reset(new Graph(flib_def_.get()));
  CopyGraph (* graph_, ng.get ());

  // 2. prune the client graph
  subgraph::RewriteGraphForExecution(
    ng.get(), options.feed_endpoints, options.fetch_endpoints,
    options.target_nodes, local_device_attr(),
    options.use_function_convention);
  }

  // 3. create SimpleClientGraph, and return it.
  std::unique_ptr<SimpleClientGraph> dense_copy(
      new SimpleClientGraph(std::move(flib)));
  CopyGraph (* ng, & dense_copy-> graph);
  *out = std::move(dense_copy);

  return Status::OK();
}
\end{c++}
\end{leftbar}

Therefore, the \code{ClientGraph} process is constructed with the key path \code{RewriteGraphForExecution}, the pruning algorithm. The pruning algorithm traverses \ascii{FullGraph} in the inverse of the input/output list to find the smallest dependent subgraph \code{ClientGraph}.

In general, for the \code{ClientGraph} input node, the starting node is played; and the output node acts as the terminating node. So, with regard to input and output, there are two more difficult issues:

\begin{enum}
  \eitem{Input: How to pass \code{Tensor} to the input node before the start of the \code{ClientGraph} calculation;}
  \eitem{Output: When the \code{ClientGraph} is evaluated, how does the external runtime get \code{Tensor} from the output node.}
\end{enum}

There are two kinds of media: \code{FunctionCallFrame} and \code{Rendezvous}, and the external runtime and the input/output node can exchange data using one of the media.

\code{FunctionCallFrame} is used for the \code{Arg/RetVal} function call's \ascii{OP}, which is used to pass function argument values ​​when the function is called, and its return function value. However, they only apply to the single-process runtime environment.

\code{Rendezvous} is used for \code{Send/Recv} messaging \ascii{OP}, which is a more general way of communicating for distributed runtime environments.


\subsection{Based on Rendezvous}
As shown by \refig{client-prune-graph}, according to the \code{fetches} list, the dependent nodes are searched backwards until \code{feeds}, and the sub-graph with the smallest dependency is calculated.

Perform pruning on the edge of \code{Feed}, such as the pruning \code{ina:0}) edge, and insert the node \code{Recv} here, and name the node by the name of the input side, for example \code {\_recv\_ina\_0}.

Similarly, pruning is also performed on the edges of \code{Fetch}, such as pruning the \code{f:0} edge, where the node \code{Send} node is inserted, and the node is named after the output edge. , for example, \code{\_send\_f\_0}.

Finally, by inserting the \code{Source/Sink} node, the subgraphs of each Unicom are summarized after pruning to form a complete \ascii{DAG} map.

\begin{figure}[H]
  \centering
  \includegraphics[width=0.9\textwidth]{figures/client-prune-graph.png}
  \caption{Figure pruning: insert Send/Recv node}
  \label{fig:client-prune-graph}
\end{figure}


\subsection{Based on FunctionCallFrame}
However, there may be a performance bottleneck in the input/output exchange of data via \code{Rendezvous}. Because the \code{Tensor} to be sent needs to carry the sending device, the receiving device, and \code{TensorId}, together form a unique string identifier, and the data transmission and reception takes a long time to parse the string.

In particular, for local mode, there is an unnecessary performance penalty for exchanging data using \code{Rendezvous} within the same process. It can be replaced with a call based on the \code{FunctionCallFrame} function.

Therefore, in local mode, you can use \code{Arg/RetVal} instead of the \code{Send/Recv} node, which implements the way the function calls exchange data, replacing the original way of interacting with data based on \code{Rendezvous}. .

As shown in \refig{client-prune-graph-function-ops}. Pruning the edge of \code{Feed}, such as the pruning \code{ina:0}) edge, and inserting the node \code{Arg} here, and naming the node by the name of the input side, such as \code {\_arg\_ina\_0}.

Similarly, pruning is also performed on the edges of \code{Fetch}, such as pruning the \code{f:0} edge, where the node \code{RetVal} node is inserted, and the node is named after the output edge. , for example, \code{\_retval\_f\_0}.

Finally, by inserting the \code{Source/Sink} node, the subgraphs of each Unicom are summarized after pruning to form a complete \ascii{DAG} map.

\begin{figure}[H]
  \centering
  \includegraphics[width=0.9\textwidth]{figures/client-prune-graph-function-ops.png}
  \caption{Figure pruning: insert Arg/RetVal node}
  \label{fig:client-prune-graph-function-ops}
\end{figure}


\subsection{Pruning algorithm implementation}
The pruning algorithm is mainly done by \code{RewriteGraphForExecution}, which mainly includes \ascii{3} sub-processes.

\begin{enum}
  \eitem{Additional input node}
  \eitem{Additional output node}
  \eitem{Reverse pruning}
\end{enum}

\begin{leftbar}
\begin{c++}
void RewriteGraphForExecution(Graph* g, bool use_function, 
    const ArraySlice<string>& fed_outputs,
    const ArraySlice<string>& fetch_outputs,
    const ArraySlice<string>& target_node_names,
    const DeviceAttributes& device_info) {
  FeedInputs(g, use_function, device_info, fed_outputs);

  std::vector<Node*> fetch_nodes;
  FetchOutputs(g, use_function, device_info, 
    fetch_outputs, &fetch_nodes);

  PruneForTargets(g, fetch_nodes, target_node_names);
}
\end{c++}
\end{leftbar}


\subsubsection{Add input node}
As shown in \refig{local-prune-feed}, when pruning is performed on any input edge, insert the corresponding \code{Arg} or \code{Recv} node, delete the existing edge, and reconnect the corresponding side.

In the calculation graph, an edge is uniquely identified by \code{TensorId}, which consists of a \code{op:src\_output} binary group. The former represents the upstream node of the edge, and the latter represents the edge of the upstream node.

The sample code removes some of the less important logic and relocates some of the functions, and tries partial function extraction locally to better restore the logic of the algorithm. Among them, suppose \code{Graph} can index nodes and edges according to \code{TensorId}.

\begin{figure}[H]
  \centering
  \includegraphics[width=0.6\textwidth]{figures/local-prune-feed.png}
  \caption{Pruning: Input Edge}
  \label{fig:local-prune-feed}
\end{figure}

\begin{leftbar}
\begin{c++}
namespace {
  DataType data_type(Graph& g, const TensorId& tensor_id) {
    Node* upstream_node = g.upstream_node(tensor_id);
    return BaseType(upstream_node->output_type(tensor_id.src_output()));
  }

  Node* AppendRecvNode(Graph& g, 
    const TensorId& tensor_id, const DeviceAttributes& device_info) {
      Node * recv_node;
      NodeBuilder(strings::StrCat(
        "_recv_", tensor_id.op(), "_", tensor_id.src_output()), "_Recv")
        .Attr("tensor_type", data_type(g, tensor_id))
        .Attr("tensor_name", tensor_id.name())
        .Attr("send_device", device_info.name())
        .Attr("recv_device", device_info.name())
        .Attr("send_device_incarnation", device_info.incarnation())
        .Attr("client_terminated", true)
        .Finalize(g, &recv_node);
      return recv_node;
  }

  Node* AppendArgNode(Graph& g, size_t index, 
    const TensorId& tensor_id, const DeviceAttributes& device_info) {
    Node * arg_node;
    NodeBuilder(strings::StrCat(
      "_arg_", tensor_id.op(), "_", tensor_id.src_output()), "_Arg")
      .Attr("T", data_type(g, tensor_id))
      .Attr("index", index)
      .Finalize(g, &arg_node);
    return arg_node;
  }

  // 1. append arg/recv node
  Node* AppendNewNode(Graph& g, bool use_function, size_t index, 
    const TensorId& tensor_id,const DeviceAttributes& device_info) {
    if (use_function) {
      return AppendArgNode(g, index, tensor_id, device_info);
    } else {
      return AppendRecvNode(g, tensor_id, device_info);
    }
  }

  void AppendNewEdges(Graph& g, 
    Node* new_node, const TensorId& tensor_id) {
    // 2. add control edge between source node and new node.
    g.AddControlEdge(g.source_node(), new_node);

    Edge* old_edge = g.edge(tensor_id);
    
    // 3. add edge between new node and downstream node.
    g.AddEdge(new_node, 0, old_edge->dst(), old_edge->dst_input());
    
    // 4. remove old edge.
    g.RemoveEdge(old_edge);
  }
}

void FeedInputs(Graph& g, bool use_function,
  const DeviceAttributes& device_info,
  const ArraySlice<TensorId>& feeds) {
  for (size_t i = 0; i < feeds.size(); ++i) {
    Node* new_node = AppendNewNode(use_function, i, feeds[i]);
    AppendNewEdges(g, new_node, feeds[i]);
  }
}
\end{c++}
\end{leftbar}


\subsubsection{Add output node}
When pruning is performed on any of the output edges, insert the corresponding \code{RetVal} or \code{Send} node and connect it to the \code{Sink} node by controlling the dependent edges.

Perform pruning on the output side as shown in \refig{local-prune-fetch}. The connection relationship between the new node and the upstream node is specified by \code{Input} when constructing the new node. In addition, the function directly returns the new node (\code{RetVal/Send}) as the termination node, so there is no need to delete the original edge, and the algorithm has a subtle difference from the processing of the input edge.

\begin{figure}[H]
  \centering
  \includegraphics[width=0.6\textwidth]{figures/local-prune-feed.png}
  \caption{Pruning: Output Edge}
  \label{fig:local-prune-fetch}
\end{figure}

\begin{leftbar}
\begin{c++}
namespace {
  Node* AppendSendNode(Graph& g, 
    const TensorId& tensor_id, const DeviceAttributes& device_info) {
    Node* send_node;
    NodeBuilder(strings::StrCat(
      "_send_", tensor_id.op(), "_", id.src_output()), "_Send")
      // 2. add edge between upstream node and send node.
      .Input(g.upstream_node(tensor_id), tensor_id.src_output())
      .Attr("tensor_name", tensor_id.name())
      .Attr("send_device", device_info.name())
      .Attr("recv_device", device_info.name())
      .Attr("send_device_incarnation",
            device_info.incarnation())
      .Attr("client_terminated", true)
      .Finalize(g, &send_node);
    return send_node;
  }

  Node* AppendRetvalNode(Graph& g, size_t index, 
    const TensorId& tensor_id, const DeviceAttributes& device_info) {
    Node* retval_node;
    NodeBuilder(strings::StrCat(
      "_retval_", tensor_id.op(), "_", tensor_id.src_output(), "_", index), 
      "_Retval")
      // 2. add edge between upstream node and retval node.
      .Input(g.upstream_node(tensor_id), tensor_id.src_output())
      .Attr("T", data_type(g, tensor_id))
      .Attr("index", index)
      .Finalize(g, &retval_node))
    return retval_node;
  }

  // 1. append retval/send node
  Node* AppendNewNode(Graph& g, bool use_function, size_t index, 
    const TensorId& tensor_id,const DeviceAttributes& device_info) {
    if (use_function) {
      return AppendRetvalNode(g, index, tensor_id, device_info);
    } else {
      return AppendSendNode(g, tensor_id, device_info);
    }
  }
}

void FetchOutputs(Graph& g, bool use_function,
  const DeviceAttributes& device_info,
  const ArraySlice<TensorId>& fetches,
  std::vector<Node*>& fetch_nodes) {
  for (size_t i = 0; i < fetches.size(); ++i) {
    Node* new_node = AppendNewNode(use_function, i, fetches[i]);
    
    // 3. add control edge between new node and sink node. 
    g->AddControlEdge(new_node, g->sink_node());

    fetch_nodes.push_back(new_node);
  }
}
\end{c++}
\end{leftbar}


\subsubsection{Reverse pruning}
The pruning operation is essentially the \ascii{DAG} reverse width-first traversal algorithm. First, a queue is created with a \code{visited} array that records the nodes that have been traversed. At initialization, the queue contains only the output node and the input node (\code{targets}). When the graph is traversed, the node in \code{visited} is no longer present, indicating that the execution is not dependent on it, and the node and its associated edges should be removed from the graph.

After pruning, several \ascii{DAG} subgraphs will be formed. The node with the intrinsic degree of \code{0} is connected to the \code{Source} node by controlling the dependent edge; the node with the degree of \ascii{0} is connected to the \code{Sink} node by controlling the dependent edge. , eventually form a complete \ascii{DAG} map.

\begin{leftbar}
\begin{c++}
namespace {
  void ReverseBFS(
    Graph* g, std::unordered_set<const Node*>& visited) {
    std::deque<const Node*> queue(visited.begin(), visited.end());
    while (!queue.empty()) {
      const Node* n = queue.front();
      queue.pop_front();
      for (const Node* in : n->in_nodes()) {
        if (visited.insert(in).second) {
          queue.push_back(in);
        }
      }
    }
  }

  void RemoveUnvisitedNodes(
    Graph* g, std::unordered_set<const Node*>& visited) {
    for (Node* n : g->nodes()) {
      if (visited.count(n) == 0 && !n->IsSource() && !n->IsSink()) {
        g->RemoveNode(n);
      }
    }
  }

  void PruneForReverseReachability(
    Graph* g, std::unordered_set<const Node*>& visited) {
    ReverseBFS(g, visited);
    RemoveUnvisitedNodes(g, visited);
  }

  void FixupSourceEdges(Graph* g, Node* n) {
    if (!n->IsSource() && n->in_edges().empty()) {
      g->AddControlEdge(g->source_node(), n);
    }  
  }

  void FixupSinkEdges(Graph* g, Node* n) {
    if (!n->IsSink() && n->out_edges().empty()) {
      g->AddControlEdge(n, g->sink_node());
    }  
  }

  void FixupSourceAndSinkEdges(Graph* g) {
    for (Node* n : g->nodes()) {
      FixupSourceEdges(g, n);
      FixupSinkEdges(g, n);
    }
  }

  void AppendTargetNodes(Graph& g, 
    const ArraySlice<string>& target_names,
    std::unordered_set<const Node*>& targets) {
    for (auto name : target_names) {
      Node* target = g.GetNodeBy(name);
      targets.insert(target);
    }
  }  
}

void PruneForTargets(Graph* g, 
  std::vector<Node*>& fetch_nodes,
  const ArraySlice<string>& target_names) {
  std::unordered_set<const Node*> targets(
    begin(fetch_nodes), end(fetch_nodes));

  AppendTargetNodes(g, target_names, targets);
  PruneForReverseReachability(g, targets);
  FixupSourceAndSinkEdges(g);
}
\end{c++}
\end{leftbar}


\section{Split}
\label{sec:graph-operation-split}
As shown in \refig{local-graph-split-by-device}, if the \code{d} node is placed on \ascii{GPU0}, the other nodes are placed on \ascii{CPU0}. Among them, node \code{a} and \code{b} input data through \code{Arg}; node \code{f} outputs its result to \code{RetVal} node.

\begin{figure}[H]
  \centering
  \includegraphics[width=0.9\textwidth]{figures/local-graph-split-by-device.png}
  \caption{Execute map split by local device set}
  \label{fig:local-graph-split-by-device}
\end{figure}

Therefore, there are several edges across the device in the calculation graph. For the edge across the device, the runtime splits it and inserts it into the \code{Send/Recv} side, which is used to send data on the original device and accept data on the target device to complete data exchange between devices. As shown in \refig{local-graph-split-insert-send-recv}.

The \code{Arg/RetVal} node exchanges data via the medium \code{FunctionCallFrame}; the \code{Send/Recv} node exchanges data via the medium \code{Rendezvous}.

\begin{figure}[H]
  \centering
  \includegraphics[width=1.0\textwidth]{figures/local-graph-split-insert-send-recv.png}
  \caption{Insert a Send/Recv node between device OPs}
  \label{fig:local-graph-split-insert-send-recv}
\end{figure}


\subsection{Case 1}
In the simplest case, \code{src} is in the same \code{Partition} as \code{dst}. Therefore, you can directly assign it to the same \code{Partition}.

\begin{figure}[H]
  \centering
  \includegraphics[width=0.6\textwidth]{figures/split-graph-1.png}
  \caption{Case 1: src and dst are in the same Partition}
  \label{fig:split-graph-1}
\end{figure}


\subsection{Case 2}
If \code{src} is not in the same \code{Partition} as \code{dst}, but the two are originally connected by ordinary edges. Therefore, you only need to add the \code{Send} and \code{Recv} nodes to them in two different \code{Partition}.

\begin{figure}[H]
  \centering
  \includegraphics[width=0.7\textwidth]{figures/split-graph-2.png}
  \caption{Case 2: src and dst are not in the same Partition, but the normal edge between the two}
  \label{fig:split-graph-2}
\end{figure}


\subsection{Case 3}
If \code{src} is not in the same \code{Partition} as \code{dst}, but the two are originally connected by controlling the dependent edges.

At this point, you need to add a \code{DummyNode} of \code{Const} on the \code{src} side, and connect it as a downstream of \code{src} by controlling the dependency edges; finally, by \code{Send} Its value is sent to the peer.

On the \code{dst} side, \code{Recv} receives the value and consumes it using \code{Identity}; finally, the \code{Identity} and \code{dst} are connected to control the dependent edge.

Here, \code{Const} plays the producer and \code{Identity} plays the consumer role. It satisfies the requirements of communication between devices and the control dependency between \code{src} and \code{dst}. However, the downside is that there are subtle performance overheads.

\begin{figure}[H]
  \centering
  \includegraphics[width=0.8\textwidth]{figures/split-graph-3.png}
  \caption{Case 3: src and dst are not in the same Partition, but between them is the control dependent edge}
  \label{fig:split-graph-3}
\end{figure}


\subsection{Split algorithm implementation}
The splitting algorithm is also an algorithm that reverses the traversal map. For the currently traversed node, mark it as \code{dst}; then look for all input edges of \code{dst}; iterate over all the input edges to find the source node that is connected to the changed edge, marking it as \code {src}.

Therefore, in the three cases discussed above, iteratively implements the \code{Partition} partitioning algorithm before \code{src} and \code{dst}.

\begin{leftbar}
\begin{c++}
namespace {
  
  using Edges = std::vector<const Edge*>;
  using Partitions = std::unordered_map<string, GraphDef>;

  void AddInput(NodeDef* dst, StringPiece src_name, int src_slot) {
    if (src_slot == Graph::kControlSlot) {
      dst->add_input(strings::StrCat("^", src_name));
    } else if (src_slot == 0) {
      dst->add_input(src_name.data(), src_name.size());
    } else {
      dst->add_input(strings::StrCat(src_name, ":", src_slot));
    }
  }

  Edges InputsOf(const Node* dst) {
    Edges inputs(dst->num_inputs(), nullptr);
    for (auto edge : dst.in_edges()) {
      if (edge->IsControlEdge()) {
        inputs.push_back(e);
      } else {
        inputs[edge->dst_input()] = edge;
      }
    }
    return inputs;
  }

  NodeDef* InitDstNodeDef(const Node& dst, NodeDef* dst_def) {
    dst_def = dst.def();
    dst_def->set_device(dst.assigned_device_name());
    dst_def->clear_input();
    return dst_def;  
  }

  NodeDef* AddDummyConst(const PartitionOptions& opts, GraphDef* gdef,
                         const Edge* edge, Status* status) {
    const Node* src = edge->src();
    Tensor tensor(DT_FLOAT, TensorShape({0}));
    NodeDef* result = gdef->add_node();
    *status = NodeDefBuilder(opts.new_name(src->name()), "Const")
                  .Device(src->assigned_device_name())
                  .Attr("dtype", DT_FLOAT)
                  .Attr("value", tensor)
                  .Finalize(result);
    return result;
  }

  NodeDefBuilder::NodeOut BuildSendFrom(
      const PartitionOptions& opts,
      GraphDef* src_graph,
      const Edge* edge,
      NodeDefBuilder::NodeOut& send_from) {
    if (edge->IsControlEdge()) {
      // Case 3: DummyNode(Const) -ctrl-> src -> send  
      NodeDef* dummy = AddDummyConst(opts, src_graph, edge);
      AddInput(dummy, edge->src()->name(), Graph::kControlSlot);
      send_from.Reset(dummy->name(), 0, DT_FLOAT);
    } else {
      // Case 2: src -> send  
      send_from.Reset(edge->src()->name(),
                      edge->src_output(), 
                      EdgeType(edge));
    }
  }

  void SetSendRecvAttrs (
      const PartitionOptions& opts, 
      const Edge* edge,
      NodeDefBuilder* builder) {
    builder->Attr("tensor_name",
                  strings::StrCat("edge_", edge->id(), "_", edge->src()->name()));
    builder->Attr("send_device", edge->src()->assigned_device_name());
    builder->Attr("send_device_incarnation",
                  static_cast<int64>(
                      opts.get_incarnation(edge->src()->assigned_device_name())));
    builder->Attr("recv_device", edge->dst()->assigned_device_name());
    builder->Attr("client_terminated", false);
  }

  NodeDef* AddSend(
      const PartitionOptions& opts, 
      GraphDef* gdef, 
      const Edge* edge,
      NodeDefBuilder::NodeOut send_from) {
    NodeDef* send = gdef->add_node();
    NodeDefBuilder builder(opts.new_name(edge->src()->name()), "_Send");
    SetSendRecvAttrs (opts, edge, & builder);
    builder.Device(edge->src()->assigned_device_name())
           .Input(send_from)
           .Finalize(send);
    return send;
  }

  NodeDef* AddRecv(const PartitionOptions& opts, const GraphInfo& g_info,
                   GraphDef* gdef, const Edge* edge, NodeDef** real_recv,
                   Status* status) {
    NodeDef* recv = gdef->add_node();
    NodeDefBuilder builder(opts.new_name(src->name()), "_Recv");
    SetSendRecvAttrs (opts, edge, & builder);
    builder.Device(dst->assigned_device_name())
           .Attr("tensor_type", EdgeType(edge))
           .Finalize(recv);
    return recv;

    if (edge->IsControlEdge()) {
      // Case 3: Recv -> Identity -contrl-> dst
      NodeDef* id = gdef->add_node();
      NodeDefBuilder(opts.new_name(src->name()), "Identity")
          .Device(dst->assigned_device_name())
          .Input(recv->name(), 0, cast_dtype)
          .Finalize(id);
      return id;
    } else {
      return recv;
    }
  }

  void InsertSendRecv(
      const PartitionOptions& opts,
      GraphDef* src_graph, 
      Edge* edge, 
      GraphDef* dst_graph, 
      NodeDef * dst_def) {
    NodeDefBuilder::NodeOut send_from;
    BuildSendFrom(opts, src_graph, edge, send_from);

    NodeDef* send = AddSend(opts, src_graph, edge, send_from);
    NodeDef* recv = AddRecv(opts, dst_graph, edge);

    if (edge->IsControlEdge()) {
      // Case 3: In fact, recv is identity.
      AddInput(dst_def, recv->name(), Graph::kControlSlot);
    } else {
      AddInput(dst_def, recv->name(), 0);
    }
  }
}

Status Partition(const PartitionOptions& opts, 
                 Partitions& partitions, Graph& client_graph) {
  for (const Node* dst : client_graph.op_nodes()) {
    // 1. find dst node
    GraphDef* dst_graph = &partitions[opts.node_to_loc(dst)];
    NodeDef* dst_def = InitDstNodeDef(*dst, dst_graph->add_node());
    
    // 2. search all input edges.
    for (const Edge* edge : InputsOf(dst)) {
      // 3. find src node: edge->src()
      GraphDef* src_graph = &partitions[opts.node_to_loc(src)];

      // skip sink/source nodes.
      if (!edge->src()->IsOp()) 
        continue;  

      // Case 1: same partition
      if (src_graph == dst_graph) {
        AddInput(dst_def, src->name(), edge->src_output());
        continue;
      }

      // Case 2-3: different partition
      InsertSendRecv(opts, src_graph, edge, dst_graph, dst_def);
    }
  }
}
\end{c++}
\end{leftbar}


\subsection{Callback function}
There are two important callback functions in \code{PartitionOptions}. \code{NodeToLocFunc} is used for graph splitting; \code{NewNameFunc} is used to name newly added nodes, such as \code{Send/Recv}.

\begin{leftbar}
\begin{c++}
struct PartitionOptions {
  typedef std::function<string(const Node*)> NodeToLocFunc;
  NodeToLocFunc node_to_loc = nullptr;

  typedef std::function<string(const string&)> NewNameFunc;
  NewNameFunc new_name = nullptr;

  // ignore others...
};
\end{c++}
\end{leftbar}

For graph splitting, there are two basic methods of splitting.

\begin{leftbar}
\begin{c++}
string SplitByDevice(const Node* node) {
  return node->assigned_device_name();
}

string SplitByWorker(const Node* node) {
  string task, device;
  DeviceNameUtils::SplitDeviceName(
      node->assigned_device_name(), &task, &device);
  return task;
}
\end{c++}
\end{leftbar}

In local mode, \code{NodeToLocFunc} is configured as \code{SplitByDevice}. As shown in \code{intraprocess-splity-by-device}.

\begin{figure}[H]
  \centering
  \includegraphics[width=0.7\textwidth]{figures/intraprocess-splity-by-device.png}
  \caption{Local Mode: SplitByDevice}
  \label{fig:intraprocess-splity-by-device}
\end{figure}


In distributed mode, \code{NodeToLocFunc} of \code{Master} is configured as \code{SplitByWorker}; and \code{Worker}
\code{NodeToLocFunc} is configured as \code{SplitByDevice}.

Therefore, in distributed mode, the graph split undergoes two levels of separation. The first level is split according to \code{SplitByWorker}, and the picture is divided into various \code{Worker}; the second level is based on \code{SplitByDevice}, and then the picture is divided into various computing devices.

\begin{figure}[H]
  \centering
  \includegraphics[width=0.7\textwidth]{figures/interprocess-splity-by-worker.png}
  \caption{Distributed mode: two-level split}
  \label{fig:interprocess-splity-by-worker}
\end{figure}


\section{Execution}
\label{sec:graph-operation-exec}
Next, the runtime will execute each \code{PartitionGraph} concurrently. As shown in \refig{local-graph-execution}, each \code{PartitionGraph} starts a \code{Executor} to implement the calculation of the concurrent execution graph.

Each \code{Executor} will execute the topological sorting algorithm of \code{PartitionGraph}, append \ascii{OP} with degree \ascii{0} to \code{ready\_queue} and associate it with The degree of \ascii{OP} is reduced by \ascii{1}. Scheduler dispatch \code{ready\_queue}\ascii{OP
}, and put it into \code{ThreadPool} to execute the corresponding \ascii{Kernel} implementation.

Before all \code{Partition} starts concurrent execution, you need to pass its input to the corresponding \code{Arg} node externally; when all \code{Partition} finishes the calculation, the external is taken from the \code{RetVal} node. Take the data. Among them, the data between the \code{Arg/RetVal} nodes is interactive through \code{FunctionCallFrame}.

If \code{PartitionGraph} needs to exchange data across devices, the producer places it in the \code{Send} node, and the consumer gets the data through the \code{Recv} node. The sender does not block; if the receiver does not arrive, the blocking occurs until timeout. In addition, the data between the \code{Send/Recv} nodes is interactive through \code{Rendezvous}.

\begin{figure}[H]
  \centering
  \includegraphics[width=1.0\textwidth]{figures/local-graph-execution.png}
  \caption{Execution graph}
  \label{fig:local-graph-execution}
\end{figure}

Therefore, the implementation of graph calculation needs to solve the following \ascii{3} core issues:

\begin{enum}
  \eitem{Input/output processing}
  \eitem{Inter-device data exchange} 
  \eitem{Execute \code{PartitionGraph}}
\end{enum}


\subsection{Input}
On a device, the starting node of \code{PartitionGraph} is the \code{Arg} node and the ending node is the \code{RetVal} node. The whole process can be seen as a function call procedure, \code{Arg} is used to pass function arguments, and \code{RetVal} is used to return function values.

More precisely, \code{Arg} completes the input of \code{PartitionGraph}, and \code{RetVal} completes
The output of \code{PartitionGraph}. For the \code{Arg} node, the calling sequence is: \code{set\_arg -> get\_arg}. Among them, the former is passed by \code{DirectSession} before calling the \code{Executor} list, passing the value of the input parameter list by calling \code{FunctionCallFrame.SetArgs(feeds)}; the latter is \ascii by \code{Arg} {Kernel} implementation call.

\begin{leftbar}
\begin{c++}
Status DirectSession::Run(
  const RunOptions& run_options,
  const NamedTensorList& inputs,
  const std::vector<string>& output_names,
  const std::vector<string>& target_nodes,
  std::vector<Tensor>* outputs,
  RunMetadata* run_metadata) {

  // 1. prune graph
  // client\_graph = prune(full\_graph, inputs, outputs)
   
  // 2. split graph into partition by devices 
  // executors\_and\_partitions = split(client\_graph, devices)
  ExecutorsAndKeys* executors_and_keys = ... // ignore implements...
  
  // 3. lauch executor per partition
  // def run\_partitions(executors\_and\_partitions, inputs, outputs):
  // \ \ frame = FunctionCallFrame()
  // \ \ frame.set\_args(inputs)
  // \ \ for (executor, partition) in executors\_and\_partitions: 
  // \ \ \ \ exec.run(part)
  // \ \ frame.get\_ret\_vals(outputs)

  // 3.1 construct FunctionCallFrame
  FunctionCallFrame call_frame(
    executors_and_keys->input_types,
    executors_and_keys->output_types);
  
  // 3.2 frame.set\_args(inputs)
  // 3.2.1 construct feeds list
  gtl::InlinedVector<Tensor, 4> feed_args(inputs.size());
  for (const auto& it : inputs) {
    // (first, second) => (tensor\_name, tensor)
    feed_args[executors_and_keys->input_name_to_index[it.first]] = it.second;
  }

  // 3.2.2 frame.set\_args(feeds)
  call_frame.SetArgs(feed_args);
  
  // 3.3 execution competitor
  // for (executor, partition) in executors\_and\_partitions:
  // \ \ executor.run(partition) 

  // 3.4 fetch outputs.
}
\end{c++}
\end{leftbar}

\code{frame.get\_arg} has \code{Arg} to get it, and \code{Arg} outputs it to the first compute node in \code{PartitionGraph}.

\begin{leftbar}
\begin{c++}
struct ArgOp : OpKernel {
  explicit ArgOp(OpKernelConstruction* ctx) : OpKernel(ctx) {
    ctx->GetAttr("T", &dtype_);
    ctx->GetAttr("index", &index_);
  }

  void Compute(OpKernelContext* ctx) override {
    auto frame = ctx->call_frame();

    Tensor val;
    frame->GetArg(index_, &val);

    // put it into downsteram op's input.
    ctx->set_output(0, val); 
  }

 private:
  int index_;
  DataType dtype_;
};
\end{c++}
\end{leftbar}


\subsection{Concurrent execution}
After the graph is split, the run starts a \code{Executor} for each \code{Partition}. In order to monitor whether all \code{Executor} is completed, a \code{ExecutorBarrier} is created. And after starting all \code{Executor}, call \code{executors\_done.Wait()} to block and wait for all \code{Executor} to complete execution.

When done in a \code{Executor}, the \code{ExecutorBarrier} calculator minus \ascii{1} (initial value is \code{num\_executors}) until it is \ascii{0}, it will be called to complete The hook finally triggers \code{executors\_done.Notify()}.

\begin{leftbar}
\begin{c++}
Status DirectSession::Run(
  const RunOptions& run_options,
  const NamedTensorList& inputs,
  const std::vector<string>& output_names,
  const std::vector<string>& target_nodes,
  std::vector<Tensor>* outputs,
  RunMetadata* run_metadata) {

  // 1. prune graph
  // client\_graph = prune(full\_graph, inputs, outputs)
   
  // 2. split graph into partition by devices 
  // executors\_and\_partitions = split(client\_graph, devices)
  ExecutorsAndKeys* executors_and_keys = ... // ignore implements...
  
  // 3. lauch executor per partition
  // def run\_partitions(executors\_and\_partitions, inputs, outputs):
  // \ \ frame = FunctionCallFrame()
  // \ \ frame.set\_args(inputs)
  // \ \ for (executor, partition) in executors\_and\_partitions: 
  // \ \ \ \ exec.run(part)
  // \ \ frame.get\_ret\_vals(outputs)

  // 3.1 construct FunctionCallFrame
  FunctionCallFrame call_frame(
    executors_and_keys->input_types,
    executors_and_keys->output_types);
  
  // 3.2 frame.set\_args(inputs)
  // 3.2.1 construct feeds list
  gtl::InlinedVector<Tensor, 4> feed_args(inputs.size());
  for (const auto& it : inputs) {
    // (first, second) => (tensor\_name, tensor)
    feed_args[executors_and_keys->input_name_to_index[it.first]] = it.second;
  }

  // 3.2.2 frame.set\_args(feeds)
  call_frame.SetArgs(feed_args);
  
  // 3.3 execution competitor
  // barrier = ExecutorBarrier(executors\_and\_partitions.size())
  // for (executor, partition) in executors\_and\_partitions:
  // \ \ executor.run(partition) 
  // barrier.wait()
  RunState run_state(&devices_);
  run_state.rendez = new IntraProcessRendezvous(device_mgr_.get());
  
  // 3.3.1 notify when finished.
  size_t num_executors = executors_and_keys->items.size();
  ExecutorBarrier* barrier = new ExecutorBarrier(
      num_executors, run_state.rendez, [&run_state](const Status& ret) {
        {
          mutex_lock l(run_state.mu_);
          run_state.status.Update(ret);
        }
        run_state.executors_done.Notify();
      });

  Executor::Args args;
  args.call_frame = &call_frame;
  args.rendezvous = run_state.rendez;
  args.runner = [this, pool](Executor::Args::Closure c) {
    SchedClosure(pool, std::move(c));
  };

  // 3.3.2 lauch all executors.
  for (const auto& item : executors_and_keys->items) {
    item.executor->RunAsync(args, barrier->Get());
  }

  // 3.3.3 wait until all executors finished.
  WaitForNotification(&run_state, 
      &step_cancellation_manager,
      GetTimeoutInMs(run_options));

  // 3.4 fetch outputs.
}
\end{c++}
\end{leftbar}


\subsection{Output}
Similarly, for the \code{RetVal} node, the calling sequence is: \code{set\_ret\_val -> get\_ret\_val}. The former is done by \code{RetVal} and the latter by \code{DirectSession}.

\begin{leftbar}
\begin{c++}
struct RetvalOp : OpKernel {
  explicit RetvalOp(OpKernelConstruction* ctx) : OpKernel(ctx) {
    ctx->GetAttr("T", &dtype_);
    ctx->GetAttr("index", &index_);
  }

  void Compute(OpKernelContext* ctx) override {
    // get upstream op's output.
    const Tensor& val = ctx->input(0); 

    auto frame = ctx->call_frame();
    frame->SetRetval(index_, val);
  }

 private:
  int index_;
  DataType dtype_;
};
\end{c++}
\end{leftbar}

After all \code{Executor} has finished running, \code{DirectSession} can take all output values ​​from \code{FunctionCallFrame} and place them in \code{outputs} and return \ascii{Client}.

\begin{leftbar}
\begin{c++}
Status DirectSession::Run(
  const RunOptions& run_options,
  const NamedTensorList& inputs,
  const std::vector<string>& output_names,
  const std::vector<string>& target_nodes,
  std::vector<Tensor>* outputs,
  RunMetadata* run_metadata) {
  
  // 1. prune graph
  // client\_graph = prune(full\_graph, inputs, outputs)
   
  // 2. split graph into partition by devices 
  // executors\_and\_partitions = split(client\_graph, devices)
  executors_and_keys = ... // ignore implements...
  
  // 3. lauch executor per partition
  // def run\_partitions(executors\_and\_partitions, inputs, outputs):
  // \ \ frame = FunctionCallFrame()
  // \ \ frame.set\_args(inputs)
  // \ \ for (executor, partition) in executors\_and\_partitions: 
  // \ \ \ \ exec.run(part)
  // \ \ frame.get\_ret\_vals(outputs)

  // 3.1 construct FunctionCallFrame
  FunctionCallFrame call_frame(
    executors_and_keys->input_types,
    executors_and_keys->output_types);
  
  // 3.2 frame.set\_args(inputs)
  // 3.2.1 construct feeds list
  gtl::InlinedVector<Tensor, 4> feed_args(inputs.size());
  for (const auto& it : inputs) {
    // (first, second) => (tensor\_name, tensor)
    feed_args[executors_and_keys->input_name_to_index[it.first]] = it.second;
  }

  // 3.2.2 frame.set\_args(feeds)
  call_frame.SetArgs(feed_args);
  
  // 3.3 execution competitor
  // barrier = ExecutorBarrier(executors\_and\_partitions.size())
  // for (executor, partition) in executors\_and\_partitions:
  // \ \ executor.run(partition) 
  // barrier.wait()
  RunState run_state(&devices_);
  run_state.rendez = new IntraProcessRendezvous(device_mgr_.get());
  
  // 3.3.1 notify when finished.
  size_t num_executors = executors_and_keys->items.size();
  ExecutorBarrier* barrier = new ExecutorBarrier(
      num_executors, run_state.rendez, [&run_state](const Status& ret) {
        {
          mutex_lock l(run_state.mu_);
          run_state.status.Update(ret);
        }
        run_state.executors_done.Notify();
      });

  Executor::Args args;
  args.call_frame = &call_frame;
  args.rendezvous = run_state.rendez;
  args.runner = [this, pool](Executor::Args::Closure c) {
    SchedClosure(pool, std::move(c));
  };

  // 3.3.2 lauch all executors.
  for (const auto& item : executors_and_keys->items) {
    item.executor->RunAsync(args, barrier->Get());
  }

  // 3.3.3 wait until all executors finished.
  WaitForNotification(&run_state, 
      &step_cancellation_manager,
      GetTimeoutInMs(run_options)); 

  // 3.4 fetch outputs. 
  // 3.4.1 frame.get\_get\_ret\_vals
  std::vector<Tensor> sorted_outputs;
  Status s = call_frame.ConsumeRetvals(&sorted_outputs);

  // 3.4.2 emplace to outputs, and return to client.
  outputs->reserve(sorted_outputs.size());
  for (int i = 0; i < output_names.size(); ++i) {
    const string& output_name = output_names[i];
    outputs->emplace_back(
      std::move(sorted_outputs[
        executors_and_keys->output_name_to_index[output_name]]));
  }
}
\end{c++}
\end{leftbar}

At this point, the entire \code{DirectSession.Run} has been interpreted. However, how does the \code{Send/Recv} between \code{Partition} work in the \code{Partition} how the nodes are scheduled to execute?

Therefore, in the last mile, we must explore three things.

\begin{enum}
  \eitem{How \code{SendOp} works with \code{RecvOp}}
  \eitem{How \code{IntraProcessRendezvous} works}
  \eitem{An \code{Executor} scheduling algorithm}
\end{enum}


\section{Inter-device communication}
\code{SendOp/RecvOp} exchanges data via \code{Rendezvous}; it implements message delivery/acceptance and decouples from specific messaging. For example, in a single process, \code{SendOp/RecvOp} passes data based on \code{IntraProcessRendezvous}; in a multi-process environment, \code{SendOp/RecvOp} can pass data based on \code{GrpcRendezvous}.

First, explore the workings of these two \ascii{OP}; then, explore the working principle of \code{IntraProcessRendezvous} in local mode.


\subsection{SendOp implementation}
As shown in \refig{local-send-recv-ops}, the in-process \code{Send/Recv} exchanges data with a unique identifier \code{ParsedKey}.

\begin{figure}[H]
  \centering
  \includegraphics[width=0.8\textwidth]{figures/local-send-recv-ops.png}
  \caption{Data exchange between \code{SendOp} and \code{RecvOp}}
  \label{fig:local-send-recv-ops}
\end{figure}

Referring to the \ascii{Kernel} implementation of \code{SendOp}, it looks very complicated, but it actually does one thing. First, it constructs the keyword \code{ParsedKey} for communication between devices, then calls the \code{Rendezvous.Send} operation to send the upstream \ascii{OP} to the \code{Tensor} of \code{SendOp} to \ In the code{Rendezvous} cache, this operation is non-blocking.

Among them, \code{ParsedKey} includes: the sending device, the receiving device, the device global identifier, and the identifier of the \code{Tensor} to be sent (\code{src:output\_index}).

\begin{leftbar}
\begin{c++}
struct SendOp : OpKernel {
  explicit SendOp(OpKernelConstruction* ctx) : OpKernel(ctx) {
    string send_device;
    ctx->GetAttr("send_device", &send_device);

    string recv_device;
    ctx->GetAttr("recv_device", &recv_device);

    uint64 send_device_incarnation;
    ctx->GetAttr(
        "send_device_incarnation",
        reinterpret_cast<int64*>(&send_device_incarnation));

    string tensor_name;
    ctx->GetAttr("tensor_name", &tensor_name);

    key_prefix_ = GetRendezvousKeyPrefix (
        send_device, recv_device,
        send_device_incarnation, tensor_name);

    GetRendezvousKey(key_prefix_, {0, 0}, &parsed_key_.buf_);
    Rendezvous::ParseKey(parsed_key_.buf_, &parsed_key_);

    if (!ctx->GetAttr("_hostmem_sendrecv", &hostmem_sendrecv_).ok()) {
      hostmem_sendrecv_ = false;
    }
  }

  void Compute(OpKernelContext* ctx) override {
    Rendezvous::Args args;
    args.device_context = ctx->op_device_context();
    args.alloc_attrs = ctx->input_alloc_attr(0);
    
    // get it from upstream op's output, and as this op's input.
    ctx->rendezvous()->Send(
        CreateParsedkey(ctx), args, ctx->input(0),
        ctx->is_input_dead());
  }
 
 private:
  Rendezvous::ParsedKey CreateParsedkey(OpKernelContext* ctx) {
    FrameAndIter frame_iter = GetFrameAndIter (ctx, hostmem_sendrecv_);
    if (frame_iter == FrameAndIter(0, 0)) {
      return parsed_key_;
    } else {
      Rendezvous::ParsedKey in_loop_parsed;
      GetRendezvousKey(key_prefix_, frame_iter, &in_loop_parsed.buf_);
      Rendezvous::ParseKey(in_loop_parsed.buf_, &in_loop_parsed);
      return in_loop_parsed;
    }  
  }

 private:
  string key_prefix_;
  Rendezvous::ParsedKey parsed_key_;
  bool hostmem_sendrecv_;
};
\end{c++}
\end{leftbar}


\subsection{RecvOp implementation}
Similarly, you can guess the implementation of \ascii{Kernel} for \code{Recv}. It first constructs the \code{ParsedKey} of \code{Rendezvous}, then calls the \code{Rendezvous.RecvAsync} operation and fetches the corresponding \code{Tensor} from \code{Rendezvous}.

This is an asynchronous operation. When the data in \code{Rendezvous} is available, it starts executing the callback function \code{done\_cb}, which outputs the resulting \code{Tensor} to the downstream \ascii{OP}.

\begin{leftbar}
\begin{c++}
struct RecvOp : AsyncOpKernel {
  explicit RecvOp(OpKernelConstruction* ctx) : AsyncOpKernel(ctx) {
    string send_device;
    ctx->GetAttr("send_device", &send_device);
  
    string recv_device;
    ctx->GetAttr("recv_device", &recv_device);

    uint64 send_device_incarnation;
    ctx->GetAttr(
        "send_device_incarnation",
        reinterpret_cast<int64*>(&send_device_incarnation));
  
    string tensor_name;
    ctx->GetAttr("tensor_name", &tensor_name);

    key_prefix_ = GetRendezvousKeyPrefix (
        send_device, recv_device,
        send_device_incarnation, tensor_name);
  
    GetRendezvousKey(key_prefix_, {0, 0}, &parsed_key_.buf_);
    Rendezvous::ParseKey(parsed_key_.buf_, &parsed_key_));
    if (!ctx->GetAttr("_hostmem_sendrecv", &hostmem_sendrecv_).ok()) {
      hostmem_sendrecv_ = false;
    }
  }

  void ComputeAsync(OpKernelContext* ctx, DoneCallback done) override {
    Rendezvous::Args args;
    args.device_context = ctx->op_device_context();
    args.alloc_attrs = ctx->output_alloc_attr(0);

    ctx->rendezvous()->RecvAsync(
      CreateParsedKey(ctx), args, CreateDoneCallback(ctx));
  }

 private:
  Rendezvous::ParsedKey CreateParsedKey(OpKernelContext* ctx) {
    FrameAndIter frame_iter = GetFrameAndIter (ctx, hostmem_sendrecv_);
    if (frame_iter == FrameAndIter(0, 0)) {
      return parsed_key_;
    } else {
      Rendezvous::ParsedKey in_loop_parsed;
      GetRendezvousKey(key_prefix_, frame_iter, &in_loop_parsed.buf_);
      Rendezvous::ParseKey(in_loop_parsed.buf_, &in_loop_parsed);
      return in_loop_parsed;
    }  
  }

  Rendezvous::DoneCallback CreateDoneCallback(OpKernelContext* ctx) {
    using namespace std::placeholders;
    return std::bind([ctx](DoneCallback done, const Status& s, 
        const Rendezvous::Args&, const Rendezvous::Args&, 
        const Tensor& val, bool is_dead) {
          ctx->SetStatus(s);
          if (s.ok()) {
            if (!is_dead) {
              // put it into downstream op's input.
              ctx->set_output(0, val);  
            }
            *ctx->is_output_dead() = is_dead;
          }
          done();
        },
        std::move(done), _1, _2, _3, _4, _5);  
  }

 private:
  string key_prefix_;
  Rendezvous::ParsedKey parsed_key_;
  bool hostmem_sendrecv_;
};
\end{c++}
\end{leftbar}

\begin{savequote}[45mm]
\ascii{Any fool can write code that a computer can understand. Good programmers write code that humans can understand.}
\qauthor{\ascii{- Martin Flower}}
\end{savequote}

\chapter{Distributed TensorFlow} 
\label{ch:distributed}

\begin{content}

\tf{} can be run in a distributed environment to complete the execution of the calculation graph. This chapter will focus on the basic architecture and operation mechanism of distributed runtime; focus on the interaction between various service processes; and deeply analyze the key operations in graph operations in distributed environments and their session lifecycle control;

\end{content}

\section{Distributed mode}

\begin{content}

In distributed mode, \ascii{Client} is responsible for computing the construction of the graph, and then starts the execution of the graph by calling \code{Session.run}.

After the \ascii{Master} process receives the message of the calculation graph execution, it starts the pruning, splitting, optimization, etc. of the calculation graph; finally, the subgraph distribution is registered to each \ascii{Worker} process, and then each \ascii{ The Worker} process executes the subgraph concurrently.

After the \ascii{Worker} process receives the message of the sub-picture registration, according to the local computing device resources, the calculation sub-picture is subjected to secondary splitting, the sub-picture is allocated to each computing device, and finally each computing device is started concurrently. Figure; if there is data exchange between \ascii{Worker}, the interaction can be done through interprocess communication.

\begin{figure}[H]
\centering
\includegraphics[width=0.8\textwidth]{figures/distributed.png}
\caption{distributed mode}
 \label{fig:distributed}
\end{figure}

\subsection{Figure operation}

As shown in \refig{dist-runtime}, in the execution of \code{run\_step}, it involves the pruning, splitting, and execution of three important graph operations. Among them, in the distributed operation, the graph split undergoes a two-stage splitting process.

\begin{enum}
  \eitem{level split: done by \code{MasterSession}, complete the graph split process according to \code{SplitByWorker} or \code{SplitByTask};
  \eitem{Secondary split: completed by \code{WorkerSession}, complete the graph splitting process according to \code{SplitByDevice}. }
\end{enum}

In the distributed mode, the graph pruning also reflects the design concept of the \tf{} part of the implementation; and the splitting and execution of the graph also reflects the design concept of \tf{} concurrent execution. Among them, the graph pruning only occurs on \ascii{Master}, does not occur on \ascii{Worker}; and the graph split occurs on \ascii{Master} and \ascii{Worker}; graph execution only occurs in \ascii On {Worker}, it does not happen on \ascii{Master}.

\begin{figure}[H]
\centering
\includegraphics[width=1.0\textwidth]{figures/dist-runtime.png}
\caption{Distributed: Figure Operation}
 \label{fig:dist-runtime}
\end{figure}

\subsubsection{Figure split}

To better understand how the distributed runtime works, a simple example illustrates the specific process of the graph operation. As shown in \refig{dist-exp-1}, if there is a simple calculation graph, and \code{f, c, a} is deployed on \code{/job:ps/task:0}, and Arranged on \code{CPU0, CPU1, CPU2}; \code{g, h} is deployed on \code{/job:worker/task:0} and is also arranged on \code{GPU0}; \code {b, d, e} is deployed on \code{/job:worker/task:1}, and \code{d, e} is arranged on \code{GPU0}, and \code{b} is arranged to On \code{GPU1}.

\begin{figure}[H]
\centering
\includegraphics[width=1.0\textwidth]{figures/dist-exp-1.png}
\caption{distributed: split graph}
 \label{fig:dist-exp-1}
\end{figure}

\subsubsection{Data Exchange}

As shown in \refig{dist-exp-2}, for the edge across devices, the runtime automatically splits the edges, inserting two end nodes, \code{Send} and \code{Recv}, on the sender and receiver respectively. .

The \code{Send} and \code{Recv} nodes between processes exchange data through \code{GrpcRemoteRendezvous}. For example, \code{/job:ps/task:0} and \code{/job:worker/task:0},\code{/job:ps/task:0} and \code{/job:worker/task :1}, or \code{/job:worker/task:0} and \code{/job:worker/task:1} are exchanged via \code{GrpcRemoteRendezvous}.

The \code{Send} and \code{Recv} nodes in the process exchange data through \code{IntraProcessRendezvous}. For example, there are two \code{GPU}s in \code{/job:worker/task:1}, which use \code{IntraProcessRendezvous} for data exchange. The specific implementation process of \code{Rendezvous} will be highlighted below.

\begin{figure}[H]
\centering
\includegraphics[width=0.8\textwidth]{figures/dist-exp-2.png}
\caption{Distributed: Data Exchange}
 \label{fig:dist-exp-2}
\end{figure}

\subsection{formalization}

In real system implementations, the distributed runtime is implemented using \ascii{C++}. The key path to the \tf{} runtime is \code{run\_step}. Because the real system implementation involves too much detail, it is not easy to find the backbone and logic of the algorithm. To simplify the description of the problem, the implementation of \code{run\_step} will be described formally.

\ Subsubsection {Master :: RunStep}

On \ascii{Master}, the pruning operation of \code{FullGraph} is mainly completed, and \code{ClientGraph} is generated; then, \code{ClientGraph} is split into multiple \code{PartitionGraph} according to \ascii{Worker} Finally, register the \code{PartitionGraph} list with each \ascii{Worker} and launch each \ascii{Worker} concurrently to execute the \code{PartitionGraph} list.

\begin{leftbar}
\ Begin {python}
def run_step(workers, full_graph, inputs, outputs):
  client_graph = prune(full_graph, inputs, outputs)
  partition_graphs = split(client_graph, workers)
  register_graphs(partition_graphs, inputs, outputs)
  run_graphs(partition_graphs, inputs, outputs)
\end{python}
\end{leftbar}

\subsubsection{Worker::RunStep}

On a particular \ascii{Worker} node, when the \code{RegisterGraphRequest} message is received, the computed graph is split into multiple \code{PartitionGraph}s according to the local device set. Then, start a \code{Executor} on each computing device to execute the \code{PartitionGraph} assigned to it.

After a certain computing device executes the allocated \code{PartitionGraph}, the counter of \code{ExecutorBarrier} is added to \ascii{1} until all devices complete the execution of the \code{PartitionGraph} list, \code{barrier.wait ()} Blocking operation exits.

There may be data dependencies between \code{PartitionGraph} across devices, and they interact by inserting \code{Send/Recv} nodes. In fact, in a distributed runtime, \code{Send/Recv} does the data exchange via \code{RpcRemoteRendezvous}.

The %\code{Send} node will call \code{RpcRemoteRendezvous::Send}, which delegates the \code{LocalRendezvous} data locally. The \code{Recv} node gets the data according to the identifier call \code{RpcRemoteRendezvous::Recv}. At this point, there may be two situations.

% \begin{enum}
% \eitem{The original device is in the same \ascii{Worker} as the target device: \code{RpcRemoteRendezvous::Recv} will directly delegate \code{LocalRendezvous::Recv} locally;
% \eitem{The original device is not in the same \ascii{worker} as the target device: \code{RpcRemoteRendezvous::Recv} sends a \code{RecvTensorRequest} request to the \ascii{Worker} where the target device is located; target\ascii{Worker } will get the data locally from \code{LocalRendezvous::Recv} and return a \code{RecvTensorResponse} message to the peer. }
% \end{enum}

\begin{leftbar}
\ Begin {python}
def send_inputs(remote_rendezvous, inputs):
  for (key, tensor) in inputs:
    remote_rendezvous.send(key, tensor)

def do_run_partitions(executors_and_partitions):
  barrier = ExecutorBarrier(executors_and_partitions.size())
  for (executor, partition) in executors_and_partitions:
    executor.run(partition, barrier.on_done())  
  barrier.wait()

def recv_outputs(remote_rendezvous, outputs):
  for (key, tensor) in outputs:
    remote_rendezvous.recv(key, tensor)

def run_partitions(executors_and_partitions, inputs, outputs):
  remote_rendezvous = RpcRemoteRendezvous()
  send_inputs(remote_rendezvous, inputs)
  do_run_partitions(executors_and_partitions)
  recv_outputs(remote_rendezvous, outputs)

def run_step(devices, full_graph, inputs, outputs):
  executors_and_partitions = split(full_graph, devices)
  run_partitions(executors_and_partitions, inputs, outputs)
\end{python}
\end{leftbar}

\subsection{domain model}

As shown in \refig{cc-dist-model}, there is a sophisticated domain model in the \tf{} distributed runtime.

\begin{figure}[H]
\centering
\includegraphics[width=0.7\textwidth]{figures/cc-dist-model.png}
\caption{Distributed: Domain Model}
 \label{fig:cc-dist-model}
\end{figure}

\subsubsection{Cluster}

\ascii{Cluster} is described using \ascii{ClusterSpec}, which can be divided into one or more \ascii{Job}, one \ascii{Job} contains one or more \ascii{Task}. That is, the \ascii{TensorFlow} cluster is composed of the task set \ascii{(Task Set)} that executes the calculation graph.

Each \ascii{Task} can be run on a separate machine, or multiple \ascii{Task} can be run on a single machine (for example, stand-alone multi-ascii{CPU}, or stand-alone multi-ascii{GPU} ).

\subsubsection{Job}

Assign the same \ascii{Task} to the same \ascii{Job}. Each \ascii{Job} uses a unique identifier of \code{job\_id}.

In general, there are two basic \ascii{Job} types in the model training process of distributed deep learning:

\begin{enum}
  \eitem{\ascii{ps}: responsible for storing and updating model parameters;}
  \eitem{\ascii{worker}: Responsible for computationally intensive model training and reasoning. }
\end{enum}

\begin{figure}[H]
\centering
\includegraphics[width=0.5\textwidth]{figures/py-dist-ps-worker.png}
\caption{Distributed Model Training: Interaction between PS and Worker}
 \label{fig:py-dist-ps-worker}
\end{figure}

\subsubsection{Task}

In general, in a distributed runtime, \ascii{Task} runs in a separate process and runs a \code{tf.train.Server} instance on it. Where \ascii{Task} uses the binary unique identifier of \code{job\_id:task\_index}.

\subsubsection{Server}

\ascii{Server} represents the \ascii{Task} service process, which provides \code{MasterService} and \code{WorkerService} services. In other words, \ascii{Server} can play both \ascii{Master} and \ascii{Worker} roles.

\subsection{Building a cluster}

In the distributed \tf{} runtime, each \ascii{Task} launches a \ascii{Server} and provides the \code{MasterService} service and the \code{WorkerService} service. Among them, the formation of \ascii{TensorFlow} cluster consists of two basic steps:

\begin{enum}
  \eitem{Create\code{tf.train.ClusterSpec}, which describes the deployment information of \ascii{Task} in the cluster and is organized as \ascii{Job};
  \eitem{For each \ascii{Task}, start a \code{tf.train.Server} instance. }
\end{enum}

\subsubsection{cluster configuration}

\code{ClusterSpec} describes the deployment information for \ascii{Task} in the cluster and is organized as \ascii{Job}. In general, in a distributed execution mode, a process is started for each \ascii{Task}. Therefore, \code{ClusterSpec} also describes the process distribution of the \ascii{TensorFlow} distributed runtime.

For example, there is a \ascii{TensorFlow} cluster consisting of \code{ps} and \code{worker}\ascii{Job}. Where \code{ps} is deployed on \code{ps0:2222, ps1:2222}; \code{worker} is deployed on \code{worker0:2222, worker1:2222, worker2:2222}.

\begin{leftbar}
\ Begin {python}
tf.train.ClusterSpec({
  "worker": [
    "worker0:2222",   # /job:worker/task:0
    "worker1:2222",   # /job:worker/task:1
    "worker2:2222"    # /job:worker/task:2
  ],  
  "ps": [
    "ps0:2222",       # /job:ps/task:0
    "ps1:2222"        # /job:ps/task:0
  ]})
\end{python}
\end{leftbar}

In this case, the index of \ascii{Task} is not explicitly specified. By default, in the \ascii{Task} collection of \ascii{Job}, the \ascii{Task} index is incremented sequentially from \ascii{0}.

\subsubsection{ProtobufDescription}

\begin{leftbar}
\ Begin {python}
message JobDef {
  string name = 1;
  map<int32, string> tasks = 2;
}

message ClusterDef {
  repeated JobDef job = 1;
}
\end{python}
\end{leftbar}

The \code{tasks} keyword represents \code{task\_index} and the value represents \code{host:port}.

\end{content}

\section{Master service}

\begin{content}

\code{MasterService} is a \ascii{RPC} service. When \ascii{Client} accesses the \ascii{Server} instance according to \code{target}, \ascii{Server} plays the role of \ascii{Master} and provides the \code{MasterService} service.

The interaction between \ascii{Client} and \ascii{Master} follows the interface specification defined by \code{MasterService}. In other words, \code{MasterService} defines the public contract for \ascii{Client} access\ascii{Master}, which is responsible for coordinating and controlling the execution of multiple \code{WorkerService}.

\subsection{interface definition}

In the \code{master\_service.proto} file, all interfaces of \code{MasterService} are defined; in the \code{master.proto} file, the body of each interface is defined.

\begin{leftbar}
\begin{c++}
service MasterService {
  rpc CreateSession(CreateSessionRequest) 
      returns (CreateSessionResponse);
  
  rpc ExtendSession(ExtendSessionRequest) 
      returns (ExtendSessionResponse);

  rpc PartialRunSetup(PartialRunSetupRequest) 
      returns (PartialRunSetupResponse);

  rpc RunStep (RunStepRequest) 
      returns (RunStepResponse);
  
  rpc CloseSession(CloseSessionRequest) 
      returns (CloseSessionResponse);
  
  rpc ListDevices(ListDevicesRequest) 
      returns (ListDevicesResponse);

  rpc Reset(ResetRequest) 
      returns (ResetResponse);
}
\end{c++}
\end{leftbar}

\subsection{Access Service}

In general, \ascii{Client} uses the interface \code{MasterInterface} to get the remote \code{MasterService} service. In particular, all interfaces of \code{MasterInterface} are synchronous interfaces, making \ascii{Client} access to the remote \code{MasterService} service as if it were a local function.

Note that the \code{RunStepRequest/RunStepResponse} message may contain a large \code{Tensor} instance. In order to avoid unnecessary copying of objects, specialization implements a message wrapper.

\begin{leftbar}
\begin{c++}
// Abstract interface for communicating with the TensorFlow Master service.
//
// This interface supports both RPC-based master implementations, and
// in-process master implementations that do not require an RPC roundtrip.
struct MasterInterface {
  virtual ~MasterInterface() {}
  
  virtual Status CreateSession(
      CallOptions * call_options,
      const CreateSessionRequest* request,
      CreateSessionResponse* response) = 0;

  virtual Status ExtendSession(
      CallOptions * call_options,
      const ExtendSessionRequest* request,
      ExtendSessionResponse* response) = 0;

  virtual Status PartialRunSetup(
      CallOptions * call_options,
      const PartialRunSetupRequest* request,
      PartialRunSetupResponse* response) {
    return errors::Unimplemented(
      "Partial run not implemented for master");
  }

  Virtual Status RunStep (
      CallOptions * call_options,
      RunStepRequestWrapper * request,
      MutableRunStepResponseWrapper* response) = 0;

  // Wrapper classes for the `MasterService.RunStep` message.
  //
  // The `RunStepRequest/RunStepResponse` message can contain 
  // potentially large tensor data as part of its `feed/fetch` 
  // submessages.
  Virtual Status RunStep (
    CallOptions * call_options,
    const RunStepRequest* request,
    RunStepResponse * response) {
    std::unique_ptr<RunStepRequestWrapper> wrapped_request(
        new ProtoRunStepRequest(request));
    std::unique_ptr<MutableRunStepResponseWrapper> wrapped_response(
        new NonOwnedProtoRunStepResponse(response));
    return RunStep(call_options, 
        wrapped_request.get(), 
        wrapped_response.get());
  }

  // Returns a request object for use in calls to
  // `RunStep()`. Ownership is transferred to the caller.
  virtual MutableRunStepRequestWrapper* CreateRunStepRequest() {
    return new MutableProtoRunStepRequest;
  }

  // Returns a response object for use in calls to
  // `RunStep()`. Ownership is transferred to the caller.
  virtual MutableRunStepResponseWrapper* CreateRunStepResponse() {
    return new OwnedProtoRunStepResponse;
  }

  virtual Status CloseSession(
    CallOptions * call_options,
    const CloseSessionRequest* request,
    CloseSessionResponse* response) = 0;

  virtual Status ListDevices(
    CallOptions * call_options,
    const ListDevicesRequest* request,
    ListDevicesResponse* response) = 0;

  virtual Status Reset(
    CallOptions* call_options, const ResetRequest* request,
    ResetResponse* response) = 0;
};
\end{c++}
\end{leftbar}

As shown by \refig{dist-master-interface}, there are two basic implementations of \code{MasterInterface}.

\begin{enum}
  \eitem{distributed: \code{GrpcRemoteMaster} based on \ascii{gRPC}, \ascii{Client} and \ascii{Master} are deployed in two different processes;
  \eitem{Local mode: Based on the \code{LocalMaster} implementation of the function call, \ascii{Client} is in the same process as \ascii{Master}. }
\end{enum}

\begin{figure}[H]
\centering
\includegraphics[width=0.6\textwidth]{figures/dist-master-interface.png}
\caption{\code{MasterInterface}}
 \label{fig:dist-master-interface}
\end{figure}

In distributed mode, \code{GrpcRemoteMaster} uses pseudo-code similar to the following and gets the remote \code{MasterService} service via \ascii{gRPC}.

\begin{leftbar}
\begin{c++}
stub = NewStub("/job:worker/replica:0/task:0")
handle = stub->CreateSession({graph_def})
do {
  stub->RunStep(handle, feeds, fetches);
} while (!should_stop());
stub->CloseSession({handle})
\end{c++}
\end{leftbar}

\subsection{RPC procedure}

As shown by \refig{dist-client-master-interaction}, \ascii{Client} gets the service of remote\ascii{MasterService} via \code{MasterInterface}.

\begin{figure}[H]
\centering
\includegraphics[width=0.8\textwidth]{figures/dist-client-master-interaction.png}
\caption{Client to get the principle of MasterService}
 \label{fig:dist-client-master-interaction}
\end{figure}

Among them, \code{GrpcRemoteMaster} is an implementation of the \ascii{gRPC} client, which finally obtains the \code{GrpcMasterService} service on the remote \ascii{Master} via \code{Stub}, making its behavior behave as if Local function calls are generic. Among them, \code{GrpcMasterService} implements all the service interfaces defined by \code{MasterService}, which is the real service entity of \code{MasterService}.

\begin{remark}
Strictly speaking, \script{GrpcSession, ClientMaster, GrpcRemoteMaster} are all part of the \ascii{Client} implementation. Rather than commonly understood, the \ascii{Python} front-end system is a full \ascii{Client} implementation, and the backend \ascii{C++} backend system does not include any implementation of \ascii{Client}.
\end{remark}

\subsection{message definition}

Next, we'll take a closer look at the message definitions for each interface. Among them, the most important thing is to identify the identity of each service. For example, \ascii{Master} can be accessed by multiple \ascii{Client} and generate a corresponding \code{MasterSession} instance for each \ascii{Client}. So \code{GrpcSession} holds the \code{MasterSession} handle and implements \ascii{Client} to get the \ascii{Master} service.

\subsubsection{CreateSession}

As shown in \refig{dist-ms-create-sess-req}, the \code{CreateSessionRequest} message carries the initial calculation graph and establishes a connection with \ascii{Master} specified by \code{target}. When \ascii{Master} receives the request message, it creates a corresponding \code{MasterSession} instance and uniquely identifies the \code{MasterSession} instance using \code{session\_handle}.

After the \ascii{Master} logic is processed, return the message \code{CreateSessionResponse} to \ascii{Client}. The \code{CreateSessionResponse} message carries \code{session\_handle}, and the \code{GrpcSession} on the \ascii{Client} side is associated with \code{MasterSession} on the \ascii{Master} side. Subsequently, in all interactions between \ascii{Client} and \ascii{Master}, the \ascii{MasterSession} corresponding to it is indexed in the request message by carrying \code{session\_handle}, \ascii{Master} Example.

In addition, \code{CreateSessionResponse} also carries the initial \code{graph\_version} for subsequent initiating \code{ExtendSession} operations, adding new nodes to the original calculation graph.

\begin{figure}[H]
\centering
\includegraphics[width=0.6\textwidth]{figures/dist-ms-create-sess-req.png}
\caption{\code{CreateSession}}
 \label{fig:dist-ms-create-sess-req}
\end{figure}

\begin{leftbar}
\begin{c++}
message CreateSessionRequest {
  GraphDef graph_def = 1;
  ConfigProto config = 2;
  string target = 3;
}

message CreateSessionResponse {
  string session_handle = 1;
  int64 graph_version = 2;
}
\end{c++}
\end{leftbar}

\subsubsection{ExtendSession}

After \ascii{CreateSession} succeeds, the subsequent \ascii{Client} can pass the subgraph to be expanded to \ascii{Master} via \code{ExtendSession}, increasing the size of the original calculation graph (only submaps can be added) Cannot modify or delete nodes).

As shown in \refig{dist-ms-extend-sess-req}, you need to carry \code{current\_graph\_version}, \ascii{Master} for version matching verification in the request message; wait for \code{ExtendSession} After the logical processing is completed, \code{new\_graph\_version} is carried in the response message for the next \code{ExtendSession} operation. The initial \code{graph\_version} is carried by \code{CreateSessionResponse} to \ascii{Client}.

\begin{figure}[H]
\centering
\includegraphics[width=0.7\textwidth]{figures/dist-ms-extend-sess-req.png}
\caption{\code{ExtendSession}}
 \label{fig:dist-ms-extend-sess-req}
\end{figure}

\begin{leftbar}
\begin{c++}
message ExtendSessionRequest {
  string session_handle = 1;

  // REQUIRED: The nodes to be added to the session's graph. 
  // If any node has the same name as an existing node, 
  // the operation will fail with ILLEGAL\_ARGUMENT.
  GraphDef graph_def = 2;

  // REQUIRED: The version number of the graph to be extended. 
  // This will be tested against the current server-side version 
  // number, and the operation will fail with FAILED\_PRECONDITION 
  // if they do not match.
  int64 current_graph_version = 3;
}

message ExtendSessionResponse {
  // The new version number for the extended graph, 
  // to be used in the next call to ExtendSession.
  int64 new_graph_version = 4;
}
\end{c++}
\end{leftbar}

\ Subsubsection {} RunStep

In general, \code{RunStep} is iteratively executed on the client side. As shown in \refig{dist-ms-run-step-req}, during each \code{RunStep} execution, \ascii{Client} carries \code{feed, fetch, target} in the request message, respectively Indicates the list of \ascii{NamedTensor} entered, the name list of \ascii{Tensor} to be output, the name list of \ascii{OP} to be executed, and \code{tensor} in the response message, corresponding to \code{fetch A list of names for }, a list of \ascii{Tensor} output.

\begin{figure}[H]
\centering
\includegraphics[width=0.7\textwidth]{figures/dist-ms-run-step-req.png}
\ Caption {\ code RunStep {}}
 \label{fig:dist-ms-run-step-req}
\end{figure}

\begin{leftbar}
\begin{c++}
message RunStepRequest {
  string session_handle = 1;

  repeated NamedTensorProto feed = 2;
  repeated string fetch = 3;
  repeated string target = 4;

  RunOptions options = 5;
  string partial_run_handle = 6;
}

message RunStepResponse {
  repeated NamedTensorProto tensor = 1;
  RunMetadata metadata = 2;
}
\end{c++}
\end{leftbar}

\subsubsection{CloseSession}

When the calculation is complete, you need to close the session and release the system computing resources. As shown in \refig{dist-ms-closs-sess}, \ascii{Client} initiates the release of computing resources by sending \code{CloseSession} to \ascii{Master}.

\begin{figure}[H]
\centering
\includegraphics[width=0.5\textwidth]{figures/dist-ms-closs-sess.png}
\caption{\code{CloseSession}}
 \label{fig:dist-ms-closs-sess}
\end{figure}

\begin{leftbar}
\begin{c++}
message CloseSessionRequest {
  string session_handle = 1;
}

message CloseSessionResponse {
}
\end{c++}
\end{leftbar}

\end{content}

\section{Worker service}

\begin{content}

\code{WorkerService} is also a \ascii{gRPC} service that dispatches local device sets to execute local submaps. It defines the interface specification for accessing \ascii{Worker}, which is defined in \code{master\_service.proto}.

\ascii{Master} finds other \ascii{Server} instances in the cluster based on the \code{ClusterSpec} information, at which point these \ascii{Server} instances will play the role of \ascii{Worker}. \ascii{Master} distributes the submap to each \ascii{Worker} node and starts the execution of the subgraph calculation for each \ascii{Worker} node.

If there is a data dependency between \ascii{Worker}, the interaction is done through interprocess communication. Among them, between \ascii{Master} and \ascii{Worker}, the interaction between \ascii{Worker} and \ascii{Worker} follows the interface specification defined by \code{WorkerService}.

\subsection{interface definition}

In the \code{worker\_service.proto} file, all interfaces of \code{WorkerService} are defined; in the \code{worker.proto} file, the body of each interface is defined.

\begin{leftbar}
\begin{c++}
service WorkerService {
  rpc GetStatus(GetStatusRequest) 
      returns (GetStatusResponse);

  rpc CreateWorkerSession(CreateWorkerSessionRequest)
      returns (CreateWorkerSessionResponse);

  rpc RegisterGraph(RegisterGraphRequest) 
      returns (RegisterGraphResponse);

  rpc DeregisterGraph(DeregisterGraphRequest) 
      returns (DeregisterGraphResponse);

  rpc RunGraph(RunGraphRequest) 
      returns (RunGraphResponse);

  rpc CleanupGraph(CleanupGraphRequest) 
      returns (CleanupGraphResponse);

  rpc CleanupAll(CleanupAllRequest) 
      returns (CleanupAllResponse);

  rpc RecvTensor(RecvTensorRequest) 
      returns (RecvTensorResponse) {
  }

  rpc Logging(LoggingRequest) 
      returns (LoggingResponse);

  rpc Tracing(TracingRequest) 
      returns (TracingResponse);
}
\end{c++}
\end{leftbar}

\subsection{Access Service}

In general, \ascii{Master/Worker} uses the interface \code{WorkerInterface} to get the service of the remote \code{WorkerService}. Where \code{WorkerInterface} defines the interface for asynchronous access to \code{WorkerService}; similar to \code{MasterInterface}, since \code{RunGraphRequest/RunGraphResponse} may contain a larger \code{Tensor}, in order to avoid A necessary copy of the object, specializing in the wrapper that implements the message.

\begin{leftbar}
\begin{c++}
struct WorkerInterface {
  // async interfaces.
  virtual void GetStatusAsync(
      const GetStatusRequest* request,
      GetStatusResponse* response,
      StatusCallback done) = 0;

  virtual void CreateWorkerSessionAsync(
      const CreateWorkerSessionRequest* request,
      CreateWorkerSessionResponse* response, 
      StatusCallback done) = 0;

  virtual void RegisterGraphAsync(
      const RegisterGraphRequest* request,
      RegisterGraphResponse* response,
      StatusCallback done) = 0;

  virtual void DeregisterGraphAsync(
      const DeregisterGraphRequest* request,
      DeregisterGraphResponse* response,
      StatusCallback done) = 0;

  virtual void RunGraphAsync(
      CallOptions * opts, 
      RunGraphRequestWrapper* request,
      MutableRunGraphResponseWrapper* repsonse,
      StatusCallback done) = 0;

  // Wrapper classes for the `WorkerService.RunGraph` message.
  //
  // The `RunGraphRequest/RunGraphResponse` message can contain 
  // potentially large tensor data as part of its `send/response`
  // submessages.
  virtual void RunGraphAsync(
      CallOptions * opts, 
      const RunGraphRequest* request,
      RunGraphResponse* response, 
      StatusCallback done) {
    RunGraphRequestWrapper* wrapped_request = 
        new ProtoRunGraphRequest(request);
    MutableRunGraphResponseWrapper* wrapped_response =
        new NonOwnedProtoRunGraphResponse(response);
    RunGraphAsync(opts, wrapped_request, wrapped_response,
        [wrapped_request, wrapped_response, done](const Status& s) {
            done(s);
            delete wrapped_request;
            delete wrapped_response;
        });
  }

  // Returns a request object for use in calls to
  // `RunGraphAsync()`. Ownership is transferred to the caller.
  virtual MutableRunGraphRequestWrapper* CreateRunGraphRequest() {
    return new MutableProtoRunGraphRequest;
  }

  // Returns a response object for use in calls to
  // `RunGraphAsync()`. Ownership is transferred to the caller.
  virtual MutableRunGraphResponseWrapper* CreateRunGraphResponse() {
    return new OwnedProtoRunGraphResponse;
  }

  virtual void CleanupGraphAsync(
      const CleanupGraphRequest* request,
      CleanupGraphResponse* response,
      StatusCallback done) = 0;

  virtual void CleanupAllAsync(
      const CleanupAllRequest* request,
      CleanupAllResponse* response,
      StatusCallback done) = 0;

  virtual void RecvTensorAsync(
      CallOptions * opts,
      const RecvTensorRequest* request,
      TensorResponse* response,
      StatusCallback done) = 0;

  virtual void LoggingAsync(
      const LoggingRequest* request,
      LoggingResponse* response, 
      StatusCallback done) = 0;

  virtual void TracingAsync(
      const TracingRequest* request,
      TracingResponse* response, 
      StatusCallback done) = 0;
};
\end{c++}
\end{leftbar}


\code{WorkerInterface} also defines a synchronous access interface. The synchronization interface is implemented indirectly on the asynchronous interface through the adapter of \code{CallAndWait}. In particular, the synchronous interface makes \ascii{Master/Worker} call the remote \code{WorkerService} as if it were a local function.

\begin{leftbar}
\begin{c++}
struct WorkerInterface {
  // sync interfaces.
  Status GetStatus(
      const GetStatusRequest* request,
      GetStatusResponse* response) {
    return CallAndWait(&ME::GetStatusAsync, request, response);
  }

  Status CreateWorkerSession(
      const CreateWorkerSessionRequest* request,
      CreateWorkerSessionResponse* response) {
    return CallAndWait(&ME::CreateWorkerSessionAsync, request, response);
  }

  Status RegisterGraph(
      const RegisterGraphRequest* request,
      RegisterGraphResponse* response) {
    return CallAndWait(&ME::RegisterGraphAsync, request, response);
  }

  Status DeregisterGraph(
      const DeregisterGraphRequest* request,
      DeregisterGraphResponse* response) {
    return CallAndWait(&ME::DeregisterGraphAsync, request, response);
  }

  Status CleanupGraph(
      const CleanupGraphRequest* request,
      CleanupGraphResponse* response) {
    return CallAndWait(&ME::CleanupGraphAsync, request, response);
  }

  Status CleanupAll(
      const CleanupAllRequest* request,
      CleanupAllResponse* response) {
    return CallAndWait(&ME::CleanupAllAsync, request, response);
  }

  Status Logging(
      const LoggingRequest* request, 
      LoggingResponse* response) {
    return CallAndWait(&ME::LoggingAsync, request, response);
  }

  Status Tracing(
      const TracingRequest* request, 
      TracingResponse* response) {
    return CallAndWait(&ME::TracingAsync, request, response);
  }
 
 private:
  typedef WorkerInterface ME;

  template <typename Method, typename Req, typename Resp>
  Status CallAndWait(Method func, const Req* req, Resp* resp) {
    Status right;
    Notification n;
    (this->*func)(req, resp, [&ret, &n](const Status& s) {
      right = s;
      n.Notify();
    });
    n.WaitForNotification();
    return right;
  }
};
\end{c++}
\end{leftbar}

In particular, instances generated by \code{WorkerInterface} are deleted by \code{WorkerCacheInterface::ReleaseWorker}. Therefore, in order to avoid external illegal deletion of the \code{WorkerInterface} instance, limit the destructor of \code{WorkerInterface} to \code{protected} and declare \code{WorkerCacheInterface} as a friend.

\begin{leftbar}
\begin{c++}
struct WorkerInterface {
 protected:
  virtual ~WorkerInterface() {}
  friend class WorkerCacheInterface;
};
\end{c++}
\end{leftbar}

As shown by \refig{dist-worker-interface}, there are two implementations of \code{WorkerService}. Among them, in local mode, \code{GrpcWorker} is used directly; in distributed mode, \ascii{Worker} is deployed in a different process, using \code{GrpcRemoteWorker}.

\begin{figure}[H]
\centering
\includegraphics[width=0.6\textwidth]{figures/dist-worker-interface.png}
\caption{\code{WorkerInterface}interface}
 \label{fig:dist-worker-interface}
\end{figure}

\subsection{RPC procedure}

As shown in \refig{dist-worker-interaction}, in distributed mode, \code{GrpcRemoteWorker} is an implementation of the \ascii{gRPC} client, which eventually gets the remote \ascii{ via \code{Stub} The \code{GrpcWorkerService} service on Worker} makes its behavior behave like a native function call. Among them, \code{GrpcWorkerService} implements all the service interfaces defined by \code{WorkerService}.

\begin{remark}
Strictly speaking, \script{GrpcRemoteWorker} is part of the \ascii{Master} or peer\ascii{Worker} implementation.
\end{remark}

In the local mode, the \ascii{WorkerService} service is directly obtained through the function call of \code{GrpcWorker}, which avoids extra network transmission overhead.

\begin{figure}[H]
\centering
\includegraphics[width=0.7\textwidth]{figures/dist-worker-interaction.png}
\caption{Get the RPC process of \code{MasterService}}
 \label{fig:dist-worker-interaction}
\end{figure}

\subsection{message definition}

Next, we will take a closer look at the message definitions for each interface of \code{WorkerService}. Among them, the most important thing is to identify the identity of each service. When \code{WorkerSession} is created, the \code{MasterSession} identifier is passed to \ascii{Worker}, which implements \code{MasterSession} to uniformly manage multiple dependent \code{WorkerSession} instances.

When \code{Worker} first completes \code{RegisterGraph}, it returns a unique \code{graph\_handle} to \code{Master} to identify the graph instance. Therefore, the \code{(session\_handle, graph\_handle)} binary group can be used to uniquely identify the graph instance within the cluster.

When \ascii{Master} broadcasts each \ascii{Worker} concurrently to \code{RunGraph}. To distinguish between different \code{step}, \ascii{Master} generates a globally unique \code{step\_id} and passes it to each \ascii{Worker} via \code{RunGraph}.

\begin{enum}
  \eitem{\code{session\_handle}: Automatically generated when creating a \code{MasterSession} instance, carried to \ascii{Client} via \code{CreateSessionResponse}; \ascii{Worker} via \code{CreateWorkerSessionRequest}; }  
  \eitem{\code{graph\_id}: The first time \code{RegisterGraph} is generated by \code{Worker}, passed to \code{Master} via \code{RegisterGraphResponse};
  \eitem{\code{step\_id}: Each time \code{RunStep}, a unique identifier is generated by \code{Master}, which is carried to \ascii{Worker} via \code{RunGraphRequest}. }
\end{enum}

\subsubsection{CreateWorkerSession}

As shown in \refig{dist-worker-create-worker-sess}, the \code{CreateWorkerSessionRequest} message carries the \code{session\_handle} assigned by \code{MasterSession}. When \ascii{Worker} receives the request message, it generates a \code{WorkerSession} instance and uniquely identifies the instance within \ascii{Worker} using \code{session\_handle}.

In the same cluster, for a \code{MasterSession} instance, the other \ascii{Worker} receives the same \code{session\_handle}. In this way, the \code{MasterSession} instance can uniformly manage all \code{WorkerSession} instances that belong to it.

\begin{figure}[H]
\centering
\includegraphics[width=0.6\textwidth]{figures/dist-worker-create-worker-sess.png}
\caption{\code{CreateWorkerSession}}
 \label{fig:dist-worker-create-worker-sess}
\end{figure}

\begin{leftbar}
\begin{c++}
message CreateWorkerSessionRequest {
  string session_handle = 1;
  ServerDef server_def = 2;
}

message CreateWorkerSessionResponse {
}
\end{c++}
\end{leftbar}

\subsubsection{RegisterGraph}

As shown in \refig{dist-worker-register-graph}, the \code{RegisterGraphRequest} message carries the \code{session\_handle} assigned by \code{MasterSession}, and its subgraph instance \ascii{graph\_def} . When \code{Worker} completes the subgraph registration and its initialization, it returns \code{graph\_handle} of the submap to \ascii{Master}.

It should be noted that \code{Master} will only execute \code{RegisterGraph} once, unless the node of the calculation graph is reorganized, or the \code{Master} process is restarted.

\begin{figure}[H]
\centering
\includegraphics[width=0.6\textwidth]{figures/dist-worker-register-graph.png}
\caption{\code{RegisterGraph}}
 \label{fig:dist-worker-register-graph}
\end{figure}

\begin{leftbar}
\begin{c++}
message RegisterGraphRequest {
  string session_handle = 1;

  GraphDef graph_def = 2;
  bool has_control_flow = 3 [deprecated = true];

  GraphOptions graph_options = 4;
  DebugOptions debug_options = 5;
}

message RegisterGraphResponse {
  string graph_handle = 1;
}
\end{c++}
\end{leftbar}


\subsubsection{DeregisterGraph}

As shown in \refig{dist-worker-deregister-graph}, when the submap on the \code{Worker} node is no longer needed (for example, the calculation graph is rescheduled, the nodes in the graph are rearranged), at this time\ Code{Master} sends a \code{DeregisterGraph} message to \ascii{Worker} so that \code{Worker} unregisters the submap instance.

\begin{figure}[H]
\centering
\includegraphics[width=0.6\textwidth]{figures/dist-worker-deregister-graph.png}
\caption{\code{DeregisterGraph}}
 \label{fig:dist-worker-deregister-graph}
\end{figure}

\begin{leftbar}
\begin{c++}
message DeregisterGraphRequest {
  string session_handle = 2;
  string graph_handle = 1;
}

message DeregisterGraphResponse {
}
\end{c++}
\end{leftbar}

\subsubsection{RunGraph}

When executing the submap registered on the \ascii{Worker} node, in order to distinguish between different \code{step}, \ascii{Master} generates a unique \code{step\_id} and passes it to each \ascii{Worker}, each \ascii {Worker} implements data collaboration through \code{step\_id}.

In addition, \code{RunGraphRequest} carries \code{send, recv\_key}, which respectively represent the \code{Tensor} identifier and data entered by the submap, and the \code{Tensor} identifier of the submap output. \code{RunGraphResponse} returns a list of \code{Tensor} corresponding to \code{recv\_key}.

\begin{figure}[H]
\centering
\includegraphics[width=0.6\textwidth]{figures/dist-worker-run-graph.png}
\caption{\code{RunGraph}}
 \label{fig:dist-worker-run-graph}
\end{figure}

\begin{leftbar}
\begin{c++}
message RunGraphRequest {
  string session_handle = 8;
  string graph_handle = 1;
  int64 step_id = 2;

  ExecutorOpts exec_opts = 5;

  repeated NamedTensorProto send = 3;
  repeated string recv_key = 4;

  bool is_partial = 6;
  bool is_last_partial_run = 7;
}

message RunGraphResponse {
  repeated NamedTensorProto recv = 1;

  // execution stats
  StepStats step_stats = 2;
  CostGraphDef cost_graph = 3;
  repeated GraphDef partition_graph = 4;
}
\end{c++}
\end{leftbar}

\subsubsection{RecvTensor}

When executing \code{step} one time, if two \ascii{Worker} need to interact with data, the consumer sends a \code{RecvTensorRequest} message to the producer, by carrying \code{(step\_id, rendezvous\_key) } Binary group, request the corresponding \code{Tensor} data of the peer \ascii{Worker} and return it via \code{RecvTensorResponse}.

\begin{figure}[H]
\centering
\includegraphics[width=0.6\textwidth]{figures/dist-worker-recv-tensor.png}
\caption{\code{RecvTensor}}
 \label{fig:dist-worker-recv-tensor}
\end{figure}

\begin{leftbar}
\begin{c++}
message RecvTensorRequest {
  int64 step_id = 1;
  string rendezvous_key = 2;

  // If true, use an out-of-band DMA mechanism to transfer the
  // received tensor.
  bool dma_ok = 3;

  // Optional information on client-side device locality.
  DeviceLocality client_locality = 4;

  // Optional information on server-side device locality.
  DeviceLocality server_locality = 5;

  // Optional information needed by the RPC subsystem.
  google.protobuf.Any transport_options = 6;
}

message RecvTensorResponse {
  // The tensor as a proto.
  TensorProto tensor = 1;

  // If true, this tensor was the output of a dead node, and the
  // content is invalid.
  bool is_dead = 2;

  // The time at which tensor was available and started to be returned.
  int64 send_start_micros = 3;

  // Optional additional information about how to receive the tensor,
  // e.g. in the event that `RecvTensorRequest.dma\_ok` was true.
  google.protobuf.Any transport_options = 4;
}
\end{c++}
\end{leftbar}

\end{content}

\section{server}

\begin{content}

\code{Server} is a server based on \ascii{gRPC} that manages the local device set. It provides the \code{MasterService} service and the \code{WorkerService} service externally, with the roles of \ascii{Master} and \ascii{Worker}.

\subsection{domain model}

As shown by \refig{cc-server-model}, \code{GrpcServer} provides the \code{MasterService} service when it plays the role of \ascii{Master}; where it is the \ascii{Client for each access. } Start a \code{MasterSession} instance and identify it with a globally unique \code{session\_handle}. In other words, \ascii{Master} can access multiple \ascii{Client}, while a \ascii{Client} can only access a specific \ascii{Master}.

\code{GrpcServer} provides the \code{WorkerService} service when playing the role of \ascii{Worker}; where each \ascii{Worker} can provide computing services for multiple \ascii{Master}, it is for each Generate a corresponding \code{WorkerSession} instance from the \code{MasterSession} requesting the computing service, and wait for the corresponding \code{MasterSession} to issue the \emph{registration} and \emph{execute} commands for the calculation graph.

The entire \code{GrpcServer} instance is hosted on the \code{grpc::Server} process, which listens for messages on specific ports and automatically dispatches the corresponding message to \code{MasterService} or \code{WorkerService} when the message arrives. The callback function being processed.

\begin{figure}[H]
\centering
\includegraphics[width=1.0\textwidth]{figures/cc-server-model.png}
\caption{Server domain model}
 \label{fig:cc-server-model}
\end{figure}

\subsubsection{ProtobufDescription}

When \code{protocol} is \code{grpc}, the \code{GrpcServer} instance based on \ascii{gRPC} is enabled when the system is running. In addition, the configuration of runtime parameters can be implemented via \code{ConfigProto}. In other words, the \tf{} architecture is open to the public. For example, by extending \code{protocol} to support new communication protocols, implement a \code{Server} instance based on the new protocol.

\begin{leftbar}
\ Begin {python}
message ServerDef {
  ClusterDef cluster = 1;
  
  string job_name = 2;
  int32 task_index = 3;

  ConfigProto default_session_config = 4;
  string protocol = 5;
}
\end{python}
\end{leftbar}

\subsubsection{Service Interconnect}

As shown by \refig{cc-server-interact}, an \ascii{Server} instance is interconnected with other \ascii{Server} instances in the cluster via \code{tf.train.ClusterSpec}.

\begin{figure}[H]
\centering
\includegraphics[width=0.5\textwidth]{figures/cc-server-interact.png}
\caption{service interconnect}
 \label{fig:cc-server-interact}
\end{figure}

As shown in \refig{cc-server-interact-1}, when \ascii{Client} accesses one of \ascii{Server}, it plays the role of \ascii{Master}, and other \ascii{Server} Then played the role of \ascii{Worker}. In particular, \ascii{Server} accessed by \ascii{Client} also plays the role of \ascii{Worker}.

\begin{figure}[H]
\centering
\includegraphics[width=0.5\textwidth]{figures/cc-server-interact-1.png}
\caption{Single Client Access Cluster}
 \label{fig:cc-server-interact-1}
\end{figure}

As shown by \refig{cc-server-interact-2}, there may be multiple \ascii{Client} accessing different \ascii{Server} instances. At this point, the \ascii{Server} instance accessed by \ascii{Client} plays the \ascii{Master} role. However, the \ascii{Server} instance plays the \ascii{Worker} role relative to the other \ascii{Server} instances in the cluster.

\begin{figure}[H]
\centering
\includegraphics[width=0.5\textwidth]{figures/cc-server-interact-2.png}
\caption{Multi-Client Access Cluster}
 \label{fig:cc-server-interact-2}
\end{figure}

In particular, \ascii{Client} and \ascii{Master} can be deployed in the same process. At this point, the interaction between \ascii{Client} and \ascii{Master} is much simpler, and the two directly use function calls, avoiding the extra overhead of \ascii{gRPC} interaction. And so on, in the same \ascii{Server}, \ascii{Master} and \ascii{Worker} can be deployed in the same process. At this point, the function call is used directly between \ascii{Master} and \ascii{Worker}.

\subsection{state machine}

As shown by \refig{dist-grpc-server-state-machine}, \code{GrpcServer} is a server based on \code{grpc::Server} that manages and maintains a simple state machine.

\code{GrpcServer} starts the \code{grpc::Server} service on the \code{New} state, but does not provide services externally; instead, it starts the service on the \code{Started} state and provides \code{MasterService } and \code{WorkerService}'s \code{RPC} message service; eventually, the \code{MasterService} and \code{WorkerService} services are stopped in the \code{Stopped} state.

\begin{figure}[H]
\centering
\includegraphics[width=0.7\textwidth]{figures/dist-grpc-server-state-machine.png}
\caption{GrpcServer state machine}
 \label{fig:dist-grpc-server-state-machine}
\end{figure}

\subsubsection{Create Service}

\begin{figure}[H]
\centering
\includegraphics[width=0.7\textwidth]{figures/dist-grpc-server-factory.png}
\caption{Polymorphic creation of Server instance}
 \label{fig:dist-grpc-server-factory}
\end{figure}

\begin{leftbar}
\begin{c++}
struct GrpcServerFactory : ServerFactory {
  bool AcceptsOptions(const ServerDef& server_def) override {
    return server_def.protocol() == "grpc";
  }

  Status NewServer(const ServerDef& server_def,
      std::unique_ptr<ServerInterface>* out_server) override {
    GrpcServer::Create(server_def, Env::Default(), out_server);
    return Status::OK();
  }
};
\end{c++}
\end{leftbar}

\begin{leftbar}
\begin{c++}
void GrpcServer::Create(
    const ServerDef& server_def, Env* env,
    std::unique_ptr<ServerInterface>* out_server) {
  auto ret = std::make_unique<GrpcServer>(server_def, env);
  direction> Init ();
  *out_server = std::move(ret);
}
\end{c++}
\end{leftbar}

As shown by \refig{cc-server-model}, \code{GrpcServer::Init} will complete the initialization of the \code{GrpcServer} domain object, including the following \ascii{3} basic procedures.

\begin{enum}
  \eitem{initialize \code{MasterEnv} instance;}  
  \eitem{Initialize \code{WorkerEnv} instance;}  
  \eitem{Create and start \code{grpc::Server}}    
    \begin{enum}
    \eitem{initialize\code{MasterService}}      
    \begin{nitemize}
      \eitem{Create a \code{Master} instance;}  
      \eitem{Create a \code{MasterService} instance;}
    \end{nitemize}
    \eitem{initialize\code{WorkerService}}          
    \begin{nitemize}          
      \eitem{Create a \code{Worker} instance;}  
      \eitem{Create a \code{WorkerService} instance. }
    \end{nitemize}      
    \end{enum}
\end{enum}

In order to better understand the initialization process of the entire \code{GrpcServer} instance, the implementation is partially refactored here. First, it initializes the \code{MasterEnv, WorkerEnv} instance; then, creates and starts the \code{grpc::Server} server.

\begin{leftbar}
\begin{c++}
void GrpcServer::Init() {
  InitMasterEnv();
  InitWorkerEnv ();
  StartGrpcServer ();
}
\end{c++}
\end{leftbar}

\subsubsection{Initialize MasterEnv}

\code{MasterEnv} holds the context of the \code{Master} runtime, which has the same lifecycle as \code{GrpcServer}, so the entire \code{Master} runtime is visible.

As shown by \refig{dist-master-env}, \code{LocalDevices} is used to get the local device set; \code{WorkerCacheFactory} is used to create the \code{WorkerCacheInterface} instance; \code{WorkerCacheInterface} is used to create the \code {MasterInterface} instance, the latter is used to call the remote \code{MasterService} service; \code{MasterSessionFactory} is used to create the \code{MasterSession} instance; \code{OpRegisteryInterface} is used to query the specific \code{OP} element Data; \code{Env} is used to get the cross-platform \ascii{API} interface. Among them, the following will focus on the creation process of \code{WorkerCacheInterface}.

\begin{figure}[H]
\centering
\includegraphics[width=0.5\textwidth]{figures/dist-master-env.png}
\caption{\code{MasterEnv}模型}
 \label{fig:dist-master-env}
\end{figure}

\subsubsection{Initialize WorkerEnv}

\code{WorkerEnv} holds the context of the \code{Worker} runtime, which has the same lifecycle as \code{GrpcServer}, so the entire \code{Worker} runtime is visible.

As shown in \refig{dist-worker-env}, \code{LocalDevices} is used to get the local device set; \code{DeviceManager} is used to manage the local device set and the remote device set; \code{SessionManager} is used for management \code{WorkerSession} collection; \code{RendezvousManager} is used to manage the \code{Rendezvous} instance set; \code{ThreadPool} will automatically assign a thread from the compute pool, starting \ascii{OP}\ascii{ The execution of the Kernel} operator; \code{Env} is used to get the cross-platform \ascii{API} interface.

\begin{figure}[H]
\centering
\includegraphics[width=0.5\textwidth]{figures/dist-worker-env.png}
\caption{\code{WorkerEnv}模型}
 \label{fig:dist-worker-env}
\end{figure}

\subsubsection{Start grpc::Server}

The system implementation uses the builder to create a \code{grpc::Server} instance. First, configure the \code{grpc::Server} service options; then, build the \code{MasterService} instance and the \code{WorkerService} instance respectively. Finally, call the \code{builder.BuildAndStart} method to start the \code{grpc::Server} server.

It should be noted that \code{GrpcServer} is still in the \code{New} state when \code{grpc::Server} is started.
\code{grpc::Server} has not yet provided the \code{MasterService} service and the \code{WorkerService} service. Until \code{GrpcServer} is migrated to the \code{Started} state location, \code{grpc::Server} will actually provide the \code{MasterService} service and \code{WorkerService} service.

\begin{leftbar}
\begin{c++}
void InitServerBuilder(::grpc::ServerBuilder& builder) {
  builder.AddListeningPort(
    strings::StrCat("0.0.0.0:", GetRequestedPort()),
    GetServerCredentials(server_def_), &bound_port_);
  builder.SetMaxMessageSize(std::numeric_limits<int32>::max());
  builder.SetOption(
      std::unique_ptr<::grpc::ServerBuilderOption>(new NoReusePortOption));
}

void GrpcServer :: StartGrpcServer () {
  ::grpc::ServerBuilder builder;

  InitServerBuilder(builder);
  InitMasterService(builder);
  InitWorkerService(builder);

  server_ = builder.BuildAndStart();  
}
\end{c++}
\end{leftbar}

It's easy to see that the \code{grpc::Server} entity that provides the \code{MasterService} service is the \code{GrpcMasterService} instance. When the message arrives, the corresponding message handler in the \code{GrpcMasterService} instance is automatically called back. Among them, in the message processing function, the processing of its business logic is completely dependent on the domain object of \code{Master}.

\begin{leftbar}
\begin{c++}
std::unique_ptr<Master> GrpcServer::CreateMaster(
    MasterEnv* master_env) {
  return std::make_unique<Master>(master_env);
}

AsyncServiceInterface* NewGrpcMasterService(
    Master* master, ::grpc::ServerBuilder* builder) {
  return new GrpcMasterService(master, builder);
}

void GrpcServer::InitMasterService() {
  master_impl_ = CreateMaster(&master_env_);
  master_service_ = NewGrpcMasterService(
      master_impl_.get(), &builder);  
}
\end{c++}
\end{leftbar}

By analogy, \code{grpc::Server} provides the \code{WorkerService} service entity to the \code{GrpcWorkerService} instance. When the message arrives, the corresponding message handler in the \code{GrpcWorkerService} instance is automatically called back. Among them, in the message processing function, the processing of its business logic is completely dependent on the domain object of \code{GrpcWorker}.

\begin{leftbar}
\begin{c++}
std::unique_ptr<GrpcWorker> NewGrpcWorker(WorkerEnv* env) {
  return std::unique_ptr<GrpcWorker>(new GrpcWorker(env));
}

AsyncServiceInterface* NewGrpcWorkerService(
    GrpcWorker* worker, ::grpc::ServerBuilder* builder) {
  return new GrpcWorkerService(worker, builder);
}

void GrpcServer::InitWorkerService(::grpc::ServerBuilder& builder) {
  worker_impl_ = NewGrpcWorker(&worker_env_);
  worker_service_ = NewGrpcWorkerService(
    worker_impl_.get(), &builder);
}
\end{c++}
\end{leftbar}

\subsubsection{Starting Service}

In the \code{New} state, \code{grpc::Server} has been started, but the \code{MasterService} service and \code{WorkerService} service are not provided externally. After calling the \code{GrpcServer::Start} method, the state of \code{GrpcServer} is migrated from \code{New} to the \code{Started} state, and two separate threads are started, starting \code{MasterService} respectively. And the message processor of \code{WorkerService}. At this point, \code{GrpcServer} officially provides \code{MasterService} and \code{WorkerService}.

\begin{leftbar}
\begin{c++}
Status GrpcServer::Start() {
  mutex_lock l(mu_);
  switch (state_) {
    case NEW: {
      master_thread_.reset(
          env_->StartThread(ThreadOptions(), "TF_master_service",
                            [this] { master_service_->HandleRPCsLoop(); }));
      worker_thread_.reset(
          env_->StartThread(ThreadOptions(), "TF_worker_service",
                            [this] { worker_service_->HandleRPCsLoop(); }));
      state_ = STARTED;
      return Status::OK();
    }
    case STARTED:
      LOG(INFO) << "Server already started(" << target() << ")";    
      return Status::OK();
    case STOPPED:
    default:
      CHECK(false);
  }
}
\end{c++}
\end{leftbar}

\subsubsection{Wait for terminating service}

In order to permanently provide the \code{MasterService} service and the \code{WorkerService} service, you need to implement the \code{join} operation on the threads \code{TF\_master\_service} and \code{TF\_worker\_service} respectively. Causes the main thread to hang until the two threads terminate.

By calling the \code{GrpcServer::Join} method, when \code{GrpcServer} is in the \code{Started} or \code{Stoped} state, it will automatically call the \code{Thread} destructor.

\begin{leftbar}
\begin{c++}
Status GrpcServer::Join() {
  mutex_lock l(mu_);
  switch (state_) {
    case NEW:
      // Prevent the server from being started subsequently.
      state_ = STOPPED;
      return Status::OK();
    case STARTED:
    case STOPPED:
      master_thread_.reset();
      worker_thread_.reset();
      return Status::OK();
    default:
      CHECK(false);
  }
}
\end{c++}
\end{leftbar}

For example, in \code{StdThread} based on the \code{C++} standard library implementation, its destructor will call the \code{join} method of \code{std::thread}.

\begin{leftbar}
\begin{c++}
struct StdThread : Thread {
  StdThread(const ThreadOptions&, const string&, 
      std::function<void()> fn)
    : thread_(fn) {
  }

  ~StdThread() override { 
    thread_.join(); 
  }

 private:
  std::thread thread_;
};
\end{c++}
\end{leftbar}

\subsubsection{Terminate service}

Unfortunately, the current \code{GrpcServer} does not gracefully exit. Therefore, in the engineering practice environment, the distributed runtime of \tf{} often requires the help of \code{Kubernetes} to implement automatic management of the \code{GrpcServer} service.

\begin{leftbar}
\begin{c++}
Status GrpcServer::Stop() {
  mutex_lock l(mu_);
  switch (state_) {
    case NEW:
      state_ = STOPPED;
      return Status::OK();
    case STARTED:
      return errors::Unimplemented(
          "Clean shutdown is not currently implemented");
    case STOPPED:
      LOG(INFO) << "Server already stopped(" << target() << ")";
      return Status::OK();
    default:
      CHECK(false);
  }
}
\end{c++}
\end{leftbar}

\subsection{Create WorkerCacheInterface}

After introducing the \code{GrpcServer} state machine model, go back to a previous question. \code{MasterEnv} holds the \code{WorkerCacheInterface} instance, which is used to query or delay the creation of \code{WorkerInterface}; where \code{WorkerInterface} is used to access the remote \code{WorkerService} service.

\subsubsection{Factory method: GrpcServer::WorkerCacheFactory}

When initializing \code{MasterEnv}, create a \code{WorkerCacheInterface} instance by calling the factory method \code{GrpcServer::WorkerCacheFactory}. Where \code{WorkerCacheFactoryOptions} is equivalent to \code{ServerDef}, which contains \code{ClusterDef} and its \code{job\_name:task\_index} information. Therefore, the \code{GrpcChannelSpec} instance obtained via \code{ParseChannelSpec} is equivalent to \code{ClusterSpec}, which contains the basic configuration information of the cluster.

\begin{leftbar}
\begin{c++}
Status GrpcServer::WorkerCacheFactory(
    const WorkerCacheFactoryOptions& options,
    WorkerCacheInterface** worker_cache) {

  GrpcChannelSpec channel_spec;
  TF_RETURN_IF_ERROR(ParseChannelSpec(options, &channel_spec));

  std::unique_ptr<GrpcChannelCache> channel_cache(
      NewGrpcChannelCache(channel_spec, GetChannelCreationFunction()));

  string name_prefix = strings::StrCat(
      "/job:", *options.job_name, "/replica:0",
      "/task:", options.task_index);

  *worker_cache = NewGrpcWorkerCacheWithLocalWorker(
      channel_cache.release(), worker_impl_.get(), name_prefix);
  return Status::OK();
}
\end{c++}
\end{leftbar}

\subsubsection{Factory method: NewGrpcChannelCache}

\code{NewGrpcChannelCache} is used to create \code{GrpcChannelCache} instances, and \code{GrpcChannelCache} can get or delay the creation of the corresponding \code{grpc::Channel} instance based on the name of \code{Worker}. Among them, a \ascii{Job} creates a \code{SparseGrpcChannelCache} instance, \code{MultiGrpcChannelCache} holds multiple \code{SparseGrpcChannelCache}, which is a typical combination mode application, which will be explained in detail below. The design of GrpcChannelCache}.

\begin{leftbar}
\begin{c++}
GrpcChannelCache* NewGrpcChannelCache(
    const GrpcChannelSpec& spec,
    ChannelCreationFunction channel_func) {
  std::vector<GrpcChannelCache*> caches;
  for (auto& job : spec.host_ports_jobs()) {
    caches.push_back(
        new SparseGrpcChannelCache(
            job.job_id, job.host_ports, channel_func));
  }
  return new MultiGrpcChannelCache(caches);
}
\end{c++}
\end{leftbar}

\subsubsection{Factory method: NewGrpcWorkerCacheWithLocalWorker}

The factory method \code{NewGrpcWorkerCacheWithLocalWorker} is used to create a \code{GrpcWorkerCache} instance with local \code{Worker}.

\begin{leftbar}
\begin{c++}
WorkerCacheInterface* NewGrpcWorkerCacheWithLocalWorker(
    GrpcChannelCache* cc, WorkerInterface* local_worker,
    const string& local_target) {
  return new GrpcWorkerCache(cc, local_worker, local_target);
}
\end{c++}
\end{leftbar}

\subsubsection{Factory method: GrpcServer::GetChannelCreationFunction}

The design of \code{GetChannelCreationFunction} uses the idea of ​​\ascii{C++} functional programming, which returns a function object for creating a \code{grpc::Channel} instance. Unfortunately, the existing \code{NewHostPortGrpcChannel} function does not match the \code{ChannelCreationFunction} interface. Therefore, an adapter called \code{ConvertToChannelCreationFunction} is used here to transform \code{NewHostPortGrpcChannel} into \code{ChannelCreationFunction}.

\begin{leftbar}
\begin{c++}
using SharedGrpcChannelPtr = std::shared_ptr<::grpc::Channel>;
using ChannelCreationFunction = std::function<SharedGrpcChannelPtr(string)>;

Status NewHostPortGrpcChannel(const string& target,
    SharedGrpcChannelPtr* channel) {
  ::grpc::ChannelArguments args;
  args.SetInt("grapc.arg.max.message_length", 
              std::numeric_limits<int32>::max());
  args.SetInt("grpc.testing.fixed_reconnect_backoff_ms", 
              1000);

  *channel = ::grpc::CreateCustomChannel(
      "dns:///" + target, ::grpc::InsecureChannelCredentials(), args);
  return Status::OK();
}

ChannelCreationFunction ConvertToChannelCreationFunction(
  const std::function<Status(string, SharedGrpcChannelPtr*)>& new_channel) {
  return [new_channel_func](const string& target) -> SharedGrpcChannelPtr {
    SharedGrpcChannelPtr channel_ptr;
    if (new_channel(target, &channel_ptr).ok()) {
      return channel_ptr;
    } else {
      return nullptr;
    }
  };
}

ChannelCreationFunction GrpcServer::GetChannelCreationFunction() const {
  return ConvertToChannelCreationFunction(NewHostPortGrpcChannel);
}
\end{c++}
\end{leftbar}

At this point, we have straightened out the creation process of \code{GrpcChannelCache} and \code{WorkerCacheInterface}, but what are they used for? In fact, \code{WorkerCacheInterface} is used to get the \code{WorkerInterface} instance, which is used to access the remote \code{WorkerSerivice} service, which works very simply.

\begin{enum}
  \eitem{Get a list of all \code{Worker} names in the cluster;}
  \eitem{Create \ascii{RPC} channel according to the name of \code{Worker};}  
  \eitem{Create a \code{GrpcRemoteWorker} instance from the \ascii{RPC} channel of \code{Worker}. }
\end{enum}

Where \code{GrpcRemoteWorker} is a concrete implementation of \code{WorkerInterface}; \code{GrpcChannelCache} is responsible for getting the name of \code{Worker} and its corresponding \code{grpc::Channel} for creating \code{Worker} .

\subsection{Create Worker's RPC Channel}

\code{GrpcChannelCache} is used to get or create the \ascii{RPC} channel of the remote \code{Worker} in the cluster. Where \code{ListWorkers} is used to return a list of names for \code{Worker} in the cluster. \code{TranslateTask} is used to convert the name of \code{Worker} to the address information of \code{host:port}. \code{FindWorkerChannel} looks for a \code{grpc::Channel} instance from the cache; if not found, dynamically creates a \code{grpc::Channel} instance based on the address information and adds it to the cache.

\begin{leftbar}
\begin{c++}
typedef std::shared_ptr<::grpc::Channel> SharedGrpcChannelPtr;

struct GrpcChannelCache {
  virtual ~GrpcChannelCache() {}
  virtual void ListWorkers(std::vector<string>* workers) const = 0;
  virtual SharedGrpcChannelPtr FindWorkerChannel(const string& target) = 0;
  virtual string TranslateTask(const string& task) = 0;
};
\end{c++}
\end{leftbar}

\subsubsection{implicit tree}

As shown in \refig{dist-grpc-channel-cache}, the \code{GrpcChannelCache} class hierarchy follows an implicit "tree" structure, \code{SparseGrpcChannelCache} is a tree node, and each instance corresponds to a \ascii {Job} instance. And \code{MultiGrpcChannelCache} holds multiple \code{SparseGrpcChannelCache} instances, each of which corresponds to multiple \code{Job} instances.

\begin{figure}[H]
\centering
\includegraphics[width=1.0\textwidth]{figures/dist-grpc-channel-cache.png}
\caption{Combination to create GRPC channel}
 \label{fig:dist-grpc-channel-cache}
\end{figure}

\subsubsection{Cache Mechanism}

To avoid the overhead of creating a \code{grpc::Channel} instance each time in real time, \code{CachingGrpcChannelCache} was introduced, which uses caching techniques in the process of looking for \code{grpc::Channel}. When the lookup in the cache fails, the \code{grpc::Channel} instance is dynamically created by calling \code{FindChannelOnce} and added to the cache.

\begin{leftbar}
\begin{c++}
struct CachingGrpcChannelCache : GrpcChannelCache {
  SharedGrpcChannelPtr FindWorkerChannel(const string& target) override {
    SharedGrpcChannelPtr ch = nullptr;
    {
      mutex_lock l(mu_);
      ch = gtl::FindPtrOrNull(channels_, target);
      if (ch) {
        return ch;
      }
    }
    ch = FindChannelOnce(target);
    if (ch) {
      mutex_lock l(mu_);
      channels_.insert({target, ch});
    }
    return ch;
  }

 protected:
  virtual SharedGrpcChannelPtr FindChannelOnce(const string& target) = 0;

 private:
  mutex mu_;
  std::unordered_map<string, SharedGrpcChannelPtr> channels_;
};
\end{c++}
\end{leftbar}

\subsubsection{Foliage node}

Each instance of \code{SparseGrpcChannelCache} corresponds to a \ascii{Job} instance, which creates a corresponding \code{grpc::Channel} instance set for a \ascii{Job}, one for each \ascii{Task} \code{grpc::Channel}.

Where \code{FindChannelOnce} extracts the corresponding \code{task\_id} from the \code{Worker} name by calling \code{TranslateTask}, and then indexes it from \code{host\_ports\_}\ The address information of code{host:port}, call the factory method \code{channel\_func\_} to create the corresponding \code{grpc::Channel} instance. Therefore, it is mainly responsible for the following three responsibilities:

\begin{enum}
  \eitem{returns the list of \ascii{Task} names corresponding to \ascii{Job} via \code{ListWorkers}; for example, \code{/job:ps} returns \code{[/job:ps/replica:0/ Task:0, /job:ps/replica:0/task:1]};}
  \eitem{Index the address information of \code{host:port} by \code{TranslateTask} and according to the specific \ascii{Task} name; for example, \code{/job:ps/replica:0/task:0} The address of the index is \code{ps0:2222};}
  \eitem{Create a corresponding \code{grpc::Channel} instance via \code{FindChannelOnce} and based on the specific \ascii{Task} name. For example, \code{/job:ps/replica:0/task:0} creates a \code{grpc::Channel} instance with the address \code{ps0:2222}. }
\end{enum}

\begin{leftbar}
\begin{c++}
static string MakeAddress(const string& job, int task) {
  return strings::StrCat("/job:", job, "/replica:0/task:", task);
}

struct SparseGrpcChannelCache : CachingGrpcChannelCache {
  SparseGrpcChannelCache(
      const string& job_id,
      const std::map<int, string>& host_ports,
      ChannelCreationFunction channel_func)
      : job_id_(job_id), host_ports_(host_ports),
        channel_func_(std::move(channel_func)) {
  }

  void ListWorkers(std::vector<string>* workers) const override {
    workers->reserve(workers->size() + host_ports_.size());
    for (const auto& id_host_port : host_ports_) {
      workers->emplace_back(MakeAddress(job_id_, id_host_port.first));
    }
  }

  string TranslateTask(const string& target) override {
    DeviceNameUtils::ParsedName parsed;
    if (!DeviceNameUtils::ParseFullName(target, &parsed)) {
      return "";
    }
    auto iter = host_ports_.find(parsed.task);
    return iter == host_ports_.end() ? "" : iter->second;
  }

 protected:
  SharedGrpcChannelPtr FindChannelOnce(const string& target) override {
    auto host_port = TranslateTask(target);
    if (host_port.empty()) {
      return nullptr;
    }
    return channel_func_(host_port);
  }

 private:
  const string job_id_;
  const std::map<int, string> host_ports_;
  const ChannelCreationFunction channel_func_;
};
\end{c++}
\end{leftbar}

\subsubsection{non-leaf node}

\code{MultiGrpcChannelCache} holds multiple \code{SparseGrpcChannelCache} instances via \code{caches\_} to create a combination of \code{grpc::Channel} for all \ascii{Worker} nodes of the entire cluster. To further improve the lookup process for the \code{SparseGrpcChannelCache} instance, \code{MultiGrpcChannelCache} caches the visited \code{SparseGrpcChannelCache} instance; only when the cache finds a \code{SparseGrpcChannelCache} instance fails, it tries to get from \code The corresponding \code{SparseGrpcChannelCache} instance is indexed in the {caches\_} list and automatically added to the cache.

\begin{leftbar}
\begin{c++}
class MultiGrpcChannelCache : public CachingGrpcChannelCache {
 public:
  explicit MultiGrpcChannelCache(
      const std::vector<GrpcChannelCache*>& caches) 
      : caches {}

  ~MultiGrpcChannelCache() override {
    for (auto cache : caches_) {
      delete cache;
    }
  }

  void ListWorkers(std::vector<string>* workers) const override {
    for (auto cache : caches_) {
      cache->ListWorkers(workers);
    }
  }

  string TranslateTask(const string& target) override {
    mutex_lock l(mu_);  // could use reader lock
    auto cache = gtl::FindPtrOrNull(target_caches_, target);
    if (cache == nullptr) {
      for (auto c : caches_) {
        string r = c->TranslateTask(target);
        if (!r.empty()) {
          target_caches_.insert({target, c});
          cache = c;
          break;
        }
      }
    }
    return cache->TranslateTask(target);
  }

 protected:
  SharedGrpcChannelPtr FindChannelOnce(const string& target) override {
    for (auto cache : caches_) {
      auto ch = cache->FindWorkerChannel(target);
      if (ch) {
        mutex_lock l(mu_);
        target_caches_.insert({target, cache});
        return ch;
      }
    }
    return nullptr;
  }

 private:
  // List of channels used by this MultiGrpcChannelCache.
  const std::vector<GrpcChannelCache*> caches_;

  mutex mu_;
  // The same GrpcChannelCache can appear multiple times in the cache.
  std::unordered_map<string, GrpcChannelCache*> target_caches_;
};
\end{c++}
\end{leftbar}

\subsection{Create WorkerInterface}

As shown by \refig{dist-worker-cache-interface} , \code{GrpcWorkerCache} holds the \code{GrpcChannelCache} object and creates a \code{grpc::Channel} instance from it, thus implementing \code{GrpcRemoteWorker } Dynamic creation of instances.

\begin{figure}[H]
\centering
\includegraphics[width=1.0\textwidth]{figures/dist-worker-cache-interface.png}
\caption{polymorphic creation\code{WorkerInterface} instance}
 \label{fig:dist-worker-cache-interface}
\end{figure}

\begin{leftbar}
\begin{c++}
struct GrpcWorkerCache : WorkerCachePartial {
  GrpcWorkerCache(
      GrpcChannelCache* channel_cache,
      WorkerInterface* local_worker,
      const string& local_target)
      : local_target_(local_target),
        local_worker_(local_worker),
        channel_cache_(channel_cache) {}

  ~GrpcWorkerCache() override {
    live_rpc_counter_.WaitUntilUnused();
    delete channel_cache_;
  }

  void ListWorkers(std::vector<string>* workers) const override {
    channel_cache_->ListWorkers(workers);
  }

  WorkerInterface* CreateWorker(const string& target) override {
    if (target == local_target_) {
      return local_worker_;
    } else {
      auto channel = channel_cache_->FindWorkerChannel(target);
      if (!channel) return nullptr;
      return new GrpcRemoteWorker(&live_rpc_counter_, std::move(channel),
                                  &completion_queue_, &logger_);
    }
  }

  void ReleaseWorker(const string& target, 
      WorkerInterface* worker) override {
    if (target != local_target_) {
      WorkerCacheInterface::ReleaseWorker(target, worker);
    }
  }

 private:
  string local_target_;
  WorkerInterface* local_worker_;  // Not owned.
  GrpcCounter live_rpc_counter_;
  GrpcChannelCache* channel_cache_;  // Owned.
  ::grpc::CompletionQueue completion_queue_;
  WorkerCacheLogger logger_;
};
\end{c++}
\end{leftbar}

\end{content}

\section{session control}

\begin{content}

\emph{session control} is the core of the \tf{} distributed runtime and the critical path for the entire \tf{} execution engine. In order to streamline the context of session control, the next article will focus on the detailed process of the entire session control.


\subsection{session collaboration}

As shown in \refig{dist-session-overview}, in distributed mode, session control is implemented through the synergy between \code{GrpcSession, MasterSession, WorkerSession}, which reside in \code{Client, Master, respectively. On Worker}, use the same \code{session\_handle} to work together.

Among them, \code{tf.Session} is implemented using \ascii{Python}, which is \ascii{API} provided by \tf{}. It is in the same process as \code{GrpcSession} and directly holds the handle (or pointer) of \code{GrpcSession}.

\begin{figure}[H]
\centering
\includegraphics[width=0.6\textwidth]{figures/dist-session-overview-1.png}
\caption{session collaboration}
 \label{fig:dist-session-overview}
\end{figure}

As shown in \refig{dist-multi-client-conn}, in distributed mode, there may be multiple \ascii{Client} connected to one \ascii{Master} at the same time, \ascii{Master} for each connection The \ascii{Client} entered creates a \code{MasterSession} instance. \ascii{Worker} may also provide computing services for multiple \ascii{Master}, and \ascii{Worker} creates a \code{WorkerSession} instance for each of the requested \ascii{Master}. To distinguish between different \ascii{Client} computing services, use a different \code{session\_handle} distinction.

\begin{figure}[H]
\centering
\includegraphics[width=0.9\textwidth]{figures/dist-multi-client-conn.png}
\caption{session control: domain model}
 \label{fig:dist-multi-client-conn}
\end{figure}

\subsection{Lifecycle}

\code{GrpcSession} controls the session lifecycle of \ascii{Client}, \code{MasterSession} controls the session lifecycle of \ascii{Master}, \code{WorkerSession} controls the session lifecycle of \ascii{Worker}, they are Synergy is achieved through \code{session\_handle}.

\subsubsection{GrpcSession Lifecycle}

In distributed mode, the runtime of \code{Client} is controlled by \code{GrpcSession}, and the lifecycle of \code{GrpcSession} is shown in \refig{dist-grpc-session-life-cycle}.

\begin{figure}[H]
\centering
\includegraphics[width=0.8\textwidth]{figures/dist-grpc-session-life-cycle.png}
\caption{\code{GrpcSession}Lifecycle}
 \label{fig:dist-grpc-session-life-cycle}
\end{figure}

\subsubsection{MasterSession Lifecycle}

In distributed mode, the runtime of \code{Master} is controlled by \code{MasterSession}, and the \code{MasterSession} lifecycle process is shown in \refig{dist-master-session-life-cycle}.

\begin{figure}[H]
\centering
\includegraphics[width=0.8\textwidth]{figures/dist-master-session-life-cycle.png}
\caption{\code{MasterSession}Lifecycle}
 \label{fig:dist-master-session-life-cycle}
\end{figure}

\subsubsection{WorkerSession Lifecycle}

In distributed mode, the runtime of \code{Worker} is controlled by \code{WorkerSession}, and the \code{WorkerSession} lifecycle process is shown in \refig{dist-worker-session-life-cycle}.

\begin{figure}[H]
\centering
\includegraphics[width=0.8\textwidth]{figures/dist-worker-session-life-cycle.png}
\caption{\code{WorkerSession}Lifecycle}
 \label{fig:dist-worker-session-life-cycle}
\end{figure}

\subsection{session process}

In the user programming environment, \ascii{Client} starts from \code{tf.Session(target)} and starts iterative execution via \code{Session.run}. After the final calculation is completed, call \code{Session.close} to close. Conversation. However, in the implementation of a distributed execution engine, the process is much more complicated.

\begin{nitemize}
  \eitem{Create session}    
    \begin{enum}
      \ item {创建 \ code {GrpcSession};}  
      \eitem{Get the remote device set;} 
      \eitem{Create\code{MasterSession};}
      \eitem{Create\code{WorkerSession};}      
    \end{enum}
  \eitem{iteration execution}          
    \begin{enum}
      \eitem{Start execution;}  
      \eitem{图剪枝;}  
      \eitem{Figure split;}        
      \eitem{Register submap;}              
      \eitem{Run submap;}                         
    \end{enum}      
  \eitem{Close session}          
    \begin{enum}          
      \ item {关闭 \ code {GrpcSession};}  
      \eitem{close\code{MasterSession};}
      \eitem{关闭\code{WorkerSession};}      
    \end{enum}  
\end{nitemize}

\end{content}

\section{Create session}

\begin{content}

Before starting the calculation, you need to create a \code{GrpcSession} instance on \ascii{Client}, a \code{MasterSession} instance on \ascii{Master}, and \code{WorkerSession} on each \ascii{Worker} For example, the three implement the collaboration through \code{Master\ession}'s \code{session\_handle} to serve the \ascii{Client} instance of the access.

\subsection{Create GrpcSession}

When \ascii{Client} calls \code{tf.Session(target)}, the \code{GrpcSession} instance is triggered by calling the \ascii{C API} interface of \code{TF\_NewDeprecatedSession}. Among them, \ascii{C API} is a standard interface for multi-language programming provided by the \tf{} back-end system. Finally, \code{tf.Session} will directly hold the handle to \code{GrpcSession} as shown in \refig{dist-create-grpc-session-1}.

\begin{figure}[H]
\centering
\includegraphics[width=0.6\textwidth]{figures/dist-create-grpc-session-1.png}
\caption{Create\code{GrpcSession}: \code{tf.Session} holds \code{GrpcSession} handle}
 \label{fig:dist-create-grpc-session-1}
\end{figure}

\begin{leftbar}
\begin{c++}
Status NewSession(const SessionOptions& options, Session** out_session) {
  SessionFactory* factory;
  Status s = SessionFactory::GetFactory(options, &factory);
  if (!s.ok()) {
    *out_session = nullptr;
    return s;
  }
  *out_session = factory->NewSession(options);
  if (!*out_session) {
    return errors::Internal("Failed to create session.");
  }
  return Status::OK();
}

TF_DeprecatedSession* TF_NewDeprecatedSession(
  const TF_SessionOptions* opt, TF_Status* status) {
  Session* session;
  status->status = NewSession(opt->options, &session);
  if (status->status.ok()) {
    return new TF_DeprecatedSession({session});
  } else {
    return nullptr;
  }
}
\end{c++}
\end{leftbar}

As shown by \refig{dist-grpc-session-factory}, \code{GrpcSession} is created by \code{GrpcSessionFactory} polymorphism. When \code{target} starts with \code{grpc://}, \code{SessionFactory::GetFactory} returns the \code{GrpcSessionFactory} instance, and \code{GrpcSessionFactory::NewSession}'s factory method delegate\code{ The static factory method of GrpcSession::Create} creates a \code{GrpcSession} instance.

\begin{figure}[H]
\centering
\includegraphics[width=0.7\textwidth]{figures/dist-grpc-session-factory.png}
\caption{Polymorphic creation of GrpcSession}
 \label{fig:dist-grpc-session-factory}
\end{figure}

\begin{leftbar}
\begin{c++}
const char* kSchemePrefix = "grpc://";

struct GrpcSessionFactory : SessionFactory {
  bool AcceptsOptions(const SessionOptions& options) override {
    return StringPiece(options.target).starts_with(kSchemePrefix);
  }

  Session* NewSession(const SessionOptions& options) override {
    std::unique_ptr<GrpcSession> ret;
    Status s = GrpcSession::Create(options, &ret);
    if (s.ok()) {
      return ret.release ();
    } else {
      return nullptr;
    }
  }
};
\end{c++}
\end{leftbar}

The \code{GrpcSession::Create} static factory method is primarily responsible for creating the \code{GrpcSession} instance and completing the corresponding initialization. In the initialization process, the most important thing is to build the \code{MasterInterface} instance; where \code{MasterInterface} is used for \ascii{Client} to access the \code{MasterService} remote service on \ascii{Master}, which exists Two subclasses are implemented, which correspond to two different application scenarios:

\begin{enum}
  \eitem{\code{LocalMaster}:\ascii{Client} is in the same process as \ascii{Master}, calling \code{LocalMaster::Lookup} to get the \code{LocalMaster} instance directly;
  \eitem{\code{GrpcRemoteMaster}:\ascii{Client} is not in the same process as \ascii{Master}, call the factory method \code{NewGrpcMaster} to generate the \code{GrpcRemoteMaster} instance. }
\end{enum}

The \code{GrpcRemoteMaster} instance is a client implementation of \ascii{RPC}. When creating a \code{GrpcRemoteMaster} instance, you need to create the \ascii{Master} address and service port specified by \code{target} first. Connected \ascii{RPC} channel.

\begin{leftbar}
\begin{c++}
Status GrpcSession::Create(
    const SessionOptions& options,
    std::unique_ptr<GrpcSession>* out_session) {
  std::unique_ptr<GrpcSession> session(new GrpcSession(options));
  std::unique_ptr<MasterInterface> master;
  // intra-process between client and master.
  if (!options.config.rpc_options().use_rpc_for_inprocess_master()) {
    master = LocalMaster::Lookup(options.target);
  }
  // inter-process between client and master.
  if (!master) {
    SharedGrpcChannelPtr master_channel;
    TF_RETURN_IF_ERROR(NewHostPortGrpcChannel(
        options.target.substr(strlen(kSchemePrefix)), &master_channel));
    master.reset(NewGrpcMaster(master_channel));
  }
  session->SetRemoteMaster(std::move(master));
  *out_session = std::move(session);
  return Status::OK();
}
\end{c++}
\end{leftbar}

\subsection{Create MasterSession}

As shown in \refig{dist-create-master-session-1}, when the \code{GrpcSession} instance is created successfully, the call to \code{GprcSession::Create} will be triggered, and the initial calculation graph will be passed to \code. The {CreateSessionRequst} message is sent to \ascii{Master}; when \ascii{Master} receives the \code{CreateSessionRequst} message, it generates the corresponding \code{MasterSession} instance and uses the globally unique \code{session\_handle } Identifies the instance and finally brings it back to \code{GrpcSession} via the \code{CreateSessionResponse} message.

\begin{figure}[H]
\centering
\includegraphics[width=0.6\textwidth]{figures/dist-create-master-session-1.png}
\caption{Create\code{MasterSession}}
 \label{fig:dist-create-master-session-1}
\end{figure}

% \begin{figure}[H]
% \centering
% \includegraphics[width=1.0\textwidth]{figures/dist-create-master-session.png}
% \caption{Create \code{MasterSession}}
%  \label{fig:dist-create-master-session}
% \end{figure}

\subsubsection{GrpcSesion::Create(graph\_def)}

The \code{GrpcSession::Create(graph\_def)} method is mainly used for \code{Client} requests\code{Master} to create \code{MasterSession} instances. First, the \code{GrpcSession::Create} method finishes constructing the \code{CreateSessionRequst} message and then sends it to \ascii{Master} via \code{GrpcRemoteMaster}.

When \code{GrpcSession} receives the \code{CreateSessionResponse} message, save \code{handle} of \code{MasterSession} and its version number \code{graph\_version}. Where \code{handle} is used to identify the \code{MasterSession} instance on the \ascii{Master} side, and \code{graph\_version} is used for subsequent extended calculation graphs.

\begin{leftbar}
\begin{c++}
void GrpcSession::BuildCreateSessionReq(
    const GraphDef& graph,
    CreateSessionRequest& req) {
  *req.mutable_config() = options_.config;
  *req.mutable_graph_def() = graph;
  req.set_target(options_.target);
}

void GrpcSession::SaveCreateSessionRsp(
    CreateSessionResponse& rsp) {
  mutex_lock l(mu_);
  swap(handle_, *(resp.mutable_session_handle()));
  current_graph_version_ = resp.graph_version();
}

Status GrpcSession::CreateImpl(CallOptions* call_options,
                               const GraphDef& graph) {
  CreateSessionRequest req;
  CreateSessionResponse resp;

  BuildCreateSessionReq(graph, req);
  Status s = master_->CreateSession(call_options, &req, &resp);
  if (s.ok()) {
    SaveCreateSessionRsp (respectively);
  }
  return s;
}

Status GrpcSession::Create(const RunOptions& run_options,
                           const GraphDef& graph) {
  CallOptions call_options;
  call_options.SetTimeout(run_options.timeout_in_ms());
  return CreateImpl(&call_options, graph);
}

Status GrpcSession::Create(const GraphDef& graph) {
  CallOptions call_options;
  call_options.SetTimeout(options_.config.operation_timeout_in_ms());
  return CreateImpl(&call_options, graph);
}
\end{c++}
\end{leftbar}

\subsubsection{GrpcRemoteMaster::CreateSession}

\code{GrpcRemoteMaster} is a client implementation of \ascii{gRPC}. Its implementation is very simple, call the corresponding service interface of the remote \ascii{Master} via a \code{stub} of \ascii{gRPC}.

\begin{leftbar}
\begin{c++}
Status GrpcRemoteMaster::CreateSession(
    CallOptions * call_options,
    const CreateSessionRequest* request,
    CreateSessionResponse* response) override {
  ::grpc::ClientContext ctx;
  SetClientContext(*call_options, ctx);
  return FromGrpcStatus(stub_->CreateSession(&ctx, *request, response));
}
\end{c++}
\end{leftbar}

\subsubsection{GrpcMasterService::CreateSessionHandler}

\code{GrpcMasterService} is a \ascii{gRPC} service that implements the \ascii{RPC} service interface for \code{MasterService}. When the \code{CreateSession} message is received, it will be processed by the \code{GrpcMasterService::CreateSessionHandler} callback, which will delegate \code{Master} to process the message.

When \code{Master} is processed, the \ascii{lambda} expression at the completion of the callback is returned, and the response message of \code{CreateSessionResponse} is returned to \ascii{Client}.

\begin{leftbar}
\begin{c++}
void GrpcMasterService::CreateSessionHandler(
  MasterCall<CreateSessionRequest, CreateSessionResponse>* call) {
  master_impl_->CreateSession(
    &call->request, &call->response,
    [call](const Status& status) {
        call->SendResponse(ToGrpcStatus(status));
    });
  ENQUEUE_REQUEST(CreateSession, true);
}
\end{c++}
\end{leftbar}

\subsubsection{Master::CreateSession}

\code{Master::CreateSession} will start a thread in the thread pool and look for all \ascii{Worker} in the thread according to the \code{cluster\_spec} information to collect information about the remote device set. Finally, a \code{MasterSession} was created.

\begin{remark}
The process of finding a remote device set is described in the next section. The implementation of this part of the code is omitted from the sample code in this section.
\end{remark}

When \code{MasterSession} is created successfully, \ascii{Master} will save the binary information of \code{(handle, master\_session)} so that subsequent \ascii{Master} can be indexed by \code{handle} The \code{MasterSession} instance.

\begin{leftbar}
\begin{c++}
using RemoveDevices = unique_ptr<vector<unique_ptr<Device>>>;

void Master::CreateSession(const CreateSessionRequest* req,
                           CreateSessionResponse* resp, MyClosure done) {
  SchedClosure([this, req, resp, done]() {
    // 1. Find all remote devices. 
    WorkerCacheInterface* worker_cache = env_->worker_cache;
    RemoveDevices remote_devices(new vector<unique_ptr<Device>>());

    Status status = DeviceFinder::GetRemoteDevices(
        req->config().device_filters(), env_,
        worker_cache, remote_devices.get())

    if (!status.ok()) return;

    // 2. Build DeviceSet
    std::unique_ptr<DeviceSet> device_set(new DeviceSet);
    for (auto&& d : *remote_devices) {
      device_set->AddDevice(d.get());
    }

    int num_local_devices = 0;
    for (Device* d : env_->local_devices) {
      device_set->AddDevice(d);
      if (num_local_devices == 0) {
        // Uses the first local device as the client device.
        device_set->set_client_device(d);
      }
      num_local_devices++;
    }

    // 3. Create MasterSession
    SessionOptions options;
    options.config = req->config();
    
    MasterSession* session = env_->master_session_factory(
        options, env_, std::move(remote_devices), 
        std::move(worker_cache_ptr), std::move(device_set));

    GraphDef* gdef =
        const_cast<CreateSessionRequest*>(req)->mutable_graph_def();
    
    // ignore worker\_cache\_factory\_options implements.
    WorkerCacheFactoryOptions worker_cache_factory_options;
    Status status = session->Create(gdef, worker_cache_factory_options);
    resp->set_session_handle(session->handle());
    
    // 4. Store <handle, master\_session> pair.
    {
      mutex_lock l(mu_);
      CHECK(sessions_.insert({session->handle(), session}).second);
    }
  });
}
\end{c++}
\end{leftbar}

\subsubsection{MasterSession::Create(graph\_def)}

\code{MasterSession::Create(graph\_def)} mainly accomplishes two things.

\begin{enum}
  \eitem{Initialize the calculation graph and generate an instance of \code{SimpleGraphExecutionState};}
  \eitem{If the cluster is dynamically configured, broadcast all \ascii{Worker} to create the corresponding \code{WorkerSession} instance. }
\end{enum}

The \code{SimpleGraphExecutionState::MakeForBaseGraph} implementation is the same as the local mode and will not be repeated here.

\begin{leftbar}
\begin{c++}
Status MasterSession::Create(
    GraphDef* graph_def,
    const WorkerCacheFactoryOptions& options) {
  SimpleGraphExecutionStateOptions execution_options;
  execution_options.device_set = devices_.get();
  execution_options.session_options = &session_opts_;
  {
    mutex_lock l(mu_);
    TF_RETURN_IF_ERROR(SimpleGraphExecutionState::MakeForBaseGraph(
        graph_def, execution_options, &execution_state_));
  }

  // CreateWorkerSessions should be called only with
  // dynamic cluster membership.
  if (options.cluster_def != nullptr) {
    return CreateWorkerSessions(options);
  }
  return Status::OK();
}
\end{c++}
\end{leftbar}

\subsection{Get remote device set}

As shown in \refig{dist-worker-get-status}, \code{MasterSession} polls all \code{Worker} instances before creating \code{MasterSession}, getting all remote \code{Worker} Device information. Get the remote device set by calling \code{DeviceFinder::GetRemoteDevices} with the device finder of \code{DeviceFinder}.

It works very simply, it gets a list of all \ascii{Worker} names in the cluster according to \code{GrpcWorkerCache::ListWorkers}; then, according to the name of \code{worker\_name}, call \code{GrpcWorkerCache: The :CreateWorker} factory method creates a \code{WorkerInterface} instance, which is used to access the remote \code{WorkerService} service. Finally, the \code{GetStatusRequest} request message is broadcasted to the remote \ascii{Worker} list via \code{WorkerInterface} to obtain the remote device set.

\begin{figure}[H]
\centering
\includegraphics[width=0.9\textwidth]{figures/dist-worker-get-status.png}
\caption{Get remote device set}
 \label{fig:dist-worker-get-status}
\end{figure}

\subsubsection{Device Finder}

\code{DeviceFinder} implements a function object that implements the algorithm for remote device lookups. The process is divided into three steps:

\begin{enum}
  \eitem{\code{Start}: concurrently broadcast \code{GetStatusRequest} to all \code{Worker} instances in the cluster;}
  \eitem{\code{Wait}: Collect all \code{GetStatusResponse} messages returned by \code{Worker};}
  \eitem{\code{GetRemoteDevices}: Get the query results and return them to the client;}
\end{enum}

\begin{leftbar}
\begin{c++}
struct DeviceFinder {
  static Status DeviceFinder::GetRemoteDevices(
      MasterEnv* env,
      WorkerCacheInterface* worker_cache,
      std::vector<std::unique_ptr<Device>>* out_remote) {
    DeviceFinder finder(env, worker_cache);
    finder.Start ();
    TF_RETURN_IF_ERROR(finder.Wait());
    finder.GetRemoteDevices(env->local_devices, out_remote);
    return Status::OK();
  }
};
\end{c++}
\end{leftbar}

In order to control the \code{GetStatusResponse} message of multiple \code{Worker}, the counter of \code{num\_pending\_} is used here, and the initial value set by \code{DeviceFinder::Start} is \code The number of {Worker}.

When a \code{GetStatusResponse} message from a \code{Worker} is received, the callback \code{WhenDone} is decremented by \ascii{1}. When the counter is reduced to \ascii{0}, wake up the \code{pending\_zero\_.wait\_for} statement by calling \code{pending\_zero\_.notify\_all}, then you can pass \ Code{finder.GetRemoteDevices} gets the query results.

Among them, in \code{DeviceFinder::Start}, all \code{Worker} broadcast \code{GetStatusRequest} messages are queried for corresponding device information through \code{NewRemoteDevices}. The next section will focus on the implementation process.

\begin{leftbar}
\begin{c++}
struct DeviceFinder {
 private:
  explicit DeviceFinder(
      MasterEnv* env,
      WorkerCacheInterface* worker_cache)
      : env_(env), worker_cache_(worker_cache) {
    worker_cache->ListWorkers(&targets_);
    seen_targets_.assign(targets_.size(), false);
  }

  ~DeviceFinder() {
    for (auto dev : found_) delete dev;
  }

  void Start() {
    {
      mutex_lock l(mu_);
      num_pending_ = targets_.size();
    }

    // Talk to all workers to get the list of available devices.
    using std::placeholders::_1;
    using std::placeholders::_2;
    for (size_t i = 0; i < targets_.size(); ++i) {
      NewRemoteDevices(env_->env, worker_cache_, targets_[i],
                       std::bind(&ME::WhenFound, this, i, _1, _2));
    }
  }

  // The caller takes the ownership of returned remote devices.
  void GetRemoteDevices(
      const std::vector<Device*>& local,
      std::vector<std::unique_ptr<Device>>* remote) {
    std::unordered_set<string> names(local.size());
    for (auto dev : local) {
      names.insert(dev->name());
    }

    mutex_lock l(mu_);
    for (auto dev : found_) {
      auto& name = dev->name();
      if (names.insert(name).second) {
        remote->push_back(std::unique_ptr<Device>(dev));
      } else {
        delete dev;
      }
    }
    found_.clear();
  }

  Status Wait() {
    mutex_lock l(mu_);
    while (num_pending_ != 0) {
      pending_zero_.wait_for(l, std::chrono::milliseconds(10 * 1000));
      if (num_pending_ != 0) {
        for (size_t i = 0; i < targets_.size(); ++i) {
          if (!seen_targets_[i]) {
            LOG(INFO)
                << "CreateSession still waiting for response from worker: "
                << targets_[i];
          }
        }
      }
    }
    return status_;
  }

  void WhenFound(int target_index, const Status& s,
                 std::vector<Device*>* devices) {
    mutex_lock l(mu_);
    seen_targets_[target_index] = true;
    if (!s.ok()) {
      status_.Update(s);
    } else {
      found_.insert(found_.end(), devices->begin(), devices->end());
      devices->clear();
    }
    --num_pending_;
    if (num_pending_ == 0) {
      pending_zero_.notify_all();
    }
  }

  typedef DeviceFinder ME;
  const MasterEnv* env_;
  WorkerCacheInterface* worker_cache_;

  mutex mu_;
  int num_pending_ GUARDED_BY(mu_);
  condition_variable pending_zero_;
  std::vector<Device*> found_ GUARDED_BY(mu_);

  std::vector<string> targets_;
  std::vector<bool> seen_targets_ GUARDED_BY(mu_);
  Status status_;
};
\end{c++}
\end{leftbar}

Tips: When the \code{num\_pending\_} counter is not zero, the main thread periodically sleeps for \code{10} seconds. If you find that \ascii{Worker} has not returned a response message when you wake up, Then print the names of those \ascii{Worker}. When you see the following information printed cyclically, you should understand whether \code{Server} corresponding to \code{/job:worker/task:2} exits abnormally, or \ascii corresponding to \ascii{Master} Whether there is an abnormality in the network between {Worker}, etc., analyze and process according to the specific situation.

\begin{leftbar}
\ Begin {python}
CreateSession still waiting for response from worker: /job:worker/task:2
\end{python}
\end{leftbar}

\subsubsection{NewRemoteDevices}

\code{NewRemoteDevices} will look for the \code{WorkerInterface} instance based on \code{worker\_name} and send a \code{GetStatusRequest} message to the corresponding \code{Worker} to get its device information. When the message is returned, the function object of \code{cb} will be called back. The device information obtained from the remote \code{Worker} is not complete. It does not contain the information of \code{worker\_name}, so it needs to be manually added.

\begin{leftbar}
\begin{c++}
void NewRemoteDevices(
    Env* env, WorkerCacheInterface* worker_cache,
    const string& worker_name, NewRemoteDevicesDone done) {
  struct Call {
    GetStatusRequest req;
    GetStatusResponse resp;
  };

  WorkerInterface* wi = worker_cache->CreateWorker(worker_name);
  Call* call = new Call;
  auto cb = [env, worker_cache, &worker_name, &done, wi, call](
      const Status& status) {
    Status s = status;
    std::vector<Device*> remote_devices;
    auto cleanup = gtl::MakeCleanup(
        [worker_cache, &worker_name, wi, &done, &remote_devices, &s, call] {
          worker_cache->ReleaseWorker(worker_name, wi);
          done(s, &remote_devices);
          delete call;
        });
    if (s.ok()) {
      DeviceNameUtils::ParsedName worker_name_parsed;
      DeviceNameUtils::ParseFullName(worker_name, &worker_name_parsed);

      remote_devices.reserve(call->resp.device_attributes_size());

      for (auto& da : call->resp.device_attributes()) {
        DeviceNameUtils::ParsedName device_name_parsed;
        DeviceNameUtils::ParseFullName(da.name(), &device_name_parsed);
        
        DeviceAttributes da_rewritten = da;
        da_rewritten.set_name(DeviceNameUtils::FullName(
            worker_name_parsed.job, worker_name_parsed.replica,
            worker_name_parsed.task, device_name_parsed.type,
            device_name_parsed.id));
        auto d = new RemoteDevice(env, da_rewritten);
        remote_devices.push_back(d);
      }
    }
  };
  wi->GetStatusAsync(&call->req, &call->resp, cb);
}
\end{c++}
\end{leftbar}

\subsubsection{GrpcRemoteWorker::GetStatusAsync}

\code{GrpcRemoteWorker} is a concrete implementation of \code{WorkerInterface}, which is a client implementation of \ascii{gRPC}, which calls the corresponding service interface of the remote \code{WorkerService} via \code{stub}.

\begin{leftbar}
\begin{c++}
struct GrpcRemoteWorker : WorkerInterface {
  void GetStatusAsync(const GetStatusRequest* request,
                      GetStatusResponse* response,
                      StatusCallback done) override {
    IssueRequest(request, response, getstatus_, std::move(done));
  }
}
\end{c++}
\end{leftbar}

\subsubsection{GrpcRemoteWorker::GetStatusAsync}

\code{GrpcWorkerService} is a concrete implementation of \code{WorkerService}. When the \code{GetStatusRequest} message is received, it will be handled by the \code{GetStatusHandler} callback.

\begin{leftbar}
\begin{c++}
struct GrpcWorkerService : AsyncServiceInterface {
  void GetStatusHandler(WorkerCall<GetStatusRequest, GetStatusResponse>* call) {
    Schedule([this, call]() {
      Status s = worker_->GetStatus(&call->request, &call->response);
      call->SendResponse(ToGrpcStatus(s));
    });
    ENQUEUE_REQUEST(GetStatus, false);
  }
};
\end{c++}
\end{leftbar}

\subsubsection{Worker::GetStatusAsync}

\code{Worker::GetStatusAsync} will delegate \code{DeviceMgr} to the local device information summary, and finally return to the peer through the \code{GetStatusResponse} message.

\begin{leftbar}
\begin{c++}
void Worker::GetStatusAsync(const GetStatusRequest* request,
                            GetStatusResponse* response, StatusCallback done) {
  std::vector<DeviceAttributes> devices;
  env_->device_mgr->ListDeviceAttributes(&devices);
  response->mutable_device_attributes()->Reserve(devices.size());
  for (auto& d : devices) {
    response->add_device_attributes()->Swap(&d);
  }
  done(Status::OK());
}
\end{c++}
\end{leftbar}

\code{DeviceMgr} holds a local device set and is extremely simple to implement.

\begin{leftbar}
\begin{c++}
void DeviceMgr::ListDeviceAttributes(
    std::vector<DeviceAttributes>* devices) const {
  devices->reserve(devices_.size());
  for (auto dev : devices_) {
    devices->emplace_back(dev->attributes());
  }
}
\end{c++}
\end{leftbar}

\subsection{Create WorkerSession}

When \code{MasterSession} is created successfully, if there is no dynamic configuration of the cluster (the default distributed configuration environment), all \ascii{Worker} will not be broadcast dynamically to create \code{WorkerSession}. In fact, every \ascii{Worker} has a \code{SessionMgr} instance that holds a \code{WorkerSession} instance named \code{legacy\_session\_}. Therefore, each \ascii{Worker} has a globally unique \code{WorkerSession} instance.

\begin{leftbar}
\begin{c++}
SessionMgr::SessionMgr(
    WorkerEnv* worker_env, 
    const string& default_worker_name,
    std::unique_ptr<WorkerCacheInterface> default_worker_cache,
    WorkerCacheFactory worker_cache_factory)
    : worker_env_(worker_env),
      legacy_session_(
          default_worker_name, 
          std::move(default_worker_cache),
          std::unique_ptr<DeviceMgr>(worker_env->device_mgr),
          std::unique_ptr<GraphMgr>(
              new GraphMgr(worker_env, 
              worker_env->device_mgr))),
      worker_cache_factory_(std::move(worker_cache_factory)) {}
\end{c++}
\end{leftbar}

As shown in \refig{dist-create-worker-session}, if there is a dynamic cluster configuration, \ascii{Master} broadcasts each \ascii{Worker} to create a \code{WorkerSession} instance and uses \code{ Sessin\_handle} identifies the \code{WorkerSession}. These \code{WorkerSession} are part of this \code{MasterSession} instance because they use the same \code{session\_handle} identifier as the \code{MasterSession} instance.

Among them, \code{MasterSession} introduces the \code{BlockingCounter} counter in order to collect all the \code{CreateWorkerSessionResponse} messages returned by \ascii{Worker}. The initial value of the \code{BlockingCounter} counter is the number of \ascii{Worker}. When receiving the response message of each \ascii{Worker}, the counter is decremented by \ascii{1} until the counter is \ascii{0}. \code{done.Wait()} is woken up.

In addition, the \code{WorkerInterface} instance is queried or created via \code{WorkerCacheInterface}, which will be explained in more detail later.

\begin{figure}[H]
\centering
\includegraphics[width=0.8\textwidth]{figures/dist-create-worker-session.png}
\caption{Dynamic creation\code{WorkerSession}}
 \label{fig:dist-create-worker-session}
\end{figure}

\begin{leftbar}
\begin{c++}
struct MasterSession::Worker {
  Worker(MasterSession* sess, const string& name,
         const DeviceNameUtils::ParsedName& parsed_name,
         const WorkerCacheFactoryOptions& opts)
      : sess(sess), name(&name), worker(GetOrCreateWorker()) {
    BuildRequest(parsed_name, opts);
  }

  void CreateWorkerSession(BlockingCounter& done, Status& status) {
    auto cb = [&status, &done](const Status& s) {
      status.Update(s);
      done.DecrementCount();
    };
    // IMPORTANT: notify worker to create worker session.
    worker->CreateWorkerSessionAsync(&request, &response, cb);
  }

  void Release() {
    if (worker != nullptr) {
      sess->worker_cache_->ReleaseWorker(*name, worker);
    }
  }

 private:
  WorkerInterface* GetOrCreateWorker() {
    return sess->worker_cache_->CreateWorker(*name);
  }

  void BuildRequest(const DeviceNameUtils::ParsedName& parsed_name,
                    const WorkerCacheFactoryOptions& opts) {
    request.set_session_handle(sess->handle_);
    BuildServerDef(parsed_name, opts, request.mutable_server_def());
  }

  void BuildServerDef(const DeviceNameUtils::ParsedName& parsed_name,
                      const WorkerCacheFactoryOptions& opts,
                      ServerDef* server_def) {
    *server_def->mutable_cluster() = *opts.cluster_def;
    server_def->set_protocol(*opts.protocol);
    server_def->set_job_name(parsed_name.job);
    server_def->set_task_index(parsed_name.task);
  }

 private:
  MasterSession * sex;

  // The worker name. (Not owned.)
  const string* name;

  // The worker referenced by name. (Not owned.)
  WorkerInterface* worker = nullptr;

  // Request and responses used for a given worker.
  CreateWorkerSessionRequest request;
  CreateWorkerSessionResponse response;
};

struct MasterSession::WorkerGroup {
  WorkerGroup (MasterSession * sex): gender (sex) {}

  Status CreateWorkerSessions(const WorkerCacheFactoryOptions& opts) {
    TF_RETURN_IF_ERROR(CreateWorkers(opts));
    TF_RETURN_IF_ERROR(BroadcastWorkers());
    return Status::OK();
  }

  void ReleaseWorkers() {
    for (auto& worker : workers) {
      worker.Release();
    }
  }

 private:
  Status CreateWorkers(const WorkerCacheFactoryOptions& opts) {
    sess->worker_cache_->ListWorkers(&worker_names);
    for (auto& worker_name : worker_names) {
      TF_RETURN_IF_ERROR(AppendWorker(worker_name, opts));
    }
    return Status::OK();
  }

  // broadcast all workers to create worker session.
  Status BroadcastWorkers() {
    Status status = Status::OK();
    BlockingCounter done(workers.size());
    for (auto& worker : workers) {
      worker.CreateWorkerSession(done, status);
    }
    done.Wait();
    return status;
  }

  Status AppendWorker(const string& worker_name,
                    const WorkerCacheFactoryOptions& opts) {
    DeviceNameUtils::ParsedName parsed_name;
    TF_RETURN_IF_ERROR(ParseWorkerName(worker_name, &parsed_name));
    workers.emplace_back(Worker(sess, worker_name, parsed_name, opts));
    return Status::OK();
  }

  Status ParseWorkerName(const string& worker_name,
                         DeviceNameUtils::ParsedName* parsed_name) {
    if (!DeviceNameUtils::ParseFullName(worker_name, parsed_name)) {
      return errors::Internal("Could not parse name ", worker_name);
    }
    if (!parsed_name->has_job || !parsed_name->has_task) {
      return errors::Internal("Incomplete worker name ", worker_name);
    }
    return Status::OK();
  }

 private:
  MasterSession * sex;
  std::vector<string> worker_names;
  std::vector<Worker> workers;
};

Status MasterSession::CreateWorkerSessions(
    const WorkerCacheFactoryOptions& options) {
  CHECK(worker_cache_) << "CreateWorkerSessions should be called only with "
                       << "dynamic cluster membership.";

  WorkerGroup worker_group(this);

  // Release the workers.
  auto cleanup = gtl::MakeCleanup([&worker_group] {
    worker_group.ReleaseWorkers();
  });

  return worker_group.CreateWorkerSessions(options);
}
\end{c++}
\end{leftbar}

\subsubsection{GrpcRemoteWorker}

\code{GrpcRemoteWorker} is the \ascii{gRPC} client that accesses the remote \ascii{Worker}. It calls the corresponding \code{stub} to call the remote service.

\begin{leftbar}
\begin{c++}
struct GrpcRemoteWorker : WorkerInterface {
  void CreateWorkerSessionAsync(
      const CreateWorkerSessionRequest* request,
      CreateWorkerSessionResponse* response,
      StatusCallback done) override {
    IssueRequest(request, response, createworkersession_, std::move(done));
  }
};
\end{c++}
\end{leftbar}

\subsubsection{GrpcWorkerService::CreateWorkerSessionHandler}

On the \ascii{Worker} side, the \code{CreateWorkerSession} message is handled by the \code{CreateWorkerSessionHandler} callback. It starts a runnable thread in the thread pool and triggers \code{Worker} to dynamically create a \code{WorkerSession} instance.

\begin{leftbar}
\begin{c++}
struct GrpcWorkerService : AsyncServiceInterface {
  void CreateWorkerSessionHandler(
      WorkerCall<CreateWorkerSessionRequest, CreateWorkerSessionResponse>*
          call) {
    Schedule([this, call]() {
      Status s = worker_->CreateWorkerSession(&call->request, &call->response);
      call->SendResponse(ToGrpcStatus(s));
    });
    ENQUEUE_REQUEST(CreateWorkerSession, false);
  }
};
\end{c++}
\end{leftbar}

\subsubsection{Create a WorkerSession instance}

\code{Worker} delegates the responsibility for creating a \code{WorkerSession} instance to \code{SessionMgr}, which manages and maintains the lifecycle of all \code{WorkerSession} instances. As shown by \refig{dist-worker-session-manager}, \code{SessionMgr} may hold multiple instances of \code{WorkerSession}, and each \code{WorkerSession} instance is identified by \code{session\_handle}.

\begin{figure}[H]
\centering
\includegraphics[width=0.6\textwidth]{figures/dist-worker-session-manager.png}
\caption{\code{Session}Manager}
 \label{fig:dist-worker-session-manager}
\end{figure}

\begin{leftbar}
\begin{c++}
void Worker::CreateWorkerSessionAsync(
    const CreateWorkerSessionRequest* request,
    CreateWorkerSessionResponse* response,
    StatusCallback done) {
  Status s = env_->session_mgr->CreateSession(
      request->session_handle(),
      request->server_def());
  done(s);
}
\end{c++}
\end{leftbar}

As shown in \refig{dist-worker-session-model}, \code{WorkerSession} holds a \code{GraphMgr} instance for registering and running multiple graph instances. Among them, each graph instance uses the \code{graph\_handle} identifier. At the same time, each \code{WorkerSession} holds a \code{DeviceMgr} instance that manages the collection of local computing devices.

\begin{figure}[H]
\centering
\includegraphics[width=1.0\textwidth]{figures/dist-worker-session-model.png}
\caption{\code{WorkerSession} domain model: multiple graph instances can be registered and run}
 \label{fig:dist-worker-session-model}
\end{figure}

\begin{leftbar}
\begin{c++}
Status SessionMgr::CreateSession(const string& session,
                                 const ServerDef& server_def) {
  mutex_lock l(mu_);

  // 1. Create WorkerCacheInterface
  WorkerCacheInterface* worker_cache = nullptr;
  TF_RETURN_IF_ERROR(worker_cache_factory_(server_def, &worker_cache));

  // 2. Rename local devices  
  auto worker_name = WorkerNameFromServerDef(server_def);
  std::vector<Device*> renamed_devices;
  for (Device* d : worker_env_->local_devices) {
    renamed_devices.push_back(
        RenamedDevice::NewRenamedDevice(worker_name, d, false));
  }
  std::unique_ptr<DeviceMgr> device_mgr(new DeviceMgr(renamed_devices));

  // 3. Create GraphMgr
  std::unique_ptr<GraphMgr> graph_mgr(
      new GraphMgr(worker_env_, device_mgr.get()));
  
  // 4. Create WorkerSession
  std::unique_ptr<WorkerSession> worker_session(new WorkerSession(
      worker_name, std::unique_ptr<WorkerCacheInterface>(worker_cache),
      std::move(device_mgr), std::move(graph_mgr)));

  // 5. Store (session\_handle, WorkerSession) pair.
  sessions_.insert(std::make_pair(session, std::move(worker_session)));
  return Status::OK();
}
\end{c++}
\end{leftbar}

\end{content}

\section{iteration execution}

\begin{content}

\subsection{Start execution}

\begin{figure}[H]
\centering
\includegraphics[width=0.6\textwidth]{figures/dist-run-step-stage-1.png}
\ caption {GprcSession: 启动 RunStep}
 \label{fig:dist-run-step-stage-1}
\end{figure}

\subsubsection{GrpcSession::Run}

\begin{leftbar}
\begin{c++}
namespace {
  using TensorIndex = std::unordered_map<string, int>;

  void BuildReqOptions(const SessionOptions& sess_options,
      const RunOptions& run_options, 
      RunOptions& options) {
    options = run_options;
    if (run_options.timeout_in_ms() == 0) {
      options.set_timeout_in_ms(
          sess_options.config.operation_timeout_in_ms());
    }    
  }

  void BuildReqFeeds(const vector<pair<string, Tensor>>& inputs,
      MutableRunStepRequestWrapper* req) {
    for (auto& it : inputs) {
      req->add_feed(it.first, it.second);
    }
  }

  void BuildReqFetches(const std::vector<string>& output_names,
      MutableRunStepRequestWrapper* req) {
    for (int i = 0; i < output_names.size(); ++i) {
      req->add_fetch(output_names[i]);
  }

  void BuildReqTargets(const std::vector<string>& target_names,
      MutableRunStepRequestWrapper* req) {
    for (string& target : target_names) {
      req->add_target(target);
    }
  }

  void BuildRunStepReq(
      const SessionOptions& sess_options,
      const RunOptions& run_options,
      const vector<pair<string, Tensor>>& inputs,
      const std::vector<string>& output_names,
      const std::vector<string>& target_names,
      MutableRunStepRequestWrapper* req) {
    BuildReqOptions(sess_options, run_options, 
        req->mutable_options());
    BuildReqFeeds(inputs, req);
    BuildReqFetches(output_names, req);
    BuildReqTargets(target_names, req); 
  }

  void BuildOuputNamesIndex(
      const std::vector<string>& output_names,
      TensorIndex& tensor_index) {
    for (int i = 0; i < output_names.size(); ++i) {
      const string& name = output_names[i];
      tensor_index.insert(make_pair(name, i));
    }
  }

  void BuildCallOptions(const RunOptions& options, 
      CallOptions & call_options) {
    call_options.SetTimeout(options.timeout_in_ms());
  }

  Status DoSaveOutputs(const TensorIndex& tensor_index,
      const std::vector<string>& output_names,
      MutableRunStepResponseWrapper* resp,
      std::vector<Tensor>* outputs) {
    for (size_t i = 0; i < resp->num_tensors(); ++i) {
      auto fetch_it = tensor_index.find(resp->tensor_name(i));
      if (fetch_it == tensor_index.end()) {
        return errors::Internal(
           "unrequested fetch: ", resp->tensor_name(i));
      }

      Tensor output;
      TF_RETURN_IF_ERROR(resp->TensorValue(i, &output));
      (*outputs)[fetch_it->second] = output;
    }  
  }

  Status SaveOutputs(const TensorIndex& tensor_index,
      const std::vector<string>& output_names,
      MutableRunStepResponseWrapper* resp,
      std::vector<Tensor>* outputs) {
    if (!output_names.empty()) {
      outputs->resize(output_names.size());
    }
    return DoSaveOutputs(tensor_index, 
        output_names, rsep, outputs);
  }

  void SaveRunMetaData(MutableRunStepResponseWrapper* resp,
      RunMetadata* run_metadata) {
    if (run_metadata) {
      run_metadata->Swap(resp->mutable_metadata());
    }
  }
  
  Status SaveRspToOutputs(const TensorIndex& tensor_index,
      const std::vector<string>& output_names,
      MutableRunStepResponseWrapper* resp,
      std::vector<Tensor>* outputs,
      RunMetadata* run_metadata) {
    SaveRunMetaData(resp, run_metadata);
    return SaveOutputs(tensor_index, output_names, rsep, outputs);
  }
}

Status GrpcSession::Run(
    const RunOptions& run_options,
    const vector<pair<string, Tensor>>& inputs,
    const vector<string>& output_names,
    const vector<string>& target_names,
    std::vector<Tensor>* outputs,
    RunMetadata* run_metadata) {
  // 1. Build run step request.
  unique_ptr<MutableRunStepRequestWrapper> req(
      master_->CreateRunStepRequest());

  unique_ptr<MutableRunStepResponseWrapper> resp(
      master_->CreateRunStepResponse());

  BuildRunStepReq(options_, run_options, inputs, 
      output_names, target_names, req.get());

  // 2. Build output tensor names index.
  TensorIndex tensor_index;
  BuildOuputNamesIndex(output_names, tensor_index);

  // 3. Build call options.
  CallOptions call_options;
  BuildCallOptions(req->options(), call_options)

  // 4. Do run step.
  TF_RETURN_IF_ERROR(RunProto(&call_options, 
      req.get(), resp.get()));

  // 5. Save response to outputs.
  return SaveRspToOutputs(tensor_index, output_names, 
      resp.get(), outputs, run_metadata);
}
\end{c++}
\end{leftbar}

\begin{leftbar}
\begin{c++}
Status GrpcSession::RunProto(
    CallOptions * call_options,
    MutableRunStepRequestWrapper* req,
    MutableRunStepResponseWrapper* resp) {
  {
    mutex_lock l(mu_);
    req->set_session_handle(handle_);
  }
  return master_->RunStep(call_options, req, resp);
}
\end{c++}
\end{leftbar}

\ Subsubsection {GrpcRemoteMaster :: RunStep}

\begin{leftbar}
\begin{c++}
struct GrpcRemoteMaster : MasterInterface {
  using MasterServiceStub = ::grpc::MasterService::Stub;

  Status RunStep(CallOptions* call_options, RunStepRequestWrapper* request,
                 MutableRunStepResponseWrapper* response) override {
    ::grpc::ClientContext ctx;
    return Call(&ctx, call_options, &request->ToProto(),
                get_proto_from_wrapper(response),
                & MasterServiceStub :: RunStep);
  }
};
\end{c++}
\end{leftbar}

\ Subsubsection {GrpcMasterService :: RunStepHandler}

\begin{leftbar}
\begin{c++}
struct GrpcMasterService : AsyncServiceInterface {
  using RunStepCall = MasterCall <RunStepRequest, RunStepResponse>;
 
  void RunStepHandler (RunStepCall * call) {
    CallOptions* call_opts = CreateCallOptions(call);

    RunStepRequestWrapper* wrapped_request =
        new ProtoRunStepRequest(&call->request);

    MutableRunStepResponseWrapper* wrapped_response =
        new NonOwnedProtoRunStepResponse(&call->response);
  
    call->SetCancelCallback([call_opts]() { 
        call_opts->StartCancel(); 
    });

    master_impl_->RunStep(call_opts, wrapped_request, wrapped_response,
      [call, call_opts, wrapped_request, wrapped_response](
          const Status& status) {
        call->ClearCancelCallback();
        delete call_opts;
        delete wrapped_request;
        call->SendResponse(ToGrpcStatus(status));
      });
    ENQUEUE_REQUEST(RunStep, true);
  }

 private:
  CallOptions * CreateCallOptions (RunStepCall * call) {
    CallOptions * call_opts = new CallOptions;
    if (call->request.options().timeout_in_ms() > 0) {
      call_opts->SetTimeout(call->request.options().timeout_in_ms());
    } else {
      call_opts->SetTimeout(default_timeout_in_ms_);
    }
    return call_opts; 
  }
};
\end{c++}
\end{leftbar}

\ Subsubsection {Master :: RunStep}

\begin{leftbar}
\begin{c++}
void Master::RunStep(CallOptions* opts, 
    const RunStepRequestWrapper * req,
    MutableRunStepResponseWrapper* resp, 
    DoneClosure done) {
  auto session = FindMasterSession(req->session_handle());
  SchedClosure([this, session, opts, req, resp, done]() {
    Status status = session->Run(opts, *req, resp);
    session->Unref();
    done(status);
  });
}
\end{c++}
\end{leftbar}

\subsubsection{MasterSession::Run}

\begin{leftbar}
\begin{c++}
Status MasterSession::Run(
    CallOptions * opts, 
    const RunStepRequestWrapper & req,
    MutableRunStepResponseWrapper* resp) {
  Status status;
  if (!req.partial_run_handle().empty()) {
    status = DoPartialRun(opts, req, resp);
  } else {
    status = DoRunWithLocalExecution(opts, req, resp);
  }
  return status;
}
\end{c++}
\end{leftbar}

\begin{leftbar}
\begin{c++}
Status MasterSession::DoRunWithLocalExecution(
    CallOptions* opts, const RunStepRequestWrapper& req,
    MutableRunStepResponseWrapper* resp) {

  // 1. Prune: build ReffedClientGraph. 
  BuildGraphOptions bgopts;
  BuildBuildGraphOptions(req, &bgopts);
  
  ReffedClientGraph* rcg = nullptr;
  int64 count = 0;
  TF_RETURN_IF_ERROR(StartStep(bgopts, &count, &rcg, false));

  // 2. Build and Register partitions to workers. 
  core::ScopedUnref unref(rcg);
  TF_RETURN_IF_ERROR(BuildAndRegisterPartitions(rcg));

  // 3. Run partitions: notify all of workers to run partitions.
  uint64 step_id = (random::New64() & ((1uLL << 56) - 1)) | (1uLL << 56);
  Status s = rcg->RunPartitions(env_, step_id, count, &pss, opts, req, resp,
                                &cancellation_manager_, false);
  // 4. Cleaup Partitions: notify all of workers to clearup partitions.
  Ref();
  rcg->Ref();
  rcg->CleanupPartitionsAsync(step_id, [this, rcg](const Status& s) {
    rcg->Unref();
    Unref();
  });
  return s;
}
\end{c++}
\end{leftbar}

\subsubsection{MasterSession::BuildAndRegisterPartitions}

\begin{leftbar}
\begin{c++}
Status MasterSession::BuildAndRegisterPartitions(ReffedClientGraph* rcg) {
  PartitionOptions popts;
  popts.node_to_loc = SplitByWorker; // IMPORTANT
  popts.flib_def = rcg->client_graph()->flib_def.get();
  popts.control_flow_added = false;

  popts.new_name = [this](const string& prefix) {
    mutex_lock l(mu_);
    return strings::StrCat(prefix, "_S", next_node_id_++);
  };

  popts.get_incarnation = [this](const string& name) -> int64 {
    auto d = devices_->FindDeviceByName(name);
    return d->attributes().incarnation();
  };

  TF_RETURN_IF_ERROR(rcg->RegisterPartitions(popts));
  return Status::OK();
}
\end{c++}
\end{leftbar}

\subsubsection{ReffedClientGraph::RegisterPartitions}

\begin{leftbar}
\begin{c++}
Status ReffedClientGraph::RegisterPartitions(
    const PartitionOptions& popts) {
  { 
    mu_.lock();
    if (!init_started_) {
      init_started_ = true;
      Mu_.unlock();

      std::unordered_map<string, GraphDef> graph_defs;
      Status s = DoBuildPartitions(popts, &graph_defs);
      if (s.ok()) {
        s = DoRegisterPartitions(popts, std::move(graph_defs));
      }

      mu_.lock();
      init_result_ = s;
      init_done_.Notify();
    } else {
      Mu_.unlock();
      init_done_.WaitForNotification();
      mu_.lock();
    }
    Status result = init_result_;
    Mu_.unlock();
    return result;
  }
}
\end{c++}
\end{leftbar}

\subsection{Figure split: SplitByWorker}

\subsubsection{ReffedClientGraph::DoBuildPartitions}

\begin{leftbar}
\begin{c++}
Status MasterSession::ReffedClientGraph::DoBuildPartitions(
    PartitionOptions popts,
    std::unordered_map<string, GraphDef>* out_partitions) {
  // split full graph by worker name.
  return Partition(popts, &client_graph_->graph, out_partitions);
}
\end{c++}
\end{leftbar}

\subsection{Register Map}

\begin{figure}[H]
\centering
\includegraphics[width=0.8\textwidth]{figures/dist-run-step-stage-2.png}
\caption{RegisterGraph}
 \label{fig:dist-run-step-stage-2}
\end{figure}

\subsubsection{ReffedClientGraph::DoRegisterPartitions}

\begin{leftbar}
\begin{c++}
Status ReffedClientGraph::DoRegisterPartitions(
    const PartitionOptions& popts,
    std::unordered_map<string, GraphDef> graph_partitions) {
  partitions_.reserve(graph_partitions.size());
  Status s;
  for (auto& name_def : graph_partitions) {
    partitions_.resize(partitions_.size() + 1);
    Part* part = &partitions_.back();
    part->name = name_def.first;
    TrackFeedsAndFetches(part, name_def.second, popts);
    part->worker = worker_cache_->CreateWorker(part->name);
  }

  struct Call {
    RegisterGraphRequest req;
    RegisterGraphResponse resp;
    Status status;
  };

  const int num = partitions_.size();
  gtl::InlinedVector<Call, 4> calls(num);

  BlockingCounter done(num);
  for (int i = 0; i < num; ++i) {
    const Part& part = partitions_[i];
    Call* c = &calls[i];
    
    c->req.set_session_handle(session_handle_);
    c->req.mutable_graph_def()->Swap(&graph_partitions[part.name]);
    *c->req.mutable_graph_options() = session_opts_.config.graph_options();
    *c->req.mutable_debug_options() = debug_opts_;

    auto cb = [c, &done](const Status& s) {
      c->status = s;
      done.DecrementCount();
    };
    part.worker->RegisterGraphAsync(&c->req, &c->resp, cb);
  }
  done.Wait();

  for (int i = 0; i < num; ++i) {
    Call* c = &calls[i];
    s.Update(c->status);
    partitions_[i].graph_handle = c->resp.graph_handle();
  }
  return s;
}
\end{c++}
\end{leftbar}

\subsubsection{GrpcRemoteWorker::RegisterGraphAsync}

\begin{leftbar}
\begin{c++}
class GrpcRemoteWorker : public WorkerInterface {
  void RegisterGraphAsync(const RegisterGraphRequest* request,
                          RegisterGraphResponse* response,
                          StatusCallback done) override {
    IssueRequest(request, response, registergraph_, std::move(done));
  }

  void IssueRequest(const protobuf::Message* request,
                    protobuf::Message* response, const ::grpc::string& method,
                    StatusCallback done, CallOptions* call_opts = nullptr) {
    new RPCState<protobuf::Message>(counter_, &stub_, cq_, method, *request,
                                    response, std::move(done), call_opts);
  }
};
\end{c++}
\end{leftbar}

\subsubsection{GrpcWorkerService::RegisterGraphHandler}

\begin{leftbar}
\begin{c++}
class GrpcWorkerService : public AsyncServiceInterface {
  void RegisterGraphHandler(
      WorkerCall<RegisterGraphRequest, RegisterGraphResponse>* call) {
    Schedule([this, call]() {
      Status s = worker_->RegisterGraph(&call->request, &call->response);
      call->SendResponse(ToGrpcStatus(s));
    });
    ENQUEUE_REQUEST(RegisterGraph, false);
  }
};
\end{c++}
\end{leftbar}

\subsubsection{Worker::RegisterGraphAsync}

\begin{leftbar}
\begin{c++}
void Worker::RegisterGraphAsync(
    const RegisterGraphRequest* request,
    RegisterGraphResponse* response,
    StatusCallback done) {
  auto session = FindWorkerSession(request);
  Status s = session->graph_mgr->Register(
      request->session_handle(), 
      request->graph_def(), 
      request->graph_options(),
      response->mutable_graph_handle());
  done(s);
}
\end{c++}
\end{leftbar}

\subsubsection{GraphMgr::Register}

\begin{leftbar}
\begin{c++}
Status GraphMgr::Register(
    const string& session, 
    const GraphDef& gdef,
    const GraphOptions& graph_options,
    string* handle) {
  Item* item = new Item;
  Status s = InitItem (session, gdef, graph_options, item);
  if (!s.ok()) {
    item->Unref();
    return s;
  }

  // Generate unique graph\_handle, 
  // and register [graph\_handle, graph\_def] to table.
  {
    mutex_lock l(mu_);
    *handle = strings::Printf("%016llx", ++next_id_);
    item->handle = *handle;
    CHECK(table_.insert({*handle, item}).second);
  }
  return Status::OK();
}
\end{c++}
\end{leftbar}

\subsection{Figure split: SplitByDevice}

\begin{leftbar}
\begin{c++}
Status GraphMgr :: InitItem (
    const string& session, const GraphDef& gdef,
    const GraphOptions& graph_options,
    Item* item) {
  item->session = session;
  item->lib_def.reset(
      new FunctionLibraryDefinition(OpRegistry::Global(), gdef.library()));

  item->proc_flr.reset(new ProcessFunctionLibraryRuntime(
      device_mgr_, worker_env_->env, gdef.versions().producer(),
      item->lib_def.get(), graph_options.optimizer_options()));

  // 1. Constructs the full graph out of "gdef"
  Graph graph(OpRegistry::Global());
  GraphConstructorOptions opts;
  opts.allow_internal_ops = true;
  opts.expect_device_spec = true;
  TF_RETURN_IF_ERROR(ConvertGraphDefToGraph(opts, gdef, &graph));

  // 2. Splits "graph" into multiple subgraphs by device names.
  std::unordered_map<string, GraphDef> partitions;
  PartitionOptions popts;
  popts.node_to_loc = SplitByDevice;  // IMPORTANT.
  popts.new_name = [this](const string& prefix) {
    mutex_lock l(mu_);
    return strings::StrCat(prefix, "_G", next_id_++);
  };
  popts.get_incarnation = [this](const string& name) -> int64 {
    Device* device = nullptr;
    Status s = device_mgr_->LookupDevice(name, &device);
    if (s.ok()) {
      return device->attributes().incarnation();
    } else {
      return PartitionOptions::kIllegalIncarnation;
    }
  };
  popts.flib_def = &graph.flib_def();
  popts.control_flow_added = true;
  popts.scheduling_for_recvs = graph_options.enable_recv_scheduling();
  
  // IMPORTANT.  
  TF_RETURN_IF_ERROR(Partition(popts, &graph, &partitions));

  // 3. convert GraphDef partitions to Graph partitions.
  std::unordered_map<string, std::unique_ptr<Graph>> partition_graphs;
  for (const auto& partition : partitions) {
    std::unique_ptr<Graph> device_graph(new Graph(OpRegistry::Global()));
    GraphConstructorOptions device_opts;
    // There are internal operations (e.g., send/recv) that we now allow.
    device_opts.allow_internal_ops = true;
    device_opts.expect_device_spec = true;
    TF_RETURN_IF_ERROR(ConvertGraphDefToGraph(device_opts, partition.second,
                                              device_graph.get()));
    partition_graphs.emplace(partition.first, std::move(device_graph));
  }

  // 4. Build executors\_and\_partitions(item->units) = [(e0, p0), 
  // (e1, p1), ...], and (e\_n, p\_n) is called ExecutionUnit.
  LocalExecutorParams params;
  item->units.reserve(partitions.size());
  item->graph_mgr = this;

  for (auto& p : partition_graphs) {
    const string& device_name = p.first;
    std::unique_ptr<Graph>& subgraph = p.second;
    item->units.resize(item->units.size() + 1);
    ExecutionUnit* unit = &(item->units.back());

    // Construct the root executor for the subgraph.
    params.device = unit->device;
    params.function_library = lib;
    params.create_kernel = [session, lib, opseg](
        const NodeDef& ndef, OpKernel** kernel) {
      // Caches the kernel only if the node is stateful.
      if (!lib->IsStateful(ndef.op())) {
        return lib->CreateKernel(ndef, kernel);
      }
      auto create_fn = [lib, &ndef](OpKernel** kernel) {
        return lib->CreateKernel(ndef, kernel);
      };
      // Kernels created for subgraph nodes need to be cached.  On
      // cache miss, create\_fn() is invoked to create a kernel based
      // on the function library here + global op registry.
      return opseg->FindOrCreate(session, ndef.name(), kernel, create_fn);
    };

    params.delete_kernel = [lib](OpKernel* kernel) {
      // If the node is stateful, opseg owns it. Otherwise, delete it.
      if (kernel && !lib->IsStateful(kernel->type_string())) {
        delete kernel;
      }
    };

    unit->graph = subgraph.get();
    TF_RETURN_IF_ERROR(
        NewLocalExecutor(params, subgraph.release(), &unit->root));
  }
  return Status::OK();
}
\end{c++}
\end{leftbar}

\subsection{Run Figure}

\begin{figure}[H]
\centering
\includegraphics[width=0.8\textwidth]{figures/dist-run-step-stage-3.png}
\caption{RunGraph}
 \label{fig:dist-run-step-stage-3}
\end{figure}

\subsubsection{ReffedClientGraph::RunPartitions}

\begin{leftbar}
\begin{c++}
Status MasterSession::ReffedClientGraph::RunPartitions(
    const MasterEnv* env, int64 step_id, int64 execution_count,
    PerStepState* pss, CallOptions* call_opts, const RunStepRequestWrapper& req,
    MutableRunStepResponseWrapper* resp, CancellationManager* cm,
    const bool is_last_partial_run) {


  // 1. Prepares a number of calls to workers. 
  //    One call per partition.
  const int num = partitions_.size();
  RunManyGraphs calls(num);

  for (int i = 0; i < num; ++i) {
    const Part& part = partitions_[i];
    RunManyGraphs::Call* c = calls.get(i);
    c->req.reset(part.worker->CreateRunGraphRequest());
    c->resp.reset(part.worker->CreateRunGraphResponse());
    if (is_partial_) {
      c->req->set_is_partial(is_partial_);
      c->req->set_is_last_partial_run(is_last_partial_run);
    }
    c->req->set_session_handle(session_handle_);
    c->req->set_graph_handle(part.graph_handle);
    c->req->set_step_id(step_id);
    
    for (const auto& feed_key : part.feed_key) {
      const string& feed = feed_key.first;
      const string& key = feed_key.second;
      const int64 feed_index = feeds[feed];
      TF_RETURN_IF_ERROR(
          c->req->AddSendFromRunStepRequest(req, feed_index, key));
    }

    for (const auto& key_fetch : part.key_fetch) {
      const string& key = key_fetch.first;
      c->req->add_recv_key(key);
    }
  }

  // 2. Issues RunGraph calls.
  for (int i = 0; i < num; ++i) {
    const Part& part = partitions_[i];
    RunManyGraphs::Call* call = calls.get(i);
    part.worker->RunGraphAsync(
        &call->opts, call->req.get(), call->resp.get(),
        std::bind(&RunManyGraphs::WhenDone, &calls, i, std::placeholders::_1));
  }

  // 3. Waits for the RunGraph calls.
  call_opts->SetCancelCallback([&calls]() { calls.StartCancel(); });
  auto token = cm->get_cancellation_token();
  bool success =
      cm->RegisterCallback(token, [&calls]() { calls.StartCancel(); });
  if (!success) {
    calls.StartCancel();
  }

  calls.Wait();

  call_opts->ClearCancelCallback();
  if (success) {
    cm->DeregisterCallback(token);
  } else {
    return errors::Cancelled("Step was cancelled");
  }

  // 4. Collects fetches.
  Status status = calls.status();
  if (status.ok()) {
    for (int i = 0; i < num; ++i) {
      const Part& part = partitions_[i];
      MutableRunGraphResponseWrapper* run_graph_resp = calls.get(i)->resp.get();
      for (size_t j = 0; j < run_graph_resp->num_recvs(); ++j) {
        auto iter = part.key_fetch.find(run_graph_resp->recv_key(j));
        if (iter == part.key_fetch.end()) {
          status.Update(errors::Internal("Unexpected fetch key: ",
                                         run_graph_resp->recv_key(j)));
          break;
        }
        const string& fetch = iter->second;
        status.Update(
            resp->AddTensorFromRunGraphResponse(fetch, run_graph_resp, j));
        if (!status.ok()) {
          break;
        }
      }
    }
  }
  return status;
}
\end{c++}
\end{leftbar}

\subsubsection{GrpcRemoteWorker::RunGraphAsync}

\begin{leftbar}
\begin{c++}
struct GrpcRemoteWorker : public WorkerInterface {
  void RunGraphAsync(
      CallOptions * call_opts, 
      RunGraphRequestWrapper* request,
      MutableRunGraphResponseWrapper* response,
      StatusCallback done) override {
    IssueRequest(&request->ToProto(), 
        get_proto_from_wrapper(response),
        rungraph_, std::move(done), call_opts);
  }
};
\end{c++}
\end{leftbar}


\subsubsection{GrpcWorkerService::RunGraphHandler}

\begin{leftbar}
\begin{c++}
struct GrpcWorkerService : AsyncServiceInterface {
  void RunGraphHandler(WorkerCall<RunGraphRequest, RunGraphResponse>* call) {
    Schedule([this, call]() {
      auto wrapped_req = new ProtoRunGraphRequest(&call->request);
      auto wrapped_rsp = new NonOwnedProtoRunGraphResponse(&call->response);
      
      auto call_opts = new CallOptions;
      call->SetCancelCallback([call_opts]() { 
          call_opts->StartCancel(); 
      });

      worker_->RunGraphAsync(call_opts, wrapped_req, wrapped_rsp, 
        [call, call_opts, wrapped_req, wrapped_rsp](const Status& s) {
            call->ClearCancelCallback();
            delete call_opts;
            delete wrapped_req;
            delete wrapped_rsp;
            call->SendResponse(ToGrpcStatus(s));
        });
    });
    ENQUEUE_REQUEST(RunGraph, true);
  }
};
\end{c++}
\end{leftbar}

\subsubsection{Worker::RunGraphAsync}

\begin{leftbar}
\begin{c++}
void Worker::RunGraphAsync(
    CallOptions * opts, 
    RunGraphRequestWrapper* request,
    MutableRunGraphResponseWrapper* response,
    StatusCallback done) {
  if (request->is_partial()) {
    DoPartialRunGraph(opts, request, response, std::move(done));
  } else {
    DoRunGraph(opts, request, response, std::move(done));
  }
}
\end{c++}
\end{leftbar}

\begin{leftbar}
\begin{c++}
void Worker::DoRunGraph(
    CallOptions * opts, 
    RunGraphRequestWrapper* request,
    MutableRunGraphResponseWrapper* response,
    StatusCallback done) {
  const int64 step_id = request->step_id();

  // 1. Prepare inputs and outputs.
  GraphMgr::NamedTensors in;
  GraphMgr::NamedTensors* out = new GraphMgr::NamedTensors;
  Status s = PrepareRunGraph(request, &in, out);
  if (!s.ok()) {
    delete out;
    done(s);
    return;
  }
  
  // 2. Register Cancellation callback.
  CancellationManager* cm = new CancellationManager;
  opts->SetCancelCallback([this, cm, step_id]() {
    cm->StartCancel();
    AbortStep(step_id);
  });

  CancellationToken token;
  {
    mutex_lock l(mu_);
    token = cancellation_manager_->get_cancellation_token();
    bool already_cancelled = !cancellation_manager_->RegisterCallback(
        token, [cm]() { cm->StartCancel(); });
    if (already_cancelled) {
      opts->ClearCancelCallback();
      delete cm;
      delete out;
      done(errors::Aborted("Call was aborted"));
      return;
    }
  }

  // 3. Start Execution.
  auto session =
      FindWorkerSession(request);

  session->graph_mgr->ExecuteAsync(
      request->graph_handle(), step_id, session, 
      request->exec_opts(), response, cm, in,
      [ this, step_id, response, session, cm, 
        out, token, opts, done](Status s) {
        
        // 4. Receive output tensors from grpc remote rendezvous.
        if (s.ok()) {
          s = session->graph_mgr->RecvOutputs(step_id, out);
        }

        // 5. Unregister Cancellation callback
        opts->ClearCancelCallback();
        {
          mutex_lock l(mu_);
          cancellation_manager_->DeregisterCallback(token);
        }
        delete cm;

        // 6. Save to RunStepResponse.
        if (s.ok()) {
          for (const auto& p : *out) {
            const string& key = p.first;
            const Tensor& val = p.second;
            response->AddRecv(key, val);
          }
        }
        delete out;
        done(s);
      });
}
\end{c++}
\end{leftbar}

\subsubsection{GraphMgr}

\begin{figure}[H]
\centering
\includegraphics[width=1.0\textwidth]{figures/dist-run-step-overview.png}
\caption{Worker: RunStep Formalization}
 \label{fig:dist-run-step-overview}
\end{figure}

\begin{leftbar}
\begin{c++}
void GraphMgr::ExecuteAsync(
    const string& handle, const int64 step_id,
    WorkerSession* session, const ExecutorOpts& opts,
    MutableRunGraphResponseWrapper* response,
    CancellationManager* cancellation_manager,
    const NamedTensors& in, StatusCallback done) {
  // 1. Lookup an item. Holds one ref while executing.
  //    One item per registered graph.
  Item* item = nullptr;
  {
    mutex_lock l(mu_);
    auto iter = table_.find(handle);
    if (iter != table_.end()) {
      item = iter->second;
      item->Ref();
    }
  }

  RemoteRendezvous* rendezvous = worker_env_->rendezvous_mgr->Find(step_id);
  Status s = rendezvous->Initialize(session);

  // 2. Sends inputs to rendezvous.
  if (s.ok()) {
    s = SendInputsToRendezvous(rendezvous, in);
  }

  // 3. Start parallel executors.
  StartParallelExecutors(
      handle, step_id, item, rendezvous, collector,
      cost_graph, cancellation_manager,
      [this, item, rendezvous, done](const Status& s) {
          // 4. Recvs outputs from rendezvous.
          done(s);
          rendezvous->Unref();
          item->Unref();
      });
}
\end{c++}
\end{leftbar}

\begin{leftbar}
\begin{c++}
Status GraphMgr::SendInputsToRendezvous(
    Rendezvous* rendezvous, const NamedTensors& in) {
  Rendezvous::ParsedKey parsed;
  for (auto& p : in) {
    auto& key = p.first;
    auto& val = p.second;

    Status s = Rendezvous::ParseKey(key, &parsed);
    if (s.ok()) {
      s = rendezvous->Send(parsed, Rendezvous::Args(), val, false);
    }
    if (!s.ok()) {
      return s;
    }
  }
  return Status::OK();
}
\end{c++}
\end{leftbar}

\begin{leftbar}
\begin{c++}
void GraphMgr::StartParallelExecutors(
    const string& handle, int64 step_id,
    Item* item, Rendezvous* rendezvous,
    StepStatsCollector* collector,
    CancellationManager* cancellation_manager,
    StatusCallback done) {
  
  // 1. Wait until pending == 0, with default is num\_units,
  // `pending -= 1` when one partition graph is done. 
  int num_units = item->units.size();
  ExecutorBarrier* barrier =
      new ExecutorBarrier(
          num_units, rendezvous, [done](const Status& s) {
              done(s);
          });

  Executor::Args args;
  {
    mutex_lock l(mu_);
    args.step_id = ++next_id_;
  }
  args.rendezvous = rendezvous;
  args.cancellation_manager = cancellation_manager;
  args.stats_collector = collector;
  args.step_container = step_container;
  args.sync_on_finish = sync_on_finish_;

  using std::placeholders::_1;
  args.runner = std::bind(
      &thread::ThreadPool::Schedule, 
      worker_env_->compute_pool, _1);

  2. Broadcast all partitions to run
  for (const auto& unit : item->units) {
    unit.root->RunAsync(args, barrier->Get());
  }
}
\end{c++}
\end{leftbar}

\begin{leftbar}
\begin{c++}
Status GraphMgr::RecvOutputsFromRendezvous(
    Rendezvous* rendezvous, NamedTensors* out) {
  // Receives values requested by the caller.
  Rendezvous::ParsedKey parsed;
  for (auto& p : *out) {
    auto& key = p.first;
    auto& val = p.second;

    bool is_dead = false;
    Status s = Rendezvous::ParseKey(key, &parsed);
    if (s.ok()) {
      s = rendezvous->Recv(parsed, Rendezvous::Args(), &val, &is_dead);
    }

    if (is_dead) {
      s = errors::InvalidArgument("The tensor returned for ", key,
                                  " was not valid.");
    }

    if (!s.ok()) {
      return s;
    }
  }
  return Status::OK();
}

Status GraphMgr::RecvOutputs(int64 step_id, NamedTensors* out) {
  Rendezvous* rendezvous = worker_env_->rendezvous_mgr->Find(step_id);
  Status s = RecvOutputsFromRendezvous(rendezvous, out);
  rendezvous->Unref();
  return s;
}
\end{c++}
\end{leftbar}

\subsection{Rendzvous}

\begin{figure}[H]
\centering
\includegraphics[width=0.9\textwidth]{figures/rendezvous-hierarchy.png}
\caption{Rendezvous hierarchy}
 \label{fig:rendezvous-hierarchy}
\end{figure}

\subsubsection{polymorphic creation}

\begin{figure}[H]
\centering
\includegraphics[width=0.8\textwidth]{figures/rendezvous-remote-mgr.png}
\caption{RemoteRendezvous polymorphic creation}
 \label{fig:rendezvous-remote-mgr}
\end{figure}

\subsubsection{send}

\begin{figure}[H]
\centering
\includegraphics[width=0.8\textwidth]{figures/rendzvous-send.png}
\caption{Rendezvous send}
 \label{fig:rendzvous-send}
\end{figure}

\subsubsection{receive}

\begin{figure}[H]
\centering
\includegraphics[width=0.8\textwidth]{figures/rendezvous-recv-case-1.png}
\caption{Rendezvous receiving: distributed sender and receiver are in the same worker}
 \label{fig:rendezvous-recv-case-1}
\end{figure}

\begin{figure}[H]
\centering
\includegraphics[width=0.4\textwidth]{figures/rendezvous-recv-case-2.png}
\caption{Rendezvous receiving: The distributed sender and receiver are not in the same worker}
 \label{fig:rendezvous-recv-case-2}
\end{figure}

\subsection{Logout Map}

\end{content}

\section{Close session}

\begin{content}

\subsubsection{GrpcSession}

\begin{leftbar}
\begin{c++}
Status GrpcSession::Close() {
  CloseSessionRequest req;
  {
    mutex_lock l(mu_);
    if (handle_.empty()) {
      return errors::InvalidArgument("A session is not created yet....");
    }
    req.set_session_handle(handle_);
    handle_.clear();
  }
  CloseSessionResponse resp;
  CallOptions call_options;
  call_options.SetTimeout(options_.config.operation_timeout_in_ms());
  return master_->CloseSession(&call_options, &req, &resp);
}
\end{c++}
\end{leftbar}

\subsubsection{GrpcRemoteMaster}

\begin{leftbar}
\begin{c++}
struct GrpcRemoteMaster : MasterInterface {
  Status CloseSession(CallOptions* call_options,
                      const CloseSessionRequest* request,
                      CloseSessionResponse* response) override {
    ::grpc::ClientContext ctx;
    ctx.set_fail_fast(false);
    SetDeadline(&ctx, call_options->GetTimeout());
    return FromGrpcStatus(stub_->CloseSession(&ctx, *request, response));
  }
};
\end{c++}
\end{leftbar}

\subsubsection{GrpcMasterService}

\begin{leftbar}
\begin{c++}
struct GrpcMasterService : AsyncServiceInterface {
  void CloseSessionHandler(
      MasterCall<CloseSessionRequest, CloseSessionResponse>* call) {
    master_impl_->CloseSession(&call->request, &call->response,
                               [call](const Status& status) {
                                 call->SendResponse(ToGrpcStatus(status));
                               });
    ENQUEUE_REQUEST(CloseSession, false);
  }
};
\end{c++}
\end{leftbar}

\subsubsection{Master}

\begin{leftbar}
\begin{c++}
void Master::CloseSession(const CloseSessionRequest* req,
                          CloseSessionResponse* resp, MyClosure done) {
  MasterSession* session = nullptr;
  {
    mu_.lock();
    auto iter = sessions_.find(req->session_handle());
    if (iter == sessions_.end()) {
      Mu_.unlock();
      done(errors::Aborted(
          "Session ", req->session_handle(),
          " is not found. Possibly, this master has restarted."));
      return;
    }
    session = iter->second;
    sessions_.erase(iter);
    Mu_.unlock();
  }

  // Session Close() blocks on thread shutdown. Therefore, we need to
  // delete it in non-critical thread.
  SchedClosure([session, done]() {
    Status s = session->Close();
    session->Unref();
    done(s);
  });
}
\end{c++}
\end{leftbar}

\subsubsection{MasterSession}

\begin{leftbar}
\begin{c++}
Status MasterSession::Close() {
  {
    mutex_lock l(mu_);
    closed_ = true;  // All subsequent calls to Run() or Extend() will fail.
  }
  cancellation_manager_.StartCancel();
  std::vector<ReffedClientGraph*> to_unref;
  {
    mutex_lock l(mu_);
    while (num_running_ != 0) {
      num_running_is_zero_.wait(l);
    }
    ClearRunsTable(&to_unref, &run_graphs_);
    ClearRunsTable(&to_unref, &partial_run_graphs_);
  }
  for (ReffedClientGraph* rcg : to_unref) rcg->Unref();
  return Status::OK();
}
\end{c++}
\end{leftbar}

\subsubsection{ReffedClientGraph}

\begin{leftbar}
\begin{c++}
ReffedClientGraph::~ReffedClientGraph() { 
  DeregisterPartitions (); 
}
\end{c++}
\end{leftbar}

\begin{leftbar}
\begin{c++}
void ReffedClientGraph::DeregisterPartitions() {
  struct Call {
    DeregisterGraphRequest req;
    DeregisterGraphResponse resp.
  };
  for (Part& part : partitions_) {
    if (!part.graph_handle.empty()) {
      Call* c = new Call;
      c->req.set_session_handle(session_handle_);
      c->req.set_graph_handle(part.graph_handle);

      WorkerCacheInterface* worker_cache = worker_cache_;
      const string name = part.name;
      WorkerInterface* w = part.worker;

      auto cb = [worker_cache, c, name, w](const Status& s) {
        if (!s.ok()) {
          // This error is potentially benign, so we don't log at the
          // error level.
          LOG(INFO) << "DeregisterGraph error: " << s;
        }
        delete c;
        worker_cache->ReleaseWorker(name, w);
      };
      w->DeregisterGraphAsync(&c->req, &c->resp, cb);
    }
  }
}
\end{c++}
\end{leftbar}

\subsubsection{GrpcWorkerService}

\begin{leftbar}
\begin{c++}
struct GrpcWorkerService : AsyncServiceInterface {
  void CreateWorkerSessionHandler(
      WorkerCall<CreateWorkerSessionRequest, CreateWorkerSessionResponse>*
          call) {
    Schedule([this, call]() {
      Status s = worker_->CreateWorkerSession(&call->request, &call->response);
      call->SendResponse(ToGrpcStatus(s));
    });
    ENQUEUE_REQUEST(CreateWorkerSession, false);
  }
};
\end{c++}
\end{leftbar}

\subsubsection{Worker}

\begin{leftbar}
\begin{c++}
void Worker::DeregisterGraphAsync(const DeregisterGraphRequest* request,
                                  DeregisterGraphResponse* response,
                                  StatusCallback done) {
  WorkerSession* session =
      env_->session_mgr->WorkerSessionForSession(request->session_handle());
  Status s = session->graph_mgr->Deregister(request->graph_handle());

  done(s);
}
\end{c++}
\end{leftbar}

\subsubsection{GraphMgr}

\begin{leftbar}
\begin{c++}
Status GraphMgr::Deregister(const string& handle) {
  Item* item = nullptr;
  {
    mutex_lock l(mu_);
    auto iter = table_.find(handle);
    if (iter == table_.end()) {
      return errors::Aborted("Graph handle is not found: ", handle,
                             ". Possibly, this worker just restarted.");
    }
    item = iter->second;
    table_.erase(iter);
  }
  item->Unref();
  return Status::OK();
}
\end{c++}
\end{leftbar}

\begin{leftbar}
\begin{c++}
GraphMgr::Item::~Item() {
  for (const auto& unit : this->units) {
    delete unit.root;
    unit.device->op_segment()->RemoveHold(this->session);
  }
}
\end{c++}
\end{leftbar}

\end{content}




\part{Model training}
\begin{savequote}[45mm]
\ascii{Any fool can write code that a computer can understand. Good programmers write code that humans can understand.}
\qauthor{\ascii{- Martin Flower}}
\end{savequote}

\chapter{BP algorithm} 
\label{ch:bp}

\begin{content}

\end{content}

\section{TensorFlow implementation}

\begin{content}

\tf{} is a software system that implements automatic differentiation. First, it constructs a forward computational graph that implements the forward computation of the computational graph. When calling the \code{Optimizer.minimize} method, use the \code{compute\_gradients} method to implement the construction of the inverse calculation graph; use the \code{apply\_gradients} method to implement the submap construction of the parameter update.

\begin{leftbar}
\begin{python}
Class Optimizer(object):
  Def minimize(self, loss, var_list=None, global_step=None):
    """Add operations to minimize loss by updating var_list.
    """
    Grads_and_vars = self.compute_gradients(
      Loss, var_list=var_list)
    Return self.apply_gradients(
      Grads_and_vars, 
      Global_step=global_step)
\end{python}
\end{leftbar}

\subsection{calculation gradient}

\code{compute\_gradients} will solve the gradient of \code{var\_list=[v1, v2, ..., vn]} according to the value of \code{loss}, and the final result will be: \code{[ (grad\_v1, v1), (grad\_v2, v2), ..., (grad\_vn, vn)]}. Among them, \code{compute\_gradients} will call the \code{gradients} method to construct a submap of backpropagation.

Explain the construction process of the reverse subgraph with a simple example. First, construct a forward calculation graph.

\begin{leftbar}
\begin{python}
X = tf.placeholder("float", name="X")
Y = tf.placeholder("float", name="Y")
w = tf.Variable(0.0, name="w")
b = tf.Variable(0.0, name="b")
Loss = tf.square(Y - X*w - b)
Global_step = tf.Variable(0, trainable=False, collections=[])
\end{python}
\end{leftbar}

Construct a backpropagated subgraph with \code{compute\_gradients}.

\begin{leftbar}
\begin{python}
Sgd = tf.train.GradientDescentOptimizer(0.01)
Grads_and_vars = sgd.compute_gradients(loss)
\end{python}
\end{leftbar}

\subsubsection{structuring algorithm}

The construction algorithm of the reverse subgraph can be formally described as:

\begin{leftbar}
\begin{python}
Def gradients(loss, grad=I):
  Vrg = build_virtual_reversed_graph(loss)
  For op in vrg.topological_sort():
    Grad_fn = ops.get_gradient_function(op)
    Grad = grad_fn(op, grad)
\end{python}
\end{leftbar}

First, construct a virtual reverse subgraph based on the topological map of the forward subgraph. It is called virtual because the real reverse subgraph is much more complicated than it is; more accurately, a node in the virtual reverse subgraph corresponds to a local part of the real reverse subgraph Subgraph.

At the same time, the output of the last node of the forward subgraph has an output gradient of \code{Tensor} of \ascii{1} as the initial gradient value of the reverse subgraph, often denoted as \code{I}.

\begin{figure}[!htbp]
\centering
\includegraphics[width=0.7\textwidth]{figures/bp-back-graph-construction.png}
\caption{Construct backpropagation subgraph}
 \label{fig:bp-back-graph-construction}
\end{figure}

Next, construct a real reverse subgraph based on the virtual reverse subgraph. First, performing a topological sorting algorithm according to the reverse virtual subgraph, and obtaining a topological sorting of the virtual reverse subgraph; and then, according to the topological sorting, searching for \ascii{OP} in each forward subgraph Its "gradient function"; finally, the gradient function is called, which will construct the inverse partial subgraph corresponding to \ascii{OP}.

In summary, a positive \ascii{OP} corresponds to a reverse partial local subgraph and is constructed by the gradient function of \ascii{OP}. When the entire topological sorting algorithm is completed, each \ascii{OP} in the forward subgraph can find the corresponding local subgraph in the reverse subgraph.

For example, in the above example, the last \ascii{OP} in the forward graph: the function of finding the square is used as an example to describe how the gradient function works.

\subsubsection{gradient function prototype}

In general, the gradient function satisfies the following prototype:

\begin{leftbar}
\begin{python}
@ops.RegisterGradient("op_name")
Def op_grad_func(op, grad):
\end{python}
\end{leftbar}

Among them, the gradient function is registered by \code{ops.RegisterGradient} and placed in the repository where the gradient function is saved. Later, you can index the corresponding gradient function according to the name of the forward \ascii{OP}.

For a gradient function, the first parameter \code{op} represents the forward calculation of \ascii{OP}, according to which it can get the input and output of \ascii{OP} in the forward calculation; the second parameter \code{ Grad} is the gradient passed by the upstream node in the reverse subgraph. It is a calculated gradient value (the initial gradient value is all 1).

\subsubsection{actual combat: square function}

As a simple example, use only the input to calculate the gradient. \code{y=square(x)}, used to find the square of \code{x}. First, construct a forward calculation graph:

\begin{figure}[!h]
\centering
\includegraphics[width=0.5\textwidth]{figures/bp-square-forward-graph.png}
\caption{Square function: forward propagation subgraph}
 \label{fig:bp-square-forward-graph}
\end{figure}

Then, construct a virtual reverse subgraph in reverse. Construct a true inverse computation subgraph based on the topological ordering of the virtual inverse subgraph. If the current node is \code{Square}, find the corresponding gradient function \code{SquareGrad} from the repository based on its OP name.

\begin{figure}[!htbp]
\centering
\includegraphics[width=0.5\textwidth]{figures/bp-square-backward-graph.png}
\caption{Square function: backpropagation subgraph}
 \label{fig:bp-square-backward-graph}
\end{figure}

Because the derivative of \code{y=Square(x)} is \code{y'=2*x}. Therefore, the implementation of its gradient function \code{SquareGrad} is:

\begin{leftbar}
\begin{python}
@ops.RegisterGradient("Square")
Def SquareGrad(op, grad):
  x = op.inputs[0]
  With ops.control_dependencies([grad.op]):
    x = math_ops.conj(x)
    Return grad * (2.0 * x)
\end{python}
\end{leftbar}

After calling the gradient function, you will get \ascii{OP} of the forward \code{Square}, corresponding to the reverse subgraph \code{SquareGrad}. It needs to use the input of \code{Square} to complete the corresponding gradient calculation.

\begin{figure}[!h]
\centering
\includegraphics[width=0.5\textwidth]{figures/bp-square-backward-graph-2.png}
\caption{Square function: backpropagation subgraph}
 \label{fig:bp-square-backward-graph-2}
\end{figure}

In general, a \ascii{OP} in the forward subgraph corresponds to a local subgraph in the reverse subgraph. Because, for the gradient function implementation of \ascii{OP}, multiple \ascii{OP} may be required to complete the corresponding gradient calculation. For example, \ascii{OP} of \code{Square}, corresponding to the gradient function constructed with two \ascii{2} multiplications \ascii{OP}.

\subsubsection{actual warfare: exponential function}

To give a simple example, use only the output to calculate the gradient. \code{y=exp(x)}, exponential function; its derivative is \code{y'=exp(x)}, which is \code{y'=y}. Therefore, its gradient function is implemented as:

\begin{leftbar}
\begin{python}
@ops.RegisterGradient("Exp")
Def _ExpGrad(op, grad):
  """Returns grad * exp(x)."""
  y = op.outputs[0]
  With ops.control_dependencies([grad.op]):
    y = math_ops.conj(y)
    Return grad * y
\end{python}
\end{leftbar}

As shown in the figure below, the output of the \ascii{OP} in the forward subgraph is used for the gradient operation of the corresponding inverse local subgraph. Moreover, the local subgraph contains one node.

\begin{figure}[!h]
\centering
\includegraphics[width=0.5\textwidth]{figures/bp-exp-backward-graph.png}
\caption{Exp function: backpropagation subgraph}
 \label{fig:bp-exp-backward-graph}
\end{figure}

\subsection{Apply gradient}

To make a simple summary, when calling the \code{Optimizer.minimize} method, use the \code{compute\_gradients} method to implement the construction of the inverse computation graph; use the \code{apply\_gradients} method to implement parameter updates. Subgraph construction.

\subsubsection{structuring algorithm}

First, \code{compute\_gradients} will solve the gradient of \code{var\_list=[v1, v2, ..., vn]} at runtime based on the value of \code{loss}, and the final result will be :\code{vars\_and\_grads = [(grad\_v1, v1), (grad\_v2, v2), ..., (grad\_vn, vn)]}.

Then, \code{apply\_gradients} iteration \code{grads\_and\_vars}, for each \code{(grad\_vi, vi)}, construct a submap that updates \code{vi}. Among them, the algorithm can be formally described as:

\begin{leftbar}
\begin{python}
Def apply_gradients(grads_and_vars, learning_rate):
  For (grad, var) in grads_and_vars:
    Apply_gradient_descent(learning_rate, grad, var)
\end{python}
\end{leftbar}

Among them, \code{apply\_gradient\_descent} will construct a computational subgraph that uses the gradient descent algorithm to update the parameters. Use the binary of \code{(grad, var)} and its \code{learning\_rate}\code{Const OP} as input to \code{ApplyGradientDescent}.

\code{ApplyGradientDescent} will apply the \code{var <- var - learning*grad} algorithm to implement an in-place update of \code{var}.

\begin{figure}[!h]
\centering
\includegraphics[width=0.6\textwidth]{figures/bp-update-w.png}
\caption{Parameter update submap}
 \label{fig:bp-update-w}
\end{figure}

\subsubsection{Parameter Update Summary}

If there are multiple trained Variables, a partial subgraph of multiple update parameters is finally generated. They are aggregated together using a control dependency edge via a \code{NoOp} named \code{update}. Because each \code{Variable} is independent of each other, maximum concurrency can be achieved.

\begin{figure}[!h]
\centering
\includegraphics[width=0.9\textwidth]{figures/bp-update-all-params.png}
\caption{Parameter update summary}
 \label{fig:bp-update-all-params}
\end{figure}

\subsubsection{Exploring train\_op}

After a round of \ascii{Step} operation, the parameters are updated according to the gradient, and finally \code{global\_step} is added. The \ascii{OP} that implements \code{global\_step} plus \ascii{1} is \code{AssignAdd} and is marked as \code{train\_op}; it holds the \code{global\_step} variable The reference, then complete the in-place modification and add its value to \ascii{1}.

\begin{figure}[!h]
\centering
\includegraphics[width=0.9\textwidth]{figures/bp-train-op.png}
\caption{train\_op}
 \label{fig:bp-train-op}
\end{figure}

\subsubsection{workflow}

As shown in \refig{bp-train-pipeline}, the entire training process, a \ascii{Step} training process is calculated by forward calculation, reverse gradient calculation, parameter update, and its \code{global\_step} The four basic processes are composed.

Among them, each round \ascii{Step} starts from the start of \code{Session.run}. The output of each \ascii{OP} is obtained by the calculation of the forward subgraph and is used as the input of the downstream \ascii{OP}.

After the calculation is completed on the subgraph, the gradient of each training parameter is inversely calculated by using the initial gradient vector $I$ as input, and finally the gradient list of each training parameter is obtained, and \code{grads\_and\_vars = [( A two-tuple list representation of grad\_v1, v1), ..., (grad\_vn, vn)]}.

Subsequently, the parameter update subgraph takes \code{grads\_and\_vars} as input and performs the gradient descent update algorithm; finally, the \code{global\_step} value is added by \code{train\_op} plus \ascii{1 }, the round of \ascii{Step} is completed.

\begin{figure}[!h]
\centering
\includegraphics[width=0.9\textwidth]{figures/bp-train-pipeline.png}
\caption{Model training workflow}
 \label{fig:bp-train-pipeline}
\end{figure}

\end{content}
\begin{savequote}[45mm]
  \ascii{Any fool can write code that a computer can understand. Good programmers write code that humans can understand.}
  \qauthor{\ascii{- Martin Flower}}
\end{savequote}


\chapter{Data loading} 
\label{ch:input-pipeline}
\begin{content}
In general, \ascii{TensorFlow} enters sample data into the training/inferion subgraph to perform the operation. There are three ways to read the sample data:

\begin{enum}
  \eitem{Data injection: Pass the data to \code{Session.run} via the dictionary \code{feed\_dict} to replace the value of \code{Tensor}'s output\code{Tensor};}
  \eitem{Data Pipeline: By constructing an input subgraph, concurrently reading sample data from a file;}
  \eitem{Data preloading: For small datasets, use \code{Const} or \code{Variable} to hold data directly. }
\end{enum}

Based on large dataset training or inference tasks, the input of sample data often uses the pipeline mode of the data to ensure high throughput and improve the efficiency of training/inferencing. The process uses a queue to implement data interaction and asynchronous control between the input subgraph and the training/inference subgraph.

This chapter will focus on the workings of \ascii{Pipeline} for data loading, and gain insight into the coordination mechanism for concurrent execution of \ascii{TensorFlow} and its role in concurrent execution.
\end{content}


\section{Data injection}
\begin{content}
Data injection is the most common method of data loading. It passes the sample data to \code{Session.run} or the \code{Tensor.eval} method via the dictionary \code{feed\_dict}; where the dictionary keyword Is the name of \code{Tensor} and the value is sample data.

\ascii{TensorFlow} will replace the sample data with the value of \code{Tensor} according to the name of the \code{Tensor} in the dictionary.

\begin{leftbar}
\begin{python}
x = tf.placeholder(tf.float32, [None, 784])
y_ = tf.placeholder(tf.float32, [None, 10])

with tf.Session():
  batch_xs, batch_ys = mnist.train.next_batch(100)
  sess.run(train_step, feed_dict={x: batch_xs, y_: batch_ys})
\end{python}
\end{leftbar}

In general, \code{feed\_dict} can override any \code{Tensor} value. However, \code{Placeholder} is often used to indicate that the value of its output \code{Tensor} is undetermined and will be replaced by \code{feed\_dict}.

\end{content}


\section{Data preloading}
\begin{content}
You can use \code{Const} or \code{Variable} to hold data directly and preload data into memory to improve execution efficiency. This method is only suitable for small data sets. When the sample data set is large, the memory resource consumption is very considerable. Here we take the \ascii{mnist} data set as an example to explain how to use data preloading.

\begin{leftbar}
\begin{python}
from tensorflow.examples.tutorials.mnist import input_data

data_sets = input_data.read_data_sets('/tmp/mnist/data')
\end{python}
\end{leftbar}


\subsection{Use Const}
Since the value of \code{Const OP} output \code{Tensor} is directly inlined in the calculation graph. If the \code{Const OP} is used multiple times in the diagram, it may cause duplicate redundant data, wasting unnecessary memory resources.

\begin{leftbar}
\begin{python}
with tf.name_scope('input'):
  input_images = tf.constant(data_sets.train.images)
  input_labels = tf.constant(data_sets.train.labels)
\end{python}
\end{leftbar}


\subsection{Using Variable}
You can use an immutable, non-training \code{Variable} instead of \code{Const}. Once the \code{Variable} of this type is initialized, its value cannot be changed, thus having the attributes of \code{Const}.

There is a difference between \code{Variable} for data preloading and \code{Variable} for training. It will set \code{trainable=False} and the system will not classify it as \code{GraphKeys .TRAINABLE\_VARIABLES} in the collection. During the training process, the system does not perform an update operation on it.

Also, when constructing this type of \code{Variable}, \code{collections=[]} will also be set and will not be categorized in the \code{GraphKeys.GLOBAL\_VARIABLES} collection. The \ascii{Checkpoint} operation is not implemented on the system during training.

To create an immutable, non-training \code{Variable}, here is a simple factory method.

\begin{leftbar}
\begin{python}
def immutable_variable(initial_value):
  initializer = tf.placeholder(
    dtype=initial_value.dtype,
    shape=initial_value.shape)
  return tf.Variable(initializer, trainable=False, collections=[])
\end{python}
\end{leftbar}

\code{immutable\_variable} Constructs the type and shape information of \code{Placeholder} using the passed \code{initial\_value} and uses it as the initial value of \code{Variable}. You can use \code{immutable\_variable} to create immutable \code{Variable} for data preloading.

\begin{leftbar}
\begin{python}
with tf.name_scope('input'):
  input_images = immutable_variable(data_sets.train.images)
  input_labels = immutable_variable(data_sets.train.labels)
\end{python}
\end{leftbar}


\subsection{Batch preloading}
You can build \ascii{Pipeline} and combine the data preloading mechanism to implement batch loading of samples. First, use \code{tf.train.slice\_input\_producer} to randomize the entire sample space at the beginning of each \ascii{epoch}, each time taking a random sample from the sample set to get a training sample.

\begin{leftbar}
\begin{python}
def one(input_xs, input_ys, num_epochs)
  return tf.train.slice_input_producer(
    [input_xs, input_ys], num_epochs=num_epochs)
\end{python}
\end{leftbar}

Then, use \code{tf.train.batch} to get sample data for one batch at a time.

\begin{leftbar}
\begin{python}
def batch(x, y, batch_size)
  return tf.train.batch(
    [x, y], batch_size=batch_size)
\end{python}
\end{leftbar}

For preloading data with \code{Variable}, you can get sample data for a batch as follows.

\begin{leftbar}
\begin{python}
with tf.name_scope('input'):
  input_images = immutable_variable(data_sets.train.images)
  input_labels = immutable_variable(data_sets.train.labels)

  image, label = one(input_images, input_labels, epoch=1)
  batch_images, batch_labels = batch(image, label, batch_size=100)
\end{python}
\end{leftbar}

In fact, \code{tf.train.slice\_input\_producer} will construct the sample queue and pass the training samples one by one to the sample queue by executing the \code{Enqueue} operation concurrently via \code{QueueRunner}. At each iteration of the training start, call \code{batch\_size} batch sample data to the training submap by calling \code{DequeueMany}.

\end{content}


\section{Data Pipeline}
\begin{content}
A typical data load \ascii{Pipeline(Input Pipeline)}, including the following important data processing entities:

\begin{enum}
  \eitem{file name queue: add a list of file names to this queue;}
  \eitem{Reader: Read the file name (dequeue) from the file name queue; and select the corresponding file reader according to the data format to parse the file record;}
  \eitem{decoder: decode file records and convert to data samples;}
  \eitem{Preprocessor: Preprocessing data samples, including regularization, whitening, etc.;}
  \eitem{sample queue: Add the processed sample data to the sample queue. }
\end{enum}

Take the \ascii{mnist} data set as an example, if the data format is \code{TFRecord}. First, construct a \code{FIFOQueue} queue holding the list of file names (by executing \code{EnqueueMany OP}) using \code{tf.train.string\_input\_producer}, and at each \ascii{epoch } Randomize the list of file names in the cycle.


\subsection{Build file name queue}

\begin{leftbar}
\begin{python}
def input_producer(num_epochs):
  return tf.train.string_input_producer(
    ['/tmp/mnist/train.tfrecords'], num_epochs = num_epochs)
\end{python}
\end{leftbar}

After constructing the file name queue, use \code{tf.TFRecordReader} to get the file name from the file name queue (dequeue, execute \code{Dequeue OP} by calling), and read the sample record from the file (\ascii {Record}). Then, parse the sample data using \code{tf.parse\_single\_example}.


\subsection{Reader}

\begin{leftbar}
\begin{python}
def parse_record(filename_queue):
  reader = tf.TFRecordReader()
  _, serialized_example = reader.read(filename_queue)
  features = tf.parse_single_example(
      serialized_example,
      features={
          'image_raw': tf.FixedLenFeature([], tf.string),
          'label': tf.FixedLenFeature([], tf.int64),
      })
  return features
\end{python}
\end{leftbar}


\subsection{Decoder}

The sample data is then decoded, and its optional pre-processing process, resulting in a training sample.

\begin{leftbar}
\begin{python}
def decode_image(features):
  image = tf.decode_raw(features['image_raw'], tf.uint8)
  image.set_shape([28*28])

  # Convert from [0, 255] -> [-0.5, 0.5] floats.
  image = tf.cast(image, tf.float32) * (1. / 255) - 0.5
  return image

def decode_label(features):
  label = tf.cast(features['label'], tf.int32)
  return label

def one_example(features):
  return decode_image(features), decode_label(features)
\end{python}
\end{leftbar}


\subsection{Build sample queue}

You can use \code{tf.train.shuffle\_batch} to build a \code{RandomShuffleQueue} queue, append the parsed training samples to the queue (by executing \code{Enqueue OP}); when the iteration execution starts, Get \code{batch\_size} sample data from the batch (by executing \code{DequeueMany OP}).

\begin{leftbar}
\begin{python}
def shuffle_batch(image, label, batch_size):
    # Shuffle the examples and collect them into batch\_size
    # batches.(Uses a RandomShuffleQueue)
    images, labels = tf.train.shuffle_batch(
      [image, label], batch_size=batch_size, num_threads=2,
      capacity=1000 + 3 * batch_size,
      # Ensures a minimum amount of shuffling of examples.
      min_after_dequeue=1000)
    return images, labels
\end{python}
\end{leftbar}


\subsection{Input sub-graph}

Finally, the entire program is passed over to construct an input subgraph.

\begin{leftbar}
\begin{python}
def inputs(num_epochs, batch_size):
  with tf.name_scope('input'):
    filename_queue = input_producer(num_epochs)
    features = parse_record(filename_queue)
    image, label = one_example(features)
    return shuffle_batch(image, label, batch_size)
\end{python}
\end{leftbar}

\end{content}


\section{Data synergy}
\begin{content}
In fact, the \ascii{Pipeline} of data loading is essentially constructing an input subgraph that implements concurrent \ascii{IO} operations so that the training process is not blocked by the operation \ascii{IO}, thus implementing \ascii{GPU Increase in utilization of }.

For input subgraphs, the processing of data streams is divided into several stages (\ascii{Stage}), each stage completes a specific data processing function; each stage uses a queue as a medium to complete data coordination and interaction.

As shown in the figure below, a typical neural network training mode is described. The entire pipeline is mediated by two queues, which are divided into three phases.

\begin{figure}[!htbp]
  \centering
  \includegraphics[width=0.9\textwidth]{figures/tf-input-pipeline.png}
  \caption{model training workflow}
  \label{fig:tf-input-pipeline}
\end{figure}


\subsection{Stage 1}
\code{string\_input\_producer} constructs a queue of \code{FIFOQueue}, which is a stateful \ascii{OP}. According to the \code{shuffle} option, at the beginning of each \ascii{epoch}, a list of files is randomly generated and appended to the queue.

\begin{figure}[!htbp]
  \centering
  \includegraphics[width=0.7\textwidth]{figures/tf-input-pipeline-stage-1.png}
  \caption{Phase 1: Model Training Workflow}
  \label{fig:tf-input-pipeline-stage-1}
\end{figure}


\subsubsection{Randomization}
First, execute \code{Const OP} named \code{filenames}, and then randomize the list of file names via \code{RandomShuffle}.


\subsubsection{Epoch Control}
To implement the count of \ascii{epoch}, the implementation subtly designed a local variable called \code{epochs}. Among them, the local variable only shares data between multiple rounds of the process, and is not updated by the training submap.

Prior to \code{Session.run}, the system performs an initialization of the local variable list and implements zero initialization of \code{Variable} named \code{epochs}.

The counting function of \ascii{epoch} is done by \code{CountUpTo}, which works like \ascii{C++}\code{i++}. It holds a reference to \code{Variable} and its upper bound parameter \code{limit}. Each round of \ascii{epoch} increments its \code{Variable} by 1 until it reaches the number of \code{num\_epochs}.

Among them, \code{CountUpTo} will automatically throw a \code{OutOfRangeError} exception when the \ascii{epoch} number reaches \code{num\_epochs}. Detailed implementations can be found in \ascii{Kernel} implementation of \code{CountUpToOp}.

\begin{leftbar}
\begin{c++}
template <class T>
struct CountUpToOp : OpKernel {
  explicit CountUpToOp(OpKernelConstruction* ctxt)
    : OpKernel (ctxt) {
    OP_REQUIRES_OK(ctxt, ctxt->GetAttr("limit", &limit_));
  }

  void Compute(OpKernelContext* ctxt) override {
    T before_increment;
    {
      mutex_lock l(*ctxt->input_ref_mutex(0));
      
      // Fetch the old tensor
      Tensor tensor = ctxt->mutable_input(0, true);
      T* ptr = &tensor.scalar<T>()();      
      before_increment = *ptr;
      
      // throw OutOfRangeError if exceed limit
      if (*ptr >= limit_) {
        ctxt->SetStatus(errors::OutOfRange(
            "Reached limit of ", limit_));
        return;
      }
      // otherwise increase 1
      ++ * ptr;
    }
    // Output if no error.
    Tensor* out_tensor;
    OP_REQUIRES_OK(ctxt, ctxt->allocate_output(
        "output", TensorShape({}), &out_tensor));
    out_tensor->scalar<T>()() = before_increment;
  }

private:
  T limit_;
};
\end{c++}
\end{leftbar}


\subsubsection{Enqueue operation}
In fact, appending the list of filenames to the queue, executing \code{EnqueueMany}, similar to \code{Assign} modifying the value of \code{Variable}, \code{EnqueueMany} is also a stateful \ascii{ OP}, which holds the handle of the queue and directly completes the status update of the queue.

Here, \code{EnqueueMany} will be executed by \code{Session.run}, the system will traverse backwards, find the dependent \code{Identity}, and the control will depend on \code{CountUpTo}, which will be started once. Ascii{epoch} counts until the \code{num\_epoch} number reaches the \code{OutOfRangeError} exception. At the same time, \code{Identity} relies on \code{RandomShuffle} to get a randomized list of filenames.


\subsubsection{QueueRunner}
In addition, when calling \code{tf.train.string\_input\_producer}, a special \ascii{OP}:\code{QueueRunner} will be registered in the calculation graph and added to \code{GraphKeys. In the QUEUE\_RUNNERS} collection. Also, a \code{QueueRunner} holds one or more \ascii{OP} of type \code{Enqueue, EnqueueMany}.


\subsection{Stage 2}
\code{Reader} gets the file name from the file name queue in the order of \ascii{FIFO}, and reads the file record according to the file name. After successful, the record is decoded and preprocessed, converted into data samples, and finally Append it to the sample queue.


\subsubsection{Reader}
In fact, the implementation constructs a \code{ReaderRead}\ascii{OP} that holds the handle to the filename queue and gets the filename from the queue in the order \ascii{FIFO}.

Since the format of the file is \code{TFRecord}, ​​\code{ReaderRead} will delegate \ascii{OP} of \code{TFRecordReader} to read the file. Finally, after the operation of \code{ReaderRead}, you will get a serialized sample.


\subsubsection{Decoder}
After the serialized samples are obtained, decoding is performed using a suitable decoder to obtain a desired sample data. Optionally, the sample can be pre-processed, such as \code{reshape}.


\subsubsection{Enqueue operation}
After getting the sample data, the operation of \code{QueueEnqueue} will be started and the sample will be appended to the sample queue. Among them, \code{QueueEnqueue} is a stateful \ascii{OP}, which holds the handle of the sample queue and directly completes the update operation of the queue.

In practice, the sample queue is a \code{RandomShuffleQueue} that uses a dequeue operation to achieve random sampling.


\subsubsection{Concurrent execution}
To improve the throughput of \ascii{IO}, you can start multi-way concurrent \code{Reader} and \code{Decoder} workflows and concurrently append samples to the sample queue. Among them, \code{RandomShuffleQueue} is thread-safe and supports concurrent enqueue or dequeue operations.


\subsection{Stage 3}
When the data samples are accumulated to a \code{batch\_size}, the training/inference subgraph will take the sample data for the batch and initiate an iterative calculation (often called \ascii{Step}).


\subsubsection{Departure operation}
In fact, the training sub-graph uses \code{DequeueMany} to get sample data for a batch.


\subsubsection{Iteration execution}
In general, an iterative run consists of two basic processes: forward calculation and reverse gradient transfer. The \ascii{Worker} task uses the \ascii{PS} task to update to the local current value, and performs a forward calculation to get the loss of this iteration.

Then, based on the loss of this iteration, calculate the gradient of each \ascii{Variable} and update it to the \ascii{PS} task; the \ascii{PS} task updates the value of each \code{Variable} and will The current value is broadcast to each \ascii{Worker} task.


\subsubsection{Checkpoint}
The \ascii{PS} task periodically implements \ascii{Checkpoint} based on a fault-tolerant strategy. Persist all current \code{Variable} data, and its graph metadata, including static graph structure information, to an external storage device for subsequent recovery of the graph and all its \code{Variable} data.


\subsection{Pipeline beat}
For example, add a list of file names to the queue of \code{FIFOQueue}, at which point the subgraph calculation starting with \code{EnqueueMany} is called, including the \code{CountUpTo} that the execution depends on. When \code{CountUpTo} reaches the upper limit of \code{limit}, the \code{OutOfRangeError} exception is automatically thrown.

Plays the \code{QueueRunner} of the main program, catches the \code{OutOfRangeError} exception re-thrown by \code{coord.join}, then immediately closes the corresponding queue and exits the thread's execution. After the queue is closed, the enqueue operation becomes illegal; the dequeue operation remains legal unless the queue element is empty.

By the same token, downstream \ascii{OP} dequeues from the queue (filename queue) and automatically throws a \code{OutOfRangeError} exception once the queue element is empty. The corresponding \code{QueueRunner} at this stage will sense the occurrence of the exception, then catch the exception and close the downstream queue (sample queue), exiting the execution of the thread.

In the final stage of \ascii{Pipeline}, \code{train\_op} dequeues the batch training sample from the sample queue, the queue is empty, and the queue is closed, throwing a \code{OutOfRangeError} exception, and finally Stop the entire training task.

\end{content}

\begin{savequote}[45mm]
\ascii{Any fool can write code that a computer can understand. Good programmers write code that humans can understand.}
\qauthor{\ascii{- Martin Flower}}
\end{savequote}

\chapter{Saver} 
\label{ch:saver}

\section{Saver}

\begin{content}

In the long-term training task, \tf{} periodically performs a breakpoint check (\ascii{Checkpoint}) in order to achieve high availability of tasks.

\code{Saver} is the infrastructure for implementing breakpoint checking. It will persist all training parameters in the file system; when recovery training is required, it can recover the calculation graph from the file system and its training parameters. value. In other words, \code{Saver} assumes the following two responsibilities:

\begin{enum}
  \eitem{\code{save}: persists the current value of the training parameter to the breakpoint file;}
  \eitem{\code{restore}: Restores the value of the training parameter from the breakpoint file. }
\end{enum}

\subsection{Usage method}

For example, there is a simple calculation graph with two training parameters. First, after the initialization is performed, the results are persisted to the file system.

\begin{leftbar}
\begin{python}
# Construct graph
V1 = tf.Variable([0], name='v1')
V2 = tf.Variable([0], name='v2')

# run graph
With tf.Session() as sess:
  Sess.run(tf.global_variables_initializer())
  Saver = tf.train.Saver()
  Saver.save(sess, 'ckp')
\end{python}
\end{leftbar}

The model can then be restored based on where the breakpoint file is stored.

\begin{leftbar}
\begin{python}
With tf.Session() as sess:
  Saver = tf.import_meta_graph('ckp.meta')
  Saver.restore(sess, 'ckp')
\end{python}
\end{leftbar}

\subsection{file function}

After executing the \code{Saver.save} operation, the following files are generated in the file system:

\begin{leftbar}
\begin{python}
├── checkpoint
├── ckp.data-00000-of-00001
├── ckp.index
├── ckp.meta
\end{python}
\end{leftbar}

\subsubsection{index file}

The index (\ascii{index}) file holds data for an immutable table (\code{tensorflow::table::Table}); where the keyword is the name of \ascii{Tensor} and its value describes the \ascii Metadata information of {Tensor}, including which data (\ascii{data}) the \ascii{Tensor} is stored in, its offset in the data file, and its checksum.

\subsubsection{data file}

The data (\ascii{data}) file records the value of all variables \ascii{(Variable)}. When \code{restore} a variable, first find out which data file the corresponding variable is in the index file, and then directly obtain the value of the variable according to the index, thereby realizing the recovery of the variable data.

\subsubsection{meta file}

The metafile (\ascii{meta}) holds the persistence data of \code{MetaGraphDef}, which includes metadata such as \code{GraphDef, SaverDef}.

Separating the metadata describing the calculation graph from the data file storing the variable values, the separation of the static graph structure from the dynamic data representation is achieved. Therefore, when restoring \ascii{(Restore)}, first call \code{tf.import\_meta\_graph} to restore \code{GraphDef} first, then restore \code{SaverDef}, thus restoring the description static The \code{Graph} object of the graph structure, and its \code{Saver} object used to restore the value of the variable, and finally restore the values ​​of all variables using \code{Saver.restore}.

This is also the real reason why \code{tf.import\_meta\_graph} must be called before calling \code{Saver.restore} in the above example; otherwise, if you delete the instance of the calculation graph, you cannot talk about restoring the data. Go to the figure example.

\subsubsection{status file}

The \ascii{Checkpoint} file records the prefix of the most recent breakpoint file (\ascii{Checkpoint File}), and the corresponding index and data file can be found according to the prefix. When you call \code{tf.train.latest\_checkpoint}, you can quickly find the most recent breakpoint file.


In addition, the \ascii{Checkpoint} file also records a list of all breakpoint files, and the file list is sorted by the oldest to newest time. When the training task time period is very long, the breakpoint check will continue, which will result in the disk space being exhausted. To avoid this problem, there are two basic methods:

\begin{enum}
  \eitem{\code{max\_to\_keep}: Configure the maximum number of recently valid files. When a new breakpoint file is generated and the number of files exceeds \code{max\_to\_keep}, the oldest break is deleted. Point file; where \code{max\_to\_keep} defaults to \ascii{5};}
  \eitem{\code{keep\_checkpoint\_every\_n\_hours}: Perform a breakpoint check every \code{n} hours during training to ensure that there is only one breakpoint file; this option is turned off by default. }
\end{enum}

Since the \ascii{Checkpoint} file also records a list of breakpoint files, and the file list is sorted by the oldest to newest time. Deleting stale breakpoint files according to the above strategy will be extremely simple and effective.

\subsection{model}

\subsubsection{persistent model}

To implement the persistence feature, \code{Saver} inserts \code{SaveV2} and its associated \ascii{OP} in the calculation graph at construction time. Where \code{file\_name} is a \ascii{OP} of \ascii{Const}, specifying the name of the breakpoint file; \code{tensor\_names} is also a \ascii{OP} of \ascii{Const} , a list of \ascii{Tensor} names that specify training parameters.

\begin{figure}[!htbp]
\centering
\includegraphics[width=0.5\textwidth]{figures/py-saver-save-model.png}
\caption{Saver: Persistence Model}
 \label{fig:py-saver-save-model}
\end{figure}

\subsubsection{Recovery Model}

Similarly, in order to implement the recovery function, \code{Saver} inserts a \code{RestoreV2} and its associated \ascii{OP} for each training parameter during the construction period. Among them, the initializer \ascii{(Initializer)}, which restores the default value of the parameter from the breakpoint file, is essentially a \code{Assign}\ascii{OP}.

In addition, \code{file\_name} is a \ascii{OP} of \ascii{Const}, specifying the name of the breakpoint file; \code{tensor\_names} is also a \ascii{OP} of \ascii{Const} , a list of \ascii{Tensor} names specifying training parameters, the length of which is \ascii{1}.

\begin{figure}[!htbp]
\centering
\includegraphics[width=0.9\textwidth]{figures/py-saver-restore-model.png}
\caption{Saver: Recovery Model}
 \label{fig:py-saver-restore-model}
\end{figure}

\end{content}
\begin{savequote}[45mm]
\ascii{Any fool can write code that a computer can understand. Good programmers write code that humans can understand.}
\qauthor{\ascii{- Martin Flower}}
\end{savequote}

\chapter{MonitoredSession} 
\label{ch:monitored-session}

\begin{content}

To train a simple model, you can run \code{train\_op} several times until the model converges, and finally implement the training parameters \ascii{Checkpoint} to persist the training model. For small-scale learning models, this process can take up to several hours.

However, it takes several days for a large-scale learning model; and it may be necessary to use multiple copies of \ascii{(replica)}, which requires a more robust training process to support the training of the model. Therefore, there are three basic issues that need to be addressed:

\begin{enum}
  \eitem{When the training process is abnormally closed, or the program crashes, it can handle the exception reasonably;}
  \eitem{When the exception is closed, or the program crashes, the training process can be resumed;} 
  \eitem{The ability to monitor the entire training process via \ascii{TensorBoard}. }   
\end{enum}

In order to be able to resume the training process, the training must be implemented periodically (ascii{Checkpoint}) after the training is shut down abnormally or the program crashes. When the training process is restarted, the training process can be resumed by looking for the most recent \ascii{Checkpoint} file.

In order to be able to monitor the training process using \ascii{TensorBoard}, you can periodically run some \ascii{Summary}\ascii{OP} and append the results to the event file. \ascii{TensorBoard} monitors and parses the data of the event file, visualizing the entire training process, including showing the structure of the calculation graph.

\end{content}

\section{Introduction to MonitoredSession}

\begin{content}

\code{tf.train.MonitoredSession}, which can be customized to \code{Hook} for listening to the entire \code{Session} lifecycle; built-in \code{Coordinator} object for coordinating all running threads simultaneously Stop, listen, report, and handle exceptions; when \code{AbortedError} or \code{UnavailableError} exception occurs, you can restart \code{Session}.

\subsection{Usage method}

In general, first create a \code{Session} instance using \code{ChiefSessionCreator} and register the three most basic \code{tf.train.SessionRunHook}:

\begin{enum}
  \ item {\ code {CheckpointSaverHook}: Periodically \ ascii {Checkpoint};}
  \eitem{\code{SummarySaverHook}: Run \ascii{Summary} periodically;} 
  \eitem{\code{StepCounterHook}: Periodically counts the number of \ascii{Step} running per second. }   
\end{enum}

In order to be able to handle exceptions safely and to be able to close \code{MonitoredSession}, the context manager of \code{with} is often used.

\begin{leftbar}
\ Begin {python}
session_creator = tf.train.ChiefSessionCreator(
  checkpoint_dir = checkpoint_dir,
  master=master,
  config=config)

hooks = [
  tf.train.CheckpointSaverHook(
    checkpoint_dir = checkpoint_dir,
    save_secs=save_checkpoint_secs),
  tf.train.SummarySaverHook(
    save_secs=save_summaries_secs,
    output_dir=checkpoint_dir),
  tf.train.StepCounterHook(
    output_dir=checkpoint_dir, 
    every_n_steps=log_step_count_steps)
]

with tf.train.MonitoredSession(
  session_creator=session_creator,
  hooks=hooks) as sess:
  if not sess.should_stop():
    sess.run (train_op)
\end{python}
\end{leftbar}

\subsection{Use factory}

Using the factory method of \code{MonitoredTrainingSession}, you can simplify the creation of \code{MonitoredSession}.

\begin{figure}[!htbp]
\centering
\includegraphics[width=0.4\textwidth]{figures/py-train-monitored-training-session.png}
\caption{MonitoredTrainingSession:Factory Method}
 \label{fig:py-train-monitored-training-session}
\end{figure}

\begin{leftbar}
\ Begin {python}
with MonitoredTrainingSession(
  master=master,
  is_chief=is_chief,
  checkpoint_dir = checkpoint_dir
  config=config) as sess:
  if not sess.should_stop():
    sess.run (train_op)
\end{python}
\end{leftbar}

\subsection{decorator}

In order to get the composite function \code{MonitoredSession}, you can assemble the \code{WrappedSession} that completes the sub-functions.

\begin{enum}
  \eitem{\code{RecoverableSession}: When \code{AbortedError} or \code{UnavailableError} exception occurs, \code{Session} can be restored and rebuilt;}
  \eitem{\code{CoordinatedSession}: Built-in \code{Coordinator} object to coordinate all running threads while stopping, and listening, reporting, and handling exceptions; 
  \eitem{\code{HookedSession}: Customize \code{Hook} to listen to the entire \code{Session} lifecycle. }   
\end{enum}

\begin{figure}[!htbp]
\centering
\includegraphics[width=0.9\textwidth]{figures/py-train-monitored-session-decorator.png}
\caption{MonitoredSession: Decorator}
 \label{fig:py-train-monitored-session-decorator}
\end{figure}

In the end, you can combine the characteristics of the three, and build \code{MonitoredSession} (pseudo code implementation, please refer to the specific implementation of \code{MonitoredSession} for details).

\begin{leftbar}
\ Begin {python}
MonitoredSession(
  RecoverableSession(
    CoordinatedSession(
      HookedSession(
        tf.Session(target, config)))))
\end{python}
\end{leftbar}

\end{content}

\section{Lifecycle}

\begin{content}

\code{MonitoredSession} has the lifecycle feature of \code{Session} (but not the \ascii{IS-A} relationship, but the \ascii{Like-A} relationship, which is a typical style of duck programming) .

During the lifecycle, a callback hook for \code{SessionRunHook} is inserted to monitor the lifecycle of \code{MonitoredSession}.

\subsection{initialization}

In the initialization phase, \code{MonitoredSession} mainly completes the following process:

\begin{enum}
  \eitem{Run the \code{begin} method of all callback hooks;}
  \eitem{Freeze the calculation graph by calling \code{scaffold.finalize()};} 
  \eitem{Create session: Create \code{Session} using \code{SessionCreator} polymorphism}   
  \eitem{The \code{after\_create\_session} method that runs all callback hooks}
\end{enum}

Among them, there are two types of procedures for creating \ascii{Session} using \code{SessionCreator} polymorphism.

\begin{enum}
  \eitem{\code{ChiefSessionCreator}: Call \code{SessionManager.prepare\_session} to complete the model initialization by restoring the model from the nearest \ascii{Checkpoing}, or by running \code{init\_op}; then, launch All \code{QueueRunner} instances;}
  \eitem{\code{WorkerSessionCreator}: Call \code{SessionManager.wait\_for\_session} and wait for \code{Chief} to complete the initialization of the model. }
\end{enum}

\begin{figure}[!htbp]
\centering
\includegraphics[width=0.9\textwidth]{figures/py-train-monitored-session-initialization.png}
\caption{MonitoredSession: Initialization}
 \label{fig:py-train-monitored-session-initialization}
\end{figure}

\subsection{execution}

In the execution phase, call the \code{before\_run} and \code{after\_run} methods of the hook before and after running \code{Session.run}. If a \code{AbortedError} or \code{UnavailableError} exception occurs during the run, the session service is restarted.

\begin{figure}[!htbp]
\centering
\includegraphics[width=0.9\textwidth]{figures/py-train-monitored-session-execution.png}
\caption{MonitoredSession: Execute}
 \label{fig:py-train-monitored-session-execution}
\end{figure}

\subsection{close}

When the training process is finished, close the \code{MonitoredSession} by calling the \code{close} method to release the computing resources of the system.

At this point, the \code{end} method of the hook will be called back and all \code{QueueRunner} instances will be stopped by calling the \code{Coordinator.request\_stop} method. Finally, I heard that the \code{tf.Session.close} method was called to release the system resources.

In addition, if a \code{OutOfRangeError} exception occurs, \code{MonitoredSession} considers the training process to terminate normally and ignores the exception.

\begin{figure}[!htbp]
\centering
\includegraphics[width=0.9\textwidth]{figures/py-train-monitored-session-close.png}
\caption{MonitoredSession: Close}
 \label{fig:py-train-monitored-session-close}
\end{figure}

\end{content}

\section{Model Initialization}

\begin{content}

\code{MonitoredSession} At initialization, use \code{SessionCreator} to complete session creation and model initialization.

In general, in a distributed environment, there are two types of \ascii{Worker}:

\begin{enum}
  \eitem{\ascii{Chief}: Responsible for the initialization of the model;}
  \eitem{\ascii{Non-Chief}: Wait for \ascii{Chief} to complete the initialization of the model. }
\end{enum}

The initialization of the model is done together through a simple coordination protocol.

\subsection{Coordination Agreement}

For \ascii{Chief}, it will try to recover the model from the \ascii{Checkpoint} file; if it is not successful, it will initialize the model completely by executing \code{init\_op}; its initialization algorithm can be formalized as :

\begin{leftbar}
\ Begin {python}
def prepare_session(master, init_op, saver, ckp_dir):
  if is_chief():
    sess = tf.Session (master)
    sess.run (init_op) if not saver.restore (sess, ckp_dir)
\end{python}
\end{leftbar}

For \ascii{Non-Chief}, it periodically checks if \ascii{Chief} has completed the initialization of the model by running \ascii{ready\_op}.

\begin{leftbar}
\ Begin {python}
def wait_for_session(master, ready_op, recovery_wait_secs):
  while True:
    sess = tf.Session (master)
    if sess.run(ready_op):
      sex return
    else:
      sess.close()
      time.sleep(recovery_wait_secs)   
\end{python}
\end{leftbar}

\subsection{SessionManager}

In fact, the above algorithm is mainly implemented by \code{SessionManager}, which is mainly responsible for the recovery of the model from the \ascii{Checkpoint} file, or the initialization of the model is completed directly by running \code{init\_op}, and finally the creation can work. The \code{Session} instance.

\begin{enum}
  \eitem{For \ascii{Chief}, complete the initialization of the model by calling the \code{prepare\_session} method;}
  \eitem{For \ascii{Non-Chief}, wait for \ascii{Chief} to complete the initialization of the model by calling the \code{wait\_for\_session} method. }
\end{enum}

For details, please refer to the specific implementation of \code{SessionManager}.

\subsection{Introduction factory}

Using the factory method, use \code{ChiefSessionCreator} and \code{WorkerSessionCreator} respectively to complete the above algorithm.

\begin{figure}[!htbp]
\centering
\includegraphics[width=0.9\textwidth]{figures/py-train-session-creator.png}
\caption{SessionManager}
 \label{fig:py-train-session-creator}
\end{figure}

\subsection{Scaffold}

To build a model training, you need \code{init\_op} to initialize the variables; you need \code{Saver} to periodically implement \ascii{Checkpoint}; you need \code{ready\_op} to see if a model has been initialized; \code{summary\_op} collects all \ascii{Summary} for visualization of the training process.

In general, these special OPs or objects are identified by \code{GraphKey} in the calculation graph so that these special \ascii{OP} or objects can be retrieved from the calculation graph.

In the special area of ​​the training model, a basic tool library is provided: \code{Scaffold}, which is used to create default values ​​for these \ascii{OP} or objects, and added to the collection of calculation graphs, and \code{Scaffold } Provides a query interface to easily get these \ascii{OP} or objects.

You can create an instance of this type by default by calling the \code{Scaffold.finalize} method. If the corresponding \ascii{OP} or object is \code{None}. Finally, the calculation graph is frozen, and then it is forbidden to add nodes to the graph.

\begin{leftbar}
\ Begin {python}
class Scaffold(object):
  def finalize(self):
    """Creates operations if needed and finalizes the graph."""
    
    # create init \ _on
    if self._init_op is None:
      def default_init_op():
        return control_flow_ops.group(
            variables.global_variables_initializer(),
            resources.initialize_resources(
              resources.shared_resources()))
      self._init_op = Scaffold.get_or_default(
          'init_op',
          ops.GraphKeys.INIT_OP,
          default_init_op)

    # create ready\_op
    if self._ready_op is None:
      def default_ready_op():
        return array_ops.concat([
            variables.report_uninitialized_variables(),
            resources.report_uninitialized_resources()
        ], 0)
      self._ready_op = Scaffold.get_or_default(
          'ready_op', 
          ops.GraphKeys.READY_OP,
          default_ready_op)
    
    # create ready\_for\_local\_init\_op
    if self._ready_for_local_init_op is None:
      def default_ready_for_local_init_op():
        return variables.report_uninitialized_variables(
            variables.global_variables())
      self._ready_for_local_init_op = Scaffold.get_or_default(
          'ready_for_local_init_op',
          ops.GraphKeys.READY_FOR_LOCAL_INIT_OP,
          default_ready_for_local_init_op)
    
    # create local\_init\_op
    if self._local_init_op is None:
      def _default_local_init_op():
        return control_flow_ops.group(
            variables.local_variables_initializer(),
            lookup_ops.tables_initializer())
      self._local_init_op = Scaffold.get_or_default(
          'local_init_on',
          ops.GraphKeys.LOCAL_INIT_OP,
          _default_local_init_op)
    
    # create summary\_op
    if self._summary_op is None:
      self._summary_op = Scaffold.get_or_default(
          'summary_op',
          ops.GraphKeys.SUMMARY_OP,
          summary.merge_all)
    
    # create Saver
    if self._saver is None:
      self._saver = training_saver._get_saver_or_default()
    self._saver.build ()

    ops.get_default_graph().finalize()
    return self
\end{python}
\end{leftbar}

As can be seen from the implementation of \code{finalize}, the following \ascii{OP} completes the functions:

\begin{enum}
  \eitem{\code{init\_op}: Complete initialization of all global variables and global resources;}
  \eitem{\code{local\_init\_op}: Complete initialization of all local variables and tables;} 
  \eitem{\code{ready\_op}: See if all global variables and global resources have been initialized; otherwise report a list of uninitialized global variables and global resources;}   
  \eitem{\code{ready\_for\_local\_init\_op}: See if all local variables and tables have been initialized; otherwise report a list of uninitialized local variables and tables;}   
  \eitem{\code{summary\_op}: summarizes the output of all \ascii{Summary};}       
\end{enum}

Among them, local variables can not be persisted to the \ascii{Checkpoint} file; of course, the value of the local variable cannot be restored from the \ascii{Checkpoint} file.

\subsection{initialization algorithm}

Understanding the complete semantics of the \code{prepare\_session} model initialization is not that difficult by observing the definition of \ascii{OP} above.

\begin{leftbar}
\ Begin {python}
class SessionManager(object):
  def prepare_session(self,
                      master,
                      saver=None
                      checkpoint_filename=None
                      init_op = None,
                      init_feed_dict=None,
                      init_fn=None):
    """Creates a Session. Makes sure the model is ready."""

    def _restore_checkpoint():
      sess = session.Session(master)
      if not saver or not checkpoint_filename):
        Sex return, False
      else:
        saver.restore(sess, checkpoint_filename)
        return sess, True

    def _try_run_init_op (sess):
      if init_op is not None:
        sess.run(init_op, feed_dict=init_feed_dict)
      if init_fn:
        init_fn (sex)
    
    sess, is_succ = self._restore_checkpoint()
    if not is_succ:
      _try_run_init_op (sess)
    self._try_run_local_init_op (sess)
    self._model_ready(sess)
    sex return
\end{python}
\end{leftbar}

Its initialization algorithm is very simple. First, try to recover from the \ascii{Checkpoint} file (some implementations are omitted here to simplify the problem); if it fails, call \code{init\_op} and \code{init\_fn} to complete global variables and resources. Initialization; then, to initialize local variables and tables; finally, verify that all global variables and resources have been initialized.

\begin{figure}[!htbp]
\centering
\includegraphics[width=0.9\textwidth]{figures/py-train-session-initialization-algo.png}
\caption{model initialization algorithm}
 \label{fig:py-train-session-initialization-algo}
\end{figure}

\subsection{local variable initialization}

For non-empty \code{local\_init\_op}, you must wait until all global variables have been initialized before initializing (by calling \code{\_ready\_for\_local\_init\_op}); otherwise, the report is not initialized List of global variables into the \code{msg} field.

That is, local variables are initialized after the global variable is initialized, and local variables are not persisted into the \ascii{Checkpoint} file.

\begin{leftbar}
\ Begin {python}
class SessionManager(object):
  def _ready_for_local_init(self, sess):
    """Checks if the model is ready to run local_init_op.
    """
    return _ready(self._ready_for_local_init_op, sess,
                  "Model not ready for local init")

  def _try_run_local_init_op(self, sess):
    """Tries to run _local_init_op, if not None, 
       and is ready for local init.
    """
    if not self._local_init_op:
      return True, None:
    
    is_ready, msg = self._ready_for_local_init(sess)
    if is_ready:
      sess.run (self._local_init_op)
      return True, None
    else:
      return False, msg
\end{python}
\end{leftbar}

\subsection{Verify model}

Finally, by executing \code{\_ready\_op}, see if all global variables and global resources have been initialized; otherwise, report the list of uninitialized variables to the \code{msg} field.

\begin{leftbar}
\ Begin {python}
class SessionManager(object):
  def _model_ready(self, sess):
    """Checks if the model is ready or not.
    """
    return _ready(self._ready_op, sess, "Model not ready")
\end{python}
\end{leftbar}

Among them, \code{\_ready} uses a function to run the corresponding \code{ready\_op} to see if the corresponding variable or resource is initialized.

\begin{leftbar}
\ Begin {python}
def _ready(op, sess, msg):
  """Checks if the model is ready or not, as determined by op.
  """
  if op is None:
    return True, None

  ready_value = sess.run(op)
  if (ready_value.size == 0):
    return True, None
  else:
    uninitialized_vars = ", ".join(
        [i.decode("utf-8") for i in ready_value])
    return False, "initialized vars: " + uninitialized_vars
\end{python}
\end{leftbar}

\end{content}

\section{Exception safe}

\begin{content}

In general, the \code{with} context manager is often used to implement the exception security and resource security release of \code{MonitoredSession}.

\subsection{Context Manager}

When the \code{with} statement is exited, all \code{QueueRunner} instances will be stopped and a secure shutdown of \code{tf.Session} will be implemented.

\begin{leftbar}
\ Begin {python}
class _MonitoredSession(object):
  def __exit__(self, exception_type, exception_value, traceback):
    if exception_type in [errors.OutOfRangeError, StopIteration]:
      exception_type = None
    self._close_internal(exception_type)
    return exception_type is None
  
  def _close_internal(self, exception_type=None):
    try:
      if not exception_type:
        for h in self._hooks:
          h.end (self.tf_sess)
    finally:
      try:
        self._sess.close()
      finally:
        self._sess = None
        self.tf_sess = None
        self.coord = None  
\end{python}
\end{leftbar}

In particular, when \code{OutOfRangeError} or \code{StopIteration} occurs, it is considered to terminate normally, ignoring the exception. If other types of exceptions are thrown, the callback hook for \code{end} will not be called.

\subsection{Stop QueueRunner}

In addition, when \code{self.\_sess.close()} is executed, the \code{close} method of \code{\_CoordinatedSession} will eventually be called. Notify all \code{QueueRunner} instances to stop running by calling \code{coord.request\_stop} and wait for all \code{QueueRunner} instances to run by calling the \code{coord.join} method.

\begin{leftbar}
\ Begin {python}
class _CoordinatedSession(_WrappedSession):
  def close(self):
    self._coord.request_stop()
    try:
      self._coord.join()
    finally:
      try:
        _WrappedSession.close(self)
      except Exception:
        pass
\end{python}
\end{leftbar}

\end{content}

\section{callback hook}

\begin{content}

You can monitor and manage the \code{MonitorSession} lifecycle process by customizing \code{SessionRunHook}.

\begin{leftbar}
\ Begin {python}
class SessionRunHook(object):
  def begin(self):
    pass

  def after_create_session(self, session, coord):
    pass

  def before_run(self, run_context):
    return None

  def after_run(self, run_context, run_values):
    pass

  def end(self, session):
    pass
\end{python}
\end{leftbar}

Among them, the most common \code{Hook} includes:

\begin{enum}
  \ item {\ code {CheckpointSaverHook}: 周期性 地 \ ascii {Checkpoint};}
  \eitem{\code{SummarySaverHook}: Run \ascii{Summary} periodically;} 
  \eitem{\code{StepCounterHook}: Periodically counts the number of \ascii{Step} running per second. }   
\end{enum}

\begin{figure}[!htbp]
\centering
\includegraphics[width=0.9\textwidth]{figures/py-train-session-run-hook.png}
\caption{SessionRunHook}
 \label{fig:py-train-session-run-hook}
\end{figure}

\end{content}







%%%%%%%%%%%%%%%%%%%%%
\appendix

\part{Appendix}
\chapter{code reading} 
\label{ch:code-reading}

\begin{content}

In the programmer's daily work, most of the time is in \emph{reading code}, not writing code. However, reading the code is often a very boring thing, especially when you encounter a design that is not beautiful, the psychology of resistance is often stronger. In fact, changing the habits, ideas and methods, code reading is actually a very enjoyable process.

Read the patterns, practices, and habits of the code, and the masters of the Greek author \ascii{Diomidis Spinellis}: \ascii{Code Reading, The Open Source Perspective}. This article starts from another perspective, talks about some of my habits of reading the code, and looks forward to finding more people's resonance.

\end{content}

\section{Working for good things, you must first sharpen it}

\begin{content}

First, prepare a good toolbox before reading the code, including \ascii{IDE}, \ascii{UML}, brain map and other tools. The programming languages ​​I mainly use include \ascii{C++, Scala, Java, Ruby, Python}; I prefer to use \ascii{JetBrains} products, many of which are very intimate to programmers.

Second, the efficient use of shortcuts, which is a good code reading habit, it greatly improves the efficiency and quality of code reading. For example, look at class hierarchy, function call chains, method reference points, and more.

\begin{remark}
Unplug the mouse and reduce the dependency on the mouse. When you find that there is no mouse and the work cannot be carried out, try to find the corresponding shortcut. Through daily accumulation of bits and pieces, work efficiency can be doubled.
\end{remark}

\end{content}

\section{力行而知知之}

\begin{content}

A common anti-pattern for reading code is to read the code in the way of \ascii{Debug}. The author does not recommend this way of reading code. First, because the switching between threads at run time can easily lead to the loss of direction. Second, understanding the code call stack is not effective for understanding system behavior, because it contains too many implementation details. It is not easy to find the nature of the problem.

But before reading the code, there are a few things that must be done. First, manually build a project and run test cases; second, hand-write several \ascii{Demo} to feel it.

The project is run first, not for the \ascii{Debug} code, but for understanding the way the project is built, and understanding the basic structure of the system, and understanding how the system is used.

If conditions permit, you can try to discover and mine the behavior of the system using \ascii{ATDD}. Through this process, thinking of yourself as a customer and thinking about the behavior of the system is the most important cornerstone for understanding the system.

\end{content}

\section{Discover domain model}

\begin{content}

Reading the code, not to understand each class, what each function does, but to mine more essential, less variable knowledge. In fact, the discovery of \emph{domain model} is the most important goal of reading code, because the domain model is the soul of the system. Through the code reading, find the knowledge of the essence of the system, and express it through its own mode, in order to truly grasp the context of the system, otherwise everything is empty talk.

For example, when reading the client code implemented by \tf{}'s \ascii{Python}, it is extremely important to understand the domain model of the computation graph and understand the programming model of \ascii{TensorFlow} and its runtime behavior.

\begin{figure}[!htbp]
\centering
\includegraphics[width=0.9\textwidth]{figures/py-graph.png}
\caption{Domain object: Graph}
 \label{fig:py-graph}
\end{figure}

\end{content}

\section{Excavation System Architecture}

\begin{content}

Reading the code is like sailing in the sea, and the system architecture diagram is a nautical chart. Reading the code can't be without the overall system concept, otherwise the results will be poor and the reading quality will be greatly discounted. You must have a systematic mindset and a clear goal so that you don't lose your way.

The first task is to find the boundaries of the system and to think about the behavioral characteristics of the external system in abstract thinking. Second, clarifying the interactions, relationships, and responsibilities between components in the system is critical to understanding the behavior of the entire system.

For example, for \ascii{TensorFlow}, \ascii{C API} is a bridge between the front and back systems. Understand the design of \ascii{C API} and basically be able to guess the behavior of the front-end system.

\begin{figure}[!h]
\centering
\includegraphics[width=0.9\textwidth]{figures/tf-architecture-simple.png}
\caption{TensorFlow System Architecture}
 \label{fig:tf-architecture-simple}
\end{figure}

\end{content}

\section{Details are the devil}

\begin{content}

The entanglement in the details will lead to a significant discount in the efficiency and quality of the code reading code. For example, log printing, solving a patch implementation of \ascii{Bug}, a compatibility scheme for a version branch, a hammer code implementation for some metamorphosis requirements, and so on.

A common anti-pattern for reading code is "annotating code." This is a practice that is highly inefficient and inefficient, with a very low input-output ratio. In general, the more elegant the system, the fewer comments; the more complex the system, the more comments are not helpful.

I have a habit of reading code, creating a separate \ascii{code-reading} branch for code reading, and reading the code while deleting the extraneous code.

\begin{leftbar}
\begin{scala}
$ git checkout -b code-reading
\end{scala}
\end{leftbar}

After removing these noises, you will find that the system is not as complicated as you might think. In reality, the complexity of the system is often the additional complexity caused by immature design and implementation. With the in-depth understanding of the system, many details will naturally surface, and all mysterious veil will be revealed and publicized.

\end{content}

\section{可可止}

\begin{content}

A common anti-pattern of reading code is "a rib is going to the end, not to the Yellow River." Programmers have a curiosity and are always interested in things that are unclear. For example, how is the message sent out? What is the principle of task scheduling? How is data storage done? Although this kind of courage is worthy of praise, it is definitely not worth encouraging when reading the code.

Another common anti-pattern is the "Tracking Function Call Stack". This is an extremely boring process that often leads to the rigidity of thinking; because you live forever in the shadow of the author, there is no self at all.

When I read the code personally, the depth of the function call stack never exceeds \ascii{3}, and then the underlying call is thought of using abstract thinking. Because I found that with age, the memory that I have been proud of now is gradually becoming my own shortcoming. When I tried to track the deep call stack, the previous reading information completely disappeared.

In other words, I am more accustomed to "wide traversal" rather than accustomed to "deep traversal" reading. In this way, I can find the "hierarchical concept" that the system conceals and rationalize the hierarchy of the system.

\end{content}

\section{Discover her beauty}

\begin{content}

Threesome, there must be my teacher. When reading code, when you find a good design, including implementation mode, idioms, etc., don't miss it; otherwise, after a while, this code reading will be of little value to you.

When I find a good design, I will try to use the class diagram, state machine, sequence diagram, etc. to express the design; if you find potential deficiencies, add your own ideas, it will be more perfect.

For example, when I read \ascii{Hamcrest}, I tried to draw a class diagram, and realized the relationship between them, and felt the beauty of the design, which also benefited a lot.

\begin{figure}[!h]
\centering
\includegraphics[width=0.9\textwidth]{figures/hamcrest.png}
\caption{Combination design}
 \label{fig:hamcrest}
\end{figure}

\end{content}

\section{Try to refactor}

\begin{content}

Because this is a process of code reading, there is no potential risk due to refactoring. In some complex logic, it can be made more explicit and intuitive through the reconstruction of equivalent transformations.

For a giant function, I often extract an abstract code hierarchy to discover its underlying logic. For example, this is a \ascii{ArrayBuffer} implemented with \ascii{Scala}, and when you need to add an element to the end, the existing design is like this.

\begin{leftbar}
\begin{python}
Def +=(elem: A): this.type = {
  If (size + 1 > array.length) {
    Var newSize: Long = array.length
    While (n > newSize)
      newSize *= 2
    newSize = math.min(newSize, Int.MaxValue).toInt
  
    Val newArray = new Array[AnyRef](newSize)
    System.arraycopy(array, 0, newArray, 0, size)
    Array = newArray
  }
  Array(size) = elem.asInstanceOf[AnyRef]
  Size += 1
  This
}
\end{python}
\end{leftbar}

This code creates a huge obstacle to reading, and I will try to find the backbone of the logic through a quick function extraction.

\begin{leftbar}
\begin{python}
Def +=(elem: A): this.type = {
  If (atCapacity)
    Grow()
  addElement(elem)
}
\end{python}
\end{leftbar}

As for how \code{atCapacity, grow, addElement} is implemented, don't care, because I have achieved the effect of reading the code.

\end{content}

\section{Formalization}

\begin{content}

When reading the code, some people are used to drawing the "flow chart" of the program. On the contrary, I almost never draw a "flowchart" because the flow chart reflects too many implementation details and does not profoundly reflect the nature of the algorithm.

I prefer to use a "formal" approach to describe the problem. It has a mathematical aesthetic, a simple expression, and its highly abstract thinking, which is critical to the nature of the problem.

For example, for the problem of \ascii{FizzBuzzWhizz}, the formalization is more simple and expressive than a lengthy textual description or flowchart. With \ascii{3, 5, 7} as input, after formalization, the essence of the problem can be clearly revealed.

\begin{leftbar}
\begin{python}
R1: times(3) => Fizz || 
    Times(5) => Buzz ||
    Times(7) => Whizz

R2: times(3) && times(5) && times(7) => FizzBuzzWhizz ||
    Times(3) && times(5) => FizzBuzz ||
    Times(3) && times(7) => FizzWhizz ||
    Times(5) && times(7) => BuzzWhizz

R3: contains(3) => Fizz

Rd: others => string of others

Spec: r3 || r2 || r1 || rd
\end{python}
\end{leftbar}

\end{content}

\section{instantiate}

\begin{content}

Instantiation is an important way to understand the problem. When the logic is very complicated, a simple example often makes you suddenly open. Ideally, instantiation can be made into automated test cases that describe the behavior of the system.

If there is an algorithm and implementation that is quite complex, it can also be used to instantiate the working principle of the algorithm, which is very beneficial for understanding the problem itself.

Take the \ascii{DAG} algorithm in \ascii{Spark} as an example. Starting with \ascii{G}, the boundaries of each \ascii{Stage} are identified in turn according to the dependency of \ascii{RDD}.

\begin{figure}[!htbp]
\centering
\includegraphics[width=0.9\textwidth]{figures/spark-stage-dag.png}
\caption{Spark: Stage partitioning algorithm}
 \label{fig:spark-stage-dag}
\end{figure}

\begin{itemize}
 Division of \item \ascii{Stage 3}
   \begin{enum}
     Between \eitem{\ascii{G} and \ascii{B} is a narrow dependency, the same as \ascii{Stage(3)};}
     Between \eitem{\ascii{B} and \ascii{A} is a wide dependency, \ascii{A} is the new starting \ascii{RDD}, which is called recursively;
     \eitem{\ascii{G} is a wide dependency with \ascii{F}, and \ascii{F} is the new starting \ascii{RDD}, which is called recursively. }
   \end{enum}

 Division of \item \ascii{Stage 1}
   \begin{enum}
     \eitem{\ascii{A}No father} RDD}, \ascii{Stage(1)} The end of the division. In particular, \ascii{Stage(1)} only contains \ascii{RDD A}. }
   \end{enum}
 Division of \item \ascii{Stage 2}
   \begin{enum}
     \eitem{Since the relationship between \ascii{RDD} is narrowly dependent, the rule is the same \ascii{Stage(2)};}
     \eitem{until \ascii{RDD C, E}, because there is no father \ascii{RDD}, \ascii{Stage(2)} ends. }
   \end{enum} 
\end{itemize}

Finally, the dependency of \ascii{Stage} is formed, and \ascii{TaskSet} to \ascii{TaskScheduler} are submitted in turn for scheduling execution.

\end{content}

\section{独乐乐, not as good as music}

\begin{content}

Sharing your experience with others, you may find more inspiration; especially for people who are familiar with the field, if it is better than \ascii{Owner}, you will definitely get unexpected surprises and gains.

You can also collect the experience of others through various channels, and combine your own thinking to find out your own understanding, so that you can put knowledge into your own bag.

Reading the code is not a one-person world; you should go out and participate in some community activities to understand the mainstream research direction, technological dynamics, and industrial development in the ecosystem, which is extremely helpful for understanding the business.


\end{content}

\chapter{Continuous learning} 
\label{ch:learning}


\section{Explain}

\begin{content}

\begin{remark}
There are three readings, which means that the heart is coming, the eyes are coming, and the mouth is coming. - Zhu Xi, "The Training Regulations"
\end{remark}

When I was born, my father named me \emph{Liu Guangyun}, Cheng \quo{光} generation, word \quo{云}. But since I went to school, I don’t know what it is. One day, the Chinese teacher said that the text explained: \quo{Cong, ears, eyes, mouth, heart is also}. I feel that I have a special liking for the \quo{聪} word, and I changed my name to \emph{刘光聪}.

It’s also a coincidence that since then, my mother has never worried about my studies.

\subsection{select}

Ear to the ear, choose the good person from it, choose the one who is not good and change it. Some people are accustomed to giant functions, big logic, why are they so, and their names are: for the effect of \ascii{(hai)} rate\ascii{(zi)}; and I prefer a layered code style. Short and lean, with clear intentions.

The experience of others is important, but we need to receive it selectively, not blindly. Don't be fascinated by the master, the master sometimes makes mistakes. The key lies in self-reflection and good at discerning. Especially in this impetuous world, those who can run on the ground are calling \quo{master}.

\subsection{abstract}

Seeing, sweeping the exceptions, straightforward. What you can see at a glance is an illusion. You can't see it. It's often the essence that you can't touch. Some people are accustomed to the logic of flattening and letting them repeat; and I prefer abstraction and regard the process of revealing the essence as a kind of enjoyment.

Abstraction, of course, has complexity. But such complexity is contextual. If you have similar experience, abstraction becomes a pattern. That is a kind of beauty, a medium of communication.

If the other party lacks context, abstraction is naturally difficult. The so-called simplicity is the revealing of the nature of the problem, and it pays the least price for it; instead of being flat and straightforward, simplicity is the beauty that those outsiders can never feel.

Too late, blind abstraction, will inevitably increase unnecessary complexity. It is like a large-scale pre-design, talk about the various needs of customers, talk about various changes in software design, blind abstraction.

\subsection{share}

Oral, preaching, teaching, and confusing. Sharing is a belief in life. When you understand sharing, you naturally understand the meaning of existence. I like to share my knowledge and use it as a learning motivation to urge myself to thoroughly understand the nature of the problem.

Because of the ability to share, knowledge naturally becomes something of its own. Daily \ascii{Code Review}, I often encourage team members to actively share, one to promote the team without differences, and the other to help the sharer thoroughly understand the nature of the problem.

The key to convincing others is to give others reasons to be convinced. While sharing, it can help to exercise your expressive ability, which takes a long time \emph{deliberate practice}.

\subsection{Comprehension}

When you are in your heart, you can learn and think, and you can get it if you don’t think about it. Only through independent thinking and summing up the knowledge that is truly your own.

I prefer to use charts to summarize knowledge. On the one hand, the expressive power of the graph is far greater than that of the text; in addition, by drawing, it is also forced to thoroughly understand the nature of the problem.

\end{content}

\section{The road to growth}

\begin{content}

\subsection{deduplication}

The code needs to eliminate duplication, and the habit of working also eliminates duplication. Don't be stuck with the inherent working conditions. Repeated working conditions often make people fall into a comfortable illusion and fall into the crisis of \emph{three-year effect}.

\subsection{Refining knowledge}

First of all, we are not learning information, but knowledge. Knowledge is valuable, and information has no value. It is only through your own screening, refining, and summarization that information can be turned into knowledge.

\subsection{Become a habit}

Knowledge is easy to forget. Only by putting knowledge into action and integrating it into your own work state can you become your own property forever.

For example, the use of shortcut keys, do not deliberately remember, but become a work habit of yourself; do not repeat the labor, use \ascii{Shell} to provide automation, let \ascii{Shell} become work efficiency The weapon will become a work habit.

\subsection{Update Knowledge}

We need to constantly update existing knowledge systems, especially in an era of knowledge explosion. I hate those who believe in dogma. To give a simple example, old code specifications often require \code{YODA Notation} idioms like \code{if (NULL != p)}. But such an expression compiler is happy, but it is very unfriendly to programmers.

\ascii{\quo{if you are at least 18 years old}} is more in line with English expression habits than \ascii{\quo{if 18 years is less than or equal to your age}}.

Some people refute this idiom, but modern compilers often report warnings for such misuse; and keep the pace of development of \ascii{TDD}, small steps forward, such low-level errors are hard to escape from the French Open.

\subsection{refactoring self}

Learn, then know enough; teach, then know sleepy. Don't stay at the origin, you should always refactor your knowledge system.

When I first started designing \ascii{OO}, I had nowhere to design patterns; because all the books I saw were about how well the design pattern was. Until I saw some ideas from evolutionary design, simple design and over-engineering, let me return to reason.

\subsection{Specialized in practice}

People's energy is limited, and one cannot grasp all the knowledge in the world. Rather than hesitating in the choice of programming language, it is better to thoroughly understand the inherent nature of methodology; rather than being unresolved in many frameworks, it is better to focus on the problem itself.

In short, Bo is not refined, it must be prevented.

\end{content}

%%%%%%%%%%%%%%%%%%%%
\backmatter
% \listoffigures
% \myclearpage

% \listoftables
% \myclearpage

\bibliographystyle{alpha}
\renewcommand\bibname{References}
\begin{thebibliography}{20}

\ascii{

\bibitem{tf-white-paper} M.\ Abadi, A.\ Agarwal.
  \newblock \emph{Tensorflow, Large-scale machine learning on heterogeneous distributed systems}.
  \ newblock arXiv preprint, 1603.04467, 2016. \\
  \url{https://arxiv.org/abs/1603.04467}.

\bibitem{theano-framework} R.\ Al-Rfou, G.\ Alain, 
  \newblock \emph{Theano: A Python framework for fast computation of mathematical expressions}. 
  \ newblock arXiv preprint, 1605.02688, 2016. \\
  \url{https://arxiv.org/abs/1605.02688}.

\bibitem{mxnet-framework} T.\ Chen, M.\ Li. 
  \newblock \emph{MXNet: A flexible and efficient machine learning library for heterogeneous distributed systems}. 
  \newblock In Proceedings of LearningSys, 2015. \\
  \url{www.cs.cmu.edu/̃muli/file/mxnet-learning-sys.pdf}.  

\bibitem{distributed-dnn} J. Dean, G. S. Corrado. 
  \newblock \emph{Large scale distributed deep networks}.
  \newblock In Proceedings of NIPS, pages 1232–1240, 2012. \\
  \url{http://research.google.com/archive/large_deep_networks_nips2012.pdf}.

\bibitem{eigen} Ga\"{e}l Guennebaud.
  \newblock \emph{Eigen: a c++ linear algebra library}. \\
  \ {Url} http://downloads.tuxfamily.org/eigen/eigen_CGLibs_Giugno_Pisa_2013.pdf.

}

\end{thebibliography}

\endinput

\end{document}
%%%% End of the body part
%%%%%%%%------------------------------------------------------------------------
