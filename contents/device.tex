\begin{savequote}[45mm]
\ascii{Any fool can write code that a computer can understand. Good programmers write code that humans can understand.}
\qauthor{\ascii{- Martin Flower}}
\end{savequote}

\chapter{device} 
\label{ch:device}

\begin{content}

\end{content}

\section{Device Specification}

\begin{content}

The device specification \ascii{(Device Specification)} is used to describe the specific location of the \ascii{OP} storage or computing device.

\subsection{formalization}

A device specification can be formally described as:

\begin{leftbar}
\ Begin {python}
DEVICE_SPEC ::= COLOCATED_NODE | PARTIAL_SPEC
COLOCATED_NODE ::= "@" NODE_NAME
PARTIAL_SPEC ::= ("/" CONSTRAINT) *
CONSTRAINT ::= ("job:" JOB_NAME)
             | ("replica:" [1-9][0-9]*)
             | ("task:" [1-9][0-9]*)
             | ( ("gpu" | "cpu") ":" ([1-9][0-9]* | "*") )
\end{python}
\end{leftbar}

\subsubsection{full designation}

As shown in the following example, a complete description of a \ascii{OP} is placed in the \ascii{PS} job, \ascii{0} backup, \ascii{0} task, \ascii{GPU 0} device.

\begin{leftbar}
\ Begin {python}
/job:ps/replica:0/task:0/device:GPU:0
\end{python}
\end{leftbar}

\subsubsection{partially specified}

Equipment specifications can also be specified in part or even empty. For example, the following example only describes the device number \code{GPU0}.

\begin{leftbar}
\ Begin {python}
/device:GPU:0
\end{python}
\end{leftbar}

In particular, when the device specification is empty, it means that the device constraint is not implemented for \ascii{OP}, and the device is automatically selected to place \ascii{OP} at runtime.

\ subsubsection {peer}

Use \code{COLOCATED\_NODE} to indicate that the \ascii{OP} is placed on the same device as the specified node. For example, the node is placed on the same device as \code{other/node}.

\begin{leftbar}
\ Begin {python}
@other/node  # colocate with "other/node"
\end{python}
\end{leftbar}

\subsubsection{DeviceSpec}

A device specification can be represented using a string, or \code{DeviceSpec}. Where \code{DeviceSpec} is a value object, using the following \ascii{5} identifiers to determine the device specification.

\begin{enum}
  \eitem{job name}
  \eitem{backup index}
  \eitem{Task index}
  \eitem{device type}
  \eitem{device index}
\end{enum}

For example, use the device specification constructed by \code{DeviceSpec}.

\begin{leftbar}
\ Begin {python}
# '/job:ps/replica:0/task:0/device:CPU:0'
DeviceSpec(job="ps", replica=0, task=0, device_type="CPU", device_index=0)
\end{python}
\end{leftbar}

\subsection{Context Manager}

The \ascii{OP} device specification is often specified using the context manager \code{device(device\_spec)}, and the \code{OP} constructed within the scope of the context will be placed on the specified device at runtime.

\begin{leftbar}
\ Begin {python}
with g.device('/gpu:0'):
  # All OPs constructed here will be placed on GPU 0.
\end{python}
\end{leftbar}

Among them, \code{device} is a method of \code{Graph}, which designs a stack structure context manager to implement the closure, merge, and overlay of device specifications.

\subsubsection{Merge}

Two different ranges of device specifications can be combined.

\begin{leftbar}
\ Begin {python}
with device("/job:ps"):
  # All OPs constructed here will be placed on PS.
  with device("/task:0/device:GPU:0"):
    # All OPs constructed here will be placed on
    # /job:ps/task:0/device:GPU:0
\end{python}
\end{leftbar}

\subsubsection{override}

When merging two identical ranges of device specifications, the internally specified device specification has a high priority to achieve coverage of the device specification.

\begin{leftbar}
\ Begin {python}
with device("/device:CPU:0"):
  # All OPs constructed here will be placed on CPU 0.
  with device("/job:ps/device:GPU:0"):
    # All OPs constructed here will be placed on
    # /job:ps/device:GPU:0
\end{python}
\end{leftbar}

\subsubsection{reset}

In particular, when the internal device specification is set to \code{None}, the definition of all external device specifications is ignored.

\begin{leftbar}
\ Begin {python}
with device("/device:GPU:0"):
  # All OPs constructed here will be placed on CPU 0.
  with device(None):
    # /device:GPU:0 will be ignored.
\end{python}
\end{leftbar}

\subsubsection{Device Specification Function}

When specifying a device specification, it is often described using a string, or \code{DeviceSpec}. You can also use the more flexible \emph{device specification function}, which provides a more flexible extension to specify device specifications. The device specification function is a callback function with the input parameter \code{Operation}, which generates a device specification in string format.

\begin{leftbar}
\ Begin {python}
def matmul_on_gpu(n):
 if n.type == "MatMul":
   return "/gpu:0"
 else:
   return "/cpu:0"

with g.device(matmul_on_gpu):
  # All OPs of type "MatMul" constructed in this context
  # will be placed on GPU 0; all other OPs will be placed
  # on CPU 0.
\end{python}
\end{leftbar}

\subsubsection{implementation}

\code{Graph.device(spec)} implements a stack-structured context manager that accepts device specifications in string format, or \emph{device specification functions}. In fact, when passing a string to \code{device} function, or \code{DeviceSpec}, a simple adaptation of the string, or \code{DeviceSpec}, is first converted to a device specification function.

\begin{leftbar}
\ Begin {python}
class Graph(object):
  def device(self, device_name_or_func):
    def to_device_func():
      if (device_name_or_func is not None
          and not callable(device_name_or_func)):
        return pydev.merge_device(device_name_or_func)
      else:
        return device_name_or_func

    try:
      self._device_function_stack.append(to_device_func())
      yield
    finally:
      self._device_function_stack.pop()
\end{python}
\end{leftbar}

When a user uses \code{device}, the graph instance is not explicitly specified, and the globally unique default graph instance is implicitly used. That is, the \code{tf.device(spec)} function is actually a simple wrapper around \code{get\_default\_graph().device(spec)}.

\begin{leftbar}
\ Begin {python}
# tensorflow/python/framework/ops.py
def device(device_name_or_function):
  return get_default_graph().device(device_name_or_function)
\end{python}
\end{leftbar}

\subsubsection{Apply}

In the \code{Graph.device} implementation, \code{pydev.merge\_device} generates a device specification function. The device function creates a new copy with the input \code{spec} and merges the existing \code{node\_def.device} device specification by calling \code{copy\_spec.merge\_from(current\_device)} And \code{node\_def.device} has a higher priority. The implementation is more obscure, why does \code{node\_def.device} have a higher priority? This depends on the need for \code{\_apply\_device\_functions} to override, merge, and reset.

\begin{leftbar}
\ Begin {python}
def merge_device(spec):
  # replace string to DeviceSpec
  if not isinstance(spec, DeviceSpec):
    spec = DeviceSpec.from_string(spec or "")

  # returns a device function that merges devices specifications
  def _device_function(node_def):
    current_device = DeviceSpec.from_string(node_def.device or "")
    copy_spec = copy.copy(spec)

    # IMPORTANT: `node\_def.device` takes precedence.
    copy_spec.merge_from(current_device)      
    return copy_spec
  return _device_function
\end{python}
\end{leftbar}

When \code{Graph.create\_op}, \code{\_apply\_device\_functions} is set to set the device specification for \code{NodeDef}. It will perform a pop operation on \code{\_device\_function\_stack} in turn, and call the corresponding device specification function, and set the result directly to the device specification of \code{NodeDef}. Thus, the inline specified \code{device} has a higher priority, which implements merging, overwriting, and resetting the peripheral-specific \code{device}.

\begin{leftbar}
\ Begin {python}
class Graph(object):
  def _apply_device_functions(self, op):
    for device_function in reversed(self._device_function_stack):
      if device_function is None:
        break
      # IMPORTANT: `node\_def.device` takes precedence.  
      op._set_device (device_function (on)) 
\end{python}
\end{leftbar}

\end{content}