\makenoidxglossaries

% in the text, use \gls{label} for singluar and \glspl{label} for plural

% English - Chinese terminology comparison:

\newglossaryentry{dataflow-graph}{name={dataflow graph},description={数据流图}}
\newglossaryentry{node}{name={node},description={节点}}
\newglossaryentry{edge}{name={edge},description={边}}
\newglossaryentry{layer}{name={layer},description={层}}

\newglossaryentry{d-a-g}{name={directed acyclic graph},description={有向无环图}}
\newglossaryentry{dist-rt}{name={distributed runtime},description={分布式运行}}
\newglossaryentry{local-ds}{name={local device set},description={本地设备集}}
\newglossaryentry{dist-sys}{name={distributed system},description={分布式系统}}

\newglossaryentry{unsuper-learn}{name={unsupervised learning},description={非监督学习}}
\newglossaryentry{rein-learn}{name={reinforced learning},description={强化学习}}
\newglossaryentry{ps}{name={parameter server},description={参数服务器}}
\newglossaryentry{mod-rep}{name={model replica},description={模型副本}}
\newglossaryentry{nfnn}{name={non-feedforward neural network},description={非前馈神经网络}}


% Acronyms:

\newacronym{DAG}{DAG}{directed acyclic graph}
\newacronym{operator}{OP}{operator}
\newacronym{PS}{PS}{Parameter Server}
