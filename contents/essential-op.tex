\begin{savequote}[45mm]
  \ascii{Any fool can write code that a computer can understand. Good programmers write code that humans can understand.}
  \qauthor{\ascii{- Martin Flower}}
\end{savequote}


\chapter{OP Essentials} 
\label{ch: essential-op}
\begin{content}
\end{content}


\section{OP registration}
\begin{content}
In the \cpp{} backend system, the system completes the registration of all \ascii{OP} when the system is initialized. The registration of \ascii{OP} is done via the \code{REGISTER\_OP} macro.


\subsection{REGISTER\_OP}
In practice, \code{REGISTER\_OP} defines a set of exquisite internal \ascii{DSL}, the system automatically completes the translation of the string representation, and converts it to the internal representation of \code{OpDef}, and finally saves it in \code{OpDef} in the repository.

\begin{figure}[!h]
  \centering
  \includegraphics[width=0.9\textwidth]{figures/cc-op-repo.png}
  \caption{REGISTER\_OP: Utility macro for registering OP}
  \label{fig:cc-op-repo}
\end{figure}


\subsection{Query Interface}

\begin{leftbar}
\begin{c++}
struct OpRegistryInterface {
  virtual ~OpRegistryInterface() {}

  virtual Status LookUp(
    const string& op_name,
    const OpRegistrationData ** on_reg_data) const = 0;
  Status LookUpOpDef(const string& op_name, const OpDef** op_def) const;
};
\end{c++}
\end{leftbar}

Among them, \code{OpRegistrationData} describes the two basic metadata of \ascii{OP}: \code{OpDef} and \code{OpShapeInferenceFn}; the former is used to describe the input/output parameter information of \ascii{OP}, attribute Name list, and its constraint relationship. The latter is used to describe the deduction rules for \ascii{Shape} of \ascii{OP}.

\begin{leftbar}
\begin{c++}
struct OpRegistrationData {
  OpDef op_def;
  OpShapeInferenceFn shape_inference_fn;
};

using OpRegistrationDataFactory = 
  std::function<Status(OpRegistrationData*)>;
\end{c++}
\end{leftbar}


\subsection{OpDef repository}
In the implementation, the technique of delay initialization is adopted. To simplify the description of the problem, here is a simple code refactoring to help you understand how \code{OpRegistry} works.

\begin{leftbar}
\begin{c++}
struct OpRegistry : OpRegistryInterface {  
  OpRegistry();
  ~OpRegistry() override;

  void Register(const OpRegistrationDataFactory& factory);

 private:
  Status LookUp(
     const string& op_name,
     const OpRegistrationData** op_reg_data) const override;

 private:
  using Registry = 
    std::unordered_map<string, OpRegistrationData*>;

  mutex mu_;
  Registry registry_;
};
\end{c++}
\end{leftbar}

\begin{leftbar}
\begin{c++}
Status OpRegistry::Register(
  const OpRegistrationDataFactory& factory) {
  auto op_reg_data(std::make_unique<OpRegistrationData>());
  Status s = factory(op_reg_data.get());
  if (s.ok()) {
    gtl::InsertIfNotPresent(&registry_, 
      op_reg_data-> op_def.name (),
      op_reg_data.get ())
  }
  if (s.ok()) {
    op_reg_data.release ();
  } else {
    op_reg_data.reset ();
  }
  return watcher_status;
}
\end{c++}
\end{leftbar}

\end{content}
