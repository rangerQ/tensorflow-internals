%%%%%%%% ------------------------------------------ ------------------------------
%%%% xeCJK related package

\usepackage{xltxtra,fontspec,xunicode}

%% \ CJKsetecglue {\ hskip 0.15em plus 0.05em minus 0.05th}
%% slanfont: Allow italic
%% boldfont: Allow bold
%% CJKnormalspaces: Only whitespace between Chinese characters is ignored, but the gap between Chinese and English is preserved. 
%% CJKchecksingle: Avoid single characters in a single line.
\usepackage[slantfont, boldfont]{xeCJK} 
% \usepackage{ctex}

%% breaks the line for Chinese
\XeTeXlinebreaklocale "zh"             

%% gives TeX a certain degree of freedom
\XeTeXlinebreakskip = 0pt plus 1pt minus 0.1pt

%%%% xeCJK setting ends                                       
%%%%%%%% ------------------------------------------ ------------------------------

%%%%%%%% ------------------------------------------ ------------------------------
%%%% xeCJK font settings

%% Set Chinese punctuation style, support quanjiao, banjiao, kaiming, etc.
\punctstyle{quanjiao}                                        
                                                     
%% set default Chinese font
\setCJKmainfont[BoldFont={Adobe Heiti Std}, ItalicFont={Adobe Kaiti Std}]{Adobe Song Std}   %  FZBaoSongZ04
%% set Chinese sans serif font
\setCJKsansfont[BoldFont={Adobe Heiti Std}, ItalicFont={Adobe Kaiti Std}]{Adobe Kaiti Std} 
%% set the monospaced font
\setCJKmonofont{Adobe Heiti Std}                           
%\setCJKmonofont{Monaco}                           

%% English serif font
\setmainfont{Lucida Bright}                                 
%% English monospaced font
%\setmonofont{Courier}
\setmonofont{Monaco}                             
%\setmonofont{Consolas}                             
%% English sans serif font
\setsansfont{Optima}                                  

%% defines new font
\setCJKfamilyfont{song}{Adobe Song Std}                     
\setCJKfamilyfont{kai}{Adobe Kaiti Std}
\setCJKfamilyfont{hei}{Adobe Heiti Std}
\setCJKfamilyfont{fangsong}{Adobe Song Std}
\setCJKfamilyfont{lisu}{LiShu}
\setCJKfamilyfont{youyuan}{Adobe Kaiti Std}

%% custom English font
\newfontfamily\couriernew{Lucida Grande}
\newfontfamily\optima{Optima}
\newfontfamily\lucida{Lucida Bright}

\newcommand{\ascii}[1]{{\sffamily #1}}
\newcommand{\speak}[1]{{\itshape #1}}
% \renewcommand{\emph}[1]{{\hei #1}}
% commented the Chinese \emph to correctly display default English \emph{text}

%% custom Song
\newcommand{\song}{\CJKfamily{song}}                       
%% custom body
\newcommand{\kai}{\CJKfamily{kai}}                         
%% custom blackbody
\newcommand{\hei}{\CJKfamily{hei}}                        
%% custom imitation
\newcommand{\fangsong}{\CJKfamily{fangsong}}              
%% custom librarian
\newcommand{\lisu}{\CJKfamily{lisu}}                       
%% custom young round
\newcommand{\youyuan}{\CJKfamily{youyuan}}                 

%%%% xeCJK font setting ends
%%%%%%%% ------------------------------------------ ------------------------------

%%%%%%%% ------------------------------------------ ------------------------------
%%%% Some redefinitions about Chinese documents

%% Redefinition of the %% mathematical formula theorem

\newtheorem{example}{example}[section]                                   
\newtheorem{algorithm}{algorithm}
%% by section number
\newtheorem{theorem}{Theorem}[section]                         
\newtheorem{definition}{definition}
\newtheorem{axiom}{Axiom}
\newtheorem{property}{nature}
\newtheorem{proposition}{proposition}
\newtheorem{lemma}{lemma}
\newtheorem{corollary}{inference}
\newtheorem{condition}{conditions}
\newtheorem{conclusion}{Conclusion}
\newtheorem{assumption}{hypothesis}

\newtheorem{principle}{principle}[section]
\newtheorem{regulation}{rules}[section]
\newtheorem{advise}{recommendation}[section]
\newtheorem{concept}{concept}[section]

\usepackage{titlesec}

\renewcommand{\partname}{}
\renewcommand{\thepart}{Section \Roman{part}}

%% chapter and other name redefinition
\renewcommand{\contentsname}{catalog}
%\renewcommand{\abstractname}{Abstract}
\renewcommand{\indexname}{index}
\renewcommand{\listfigurename}{Illustration directory}
\renewcommand{\listtablename}{table directory}
\renewcommand{\figurename}{fig.~}
\renewcommand{\tablename}{table~}
\renewcommand{\appendixname}{Appendix}
\renewcommand{\appendixpagename}{Appendix}
\renewcommand{\appendixtocname}{Appendix}
%\renewcommand\refname{References} 

%% set content environment
\newenvironment{content}{
  \setlength{\parskip}{0.5\baselineskip}
  \begin{spacing}{1.5}
}{
  \end{spacing}
  \setlength{\parskip}{-0.5\baselineskip}
  \vskip -0.5\baselineskip
}

% The syntax for inserting a short story:
% \begin{story}
%   \begin{center}
% \inlinetitle{watershed}
%   \end{center}
% \end{story}
\newenvironment{story}
{
  \setlength{\parskip}{0.5\baselineskip}
  \hbox to \textwidth{\hfil\rule{\linewidth}{0.5mm}\hfil}
  \begin{spacing}{1.5}
}{%
  \end{spacing}
  \hbox to \textwidth{\hfil\rule{\linewidth}{0.5mm}\hfil}
  \setlength{\parskip}{-0.5\baselineskip}
  \vskip -0.5\baselineskip
}

%% sets the format of chapter, section and subsection
%\titleformat{\chapter}[display]{\flushright\yihao}{\thechapter{}}{1em}{\textbf}
\titleformat{\section}[block]{\flushleft\sanhao}{\optima{\thesection}}{1em}{\textbf}
\titleformat{\subsection}{\sihao}{\optima{\thesubsection}}{0.5em}{\textbf}
\titleformat{\subsubsection}{\xiaosi}{\thesubsubsection}{0.5em}{\textbf}

%\titlespacing{\chapter}{0pt}{0pt}{-\baselineskip}
%\titlespacing{\section}{0pt}{0pt}{0\baselineskip}
%\titlespacing{\subsection}{0pt}{0.5\baselineskip}{0\baselineskip}

%% set chapter format
\usepackage{quotchap}

\renewcommand\chapterheadstartvskip{
   \vspace*{-5\baselineskip}
}

\renewcommand\chapterheadendvskip{
   \vspace*{0.5\baselineskip}
}

\usepackage{helvet}
\renewcommand\sectfont{\rmfamily\bfseries}

\newcommand\refig[1] {{\itshape\figurename\ascii{\ref{fig:#1}(page \pageref{fig:#1})}}}
\newcommand\reftbl[1]{{\itshape\tablename\ascii {\ref{tbl:#1}(page \pageref{tbl:#1})}}}

\renewcommand{\footnoterule}{\vspace*{3pt}%
  \hrule width 0.382\textwidth height 0.4pt \vspace*{2.6pt}}

% Remark
\newenvironment{remark}{\par\vskip10pt\footnotesize\itshape % Vertical white space above the remark and smaller font size
\begin{list}{}{
\leftmargin=35pt % Indentation on the left
\rightmargin=25pt}\item\ignorespaces % Indentation on the right
\makebox[-2.5pt]{\begin{tikzpicture}[overlay]
\node[draw=red!60,line width=1pt,circle,fill=red!25,font=\sffamily\bfseries,inner sep=2pt,outer sep=0pt] at (-15pt,0pt){\textcolor{red}{R}};\end{tikzpicture}}
\advance\baselineskip -1pt}{\end{list}\vskip5pt}

%%%% Chinese redefinition end
%%%%%%%% ------------------------------------------ ------------------------------