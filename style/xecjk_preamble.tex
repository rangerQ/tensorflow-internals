%%%%%%%% ------------------------------------------ ------------------------------
%%%% xeCJK related package

\usepackage{xltxtra,fontspec,xunicode}

%% \ CJKsetecglue {\ hskip 0.15em plus 0.05em minus 0.05th}
%% slanfont: Allow italic
%% boldfont: Allow bold
%% CJKnormalspaces: Only whitespace between Chinese characters is ignored, but the gap between Chinese and English is preserved. 
%% CJKchecksingle: Avoid single characters in a single line.
\usepackage[slantfont, boldfont]{xeCJK} 
% \usepackage{ctex}

%% breaks the line for Chinese
\XeTeXlinebreaklocale "zh"             

%% gives TeX a certain degree of freedom
\XeTeXlinebreakskip = 0pt plus 1pt minus 0.1pt

%%%% xeCJK setting ends                                       
%%%%%%%% ------------------------------------------ ------------------------------

%%%%%%%% ------------------------------------------ ------------------------------
%%%% xeCJK font settings

%% Set Chinese punctuation style, support quanjiao, banjiao, kaiming, etc.
\punctstyle{quanjiao}                                        
                                                     
%% set default Chinese font
\setCJKmainfont[BoldFont={Adobe Heiti Std}, ItalicFont={Adobe Kaiti Std}]{Adobe Song Std}   %  FZBaoSongZ04
%% set Chinese sans serif font
\ setCJKsubscribe [BoldFont = {Adobe Head Std}, ItalicFont = {Adobe Cash Std}] {Adobe Cash Std}  
%% set the monospaced font
\ setCJKmonofont {Adobe Name Std}                            
% \ SetCJKmonofont Monaco {}                            

%% English serif font
\ setmainfont {Lucida Bright}                                  
%% English monospaced font
%\setmonofont{Courier}
\ Setmonofont Monaco {}                             
% \ setmonofont {Consoles}                              
%% English sans serif font
\setsansfont{Optima}                                   

%% defines new font
\setCJKfamilyfont{song}{Adobe Song Std}                     
\ setCJKfamilyfont {kai} {Adobe Kaiti Std}
\ setCJKfamilyfont {hey} {Adobe Name Std}
\setCJKfamilyfont{fangsong}{Adobe Song Std}
\setCJKfamilyfont{lisu}{LiShu}
\setCJKfamilyfont{youyuan}{Adobe Kaiti Std}

%% custom English font
\ newfontfamily \ couriernew {Lucida Grande}
\ Newfontfamily \ optimal} {Optima
\ newfontfamily \ shine {Lucida Bright}

\newcommand{\ascii}[1]{{\sffamily #1}}
\ newcommand {\ speak} [1] {{\ \ n \ n \ n \ n}}}
\ Renewcommand {\ emph} [1] {{\ hex # 1}}

%% custom Song
\newcommand{\song}{\CJKfamily{song}}                       
%% custom body
\ newcommand {\ kai} {\ CJKfamily {kai}}                         
%% custom blackbody
\ Newcommand {\ hi} {\ CJKfamily {hi}}                         
%% custom imitation
\newcommand{\fangsong}{\CJKfamily{fangsong}}               
%% custom librarian
\ newcommand {\ lisu} {\ CJKfamily {press}}                       
%% custom young round
\newcommand{\youyuan}{\CJKfamily{youyuan}}                 

%%%% xeCJK font setting ends
%%%%%%%% ------------------------------------------ ------------------------------

%%%%%%%% ------------------------------------------ ------------------------------
%%%% Some redefinitions about Chinese documents

Redefinition of the %% mathematical formula theorem

\newtheorem{example}{例}[section]                                   
\newtheorem{algorithm}{algorithm}
%% by section number
\newtheorem{theorem}{Theorem}[section]                         
\newtheorem{definition}{definition}
\newtheorem{axiom}{Axiom}
\newtheorem{property}{nature}
\newtheorem{proposition}{proposition}
\newtheorem{lemma}{lemma}
\newtheorem{corollary}{inference}
\newtheorem{condition}{conditions}
\newtheorem{conclusion}{Conclusion}
\newtheorem{assumption}{hypothesis}

\newtheorem{principle}{principle}[section]
\newtheorem{regulation}{rules}[section]
\newtheorem{advise}{recommendation}[section]
\newtheorem{concept}{concept}[section]

\usepackage{titlesec}

\renewcommand{\partname}{}
\renewcommand{\thepart}{第\Roman{part} section}

%% chapter and other name redefinition
\renewcommand{\contentsname}{catalog}
%\renewcommand{\abstractname}{Abstract}
\renewcommand{\indexname}{index}
\renewcommand{\listfigurename}{Illustration directory}
\renewcommand{\listtablename}{table directory}
\renewcommand{\figurename}{图}
\renewcommand{\tablename}{表}
\renewcommand{\appendixname}{Appendix}
\renewcommand{\appendixpagename}{Appendix}
\renewcommand{\appendixtocname}{Appendix}
%\renewcommand\refname{References} 

%% set content environment
\newenvironment{content}{%
  \ setlength {\ parskip} {0.5 \ baselineskip}
  \begin{spacing}{1.5}
} {%
  \end{spacing}
  \ setlength {\ parskip} {- 0.5 \ baselineskip}
  \ vskip -0.5 \ baselineskip
}

% The syntax for inserting a short story:
% \begin{story}
%   \begin{center}
% \inlinetitle{watershed}
%   \end{center}
% \end{story}
\newenvironment{story}
{
  \ setlength {\ parskip} {0.5 \ baselineskip}
  \hbox to \textwidth{\hfil\rule{\linewidth}{0.5mm}\hfil}
  \begin{spacing}{1.5}
} {%
  \end{spacing}
  \hbox to \textwidth{\hfil\rule{\linewidth}{0.5mm}\hfil}
  \ setlength {\ parskip} {- 0.5 \ baselineskip}
  \ vskip -0.5 \ baselineskip
}

%% sets the format of chapter, section and subsection
% \ Titleformat {\ Chapter} [display] {\ flushright \ Yihao} {\ thechapter} {} {1em} {\ textbf}
\titleformat{\section}[block]{\flushleft\sanhao}{\optima{\thesection}}{1em}{\textbf}
\titleformat{\subsection}{\sihao}{\optima{\thesubsection}}{0.5em}{\textbf}
\titleformat{\subsubsection}{\xiaosi}{\thesubsubsection}{0.5em}{\textbf}

%\titlespacing{\chapter}{0pt}{0pt}{-\baselineskip}
\titlespacing{\section}{0pt}{0pt}{0\baselineskip}
\ titlespacing {\ subsection} {0pt} {0.5 \ baselineskip} {0 \ baselineskip}

%% set chapter format
\usepackage{quotchap}

\renewcommand\chapterheadstartvskip{
   \ vspace * {- 5 \ baselineskip}
}

\renewcommand\chapterheadendvskip{
   \ vspace * {0.5 \ baselineskip}
}

\usepackage{helvet}
\renewcommand\sectfont{\rmfamily\bfseries}

\newcommand\refig[1]{{\itshape \figurename\ascii{\ref{fig:#1}(第\pageref{fig:#1}页)}}}
\newcommand\reftbl[1]{{\itshape \tablename\ascii{\ref{tbl:#1}(page\pageref{tbl:#1}page}}}}

\renewcommand{\footnoterule}{\vspace*{3pt}%
  \hrule width 0.382\textwidth height 0.4pt \vspace*{2.6pt}}

% Remark
\newenvironment{remark}{\par\vskip10pt\footnotesize\itshape % Vertical white space above the remark and smaller font size
\begin{list}{}{
\leftmargin=35pt % Indentation on the left
\rightmargin=25pt}\item\ignorespaces % Indentation on the right
\makebox[-2.5pt]{\begin{tikzpicture}[overlay]
\node[draw=red!60,line width=1pt,circle,fill=red!25,font=\sffamily\bfseries,inner sep=2pt,outer sep=0pt] at (-15pt,0pt){\textcolor{red}{R}};\end{tikzpicture}}
\advance\baselineskip -1pt}{\end{list}\vskip5pt}

%%%% Chinese redefinition end
%%%%%%%% ------------------------------------------ ------------------------------